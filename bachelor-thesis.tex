% !TEX root = main.tex
% !TEX option = --shell-escape
% MAIN DOCUMENT

% In case of renaming this file:
% adjust the @default_files attribute in the .latexmkrc file

% change meta text values in ./meta/parameters.tex
% save chapters and corresponding data in ./chapters/XX/
% save images in ./images/
% remove ./example/ from the project root and clear references in this file
% add imports to ./meta/packages.tex, only if necessary

% version from 2024-10



\documentclass[a4paper, oneside, 11pt]{book}
\usepackage[utf8]{inputenc}

\newif\ifdraft
\draftfalse												% flag: marks document as draft and shows to-do-notes, comment or set to "draftfalse"

% PACKAGE IMPORTS



% compilation packages
%\usepackage{morewrites}                                  			% set more write registers, default: 16
																	% -> use when: "No room for a new \write."
%\usepackage{silence}												% suppress warnings
																	% -> use when: /



% general packages
\usepackage[top=2.5cm, bottom=2cm, left=3cm, right=2cm]{geometry}   % set page margins
\usepackage[ngerman]{babel}                                         % set language
\usepackage[T1]{fontenc}                                            % set font encoding
\usepackage[nottoc,notlot,notlof]{tocbibind}                        % remove toc, lot and lof from table of contents
\usepackage[bottom]{footmisc}                                       % set footnotes to bottom of page
\usepackage{setspace}                                               % set line spacing
\usepackage{parskip}                                                % set paragraph spacing
\usepackage{graphicx}                                               % include graphics
\usepackage{wrapfig}												% enable in-text figures
\usepackage{etoolbox}                                               % patch commands
\usepackage{fancyhdr}                                               % set header and footer
\usepackage{titlesec}                                               % set title format
\usepackage[titles]{tocloft}                                        % set table of contents format
\usepackage{csquotes}                                               % set quotation marks
\usepackage[
	backend=biber,					% biber package for compilation
	citestyle=numeric,				% citation style matching: IEEE
	bibstyle=numeric,				% citation style matching: IEEE
	sorting=none,					% sorting: as in document: IEEE
	%title={Literaturverzeichnis}	% title of bibliography
]{biblatex}															% set bibliography format
\usepackage[pdftex, pdfborder={0 0 0}]{hyperref}                    % set hyperlinks
\usepackage{lmodern}                                                % set font
\usepackage{color}                                                  % set colors
\usepackage{xcolor}                                                 % set colors
\usepackage{xspace}                                                 % set spaces
\usepackage{caption}                                                % set caption format
\usepackage{tabulary}                                               % set tables
\usepackage{longtable}                                              % set tables over multiple pages
\usepackage[acronym, toc]{glossaries}						   		% set glossaries and acronyms



% TColorbox packages
% -> documentation: https://texdoc.org/serve/tcolorbox/0
\usepackage{shellesc}												% required for shell escape
\usepackage{tcolorbox}												% required for creating colored boxes
\usepackage{fontawesome}											% required for using icons
\usepackage{minted}													% required for code listing and syntax highlighting
																	% -> requires shell escape !
\tcbuselibrary{skins}												% required for creating custom skins
\tcbuselibrary{most}												% required for using most features
\tcbuselibrary{minted}												% required for using minted
\tcbuselibrary{breakable} 										 	% required for breaking boxes



% additional packages for development
\usepackage{comment}												% enable comments in source code
\ifdraft
	\usepackage{todonotes}                                          % enable notes
\else
	\usepackage[disable]{todonotes}                                 % disable
\fi
%\usepackage{showframe} 	                                        % show layout borders									% import packages
% PARAMETERS FOR METADATA



% general metadata concerning the scientific work
\newcommand{\documentCategory}{Bachelorarbeit\xspace}
\newcommand{\course}{\xspace}
\newcommand{\courseAbbreviation}{\xspace}
\newcommand{\semester}{\xspace}
\newcommand{\projectdescription}{
    Toolchain as Code Strategie zum Überwinden des\\
    Development-Deployment-Gaps im\\
    Application Management eines großen Automobilherstellers\xspace}
\newcommand{\projectshortdescription}{Von der IDE zum Service\xspace}

% flag for single or group work
% note: there is no integrity check for the students, 
%       meaning that more than one student can be set to true although the work is set to single, or vice versa
\newboolean{singleAuthor} \setbool{singleAuthor}{true}

% flags for enabling the visibility of authors
\newboolean{studentA} \setboolean{studentA}{true}
\newboolean{studentB} \setboolean{studentB}{false}
\newboolean{studentC} \setboolean{studentC}{false}
\newboolean{studentD} \setboolean{studentD}{false}
\newboolean{studentE} \setboolean{studentE}{false}
\newboolean{studentF} \setboolean{studentF}{false}

% metadata concerning the authors
\newcommand{\studentAName}{Jan Rother\xspace}
\newcommand{\studentAMatNr}{70476536\xspace}
\newcommand{\studentBName}{Studierende oder Studierender\xspace}
\newcommand{\studentBMatNr}{00000000\xspace}
\newcommand{\studentCName}{Studierende oder Studierender\xspace}
\newcommand{\studentCMatNr}{00000000\xspace}
\newcommand{\studentDName}{Studierende oder Studierender\xspace}
\newcommand{\studentDMatNr}{00000000\xspace}
\newcommand{\studentEName}{Studierende oder Studierender\xspace}
\newcommand{\studentEMatNr}{00000000\xspace}
\newcommand{\studentFName}{Studierende oder Studierender\xspace}
\newcommand{\studentFMatNr}{00000000\xspace}

% flags for enabling the visibility of tutors
\newboolean{tutorA} \setboolean{tutorA}{true}
\newboolean{tutorB} \setboolean{tutorB}{true}

% metadata concerning the tutor
\newcommand{\tutorAName}{Prof. Dr. Hans Grönniger\space}
\newcommand{\tutorBName}{Dr. Daniel Fruhner\space}

% metadata concerning the document
\newcommand{\documentDayOfWeek}{Sonntag\xspace}
\newcommand{\documentDay}{01\xspace}
\newcommand{\documentMonth}{12\xspace}
\newcommand{\documentMonthOfYear}{Dezember\xspace}
\newcommand{\documentYear}{2024\xspace}
\newcommand{\documentDate}{\documentDayOfWeek, \documentDay.\documentMonth.\documentYear\xspace}

% metadata concerning the university
\newcommand{\university}{Ostfalia Hochschule für angewandte Wissenschaften \\
                         Hochschule Braunschweig / Wolfenbüttel\xspace}
\newcommand{\faculty}{Informatik\xspace}
\newcommand{\degree}{Bachelor of Science (B.Sc.)\xspace}
\newcommand{\studies}{Informatik\xspace}

% metadata required if the document is a thesis
\newboolean{thesis} \setboolean{thesis}{true}

% metadata concerning the document manager (default: first student)
\newcommand{\firstAuthor}{\studentAName\xspace}								% set meta parameters
% COMMANDS FOR METATEXT



% title of the document (derived from document category)
\newcommand{\documentTitle}{\documentCategory\xspace}

% subtitle of the document (derived from project title)
\newcommand{\documentSubtitle}{\projectshortdescription: \\\projectdescription\xspace}

% subject of the document (derived from course and semester)
\newcommand{\documentSubject}{\documentYear\xspace}

% authors of the document (derived from student data, if set to visible)
\newcommand{\documentAuthor}{
    \ifbool{studentA}  {\studentAName, Mat.-Nr.: \studentAMatNr}{}
    \ifbool{studentB}{\\\studentBName, Mat.-Nr.: \studentBMatNr}{}
    \ifbool{studentC}{\\\studentCName, Mat.-Nr.: \studentCMatNr}{}
    \ifbool{studentD}{\\\studentDName, Mat.-Nr.: \studentDMatNr}{}
    \ifbool{studentE}{\\\studentEName, Mat.-Nr.: \studentEMatNr}{}
    \ifbool{studentF}{\\\studentFName, Mat.-Nr.: \studentFMatNr}{}
    \xspace
}

% lecturer of the document (derived from tutor)
\newcommand{\documentTutor}{\tutor\xspace}

% signature fields for single author
\newcommand*{\SignatureAndDate}[1]{
    
	\par\noindent\makebox[52mm]{\hrulefill} \hfill\makebox[65mm]{\hrulefill}
	\par\noindent\makebox[52mm][c]{Ort, Datum} \hfill\makebox[62mm][c]{#1}
    \xspace
}

% easy quotation
\newcommand{\q}[1]{\glqq#1\grqq}
\newcommand{\qbf}[1]{\textbf{\glqq#1\grqq}}
\newcommand{\qit}[1]{\textit{\glqq#1\grqq}}									% define custom commands and environments
% PAGE LAYOUT
% requires packages from "packages.tex"



\selectlanguage{ngerman}								% set language to german

\setlength{\parindent}{0em} 							% paragraph indentation to left-justified

\onehalfspacing											% set line spacing to 1.5

\makeatletter

% configure headers
\def\maxChapterTitleLength{44}							% define maximum length of chapter title in header

\patchcmd{\@makechapterhead}{\vspace*{50\p@}}{}{}{} 	% removes space above \chapter head
\patchcmd{\@makeschapterhead}{\vspace*{50\p@}}{}{}{} 	% removes space above \chapter* head
\makeatother
\setlength{\headheight}{14.5pt}							% set header height
\pagestyle{fancy}										% set page style to fancy

\renewcommand{\chaptermark}[1]{
	\noexpandarg
	\StrLen{#1}[\chapterTitleLength]

	\ifthenelse{\chapterTitleLength > \maxChapterTitleLength}{
		\StrLeft{#1}{\maxChapterTitleLength}[\title]
        \edef\title{\title \ldots}
	}{
		\def\title{#1}
	}

	\markboth{\title\ -- \chaptername\ \thechapter\ }{}	% set chapter mark
}

\renewcommand{\sectionmark}[1]{							% set section mark
	\markright{
		{\projectshortdescription}
	}
}

\fancypagestyle{titlepage}								% set title page style
{
	\setcounter{page}{-100000}
	\fancyhf{}
	\fancyfootoffset{0pt}
	\fancyheadoffset{0pt}
	\renewcommand{\headrulewidth}{0pt}
}

% set fonts of titles
\titleformat{\chapter}[display] {\sffamily \huge}{\chaptertitlename\ \thechapter}{-5pt}{\Huge}
\titlespacing*{\chapter}{0pt}{0pt}{10pt}
\titleformat{\section}[display] {\sffamily \tiny}{}{0pt}{\LARGE \thesection\ }
\titlespacing*{\section}{0pt}{0pt}{0pt}
\titleformat{\subsection}[display] {\sffamily \tiny}{}{-15pt}{\Large \thesubsection\ }
\titlespacing*{\subsection}{0pt}{0pt}{0pt}
\titleformat{\subsubsection}[display] {\sffamily \tiny}{}{-15pt}{\large \thesubsubsection\ }
\titlespacing*{\subsubsection}{0pt}{0pt}{0pt}

% set fonts of table of contents
\renewcommand{\cftchapfont}{\bf\sffamily}
\renewcommand{\cftsecfont}{\sffamily}
\renewcommand{\cftsubsecfont}{\sffamily}

% configure glossary and acronyms
% -> styles: https://www.dickimaw-books.com/gallery/glossaries-styles/
\setglossarystyle{altlistgroup}
\setacronymstyle{long-short}

% set fonts of footnotes
\renewcommand{\cfttabfont}{\sffamily}
\renewcommand{\cftfigfont}{\sffamily}

\setcounter{secnumdepth}{3}

% set metadata of document
\hypersetup{
	pdfauthor={\firstAuthor},
	pdftitle={\documentTitle\ | \documentSubtitle},
	pdfsubject={\documentSubject},
	pdfkeywords={Betreuer: \tutorAName}
}

% set code listings
\lstset{
	showspaces=false,
	showtabs=false,
	breaklines=true,
	showstringspaces=false,
	basicstyle=\ttfamily,
	frame=lt,
	rulecolor=\color{gray},
	framerule=3pt,
	xleftmargin=6pt,
}

% set code boxes
\newtcblisting{codebox}[3][]{%
	listing engine=minted,
	minted style=colorful,
	minted language=#2,
	minted options={tabsize=2,breaklines,autogobble,linenos,numbersep=3mm},
	colback=white,
	colframe=darkgray,
	listing only,
	left=5mm,
	title=\faCode\quad #3,
	enhanced,
	overlay={\begin{tcbclipinterior}\fill[lightgray] (frame.south west)
	rectangle ([xshift=5mm]frame.north west);\end{tcbclipinterior}},
	#1
}

% style captions
\newlength\myx
\setlength\myx{\textwidth}
\addtolength\myx{-2\fboxsep}
\DeclareCaptionFont{white}{\color{white} \sffamily}
\DeclareCaptionFormat{listing}{\colorbox{gray}{\parbox{\myx}{#1#2#3}}}
\captionsetup[lstlisting]{format=listing,labelfont=white,textfont=white}
\renewcommand{\lstlistingname}{Quelltext}									% set document header and page layout
% hyphenation



\hyphenation{
    De-vel-op-ment
    De-vel-op-ment-
    De-vel-op-ment-um-ge-bung
    De-vel-op-ment-um-ge-bung-en
    De-ploy-ment
    De-ploy-ment-
    De-ploy-ment-um-ge-bung
    De-ploy-ment-um-ge-bung-en
%
    For-schungs-fra-ge
    For-schungs-fra-gen
    Li-te-ra-tur-zu-sam-men-stel-lung
    Ex-per-ten-in-ter-view
    Ex-per-ten-in-ter-views
    Pro-to-ty-pen
%
    DevOps
    GitOps
    Dotfiles
    Twelve-Factor-App
%
    Tool-chain
    Tool-chains
}
								% define hyphenation rules

%\setcounter{chapter}{-1}								% reset chapter counter to start with 0

\addbibresource{literature.bib}							% import bibliography

\makeglossaries											% create glossary
\loadglsentries{./main/glossaries.tex}					% load glossary entries

\graphicspath{{images/}}								% set graphics path

\begin{document}

	%-------------------------------------------------------------------------------
	% FRONT MATTER
	%-------------------------------------------------------------------------------

	\listoftodos
	% COVER PAGE



\frontmatter
\begin{titlepage}
\thispagestyle{titlepage}

    %---------------------------------------------------------------------------

    \newgeometry{top=2cm, bottom=3.5cm, left=1.5cm, right=1.5cm}

    \hfill
    \includegraphics[scale=0.93]{./images/logos/logo_ostfalia.jpg}
    \includegraphics[scale=1.20]{./images/logos/sublogo_wf.jpg}

    \hspace{1cm}
    \begin{minipage}{\dimexpr\textwidth-1.5cm\relax}
        {\Large\textsf{Fakultät \faculty}}
    \end{minipage}

    \vfil

    %---------------------------------------------------------------------------

    \hspace{1cm}
    \begin{minipage}{\dimexpr\textwidth-1.5cm\relax}
        \hrulefill

        \vspace{2em}

        {\Large\textbf{\textsf{\documentSubject}}}

        \vspace{2em}

        {\Huge\textbf{\textsf{\documentTitle}}}

        \vspace{2em}

        {\Large\textsf{\documentSubtitle}}

        \vspace{1em}

        \hrulefill
    \end{minipage}

    %---------------------------------------------------------------------------

    \vfil

    \begingroup
    \centering
    \ifbool{thesis}{
        \textsf{
            Zur Erlangung des akademischen Grades\\
            \textbf{\degree}\\
            im Studiengang \studies\\
            an der\\
            \university\\
        }
    }{}
    \endgroup

    %----------------------------------------------------------------------------

    \vfil

    \hspace{1cm}
    \begin{minipage}{\dimexpr\textwidth-1.5cm\relax}
        {\Large\textsf{
            \ifbool{thesis}{
                vorgelegt von \\\textbf{\documentAuthor}
            }{
                \textbf{Autor\ifbool{singleAuthor}{}{en}:} \\\documentAuthor
            }}}

        \vspace{0.5cm}

        {\Large\textsf{
            \textbf{Betreuer:}
            \ifbool{tutorA}{\\\tutorAName}{}
            \ifbool{tutorB}{\\\tutorBName}{}
        }}

        \vspace{0.5cm}

        {\Large\textsf{
            \textbf{Abgabedatum:} \\\documentDate
        }}
    \end{minipage}

    \vfil

    %---------------------------------------------------------------------------

    \enlargethispage{5\baselineskip}

    \includegraphics[scale=1.20]{./images/logos/sublogo_sz-sud-wob.jpg}

    %---------------------------------------------------------------------------

\end{titlepage}

\restoregeometry

	\ifbool{thesis}{
		\clearpage

{\LARGE\textbf{Vorwort}}
\vspace{1em}

\itshape

In diesem Abschnitt kann der Hintergrund zur Arbeit beschrieben werden. Auch ist Platz für persönliche Anmerkungen des Autors oder der Autoren.

Soll eine Danksagung ausgesprochen werden, so kann dies ebenfalls in diesem Abschnitt erfolgen.

\normalfont

\vspace{1em}
\begin{flushright}
    \textbf{\firstAuthor,\xspace \documentMonthOfYear \documentYear}
\end{flushright}

		\clearpage

\begin{center}
    {\LARGE\textbf{Kurzfassung}}
\end{center}
\vspace{1em}

Modernes Software Engineering wird durch eine Vielzahl technischer Werkzeuge und organisatorischer Methoden unterstützt, welche Entwicklerinnen und Entwickler aktiv bei der Erstellung hochqualitativer Software begleiten. In der Praxis führt dies jedoch zu einer Menge neuer Herausforderungen. Schnell wächst die Komplexität von Entwicklungs- und Bereitstellungsumgebungen. In beiden dieser Bereiche, Development und Deployment, bestehen unterschiedliche und teils sehr spezifische Anforderungen. Steigt die Menge verwendeter Technologien und Werkzeuge, die Toolchain eines Projekts, so vergrößert sich auch die Lücke zwischen den beiden Bereichen. Dieses im Rahmen dieser Arbeit als \q{Development-Deployment-Gap} bezeichnete Phänomen macht die Entwicklung zeitaufwändig und fehleranfällig.

Motiviert durch ein Projekt im Application Management eines großen Automobilherstellers beschäftigt sich die vorliegende Arbeit mit der Frage, welche Ansätze bereits bestehen, um diesem Trend entgegenzuwirken und welche Anforderungen Entwicklerinnen und Entwickler an eine Lösung stellen. Auf Basis von Literaturrecherche und Experteninterviews wird eine \q{Toolchain-as-Code Strategie} entwickelt, deren Ziel eine Verlagerung und Zentralisierung von Konfigurationen in Software Repositories ist, wodurch eine Durchgängigkeit von Toolchains erreicht werden soll. Im Kontext von Webentwicklung und Microservices Architekturen sowie unter Berücksichtigung von Sicherheits-, Wartbarkeits- und Performanzaspekten sollen Best Practices für die Entwicklung mit Software Containern erarbeitet werden. Eine auf die Erkenntnisse der Arbeit aufbauende, prototypische Umsetzung der Strategie soll Realisierbarkeit und Mehrwerte aufzeigen sowie gleichermaßen einen Leitfaden zur Implementierung bieten.

\vfil

\begin{center}
    {\LARGE\textbf{Abstract}}
\end{center}
\vspace{1em}

Modern software engineering relies on a wide range of technical tools and organizational methods that support developers in building high-quality software, yet this variety also introduces challenges, particularly with growing complexity in development and deployment environments. Each area, development and deployment, comes with its own set of sometimes very specific requirements, and as projects adopt more technologies and tools, a \q{development-deployment-gap} can emerge, making processes more time-consuming and prone to errors.

Based on a project in the application management division of a major automotive manufacturer, this study examines current strategies to address this gap and identifies developer's key requirements for an effective solution. Using a combination of literature review and expert interviews, this work develops a \q{toolchain-as-code} strategy, that moves and centralizes configuration of tools in software repositories, leading to an improvement of overall toolchain consistency. This thesis aims to establish best practices for development with software containers in the context of web development and microservice architectures, while taking security, maintainability, and performance into account. A prototype based on the findings of this study will demonstrate the strategy’s feasibility and value, providing a practical guide for implementation.

	}{}

	\chapter*{Erklärung}

\ifbool{singleAuthor}{
    Hiermit versichere ich, die vorliegende Arbeit selbständig verfasst und keine anderen als die angegebenen Quellen und Hilfsmittel verwendet zu haben. Ich versichere weiterhin, alle wörtlich oder sinngemäß aus anderen Quellen übernommenen Aussagen als solche gekennzeichnet zu haben. Dies gilt explizit auch für die Verwendung von text- oder codegenerierenden Werkzeugen der Künstlichen Intelligenz.
    Die eingereichte Arbeit ist weder vollständig noch in wesentlichen Teilen Gegenstand eines anderen Prüfungsverfahrens gewesen.
    Ich habe zur Kenntnis genommen, dass die Arbeit einer elektronischen Plagiatsprüfung unterzogen werden kann.
}{
    Hiermit versichern wir, die vorliegende Arbeit selbständig verfasst und keine anderen als die angegebenen Quellen und Hilfsmittel verwendet zu haben. Wir versichern weiterhin, alle wörtlich oder sinngemäß aus anderen Quellen übernommenen Aussagen als solche gekennzeichnet zu haben. Dies gilt explizit auch für die Verwendung von text- oder codegenerierenden Werkzeugen der Künstlichen Intelligenz.
    Die eingereichte Arbeit ist weder vollständig noch in wesentlichen Teilen Gegenstand eines anderen Prüfungsverfahrens gewesen.
    Wir haben zur Kenntnis genommen, dass die Arbeit einer elektronischen Plagiatsprüfung unterzogen werden kann.
}

\vspace*{3em}

\ifbool{studentA}{\SignatureAndDate{- \studentAName\ -}}{} \\\\
\ifbool{studentB}{\SignatureAndDate{- \studentBName\ -}}{} \\\\
\ifbool{studentC}{\SignatureAndDate{- \studentCName\ -}}{} \\\\
\ifbool{studentD}{\SignatureAndDate{- \studentDName\ -}}{} \\\\
\ifbool{studentE}{\SignatureAndDate{- \studentEName\ -}}{} \\\\
\ifbool{studentF}{\SignatureAndDate{- \studentFName\ -}}{} \\

	\tableofcontents

	\printglossary[type=\acronymtype, title=Abkürzungsverzeichnis, style=listdotted]
	\printglossary[type=\glsdefaulttype, title=Glossar] % style: see meta/header.tex

	%-------------------------------------------------------------------------------
	% CONTENT
	%-------------------------------------------------------------------------------

	\mainmatter
	
	\chapter{Einleitung und Motivation}
\label{ch:01_introduction-and-motivation}

\section{Hintergrund}
\label{sec:01-01_background}

Der Begriff \Gls{developer-experience} ist definiert als die Menge aller Erfahrungen, die Entwickler von Software mit verschiedenen Arten von Artefakten und Aktivitäten machen, die während der Entwicklung von Software durchlaufen werden \cite{017:Developer-Experience-Concept-and-Definition,100:Developer-Experience-Glueckliche-Entwickler-schreiben-besseren-Code}. \q{Experience} als Teilbegriff bezieht sich dabei nicht auf die Erfahrenheit der Beteiligten, sondern vielmehr darauf, wie ihre subjektive Wahrnehmung der Arbeit an einem Projekt ist. Darunter fallen Bereiche wie die Entwicklungsinfrastruktur, aber auch Zusammenarbeit und Wertschätzung \cite{017:Developer-Experience-Concept-and-Definition}. Wird die Infrastruktur betrachtet, so spielen viele Faktoren eine Rolle, unter ihnen Entwicklungs- und Verwaltungswerkzeuge, Programmiersprachen, Programmierbibliotheken, Plattformen und Methoden \cite{100:Developer-Experience-Glueckliche-Entwickler-schreiben-besseren-Code}. Dabei umfasst die kognitive Dimension von \Gls{developer-experience} die Wahrnehmung dieser Entwicklungsinfrastruktur. Hierzu zählen beispielsweise die Interaktionen mit Entwicklungswerkzeugen oder die Ausführung von Softwareprozessen \cite{017:Developer-Experience-Concept-and-Definition}. Ein Indikator, der in gewissem Maß als Metrik für \Gls{developer-experience} dienen kann, ist die Zeit, die benötigt wird, um Änderungen am Quellcode in der daraus resultierenden Software widerzuspiegeln. Dieser Wert hat einen Einfluss auf die Geschwindigkeit, mit der Entwickler Fehler identifizieren und beheben können \cite{100:Developer-Experience-Glueckliche-Entwickler-schreiben-besseren-Code}.

Bei Softwareentwicklung handelt es sich um eine komplexe und kognitiv herausfordernde Aktivität bestehend aus vielen verschiedenen Phasen \cite{014:Managing-Container-based-Software-Development-Environments}. In den letzten Jahren lassen sich immer wieder rasante Entwicklungen in dieser Disziplin beobachten. Das Feld beschreibt nicht mehr nur die Entwicklung von Quellcode, sondern vielmehr die Erstellung ganzer Produkte. Hinter dem Begriff Software Engineering steht eine komplexe Wertschöpfungskette. Insbesondere das \nameref{sec:02-01_web-development} entwickelt sich in einer rasanten Geschwindigkeit, deutlich steiler als das Software Engineering selbst \cite{026:Some-Trends-in-Web-Application-Development}. Die starken Entwicklungen digitaler Informationstechnologien bleiben jedoch nicht folgenlos, sondern tragen maßgeblich zu einem Anstieg der Komplexität in der Softwareentwicklung bei. Software selbst wird mittlerweile als das komplexeste Artefakt moderner Computertechnik bezeichnet. Dies hat zur Konsequenz, dass Entwicklungsinfrastrukturen permanent angepasst werden und Softwareumgebungen nur noch selten ausreifen \cite{018:Software-Development-Productivity}.

Die Softwarekonzerne \textit{GitKraken} und \textit{JetBrains} haben eine Untersuchung zur Zusammenarbeit in Entwicklungsteams durchgeführt (siehe \autoref{fig:g-00_git-collaboration-report}). Ergebnis sind unter anderem signifikante Hürden, die die Produktivität von Entwicklern negativ beeinflussen. Über ein Drittel der Entwickler benannte zu viele Kontextwechsel als größte Herausforderung \cite{213:2024-State-of-Git-Collaboration}. Um welche konkrete Art von Kontext es sich handelt, geht aus der Umfrage nicht hervor. Sowohl der Wechsel organisatorischen als auch technischen Kontexts kann also ein Hindernis sein. Diesbezüglich zeigt \autoref{fig:g-00_git-collaboration-report} außerdem, dass Probleme mit Infrastruktur oder den verwendeten Technologien als produktivitätshämmende Faktoren aufgeführt wurden \cite{213:2024-State-of-Git-Collaboration}.

\pagebreak[4]

\begin{figure}[h]
    \centering
    \begin{minipage}[b]{0.39\textwidth}
        \centering
        \includegraphics[width=\textwidth]{g-00_git-collaboration-report.png}
        \caption{Ergebnis des Git Collaboration Report 2024 zu Hindernissen in der Softwareentwicklung \cite{213:2024-State-of-Git-Collaboration}}
        \label{fig:g-00_git-collaboration-report}
    \end{minipage}
    \hfill
    \begin{minipage}[b]{0.59\textwidth}
        \centering
        \includegraphics[width=\textwidth]{g-01_stack-overflow-developer-survey-2024.png}
        \caption{Ergebnis der Stack Overflow Developer Survey 2024 zu typischen Problempunkten bei der Softwareentwicklung \cite{212:Developer-Survey}}
        \label{fig:g-01_stack-overflow-developer-survey-2024}
    \end{minipage}
\end{figure}

Die größte weltweite Gemeinschaft von privaten und professionellen Softwareentwicklern, \textit{Stack Overflow}, führt jährlich eine Umfrage zu vielen Bereichen des Software Engineering durch. Alleine 2024 haben über 65 Tausend Entwickler teilgenommen und Fragen aus insgesamt sieben Bereichen beantwortet. Dass ein eigener Bereich für \Gls{developer-experience} erst 2022 neu eingeführt wurde, zeigt deutlich den Anstieg an Relevanz in diesem Thema \cite{212:Developer-Survey}. In der Umfrage aus 2024 (siehe \autoref{fig:g-01_stack-overflow-developer-survey-2024}) gaben jeweils etwa 33 \% der Entwickler an, dass die Komplexität der Technologien für den \Gls{build} von Software und jene für das \Gls{deployment} zu Frustration bei der Entwicklung führten. Ähnlich viele empfänden Frustration bei der Verlässlichkeit von Werkzeugen und Systemen und etwa 23 \% geben die alleinige Anzahl der verwendeten Softwarewerkzeuge als Problem an \cite{206:Developer-Survey-2024}. Diese Werkzeuge spielen eine elementare Rolle in der Softwareentwicklung, wo sie Entwickler bei der Schaffung qualitativ hochwertiger Software unterstützen. Zu ihnen zählen unter anderem \glspl{ide} und \glspl{vcs} \cite{014:Managing-Container-based-Software-Development-Environments}. Verfügbar sein müssen solche Werkzeuge teilweise auf den verschiedenen Ebenen der Softwareentwicklung, wie Entwicklungs"=, Test"= oder, seltener, auch Produktivumgebungen. Dass sie dort jeweils in unterschiedlichem Umfang und unterschiedlicher Konfiguration eingesetzt werden, führt zu einem Neu- oder Rekonfigurationsbedarf nach jedem Kontextwechsel \cite{003:Infrastructure-from-Code}.

Neue Komplexität verlangt nach neuen Lösungen und Entwicklungsmethodiken haben sich im Laufe der Zeit vielfältig entwickelt und gewandelt. Eine Umfrage \textit{State of Developer Ecosystem} von \textit{JetBrains} richtet sich an Entwickler weltweit. 2023 gab über die Hälfte von ihnen an, an der Entwicklung von Infrastruktur beteiligt zu sein, ein Fünftel von ihnen habe eine Schlüsselrolle in diesem Bereich \cite{207:Developer-Ecosystem}. Während früher Wasserfallmodelle in der Softwareentwicklung üblich waren, werden Softwareteams zunehmend agiler. Vorgehensmodelle und Methodiken wie Scrum, \hyperref[sec:03-01_devops]{DevOps} oder \hyperref[sec:03-03_gitops]{GitOps} verändern Arbeitsweisen und Prozesse. \textit{Stack Overflow} ermittelte 2024, dass 69 \% der Entwickler \gls{ci} und \gls{cd} nutzen sowie 58 \% im Bereich \hyperref[sec:03-01_devops]{DevOps} unterwegs seien \cite{206:Developer-Survey-2024}. \hyperref[sec:03-03_gitops]{GitOps} ist eine relativ neue Methodik.

Der \textit{Hype Cycle of Emerging Technologies} des Marktforschungsunternehmens \textit{Gartner} liefert jährlich Daten zur Bewertung des Reifegrads und der Akzeptanz von Technologien. Unterstützen sollen diese Daten unter anderem Führungskräfte bei der Entwicklung nachhaltiger Unternehmensstrategien. Die Kurve des \textit{Gartner Hype Cycles} ist unterteilt in fünf verschiedene Phasen, die jede der auf der Kurve platzierten Technologien in eine Phase ihres Lebenszyklus einordnen. Gibt es einen Durchbruch in einem Forschungsgebiet oder löst ein Proof-of-Concept eine Öffentlichkeitswirkung aus, wird dies als \textbf{Technologischer Auslöser} (englisch \qit{Innovation Trigger}) bezeichnet. Die nächste Phase ist der \textbf{Gipfel der überzogenen Erwartungen} (englisch \qit{Peak of inflated Expactations}), während welchem öffentlich intensiv über Erfolge und Misserfolge berichtet wird. Einige Unternehmen beginnen in dieser Phase Investitionen in eine Technologie. Flachen die Erfolgs- und Misserfolgsmeldungen sowie in deren Folge auch das öffentliche Interesse ab, befindet sich die Technologie im \textbf{Tal der Enttäuschungen} (englisch \qit{Trough of Disillusionment}). Die vierte Phase, der \textbf{Pfad der Erleuchtungen} (englisch \qit{Slope of Enlightenment}) bringt ein besseres Technologieverständnis und mehr Unternehmensinvestitionen mit sich. Schlussendlich erreicht eine Technologie das \textbf{Plateau der Produktivität} (englisch \qit{Plateau of Productivity}). Die Technologie hat sich etabliert und ist gereift, erste Investitionen zahlen sich aus \cite{108:Gartner-Hype-Cycle}.

Im Jahr der Erstellung dieser Arbeit identifizierte der \textit{Gartner Hype Cycle 2024} (siehe \autoref{fig:g-02_gartner-hype-cycle-2024}) den großen Bereich Entwicklerproduktivität. Unter diesem Begriff befindet neben \gls{cloud-native}, Entwicklerportalen, \Gls{prompt-engineering} und Web Assembly auch \hyperref[sec:03-03_gitops]{GitOps} kurz vor dem Höhepunkt der Kurve, dem \textbf{Gipfel der überzohenen Erwartungen} \cite{106:Gartner-2024-Hype-Cycle-for-Emerging-Technologies}. Es ist also möglich, dass diese neue Technologie sich durchsetzt, die letzte Lebenszyklusphase erreicht und sich in Unternehmen etabliert.

\begin{figure}[h]
    \centering
    \includegraphics[width=0.75\textwidth]{g-02_gartner-hype-cycle-2024.png}
    \caption{Gartner Hype Cycle for Emerging Technologies 2024 \cite{106:Gartner-2024-Hype-Cycle-for-Emerging-Technologies}}
    \label{fig:g-02_gartner-hype-cycle-2024}
\end{figure}

Unter dem Überbegriff Entwicklerproduktivität benennt \textit{Gartner} unter anderem Produktivität und \Gls{developer-experience}. Diese Konzepte sollen die Zufriedenheit und Zusammenarbeit von Entwicklern erhöhen. Von möglichen neuen Technologien in diesem Feld wird erwartet, dass sie die Qualität von Sofwtareprodukten schnell und nachhaltig verbessern \cite{107:Spotlight-on-2024-Gartner-Hype-Cycle-for-Emerging-Technologies}.

Einen weiteren, ähnlich großen Einfluss auf das Software Engineering haben die Technologien der Cloud und Virtualisierung. Auch sie verändern Prozesse und Methoden, insbesondere beim Einsatz im Bereich der Entwicklung, also dem tatsächlichen Schreiben von Quellcode \cite{014:Managing-Container-based-Software-Development-Environments}.

\section{Projektkontext}
\label{sec:01-02_project-context}

% place content here
Inhalt

\section{Zielsetzung und Forschungsfragen}
\label{sec:01-03_objectives-and-research-questions}

Für die im Abschnitt \nameref{sec:01-01_background} beschriebenen Probleme und Herausforderungen wird der Begriff \qbf{Development-Deployment-Gap} eingeführt. \Gls{development} \glsdesc{development}. Hier besteht der größte Kontaktpunkt mit Entwicklern. \Gls{deployment} \glsdesc{deployment}. Der Begriff wird in Literatur und Praxis häufig verschieden definiert beziehungsweise interpretiert. Im Kontext dieser Arbeit meint er die Ablage erfolgreich gebauter Artefakte an einem bestimmten Ort, aber explizit nicht ihre Bereitstellung für mögliche Kunden.

In allen diesen Bereichen spielen die ebenfalls im \nameref{sec:01-01_background} angesprochenen Werkzeuge eine große Rolle. Sie werden im folgenden als \qbf{Tools} bezeichnet und meinen Software oder Softwarekomponenten, die Entwickler auf allen möglichen Ebenen zwischen \Gls{development} und \Gls{deployment} nutzen. Tools unterstützen sie bei einzelnen oder mehreren Aufgaben. Bei einer \qbf{Toolchain} handelt es sich um den Verbund mehrerer Tools, die in einem gleichen Kontext oder Projekt genutzt werden. Software und Tools formen zusammen Umgebungen: \qbf{Developmentumgebungen} im \Gls{development} und \qbf{Deploymentumgebungen} im \Gls{deployment}.

Ziel der Arbeit ist die Entwicklung einer \qbf{Toolchain-as-Code Strategie}. Die Erwartung an eine solche Strategie ist ihr Beitrag zur Beseitigung von Hindernissen und Problemräumen, zur Erhöhung von Produktivität und Zufriedenheit der Entwickler sowie zur Erreichung einer gewissen Durchgängigkeit von Tools zwischen Development- und Deploymentumgebungen.

Die Zielerreichung soll dabei über den Fokus auf vier zentrale \Glspl{rq} gelingen:

\vspace{1em}
\begin{table}[H]
    \centering
    \begin{tabular}{p{0.1\textwidth} p{0.8\textwidth}}
        \textbf{RQ-0} & Welche Anforderungen an Development- und Deploymentumgebungen haben Entwickler und Administratoren, die diese Bereiche im Rahmen von Softwareprojekten mit Microservices Architektur verantworten? \\[2em]
        \textbf{RQ-1} & Welche Ansätze für das Application Management bestehen bereits, um Software sicher, wartbar, performant und in kurzen Abständen zu entwickeln und auszuliefern? Welche Grenzen haben diese Ansätze gegebenenfalls im Kontext des Ziels dieser Arbeit? \\[2em]
        \textbf{RQ-2} & Wie kann eine Toolchain-as-Code Strategie bestehende Ansätze kombinieren, um die Lücke zwischen \Gls{development} und \Gls{deployment} zu schließen, manuelle Schritte im Rahmen der Toolchain zu reduzieren und dabei gleichermaßen eine effiziente Konfiguration von Softwareprojekten zu ermöglichen? \\[2em]
        \textbf{RQ-3} & Was sind Best Practices beim Einrichten von Development- und Deploymentumgebungen, die Sicherheit, Wartbarkeit und Performanz begünstigen? \\
    \end{tabular}
\end{table}
\vspace{1em}

Diese Fragen sollen zur Entwicklung einer Strategie wie beschrieben beitragen.

\section{Abgrenzung}
\label{sec:01-04_demarcation}

% place content here
Inhalt

\section{Methodik und Aufbau}
\label{sec:01-05_methodology-and-structure}

\subsection{Wissenschaftlicher Ansatz}
\label{subsec:01-05-01_scientific-approach}

Die Forschungen in dieser Arbeit sollen sich auf drei zentrale methodische Elemente stützen:

\begin{itemize}
    \begin{minipage}[t]{0.75\textwidth}
        \item Eine \textbf{Literaturzusammenstellung} als initialer Berührungspunkt mit dem Forschungsgebiet und zur Ermittlung einer Ausgangssituation.
        \item \textbf{Experteninterviews} zur Herleitung von Anforderungen an die durch die spätere Strategie abzudeckenden Bereiche.
        \item Die Entwicklung eines \textbf{Prototypen} zur Validierung des Ansatzes und um einen ersten Leitfaden zur Implementierung der Strategie zu liefern.
    \end{minipage}
\end{itemize}

\begin{comment}
    (A) Einführung -> 01, 02
    (B) DevOps und GitOps -> 03; RQ-1
    (C) Dev Container und Developer Experience -> 03; RQ-2
    (D) Anforderungen an Development und Deployment Umgebungen -> 04; RQ-0
    (E) Experten-Interview -> 04
    (F) Sicherheit, Wartbarkeit und Performanz von Docker Images -> 05; RQ-3

    (0) Paper -> 30 Stk.
    (1) Artikel -> 12 Stk.
    (2) Journale oder Blogs -> 14 Stk.
    (3) Dokumentationen -> 11 Stk.
    (4) Bücher -> 02 Stk.

    - Durchlauf eines Peer-Reviews
    - Veröffentlichung in einem renommierten Journal oder einer renommierten Konferenz
    - Aktualität
    - Methodik
    - Transparenz der Daten und Quellen
    - Anzahl an Zitationen in anderer Literatur
\end{comment}

Die \textbf{Literaturzusammenstellung} soll insbesondere die Erkenntnisse von \autoref{ch:02_technological-environment} und \autoref{ch:03_examination-of-existing-approaches} stützen. Die Methodik wird an dieser Stelle vorgegriffen. Als Hauptquelle dient die Literaturdatenbank \textit{IEEE Xplore} (\url{https://ieeexplore.ieee.org/}). Bei zu wenigen Ergebnissen über diese Quelle, wird zusätzlich \textit{Google Scholar} (\url{https://scholar.google.com}) herangezogen. Technische Themen, insbesondere solche, die sehr praxisnah oder sehr jung sind, werden häufig besser über \textit{Google} (\url{https://www.google.com/}) gefunden. Die Suchanfragen werden dabei auf zentralen Stichworten der Arbeit aufgebaut und zuvor entlang dem \nameref{subsec:01-05-02_structure-of-the-thesis} sowie unter Berücksichtigung von \nameref{sec:01-03_objectives-and-research-questions} gruppiert. Die Gruppe \textsc{(A) Einführung} stützt die Abschnitte zu \nameref{ch:01_introduction-and-motivation} (\autoref{ch:01_introduction-and-motivation}) und \nameref{ch:02_technological-environment} (\autoref{ch:02_technological-environment}). Die Gruppen \textsc{(B) DevOps und GitOps} sowie \textsc{(C) Dev Container und Developer Experience} liefern Ergebnisse für die \nameref{ch:03_examination-of-existing-approaches} (\autoref{ch:03_examination-of-existing-approaches}) und zur Beantwortung der Forschungsfragen \textsc{(B)} \textbf{RQ-1} beziehungsweise \textsc{(C)} \textbf{RQ-2}. Zur Ermittlung von \nameref{ch:04_requirements-for-development-and-deployment-environments} (\autoref{ch:04_requirements-for-development-and-deployment-environments}) soll Gruppe \textsc{(D) Anforderungen an Development und Deployment Umgebungen} mit Gruppe \textsc{(E) Experten-Interview} als wissenschaftliche Grundlage der methodischen Unterstützung dienen. \textsc{(F) Sicherheit, Wartbarkeit und Performanz von Docker Images} soll einen Beitrag zur \nameref{ch:05_toolchain-as-code} Strategie (\autoref{ch:05_toolchain-as-code}) und zur Beantwortung der Forschungsfrage \textsc{(F)} \textbf{RQ-3} leisten. Auch die Ergebnisse der Literaturzusammenstellung wurden während der Forschung gruppiert, sodass folgende Aussage getroffen werden kann: Insgesamt wurden 70 wissenschaftliche Quellen für die Literaturzusammenstellung herangezogen, unter ihnen \textsc{(0)} 31 \textsc{Paper}, \textsc{(1)} 12 \textsc{Artikel}, \textsc{(2)} 14 \textsc{Journale oder Blogs}, \textsc{(3)} 11 \textsc{Dokumentationen} sowie \textsc{(4)} 02 \textsc{Bücher}. Der Anteil ihrer Verwendung in der Arbeit konnte dabei über ein Schulnotensystems (von 1 bis 6) erfolgen, wobei Quellen mit der Note 6 nicht eingeflossen sind. Faktoren wie das Durchlaufen eines Peer-Reviews, die Veröffentlichung in einem renommierten Journal oder einer renommierten Konferenz, die Aktualität, die Methodik, die Transparenz der Daten und Quellen sowie die Anzahl an Zitationen in anderer Literatur wurden dabei berücksichtigt.

Für die \textbf{Experteninterviews} wird ein leitfadengestütztes Interview vorbereitet. Mit insgesamt vier Experten soll dieses Interview anschließend durchgeführt werden, wobei ein Protokoll in Stichpunkten angelegt werden soll. Die konsolidierten Ergebnisse sollen anschließend bei der Herleitung von Anforderungen an die \nameref{ch:05_toolchain-as-code} Strategie dienen. Die genaue Methodik wird zuvor im Abschnitt \nameref{subsec:04-01-02_methodology} (\autoref{subsec:04-01-02_methodology}) erarbeitet werden.

Ziel des \textbf{Prototypen} ist die Erstellung eines minimalen Projekts auf Basis der erhobenen Anforderungen und der entwickelten Strategie. Dem vorgelagert wird eine begründete Auswahl von \nameref{sec:06-01_technologies-and-tools} (\autoref{sec:06-01_technologies-and-tools}) stehen. Abschließend soll es möglich sein, das entwickelte Konzept anhand des Prototypen zu bewerten.

Zwei Herangehensweisen an wissenschaftliche Arbeiten sind unter anderem möglich: ein deduktiver und ein induktiver Ansatz (siehe \autoref{fig:g-03_methodology-of-inductive-approach}). Im \textbf{deduktiven Ansatz} ist das Ziel die Aufstellung von Theorien und Hypothesen. Typische Bestandteile sind häufig Motivation, Abgrenzung, Grundlagenaufbereitung und Hypothesenbildung. Wird der \textbf{induktive Ansatz} gewählt, sind die aufgestellten Theorien und Hypothesen anhand einer praktischen Umsetzung kritisch zu reflektieren. Nach Ausdetaillierung des entworfenen Konzepts folgen dessen Umsetzung und Anwendung. So kann das erhaltene Produkt vor dem Fazit bewertet werden \cite{400:Beitrag-Produktrepraesentation-fuer-Bedarfs-und-Kapazitaetsmanagement-digitalisierter-Fahrzeuge}.

\begin{figure}[h]
    \centering
    \includegraphics[width=0.95\textwidth]{g-03_methodology-of-inductive-approach.png}
    \caption{Methodisches Vorgehen nach dem induktiven Ansatz \acrshort{iAa} \citeauthor{400:Beitrag-Produktrepraesentation-fuer-Bedarfs-und-Kapazitaetsmanagement-digitalisierter-Fahrzeuge}}
    \label{fig:g-03_methodology-of-inductive-approach}
\end{figure}

Die Umsetzbarkeit einer \nameref{ch:05_toolchain-as-code} Strategie soll möglichst deutlich und reproduzierbar dargestellt werden. Auch Praktikabilität und Nutzen sollen aus ersten Erfahrungen abgeleitet werden. Daher wurde unter anderem die Methodik des Prototypen als ein Bestandteil der Arbeit gewählt und deshalb wird diese Arbeit einem induktiven Ansatz wie in \autoref{fig:g-03_methodology-of-inductive-approach} folgen.

\subsection{Aufbau der Arbeit}
\label{subsec:01-05-02_structure-of-the-thesis}

Sowohl dieser wissenschaftliche Ansatz als auch die gewählten Methodiken finden sich deutlich im Aufbau der Arbeit und in der Kapitelstruktur wieder, wie \autoref{fig:g-04_structure-of-thesis} zeigt.

\begin{figure}[h]
    \centering
    \includegraphics[width=0.95\textwidth]{g-04_structure-of-thesis.png}
    \caption{Aufbau und Struktur der Arbeit}
    \label{fig:g-04_structure-of-thesis}
\end{figure}

Das aktuelle \textbf{\autoref{ch:01_introduction-and-motivation}} enthält die \nameref{ch:01_introduction-and-motivation} zu dieser Arbeit. Dazu führt es zunächst in die Problemstellung ein, nennt und motiviert erste Teilthemen und erarbeitet aus den identifizierten Problemräumen vier zentrale Forschungsfragen. Außerdem enthält es Erläuterungen zum methodischen und wissenschaftlichen Vorgehen.

\pagebreak[4]

\textbf{\autoref{ch:02_technological-environment}} skizziert daraufhin ein \nameref{ch:02_technological-environment} und stützt sich stark auf die \textbf{Literaturzusammenstellung}. Wichtig in diesem Abschnitt sind eine wissenschaftliche Betrachtung des Projektumfelds sowie die Begründungen hinter der Wahl einzelner Teiltechnologien. Es enthält Zusammenfassungen der Bereiche \nameref{sec:02-01_web-development}, \nameref{sec:02-02_microservices} und \nameref{sec:02-03_containerization}.

Das \textbf{\autoref{ch:03_examination-of-existing-approaches}} widmet sich der \nameref{ch:03_examination-of-existing-approaches}. Auch hier basieren die zusammengestellten Inhalte auf der \textbf{Literaturzusammenstellung}. Es geht um eine kritische Untersuchung erster bereits existierender Ansätze, welche Lösungen für die Problemstellungen der Arbeit bieten. Betrachtet werden die Vor- und Nachteile von \hyperref[sec:03-01_devops]{DevOps} und \hyperref[sec:03-03_gitops]{GitOps}. Außerdem gibt es eine Einführung in das Konzept der \hyperref[sec:03-04_dotfiles]{Dotfiles} und in die \hyperref[sec:03-05_concept-of-twelve-factor-app]{\q{Twelve-Factor-App}}.

Auf den Erkenntnissen aufbauend werden \nameref{ch:04_requirements-for-development-and-deployment-environments} in \textbf{\autoref{ch:04_requirements-for-development-and-deployment-environments}} ermittelt. Als Grundlage hierfür sollen \textbf{Experteninterviews} dienen. Herausgefunden werden soll, welche Anforderungen Entwickler an Toolchains in den Bereichen \Gls{development} und \Gls{deployment} sowie an die Durchgängigkeit dieser Toolchains stellen. Dazu werden Interviewfragen ausgearbeitet und die Ergebnisse strukturiert ausgewertet.

Das wichtigste Kapitel dieser Arbeit ist \textbf{\autoref{ch:05_toolchain-as-code}}. In \nameref{ch:05_toolchain-as-code} entsteht das Hauptergebnis der Forschungen (siehe \autoref{fig:g-03_methodology-of-inductive-approach}). Dazu werden zunächst bisherige Ergebnisse zusammengestellt, welche anschließend durch zusätzliche Recherchen zu konkreten Technologien für \nameref{sec:02-03_containerization} ergänzt werden. Ergebnis des Kapitels ist eine ausgearbeitete \nameref{ch:05_toolchain-as-code} Strategie inklusive zugehöriger Architektur und Prozessen.

Diese Strategie soll als \textbf{Prototyp} in \textbf{\autoref{ch:06_prototypical-implementation-of-the-concept}} umgesetzt werden. Auf eine Festlegung von Basistechnologien für die Realisierung einer minimalen Toolchain auf Basis der entworfenen Ansätze folgt dort die \hyperref[ch:06_prototypical-implementation-of-the-concept]{prototypische Umsetzung des Konzepts}. Auf Grundlage der hierbei getroffenen Erkenntnisse soll eine Kurzbewertung der \nameref{ch:05_toolchain-as-code} Strategie ermöglicht werden.

\textbf{\autoref{ch:07_conclusion-and-outlook}} ist das letzte Kapitel und enthält \nameref{ch:07_conclusion-and-outlook}. Hier werden die Forschungsfragen aus \autoref{sec:01-03_objectives-and-research-questions} beantwortet. Außerdem sollen im Ausblick nicht betrachtete Aspekte des Forschungsgebiets benannt werden und es wird eine Einschätzung zu weiteren Entwicklungen auf dem Forschungsgebiet getroffen.


	\chapter{Technologisches Umfeld}
\label{ch:02_technological-environment}

\section{Web-Development}
\label{sec:02-01_web-development}

% place content here
Inhalt

\section{Microservices}
\label{sec:02-02_microservices}

Der Begriff Microservices beschreibt einen Architekturstil in der Softwareentwicklung. Ein System folgt dieser Architektur, wenn es sich aus mehreren Komponenten zusammensetzt, welche als Services bezeichnet werden. Jeder dieser Services ist verantwortlich für genau eine Aufgabe oder einen kleinen Funktionsblock. Solche Systeme nutzen leichtgewichtige Kommunikationsprotokolle zum Austausch von Daten untereinander \cite{027:Containerized-Microservices-Deployment-Approach}. Alle Services sind unabhängig voneinander. Dies hat Auswirkungen auf \Gls{development}, \Gls{deployment} und die Skalierbarkeit von Systemen \cite{028:Analyzing-Microservices-and-Monolithic-Systems}.

Das bereits in \autoref{ch:01_introduction-and-motivation} angesprochene Softwareunternehmen \textit{JetBrains} entwickelt Tools für Softwareentwickler weltweit. Zum Zeitpunkt dieser Arbeit verzeichnet es mehr als 11 Millionen Nutzer ihrer \Glspl{ide}, unter ihnen 287 Tausend Geschäftskunden. Entwickelt werden Produkte von \textit{JetBrains} durch über zweitausend Mitarbeitende an 13 internationalen Standorten \cite{210:Jet-Brains-Company}. Die \textit{JetBrains State of Developer Ecosystem} Umfrage erreichte im Jahr 2023 knapp über 23 Tausend Entwickler. Für den Bereich Microservices ergab sich, dass ein Drittel der Teilnehmenden an ihnen beteiligt sind. Fast alle von ihnen gaben an, auch einen entsprechenden Designansatz zu verwenden. Drei Viertel der Befragten organisieren die Kommunikation der einzelnen Services über \textit{\Gls{rest}} oder \textit{\Gls{rpc}}. Insgesamt greifen nur 20 \% der Teilnehmenden noch auf ein monolithisches Backend zurück \cite{207:Developer-Ecosystem}.

Bei monolithischen Architekturen befinden sich alle Funktionalitäten einer Applikation in einer einzelnen \Gls{codebase}. Eine \Gls{codebase} \glsdesc{codebase}. Das wichtigste Merkmal von Monolithen ist die enge Kopplung ihrer einzelnen Systemkomponenten. Eine steil anwachsende Komplexität solcher Systeme wird schnell zu einer Herausforderung und entschleunigt die Evolution einer Applikation \cite{028:Analyzing-Microservices-and-Monolithic-Systems}. Monolithen führen zwangsläufig zu einem sehr aufwändigen \Gls{deployment} und sind aufgrund ihrer Größe oft unflexibel. Diese Nachteile motivieren den Einsatz von Microservices als Alternative. Sie ermöglichen beispielsweise einen deutlich effizienteren Einsatz von Pipelines für \Gls{cicd} \cite{019:Advanced-DevOps-Environment-for-Microservices-based-Applications}. Weitere Vorteile von Microservices sind eine schnellere Bereitstellung von Software sowie eine deutlich stärkere Autonomie der Komponenten \cite{019:Advanced-DevOps-Environment-for-Microservices-based-Applications,027:Containerized-Microservices-Deployment-Approach}. Sie folgen einem Domain-Driven Design \cite{019:Advanced-DevOps-Environment-for-Microservices-based-Applications} und sind vor allem skalierbarer als monolithische Ansätze \cite{019:Advanced-DevOps-Environment-for-Microservices-based-Applications,027:Containerized-Microservices-Deployment-Approach,028:Analyzing-Microservices-and-Monolithic-Systems}. Die genannten Faktoren haben einen positiven Einfluss auf die Wartbarkeit von Microservices. Das bedeutet jedoch nicht, dass bei solchen verteilten Systemen keine zusätzliche Komplexität entsteht. Auch bei diesem Ansatz kann sie schnell zu einem operationalen Mehraufwand führen. Daher erfordert eine erfolgreiche Implementierung von Microservices Architekturen gute Softwareteams, die in der Lage sind, diese Komplexität zu managen.

Wirft man den Blick auf große Unternehmen, so fällt auf, dass einige weiterhin monolithische Ansätze verfolgen. In Bezug auf Datendurchsatz und Mehrläufigkeit werden Microservices mitunter von Monolithen übertroffen. Unter anderem \textit{Stack Overflow} oder \textit{Amazon} machen sich diese bessere Performanz zunutze. Dennoch vertrauen viele Konzerne, unter ihnen \textit{Amazon}, \textit{Netflix} und \textit{Spotify} auf Microservices und etablieren diese Architekturen für ihre Software \cite{028:Analyzing-Microservices-and-Monolithic-Systems}.

Noch besser einsetzbar sind Microservices in Verbindung mit \nameref{sec:02-03_containerization}.

\section{Containerization}
\label{sec:02-03_containerization}

\qit{Container sind leichtgewichtige, isolierte Umgebungen, welche eine Applikation inklusive ihrer Abhängigkeiten paketieren und es ihr dadurch ermöglichen, konsistent einsetzbar zu sein, auch über verschiedene Umgebungen hinweg.} \cite{023:Setting-up-CI-CD-Pipeline-in-the-Cloud-for-Web-Application}

Historisch war ein Grund für die Entwicklung solcher Technologien ein Anstieg an Komplexität in den Ökosystemen von Softwareentwicklern. Zunächst wurden virtuelle Maschinen genutzt, um dieses Problem anzugehen, jedoch bieten Container Technologien eine deutlich schlankere Alternative \cite{014:Managing-Container-based-Software-Development-Environments}. Mittlerweile finden sie daher eine starke Verwendung in \nameref{sec:02-02_microservices} Architekturen \cite{014:Managing-Container-based-Software-Development-Environments} und Cloudumgebungen \cite{025:Exploring-Solutions-for-Container-Image-Security}.

Die \textit{JetBrains State of Developer Ecosystem} deckte 2023 auch den Bereich \hyperref[sec:03-01_introduction-to-devops]{DevOps} ab und fand heraus, dass mit 54 \% mehr als die Hälfte der teilnehmenden Entwickler \textit{Docker} in der Softwareentwicklung nutzen \cite{207:Developer-Ecosystem}. \textit{Docker} ist eine Software, die Containerization ermöglicht und sich in diesem Bereich weitestgehend durchgesetzt hat. Knapp ein Drittel, nämlich 60 \% der Entwickler, nutzen \textit{Docker Compose}, 19 \% legen ihre damit gebauten Artefakte in \textit{Docker Hub} ab \cite{207:Developer-Ecosystem}. Dabei handelt es sich um eine \Gls{container-registry}, dies \glsdesc{container-registry}. Mit 18 \% auch häufig genutzt ist die \textit{GitHub Container Registry}.

Auch in der Literatur zeigt sich ein deutlicher Trend in Richtung von \textit{Docker}. Eine systematische Übersichtsarbeit zu Containern in der Softwareentwicklung von \citeauthor{015:Containers-in-Software-Development}, durchgeführt an zwei finnischen Universitäten, lieferte ähnliche Ergebnisse. 57 \% der dort identifizierten Paper thematisieren \textit{Docker}. Die Motivation hinter dieser Studie war kontinuierliche Softwareentwicklung. \cite{015:Containers-in-Software-Development}

\textit{Docker} selbst beschreibt in diesem Rahmen einen \q{Container-First-Approach}, bei welchem Container jeden einzelnen Kontaktpunkt mit Software beeinflussen: Infrastruktur, Anwendungen, \hyperref[sec:03-01_introduction-to-devops]{DevOps}, Sicherheit, Plattform und viele mehr werden genannt. Zentrale Vorteile dieses Ansatzes sind Standardisierung, Isolation, Wiederholbarkeit und Konsistenz von Softwareentwicklung. Entwicklungsumgebungen sind schneller einsatzbereit und Entwicklern bleibt mehr Zeit für Innovation. Dies hat eine Einsparung von Zeitressourcen für Wartung und Infrastruktur zur Folge. \cite{016:Effectively-managing-all-of-those-Applications} Die Leichtgewichtigkeit von Containern, nicht nur im \Gls{development}, sondern auch im \Gls{deployment} machen sie prädestiniert für den Einsatz in Cloudumgebungen \cite{015:Containers-in-Software-Development,024:Investiugating-Impact-of-Containerization-on-Deployment-Process-in-DevOps,025:Exploring-Solutions-for-Container-Image-Security}. Sie ermöglichen reproduzierbare Umgebungen für Software über alle Ebenen hinweg \cite{013:Role-of-Containers-in-Reproducibility,024:Investiugating-Impact-of-Containerization-on-Deployment-Process-in-DevOps}. Container besitzen die Fähigkeit, konsistente Produktiv- und Testumgebungen bereitzustellen, sie sind modular, portabel und sie ermöglichen Automatisierung und Abstraktion \cite{014:Managing-Container-based-Software-Development-Environments}.

Die Summe aller genannten Vorteile überzeugt auch Unternehmen und Organisationen, wo Containerization immer stärker Einzug halten wird. Noch 2020 lag der Anteil an Containerization in Produktivumgebungen bei etwa 30 \%, bis 2025 prognostizieren \citeauthor{020:Assessing-and-Improving-Quality-of-Docker-Artifacts} einen Anstieg auf 85 \%. \cite{020:Assessing-and-Improving-Quality-of-Docker-Artifacts}


	\chapter{Untersuchung bestehender Ansätze}
\label{ch:03_examination-of-existing-approaches}

\section{DevOps}
\label{sec:03-01_devops}

Zusammen über die Hälfte der Entwickler gaben 2023 an, in in die Entwicklung von Infrastruktur (43 \%) und in DevOps (12 \%) involviert zu sein \cite{207:Developer-Ecosystem}. DevOps ist eine Methodik im Software Engineering, deren Ziel es ist, die Lücke zwischen den beiden Bereichen \Gls{development}, also der Entwicklung von Quellcode, und Operations, also dem Betrieb der entwickelten Software, zu schließen. Dabei legt sie klare Schwerpunkte auf Kommunikation und Zusammenarbeit, \acrlong{ci}, Qualitätssicherung und automatisiertes Deployment. DevOps selbst ist jedoch keine spezifische Methodik, sondern viel mehr ein Verbund aus verschiedenen Praktiken. \cite{001:DevOps-Adoption-in-Software-Development}

DevOps betrachtet stets den gesamten Prozess der Erschaffung von Software, zentrale Strategien bauen fast immer auf agilen Konzepten auf \cite{001:DevOps-Adoption-in-Software-Development}. Die zentrale Mission ist die Auslieferung von Software in kürzeren Abständen und in weniger Zeit \cite{006:Prevalence-of-GitOps-DevOps-in-Fast-CI-CD-Cycles}. Dies soll gelingen, obwohl \Gls{development} und Operations oft gegenläufige Ziele haben: Das \Gls{development} möchte möglichst schnell neue Funktionalitäten und Änderungen in der Software für den Kunden bereitstellen, während Operations Wert auf zuverlässige und sichere Software legen, welche jedoch durch regelmäßige Änderungen vulnerabler wird. Unter DevOps werden die Aufgaben und Verantwortungen für Software innerhalb eines Teams geteilt. Das betrifft alle Bereiche, von \Gls{development} bis \Gls{deployment}, wodurch auch \Gls{development} und Operations enger zusammenarbeiten sollen. \cite{000:CI-CD-Deployment-in-DevOps-reduce-Gap-Developer-Operation}

Die Integration zwischen \Gls{development} und \Gls{deployment} muss eine kontinuierliche sein. Ein Paper von \citeauthor{005:Continous-Software-Engineering-and-Beyond} identifiziert drei Bereiche, in denen sogenannte \q{Continuous}-Aktivitäten vorkommen: Business Strategy, \Gls{development} und Operations -- also von der Planung, über die Entwicklung bis hin zum Betrieb sollte Kontinuität gegeben sein. \cite{005:Continous-Software-Engineering-and-Beyond} Hier zeigt sich, dass Continuous-Aktivitäten eine wichtige Komponente von DevOps sind \cite{000:CI-CD-Deployment-in-DevOps-reduce-Gap-Developer-Operation}. Eine Aktivität ist \q{continuous}, wenn sie ein gleiches Muster konsistent und systematisch wiederholt, in DevOps typischerweise einzelne Schritte der Entwicklung und Implementierung, des Betriebs oder der Qualitätssicherung \cite{007:Analysis-of-Declarative-and-Pull-based-Deployment-Models-on-GitOps}. Eine der bekanntesten Aktivitäten, die diese Kriterien erfüllt, ist \acrfull{ci}. Sie umfasst zusammenhängende Schritte wie die Kompilierung von Quellcode, die Ausführung von Tests, die Prüfung auf Einhaltung von Standards und das Bauen von Paketen für den \Gls{deployment}-Bereich. Eine große Rolle spielt die Regelmäßigkeit der Integration. Je häufiger integriert wird, desto schneller erhalten Entwickler Feedback zu getätigten Änderungen. Schlägt sie fehl, so sollten Artefakte wie zum Beispiel Logs möglichst transparent und übersichtlich bereitgestellt werden. Dies unterstützt Entwickler dabei, in kurzer Zeit Lösungen für die Ursachen des Problems zu finden und es zu beheben. Häufigere \Glspl{release}, eine gesteigerte Vorhersagbarkeit und eine verbesserte Kommunikation in produktiveren Entwicklungsteams sind nur ein paar der Vorteile, die DevOps im Software Engineering hat. \acrfull{cd} und \acrfull{cde} können auf \Gls{ci} folgen. \acrshort{cd} stellt die gebaute und validierte Software automatisch in einer für sie vorkonfigurierten Umgebung bereit und \acrshort{cde} macht sie zusätzlich für den Kunden verfügbar. Dabei ist \Gls{deployment} eine zwingende Vorbedingung für \Gls{delivery}, aber nicht notwendigerweise vice versa. Beide sind abhängig von den Artefakten der (erfolgreichen) Integration. \cite{005:Continous-Software-Engineering-and-Beyond} Die Kombination aus \Gls{ci} und \Gls{cd} ist \acrfull{cicd}. \Gls{cicd} ist ein sehr weit verbreitetes Konzept und zählt zu den Best Practices von DevOps. Eine typische Organisation von Softwareaktivitäten entlang eines DevOps Prozesses kann \autoref{fig:g-05_devops-workflow} entnommen werden.

\begin{figure}[h]
    \centering
    \includegraphics[width=0.95\textwidth]{g-05_devops-workflow.png}
    \caption{DevOps Prozessablauf (vereinfacht) \acrshort{iAa} \citeauthor{008:GitOps-Approach-to-Cloud-Cluster-System-Deployment}}
    \label{fig:g-05_devops-workflow}
\end{figure}

Der DevOps Ansatz folgt einigen ihm zugrundeliegenden Prinzipien, an denen sich bei der Implementierung orientiert werden kann. Softwareentwicklung mit DevOps findet in Iterationen und mit Inkrementen statt. Wiederkehrende Aufgaben werden automatisiert und Teams werden befähigt, alle notwendigen Aufgaben entlang des Lebenszyklus der Software selbst auszuführen, was eine Alternative zur Verteilung der Verantwortung darstellt und breit aufgestellte Entwickler hervorbringt. Kollaboration ist ein weiteres wichtiges Prinzip. \cite{009:GitOps-Evolution-of-DevOps} Folgen Teams diesen Prinzipien, kann eine Implementierung des Ansatzes gelingen und eine Reihe von Vorteilen mit sich bringen, von denen sowohl Entwickler als auch Kunden profitieren. DevOps erhöht die Frequenz mit der Software ausgeliefert werden kann, ohne zu Einschnitten in der Qualität des Produkts zu führen. Tatsächlich kann diese sich bei einem konsequenten \Gls{ci} Konzept sogar steigern. Insbesondere bei Administration und Wartung kann mit einer nicht unwesentlichen Zeitersparnis gerechnet werden, wodurch sich auch die Kosteneffizienz verbessert. Kommt es zu unerwarteten Fehlern, ist außerdem das Zurückrollen der Software auf eine vorherige, stabile Version deutlich einfacher. Die Einbindung der Continous-Praktiken ist für die Erreichung aller Vorteile unumgänglich. Sie können zum Beispiel durch Automatisierung von Pipelines umgesetzt werden. Dabei sind \acrfull{ci}, \acrfull{cd} und \acrfull{cde}, in dieser Reihenfolge, auf den ersten drei Plätzen bezüglich ihrer Verbreitung. \cite{001:DevOps-Adoption-in-Software-Development}

Das Konzept \Gls{iac} kann ebenfalls Teil von DevOps sein und kommt häufig in der Literatur vor \cite{001:DevOps-Adoption-in-Software-Development}. Es handelt sich dabei um ein Best Practice von DevOps, welches die Verwaltung und Provisionierung der Infrastruktur mit Hilfe von Code ermöglicht. Vorteile von \Gls{iac} sind unter anderem konsistente und skalierbare Infrastrukturlösungen, Versionierbarkeit involvierter Konfigurationen und Reproduzierbarkeit von Ergebnissen. \cite{012:Compare-and-Contrast-various-Software-Development-Methodologies}

Neben diesen Konzepten gibt es weitere Best Practices bei der Verwendung von DevOps, darunter auch Continuous Monitoring, welches sich unter den fünf meist erwähnten in der Literatur befindet. Es beschreibt vor allem Feedbackmechanismen. Entwicklungsteams werden dadurch befähigt, kontinuierlich zu iterieren und Verbesserungen der Applikation einzupflegen. Continous Monitoring hat dadurch einen positiven Einfluss auf die Robustheit von Systemen. \cite{012:Compare-and-Contrast-various-Software-Development-Methodologies}

DevOps ist mittlerweile eine Methodik, die in den meisten modernen Unternehmen zum Einsatz kommt \cite{020:Assessing-and-Improving-Quality-of-Docker-Artifacts}. Konzerne wie \textit{Google}, \textit{Apple} oder \textit{Amazon} haben die beschriebenen Praktiken erfolgreich implementiert und profitieren von ihnen \cite{001:DevOps-Adoption-in-Software-Development}.

\section{Grenzen von DevOps}
\label{sec:03-02_limitations-of-devops}

% place content here
Inhalt

\section{GitOps}
\label{sec:03-03_gitops}

Anders als bei \hyperref[sec:03-01_devops]{DevOps} stehen in GitOps, welches ursprünglich von \textit{Weaveworks} entwickelt wurde, klare Beschreibungen des Konzepts zur Verfügung. Dafür ist das Modell jedoch noch recht jung und unerforscht \cite{009:GitOps-Evolution-of-DevOps}. GitOps stellt die Weichen für einen deklarativen und automatisierten Ansatz, der auf die Verwaltung von Infrastruktur, die Verbesserung der Skalierbarkeit sowie die Reduzierung divergenter Konfigurationen anwendbar sein soll. Zentrales und namensgebendes Element von GitOps ist das \Gls{vcs} \textit{\Gls{git}}. Es hat insofern eine große Bedeutung für das Konzept, als dass sich alle Konfigurationen der Infrastruktur und Applikation im Repository befinden \cite{024:Investiugating-Impact-of-Containerization-on-Deployment-Process-in-DevOps}.

Aktuelle Entwicklungen zeigen auch für den \hyperref[sec:03-01_devops]{DevOps} Bereich einen klaren Trend zu GitOps, welches zunehmend an Popularität gewinnt \cite{024:Investiugating-Impact-of-Containerization-on-Deployment-Process-in-DevOps}. Der \textit{Gartner Hype Cycle for Emerging Technologies} betonte im Jahr 2024 die Steigerung der Produktivität von Entwicklern als einen von vier großen technologischen Trends. GitOps ist auf der Kurve (siehe \autoref{fig:g-02_gartner-hype-cycle-2024}) weit oben gelistet. \textit{Gartner} prognostiziert, dass bereits in zwei Jahren oder weniger der so genannte \qit{Peak of inflated Expectations} erreicht sein könnte \cite{106:Gartner-2024-Hype-Cycle-for-Emerging-Technologies}, welchen eine Technologie durchlaufen muss, um das \qit{Plateau of Productivity} erreichen zu können \cite{108:Gartner-Hype-Cycle}. Mit GitOps verwandte oder durch GitOps verwendete Tools erleben eine ähnliche Entwicklung. Die Anzahl der Nutzer auf \textit{\Gls{github}} ist steigend und auch die Verwendung des \Gls{cicd} Tools \textit{\Gls{github-actions}} wächst \cite{008:GitOps-Approach-to-Cloud-Cluster-System-Deployment}.

GitOps eignet sich besonders gut für die Verwendung bei \gls{cloud-native} Applikationen. Erleichtert wird die Einführung in Entwicklungsteams dadurch, dass hauptsächlich Tools zum Einsatz kommen, mit denen die Entwickler bereits vertraut sind, als wichtigstes unter ihnen selbstverständlich \textit{\Gls{git}} \cite{109:GitOps}. \textit{\Gls{git}} wird in der Philosophie von GitOps als \q{Single Source of Truth} betrachtet. Das Repository eines Projekts beschreibt den gewünschten Zustand von Infrastruktur, Abhängigkeiten und Toolchain (nachfolgend \q{(Software-)Umgebung} genannt), sodass Änderungen jederzeit zentralisiert vorgenommen werden können \cite{007:Analysis-of-Declarative-and-Pull-based-Deployment-Models-on-GitOps,010:Efficient-Application-Deployment-GitOps-for-Faster-and-Secure-CI-CD-Cycles,109:GitOps}. Eine Herangehensweise an das \Gls{deployment} von Software ist der imperative Ansatz, bei welchem eine bestimmte Spezifikation hintereinander, beispielsweise durch Entwickler oder \Gls{cicd} Pipelines, ausgeführt wird. Der deklarative Ansatz von GitOps grenzt sich ganz bewusst von imperativem \Gls{deployment} ab. Während bei letzterem die einzelnen Schritte in Form konkreter Anweisungen angegeben werden, wird in einem deklarativen Ansatz lediglich der gewünschte Zielzustand beschrieben. Dadurch kommt dieses Vorgehen oft mit einem Zehntel der benötigten \Gls{loc} aus \cite{007:Analysis-of-Declarative-and-Pull-based-Deployment-Models-on-GitOps}.

GitOps kommt mit einer Menge qualitativer Mehrwerte. Ein wichtiger Beitrag zum \Gls{deployment} ist die Erhöhung von Geschwindigkeit und Häufigkeit, ohne dass ein Wechsel der Tools erforderlich wird \cite{008:GitOps-Approach-to-Cloud-Cluster-System-Deployment,109:GitOps}. Dies ist deshalb möglich, weil verschiedene Versionen der Konfiguration auf unterschiedliche Zielumgebungen, Kunden oder Anforderungen abgestimmt und bereitgestellt werden können, wobei sich zugleich mehrere Projekte eine Konfiguration teilen können \cite{008:GitOps-Approach-to-Cloud-Cluster-System-Deployment}. Die unkomplizierte Wiederherstellung vorheriger Stände einer Umgebung ist über Revisionen in \textit{\Gls{git}} möglich \cite{008:GitOps-Approach-to-Cloud-Cluster-System-Deployment,010:Efficient-Application-Deployment-GitOps-for-Faster-and-Secure-CI-CD-Cycles,109:GitOps}. Die Verwaltung von \textit{Credentials} ist deutlich einfacher, da beim \Gls{deployment} die Tools auf das Repository zugreifen \cite{008:GitOps-Approach-to-Cloud-Cluster-System-Deployment,109:GitOps} und der Bedarf entfällt, dass Entwickler selbst Zugang zu Deploymentumgebungen benötigen \cite{010:Efficient-Application-Deployment-GitOps-for-Faster-and-Secure-CI-CD-Cycles}. Insgesamt beeinflusst die Implementierung von GitOps die Sicherheit beim \Gls{deployment} der Software posititv \cite{008:GitOps-Approach-to-Cloud-Cluster-System-Deployment,010:Efficient-Application-Deployment-GitOps-for-Faster-and-Secure-CI-CD-Cycles,109:GitOps}. Entwickler haben einen besseren und transparenteren Überblick über ein \Gls{deployment}, dessen deklarative Konfiguration überwiegend selbst-dokumentierend ist \cite{008:GitOps-Approach-to-Cloud-Cluster-System-Deployment,109:GitOps}. Klare Commit Messages können als erweiterte Dokumentation dienen \cite{008:GitOps-Approach-to-Cloud-Cluster-System-Deployment} und jede Änderung der Konfiguration im Repository führt automatisch zu einem neuen Commit \cite{010:Efficient-Application-Deployment-GitOps-for-Faster-and-Secure-CI-CD-Cycles}, wodurch eine lückenlose und dauerhaft abrufbare Historie der Softwareumgebung entsteht \cite{008:GitOps-Approach-to-Cloud-Cluster-System-Deployment,010:Efficient-Application-Deployment-GitOps-for-Faster-and-Secure-CI-CD-Cycles,109:GitOps}. Sowohl ihre Evolution als auch ihre Beschreibung ist für das gesamte Entwicklungsteam sichtbar \cite{109:GitOps}. Oftmals sind Repositories aus Diensten wie \textit{\Gls{github}} sehr intuitiv integrierbar in andere Entwicklungswerkzeuge, beispielsweise in Projektmanagementtools wie \textit{Atlassian Jira} \cite{008:GitOps-Approach-to-Cloud-Cluster-System-Deployment}. Generell führt die größere Unabhängigkeit des Designs von spezifischen Tools zu mehr Flexibilität, Austauschbarkeit und (Re-)Kombinierbarkeit von Toolchains \cite{010:Efficient-Application-Deployment-GitOps-for-Faster-and-Secure-CI-CD-Cycles}.

Auch quantitativ lassen sich Vorteile von GitOps benennen. Neben der Reduktion von \Gls{loc} in der gesamten Konfiguration eines Projekts sowie in einzelnen Dateien können Präzision und Struktur der \Gls{codebase} erhöht werden. GitOps fordert eine logischere Anordnung von Verzeichnis- und Dateistrukturen, was zu einer besseren Lesbarkeit und damit zu weniger Einarbeitungszeit auf Seite der Entwickler beiträgt. Messbar sind diese Vorteile über Metriken wie die Summe aller Dateien oder die maximale Tiefe des Verzeichnisbaums eines Projekts \cite{008:GitOps-Approach-to-Cloud-Cluster-System-Deployment}.

GitOps kann als Ergänzung des \hyperref[sec:03-01_devops]{DevOps} Ansatzes verstanden werden. Es komplettiert die einzelnen Phasen des Prozesses aus \autoref{fig:g-05_devops-workflow} durch die Abgrenzung verschiedener Bereiche und die Festlegung konkreter Verantwortlichkeiten. Dieser verbesserte Prozessablauf wird in \autoref{fig:g-06_gitops-workflow} sichtbar:

\begin{figure}[h]
    \centering
    \includegraphics[width=0.95\textwidth]{g-06_gitops-workflow.png}
    \caption{GitOps Prozessablauf (vereinfacht) \acrshort{iAa} \citeauthor{008:GitOps-Approach-to-Cloud-Cluster-System-Deployment}}
    \label{fig:g-06_gitops-workflow}
\end{figure}

Die strenge Definition von GitOps nennt zwei in den Prozess involvierte Repositories als Untergrenze. Ein \textbf{Application Repository} enthält den Quellcode der Applikation sowie die Manifeste für ihr \Gls{deployment}. In einem \textbf{Environment Configuration Repository} sind alle aktiven Manifeste abgelegt, die zum \Gls{deployment} auf die aktuelle Zielumgebung notwendig sind. Weitere Repositories können bei Bedarf, beispielsweise im Falle mehrerer Services mit getrennter \Gls{codebase}, beliebig ergänzt werden \cite{109:GitOps}.

Grundsätzlich werden bei GitOps zwei Ansätze für deklaratives \Gls{deployment} unterschieden:

\begin{itemize}
    \item \textbf{Push-based} Deployment und
    \item \textbf{Pull-based} Deployment.
\end{itemize}

Ausgangspunkt eines Durchlaufs der Pipelines sind in beiden Fällen Änderungen am Application Repository. Bis zum Ablegen der vollständig gebauten Software beziehungsweise ihrer Umgebung in Form von Container \Glspl{image} in einem Repository sind beide Ansätze identisch. Zunächst lösen Änderungen am Application Repository, beispielsweise in Form eines Pushs über das \Gls{vcs} oder eines Merge Requests die \Gls{build} Pipeline aus. Diese stellt die resultierenden Container \Glspl{image} in einer \Gls{container-registry} bereit und aktualisiert anschließend die verwandten Einträge im Environment Repository \cite{007:Analysis-of-Declarative-and-Pull-based-Deployment-Models-on-GitOps}. Der Unterschied liegt nun darin, wie Applikation und Umgebung auf Änderungen reagieren und wodurch diese Änderungen erkannt werden.

\begin{figure}[h]
    \centering
    \includegraphics[width=0.95\textwidth]{g-07_gitops-push-based-deployment.png}
    \caption{GitOps Push-based Deployment \acrshort{iAa} \citeauthor{109:GitOps}}
    \label{fig:g-07_gitops-push-based-deployment}
\end{figure}

Beim \textbf{Push-based Deployment} (siehe \autoref{fig:g-07_gitops-push-based-deployment}) lösen diese Änderungen am Deployment Repository eine Deployment Pipeline aus. Diese stellt daraufhin die gesamte Umgebung, wie sie durch die Konfiguration vorgegeben ist, bereit \cite{109:GitOps}.

\begin{figure}[h]
    \centering
    \includegraphics[width=0.95\textwidth]{g-08_gitops-pull-based-deployment.png}
    \caption{GitOps Pull-based Deployment \acrshort{iAa} \citeauthor{109:GitOps}}
    \label{fig:g-08_gitops-pull-based-deployment}
\end{figure}

Das \textbf{Pull-based Deployment} (siehe \autoref{fig:g-08_gitops-pull-based-deployment}) verfügt über einen zusätzlichen Operator, welcher in regelmäßigen Abständen die Umgebung überwacht. Dazu prüft er einen Soll-Stand aus der \Gls{image} Registry und dem Environment Repository sowie optional den Ist-Stand der aktuellen Umgebung. Im Fall von Abweichungen, ist er eigenständig in der Lage, den Soll-Stand wiederherzustellen. Hierzu kann er auch aktuelle \Glspl{image} aus der \Gls{image} Registry anwenden \cite{109:GitOps}.

Beim Push-based Deployment ist es notwendig, dass die Pipeline über Credentials zur Zielumgebung verfügt. Änderungen an der Umgebung bei diesem Ansatz werden außerdem nur dann vorgenommen, wenn sich das Application Repository ändert. Dahingegen sind Umgebungen, die auf einem Pull-based Deployment Ansatz basieren, deutlich stabiler gegenüber Änderungen in beide Richtungen. Ein großer Nachteil ist dann jedoch die Komplexität einer Pull-based Architektur, die eines zusätzlichen Operators bedarf \cite{109:GitOps}.

\hyperref[sec:03-01_devops]{DevOps} und GitOps teilen sich Konzepte wie Inkremente, Kontinuität oder Automatisierung. \hyperref[sec:03-01_devops]{DevOps} betrachtet eher die Zusammenarbeit zwischen \Gls{development} und Operations, wohingegen GitOps die Entwickler fokussiert \cite{009:GitOps-Evolution-of-DevOps}. Beide Ansätze schreiben keine konkreten Tools vor.

\section{Dotfiles}
\label{sec:03-04_dotfiles}

Verglichen mit \hyperref[sec:03-01_devops]{DevOps} oder \hyperref[sec:03-03_gitops]{GitOps} beschreiben Dotfiles ein relativ kompaktes Konzept. Der Grundgedanke soll an dieser Stelle dennoch eingeführt werden, da er eine wichtige Rolle bei der Konfiguration von Entwicklungsumgebungen spielt.

Das Konzept stammt ursprünglich aus Betriebssystemen basierend auf \textit{Unix} oder \textit{Linux}, wo Dateinamen einen Punkt (\texttt{.}, englisch \q{Dot}) als Prefix enthalten können. Die eigentliche Funktion dieses Prefix ist, dass Dateien mit führendem Punkt im Dateinamen standardmäßig aus der Ausgabe des Befehls \texttt{ls} zum Anzeigen der Dateien und Ordner eines bestimmten Verzeichnisses ausgeschlossen werden. Dabei hat sich durchgesetzt, dass diese Art von Dateien oftmals dazu verwendet wird, um die Umgebung eines Entwicklers oder Nutzers zu konfigurieren. \cite{029:Connecting-the-Dotfiles}

Der Schwerpunkt von Dotfiles liegt also auf Individualisierungsaspekten, ihr Ziel ist die Verbesserung der \Gls{developer-experience}. Sie bestehen in der Regel aus typischen Nutzerindividualisierungen, Konfigurationen für Dienstprogramme und plattformbezogenen Einstellungen. Im Verbund mit einem Installationsskript kann eine komplette Umgebung durch das Ausführen einer Befehlszeile aufgesetzt werden. Dabei enthalten Dotfiles die eigentlichen Konfigurationen, während das Installationsskript die Befehle zum Kopieren der Dotfiles an ihre Zielorte im Dateisystem beinhaltet. \cite{203:Dev-Environment-as-a-Code-with-DevContainers-Dotfiles-and-GitHub-Codespaces} Üblicherweise haben diese Installationsskripte einen der folgenden konventionellen Dateinamen: \texttt{install.sh}, \texttt{install}, \texttt{bootstrap.sh}, \texttt{bootstrap}, \texttt{script/bootstrap}, \texttt{setup.sh}, \texttt{setup}, \texttt{script/setup}. Sind Dotfiles in einem Repository abgelegt, können sie beispielsweise auf \Gls{github} gespeichert und verwaltet werden. \Gls{github} ist in der Lage, automatisch einen dieser Dateinamen zu erkennen und das Installationsskript selbst auszuführen, wenn es in ein von \Gls{github} bereitgestelltes Computing Environment injiziert wird. \cite{304:Personalizing-GitHub-Codespaces-for-your-Account} Einer der größten Vorteile von Dotfiles ist die Reproduzierbarkeit von Systemen und deren Konfiguration \cite{029:Connecting-the-Dotfiles}.

Eine Untersuchung zu den am häufigsten vorkommenden Dateinamen in öffentlichen Dotfile Repositories von \citeauthor{029:Connecting-the-Dotfiles} ergab, dass die häufigsten \Gls{mime} Typen \texttt{text/plain} und \texttt{image/x} sind, wobei \texttt{x} für die verschiedensten Subtypen steht. Neben \texttt{README.md} für die Dokumentation der Repositories sind \texttt{.gitignore} zur Angabe von Dateien in einem \textit{\Gls{git}} Repository, welche vom \Gls{vcs} ignoriert werden sollen, \texttt{.vimrc} zur Konfiguration eines Dateieditors und \texttt{.zshrc} zur Konfiguration eines Befehlszeileninterpreters die verbreitetsten Dateinamen. Generell stellen \citeauthor{029:Connecting-the-Dotfiles} fest, dass die häufigsten Dateien \texttt{.*ignore}"=, \texttt{.*rc}"= oder \texttt{.*conf*}-Dateien sind. \cite{029:Connecting-the-Dotfiles}

Eingerichtet werden können mit Hilfe von Dotfiles beispielsweise \textit{Unix}- und \textit{Linux}-Systeme, das \Gls{wsl} oder \nameref{subsec:05-01-02_dev-container} \cite{203:Dev-Environment-as-a-Code-with-DevContainers-Dotfiles-and-GitHub-Codespaces}, welche in \autoref{subsec:05-01-02_dev-container} genauer vorgestellt werden.

Befragte der oben genannten Studie von \citeauthor{029:Connecting-the-Dotfiles} gaben an, dass sie Dotfile Repositories zu 53 \% nutzen würden, um schnell neue Maschinen aufzusetzen oder bestehende zu synchronisieren. Das gilt für physische genauso wie für virtuelle Maschinen. Ebenfalls etwa die Hälfte nutzten hauptsächlich \textit{\Gls{git}} als Tool zur Verwaltung von Dotfiles. Die Umfrage richtete sich an insgesamt 1.650 Autoren öffentlicher Dotfile Repositories auf \Gls{github}. \cite{029:Connecting-the-Dotfiles}

\section{Konzept der Twelve-Factor-App}
\label{sec:03-05_concept-of-twelve-factor-app}

Die \q{Twelve-Factor-App} ist eine Methodik für Webanwendungen, so genannte Software-as-a-Service Applikationen. Sie ist unabhängig von der verwendeten Programmiersprache und auch für Applikationen bestehend aus mehreren Services anwendbar. \cite{101:The-Twelve-Factor-App}

Eine Twelve-Factor-App erfüllt die folgenden von \citeauthor{103:Creating-Cloud-native-applications-12-Factor-Applications} beschriebenen Eigenschaften \cite{103:Creating-Cloud-native-applications-12-Factor-Applications}:

\begin{itemize}
    \item Zur Automatisierung von Softwareprozessen verwendet sie ein \textbf{deklaratives Paradigma}.
    \item Sie ermöglicht die \textbf{Portabilität} der Software über verschiedene Umgebungen hinweg.
    \item \textbf{Deploybarkeit} in modernen Cloud Plattformen ist gegeben.
    \item Sie \textbf{reduziert Divergenzen} zwischen Entwicklungs- und Produktivumgebungen.
    \item Ohne größere Anpassungsbedarfe kann sie beliebige \textbf{Skalierbarkeit} erreichen.
\end{itemize}

Um diese Eigenschaften zu erfüllen, werden zwölf namensgebende Faktoren beschrieben, die in der Applikation umgesetzt sein müssen \cite{101:The-Twelve-Factor-App,102:Twelve-Factor-App-Revisited}.

\vspace{1em}
\newcounter{factorno}
\setcounter{factorno}{-1}
\newcommand{\factornumber}{\stepcounter{factorno}\Roman{factorno}}
\begin{longtable}{  |   >{\raggedleft\factornumber}p{0.025\textwidth}   % Number (centered)
                        >{\raggedright\bfseries}p{0.175\textwidth}      % Factor (left-aligned)
                    |   >{\raggedright\itshape}p{0.150\textwidth}       % Description (left-aligned)
                    |    p{0.550\textwidth}                             % Basics (block)
                    | }
    \hline
        & \upshape\textbf{Faktor} 
        & \upshape\textbf{Beschreibung} 
        & \upshape\textbf{Grundprinzipien} \\
    \hline \hline
    \endhead
    \hline
    %   (I) Codebase
        & Codebase
        & \q{One codebase tracked in revision control, with multiple deploys.}
        & Eine \Gls{codebase} \glsdesc{codebase}. Ein Deploy beschreibt die laufende Instanz einer Applikation in einer Produktivumgebung, einer Stage oder der lokalen Entwicklungsumgebung. Die Applikation sollte genau eine Codebase haben, aus der beliebig viele Deploys hervorgehen. \\
    \hline
    %   (II) Dependencies
        & Dependencies
        & \q{Dependencies must be explicitly declared and isolated.}
        & Für Bezug und Installation von Programmierbibliotheken sollten Online Repositories verwendet werden. Abhängigkeiten werden deklarativ in einem Manifest angegeben und isoliert vom umgebenden System installiert. \\
    \hline
    %   (III) Config
        & Config
        & \q{Store configuration in the environment, not in the code.}
        & Die Konfiguration kann sich zwischen verschiedenen Deploys unterscheiden, beispielsweise bei Credentials oder Variablen wie dem Namen des Hosts. Die Speicherung solcher flexiblen Konfigurationen erfolgen nicht im Code direkt, sondern in \q{Config}-Dateien außerhalb der Revision, beispielsweise in Umgebungsvariablen und \texttt{.env}-Dateien. \\
    \hline
    %   (IV) Backing Services
        & Backing Services
        & \q{Treat backing services as attached resources.}
        & Ein Backing Service ist ein Dienst, der durch die Applikation über das Netzwerk konsumiert wird, wobei nicht zwischen lokalen Diensten und solchen von Drittanbietern unterschieden werden sollte. Die Austauschbarkeit dieser Dienste muss zu jedem Zeitpunkt gewährleistet sein und ihr Zugang darf nicht in der Config gespeichert sein. \\
    \hline
    %   (V) Build, Release, Run
        & Build, Release, Run
        & \q{Strictly separate build, release, and run stages.}
        & Stages für \Gls{build}, \Gls{release} und \Gls{run} sollten getrennt werden. \Gls{build} \glsdesc{build}. \Gls{release} \glsdesc{release}. \Gls{run} \glsdesc{run}. Rollbacks der Software sind nur über vorherige \Glspl{release} möglich, aber nicht über Änderungen am Quellcode in Produktivumgebungen. \Glspl{release} sind eindeutig identifizierbar. \\
    \hline
    %   (VI) Processes
        & Processes
        & \q{Execute the app as one or more stateless processes.}
        & Die Applikation läuft als einer oder mehrere zustandslose Prozesse, die einer Share-Nothing-Architektur folgen. Zustände sind in den Backing Services gespeichert. \\
    \hline
    %   (VII) Port Binding
        & Port Binding
        & \q{Export services via port binding.}
        & Erreichbar sollte die Applikation beispielsweise über \Gls{http} oder \Gls{https} auf einem freigegebenen Port sein, auf dem sie auf Anfragen wartet. Lokale Entwicklungsumgebungen nutzen den \texttt{localhost}. Die Applikation kann dadurch als Backing Service von anderen Applikationen konsumiert werden. \\
    \hline
    %   (VIII) Concurrency
        & Concurrency
        & \q{Scale out via the process model.}
        & Die Architektur der Applikation orientiert sich am \textit{Unix Process Model}. Skalierung erfolgt über die Anzahl laufender Prozesse, Verwaltung des Workloads über Prozesstypen. \\
    \hline
    %   (IX) Disposability
        & Disposability
        & \q{Maximize robustness with fast startup and graceful shutdown.}
        & Prozesse können zu jedem Zeitpunkt gestartet oder gestoppt werden. Dies führt zu mehr Skalierbarkeit, schnellerem \Gls{deployment} von Änderungen an Code oder Config sowie einem robusten \Gls{deployment} in Produktivumgebungen. Alle Prozesse sollten eine minimale Zeit zum Hochfahren benötigen und gegen unerwartetes Beenden abgesichert sein. \\
    \hline
    %   (X) Dev / Prod Parity
        & Dev / Prod Parity
        & \q{Keep development, staging, and production as similar as possible.}
        & Die Applikation strebt eine Reduzierung der Lücken zwischen \Gls{development} und \Gls{deployment} an, um die vergehenden Zeit zwischen ihnen zu verringern. Verantwortlich für beide Bereiche sollen dabei möglichst ähnliche Personen oder Personengruppen sein. Development- und Deploymentumgebungen sollten dementsprechend möglichst gleich sein. Damit ist auch die Gleichheit der Backing Services in den verschiedenen Stages gemeint. \textit{Docker} als virtuelle Umgebung kann hierbei unterstützen. \\
    \hline
    %   (XI) Logs
        & Logs
        & \q{Treat logs as event streams.}
        & Logs sollten drei Eigenschaften erfüllen: Aggregation, Chronologie und das Auftreten in Textform. Eine Zeile sollte dabei ein Ereignis beschreiben. Die Applikation selbst sollte keine Logs schreiben, dies ist Aufgabe der einzelnen Prozesse. Sie protokollieren alle Ereignisse in \texttt{stdout}, sodass Zielspeicher und Ausgabeorte durch die Deploymentumgebung definiert werden können. \\
    \hline
    %   (XII) Admin Processes
        & Admin Processes
        & \q{Run admin or management tasks as one-off processes.}
        & Administrative Prozesse dienen der Verwaltung der Applikation. Sie sollten in der gleichen Umgebung ausgeführt werden wie alle anderen Aufgaben. \\
    \hline
    \caption{Faktoren der \q{Twelve-Factor-App} mit Kurzbeschreibung}
    \label{tab:ftwelve-factor-app-factors}
\end{longtable}
\vspace{1em}
\setcounter{factorno}{0}

Die Technologien im Cloud Bereich erfahren eine stetige Weiterentwicklung. Die Methodik der Twelve-Factor-App ist weiterhin noch anwendbar, wurde jedoch ausgeweitet, sodass mittlerweile das Konzept einer \q{15-Factor-App} existiert. Änderungen an den bestehenden Faktoren wurden dort nicht vorgenommen, jedoch werden drei zusätzliche Faktoren vorgestellt: \textbf{API First}, \textbf{Telemetry} sowie \textbf{Authentication and Authorization}. \cite{104:15-Factor-Cloud-native-Java-Applications} Diese neuen Faktoren haben einen Einfluss auf die Architektur der Applikation selbst, jedoch nicht auf deren Toolchain. Daher wird die 15-Factor-App im Rahmen dieser Arbeit nicht weiter beleuchtet.

\section{Überblick und Zusammenfassung der Ansätze}
\label{sec:03-06_overview-and-summary-of-approaches}

Das in \autoref{ch:02_technological-environment} beschriebene \hyperref[ch:02_technological-environment]{technische Umfeld} unterliegt einer stetigen Weiterentwicklung. Ziel im Software Engineering ist es, die neuen Möglichkeiten möglichst tief in bestehende Zusammenarbeits"=, Entwicklungs"= und Architekturmodelle zu integrieren. Viele aufkommende Konzepte liefern Vorschläge zur Beschleunigung von Softwareevolution und ihrer Auslieferung, decken jedoch unterschiedliche Bereiche ab und verfolgen verschiedene Ansätze.

\textbf{\hyperref[sec:03-01_devops]{DevOps}} soll die Lücke zwischen \Gls{development} und Operations schließen. Es handelt sich dabei jedoch um kein spezifisches Vorgehensmodell oder Rahmenwerk, sondern eher um eine Sammlung aus Methodiken. Eine sehr verbreitete dieser Methodiken ist \acrfull{cicd}, welches in vielen großen Unternehmen eingesetzt wird. DevOps wird teilweise unterschiedlich ausgelegt und praktiziert. Leider liefert es für den Bereich der Toolchains noch keine einheitlichen Lösungen.

\textbf{\hyperref[sec:03-03_gitops]{GitOps}} kann zwar als Weiterentwicklung von \hyperref[sec:03-01_devops]{DevOps} betrachtet werden, ergänzt es jedoch eher als es zu ersetzen. Das Konzept stellt einen deklarativen Ansatz vor, um die Herausforderungen im Toolchain Bereich zu lösen und hat eine vollständige Automatisierung der Aufgaben im \Gls{development} und im \Gls{deployment} von sowohl der Applikation selbst als auch insbesondere ihrer Zielumgebung als Idealziel. Wie der Name bereits vermuten lässt, nutzt GitOps \textit{\Gls{git}} als zentrales Tool in allen Bereichen. Es liefert auf Basis dieses \Gls{vcs} konkrete Ansätze zu Architekturen der Toolchain, fokussiert sich jedoch hauptsächlich auf Deploymentumgebungen und das \Gls{deployment}. Für Developmentumgebungen stellt es keinen Ansatz bereit.

\textbf{\hyperref[sec:03-04_dotfiles]{Dotfiles}} beschreibt ein Konzept aus dem stark individualisierten Raum von Entwicklern. Es zielt hauptsächlich auf den Bereich \Gls{development} ab und kann dabei helfen, eine weitestgehende Reproduzierbarkeit und Wiederverwendbarkeit von Developmentumgebungen zu erreichen. Meist wird darunter ein vollständiges und dediziertes Repository aus Konfigurationsdateien verstanden, das Konzept lässt sich allerdings auch in andere Ansätze integrieren.

Die \textbf{\hyperref[sec:03-05_concept-of-twelve-factor-app]{Twelve-Factor-App}} ist eine Methodik zur Entwicklung moderner und skalierbarer Softwareanwendungen. Sie definiert zwölf spezifische Prinzipien, Faktoren genannt, die Anwendungen für die Cloud optimieren sollen. Neun dieser Prinzipien stützen sich ganz oder in Teilen auf den Bereich Toolchain. Die Faktoren konzentrieren sich auf die Entwicklung wartbarer, skalierbarer und portierbarer Anwendungen und weisen an einigen Stellen Schnittmengen mit Elementen anderer in dieser Arbeit vorgestellter Konzepte auf.

Trotz vieler guter Lösungsansätze sind einige Probleme offen geblieben und es bestehen weitere Verbesserungspotentiale. Im Rahmen von \nameref{ch:05_toolchain-as-code} soll eine \nameref{sec:05-02_strategy-for-toolchains} entwickelt werden, die auch diese auf das \Gls{development} bezogenen Herausforderungen betrachtet. Zu berücksichtigende Aspekte bei der Verwendung von \nameref{sec:02-03_containerization} sollen in Form von \nameref{sec:05-03_best-practices-with-docker-and-docker-compose} ebenfalls noch untersucht werden. Diese Punkte werden aufbauend auf weitere Untersuchungen zu einem späteren Zeitpunkt in dieser Arbeit nochmals aufgegriffen.

Dieses \autoref{ch:03_examination-of-existing-approaches} beantwortet die \acrlong{rq} \textbf{RQ-1} (siehe \autoref{sec:01-03_objectives-and-research-questions}).


	\chapter{Anforderungen an Development- und Deploymentumgebungen}
\label{ch:04_requirements-for-development-and-deployment-environments}

\section{Erhebung der Anforderungen mittels Experteninterviews}
\label{sec:04-01_collection-of-requirements-using-expert-interviews}

\subsection{Ziel}
\label{subsec:04-01-01_goal}

Während der \nameref{ch:03_examination-of-existing-approaches} in \autoref{ch:03_examination-of-existing-approaches} konnten einige Ansätze identifiziert werden, die bei der Erfüllung der in \autoref{ch:01_introduction-and-motivation} angenommenen Anforderungen unterstützen können. Vor der Entwicklung einer konkreten Strategie müssen die Anforderungen an eine \nameref{ch:05_toolchain-as-code} Strategie jedoch noch konkreter und sicherer ermittelt werden. Einen ersten Vorschlag liefert die \hyperref[sec:03-05_basic-idea-of-twelve-factor-app]{Twelve-Factor-App}. Weitere und praxisnahe Anforderungen an Toolchains in den Bereichen \Gls{development} und \Gls{deployment} sollen Experten und Praktiker liefern.

Herausgefunden werden sollen eine ideale Zusammensetzung von Development- und Deploymentumgebungen, übliche Vorgehensweisen und Interaktionswege sowie typische Herausforderungen im Umgang mit aktuellen Development- und Deploymentumgebungen. Wichtig ist außerdem die Erhebung eines Ist-Stands bei Automatisierungsgrad und \Gls{developer-experience} im \Gls{development} und im \Gls{deployment}. Zusätzlich sollen die Anforderungen und Ziele für Development- und Deploymentumgebungen priorisiert werden. Das zentrale Ziel dieses Kapitels ist die Beantwortung der \acrlong{rq} \textbf{RQ-0} (siehe \autoref{sec:01-03_objectives-and-research-questions}).

Die Ergebnisse der vorausgegangenen Literaturrecherche sollen in die Untersuchungen einbezogen werden. Die hier ermittelten Anforderungen wiederum werden als Grundlage für die Entwicklung der \nameref{sec:05-02_strategy-for-toolchains} dienen.

\subsection{Methodik}
\label{subsec:04-01-02_methodology}

\subsubsection{Vorgehen}
\label{subsubsec:04-01-02-01_procedure}

Als geeignete Methodik zur Erhebung der Anforderungen wurde das Experteninterview gewählt. Die Vorteile dieser Methodik sind die relativ schnelle Gewinnung von Daten, die Abkürzung eines sonst aufwändigen Beobachtungsprozesses sowie eine tendenziell große Bereitschaft zur Teilnahme, da sich die Befragten meist im gleichen wissenschaftlichen Feld bewegen \cite{401:Das-Experteninterview}. Ein Experte im Verständnis dieser Arbeit kann dabei \acrshort{iAa} \citeauthor{401:Das-Experteninterview} verstanden werden als jemand, der über besondere Informationen oder Fähigkeiten in einem bestimmten Bereich verfügt, einen spezifischen Wissensvorsprung in einem Feld aufweist oder bei wem sich vermuten lässt, dass er bestimmtes Expertenwissen und damit Relevanz für einen Forschungsgegenstand hat \cite{401:Das-Experteninterview}. Die konstruktivistische Definition legt nahe, dass Experten innerhalb von Organisationen auch und besonders auf niedrigen Hierarchieebenen zu finden sind \cite{401:Das-Experteninterview}.

Grundsätzlich können drei verschiedene Arten des Experteninterviews unterschieden werden. Ein \textbf{exploratives Experteninterview} dient der Herstellung einer ersten Orientierung, der Schärfung des Problembewusstseins auf Seite des Forschenden oder als Vorlauf zur Erstellung eines finalen Interviewleitfadens. Es wird möglichst offen geführt und sein Leitfaden deckt nur die zentralsten Dimensionen des Forschungsgebiets ab. Das \textbf{systematische Experteninterview} fokussiert sich auf die Ermittlung von aus der Praxis gewonnenem Wissen, welches spontan durch Experten wiedergegeben werden kann. Der Experte hat hier die Funktion eines Ratgebers und der Leitfaden ist relativ ausdifferenziert. Besonders wichtig ist bei dieser Form des Experteninterviews die thematische Vergleichbarkeit. Dahingegen ist es Ziel des \textbf{theoriegenerierenden Experteninterviews}, das Expertenwissen in einer größeren Tiefe zu erschließen. \cite{401:Das-Experteninterview}

Für den vorliegenden Anwendungsfall ist diese letzte Variante des Experteninterviews eher unpassend. Eine erste Orientierung wurde bereits durch die Literaturrecherche bei der \nameref{ch:03_examination-of-existing-approaches} in \autoref{ch:03_examination-of-existing-approaches} gewonnen, weshalb auch ein exploratives Experteninterview für das Stadium der Forschungen nicht geeignet ist. Die Wahl fällt somit auf ein systematisierendes Experteninterview, wobei ein Leitfaden die Vergleichbarkeit der erhobenen Daten gewährleisten und zugleich offen genug sein soll, um zusätzliche Impulse zuzulassen.

\subsubsection{Form}
\label{subsubsec:04-01-02-02_form}

Spezifische und theoriegeleitete Vorannahmen zum Forschungsfeld bestehen bereits (\acrshort{vgl} \autoref{ch:03_examination-of-existing-approaches}). Die Wahl fällt, wie im vorherigen Abschnitt angerissen, auf ein \textbf{leitfadengestütztes Interview}.

Diese Form des Interviews ist teilstandardisiert. Konkrete Fragen werden vorab ausgearbeitet, die Orientierung erfolgt jedoch auch entlang der Aussagen des Experten. Die Fragen selbst werden dabei so angeordnet, dass sich das Interview vom Allgemeinen zum Spezifischen bewegt. \cite{205:Leitfadengestuetztes-Interview} Der Leitfaden soll eine Orientierung für den Forschenden bieten, aber \qit{nicht als zwingendes Ablaufmodell des Diskurses} \cite{401:Das-Experteninterview} angesehen werden. Ein Risiko des Leitfadens kann sein, wenn \qit{ein Experte sich in einem anderen Sprachspiel als dem des Leitfadens bewegt} \cite{401:Das-Experteninterview}. Drei wichtige Kriterien müssen durch das Interview erfüllt sein: Offenheit, Spezifität sowie Kontextualität und Relevanz. \textbf{Offenheit} fordert, dass der Interviewpartner den Sachverhalt aus seiner eigenen Sicht beschreiben können sollte. Bei Andeutungen oder ungenauen Ausführungen durch den Befragten, sollte der Forschende zu Gunsten der \textbf{Spezifität} genauer nachfragen. \textbf{Kontextualität und Relevanz} sind erfüllt, wenn die Fragen in den Kontext des Befragten passen. \cite{205:Leitfadengestuetztes-Interview}

Der zu entwickelnde Leitfaden soll genau dieser Methodik folgen. Dabei werden vor der Erstellung der \Glspl{iq} zunächst vier Fragenbereiche (von \texttt{A} bis \texttt{D}) festgelegt:

\begin{itemize}
    \item \texttt{(A)} \textbf{Offene Fragen} sollen den Ergebnisraum zunächst aus Sicht des Befragten beleuchten.
    \item \texttt{(B)} \textbf{Halb-offene Fragen} geben bereits einen gewissen Antwortraum vor und greifen auf Antworten aus \texttt{(A)} zurück.
    \item \texttt{(C)} \textbf{Geschlossene Fragen} sind besonders vergleichbar, ihre Antworten oft kurz und numerisch, meistens auf einer Skala oder mit einer vorgegebenen Einheit.
    \item \texttt{(D)} Bereits vorgegebene Konzepte sollen ganz zum Schluss beleuchtet werden, wobei deren \textbf{Bewertung durch den Befragten} durch die \textbf{Vergabe von Punkten} erfolgen soll.
\end{itemize}

In der Einführung soll dem Befragten zunächst Kontext zum Projekt gegeben werden, insbesondere zur Problematik des Development-Deployment-Gaps sowie zu den Zielen der Arbeit, beides entsprechend beschrieben in \autoref{ch:01_introduction-and-motivation}. Die Begriffe Development- und Deploymentumgebung genauso wie die Interpretation von \Gls{development} und \Gls{deployment} selbst sollen ebenfalls klargestellt werden. Anschließend wird der Kontext der Arbeit dargestellt, bestehend aus \nameref{sec:02-01_web-development}, \nameref{sec:02-02_microservices} und \nameref{sec:02-03_containerization}. Das \nameref{subsec:04-01-01_goal} des Interviews soll ebenfalls deutlich gemacht werden. Dieses ist die Ermittlung des üblichen Vorgehens im jeweiligen Bereich (\Gls{development} beziehungsweise \Gls{deployment}) sowie von Hindernissen, Wünschen oder eigene Best Practices der Befragten. Zum Abschluss soll dem Befragten Einsicht in das stichpunktförmige Protokoll gegeben werden, um eine sofortige Feedbackschleife und eine Rückversicherung zum korrekten Verständnis des Gesagten zu haben. Außerdem soll freiwillig Feedback zum Vorgehen und zum Interview selbst gegeben werden können.

Ein Interview soll insgesamt zwischen 45 und 60 Minuten dauern.

\subsubsection{Auswahl der Experten}
\label{subsubsec:04-01-02-03_selection-of-experts}

Die Auswahl von Experten stellt das Hauptproblem der Methodik dar. Sachkenntnis, Motivation und Einfluss zur praktischen Umsetzung von Ergebnissen können mögliche Faktoren sein. \cite{401:Das-Experteninterview} Sachkenntnis und praktische Erfahrung haben dabei die größte Relevanz für das \nameref{subsec:04-01-01_goal} der Interviews im Rahmen dieser Arbeit. Anforderungen an die Experten sind daher eine Tätigkeit im Projektumfeld, eine einschlägige akademische Ausbildung im Bereich Software Engineering oder \Gls{it} allgemein sowie mindestens zwei Jahre aktive Berufserfahrung. Nicht zu vernachlässigen ist die sogenannte \q{Stakeholder-Problematik}, die auftritt, wenn Interviewpartner in Maßnahmen des Forschungskontexts involviert sind \cite{401:Das-Experteninterview}. Der \nameref{sec:01-02_project-context} liegt im \Gls{am} eines großen Automobilherstellers, weshalb nur eine Hälfte der Interviewpartner aus diesem Bereich stammen soll. Die andere Hälfte soll im \Gls{sdc} des gleichen Unternehmens tätig sein. Das gewährleistet eine gewisse Diversität und beleuchtet auch die Vorgehensweisen eines anderen Bereichs. Auch wichtig ist die Berücksichtigung aller Akteursebenen \cite{401:Das-Experteninterview}, weshalb sowohl Experten aus dem Bereich \Gls{development} als auch aus dem Bereich \Gls{deployment} gesucht werden. Die Experten dürfen selbst einschätzen, welchem Bereich sie eher entsprechen und Mehrfachzuordnungen sind ebenfalls möglich. Dadurch ist gewährleistet, dass eine Grundkompetenz in allen Forschungsgegenständen vorliegt und dass bei der Auswertung der \nameref{subsec:04-01-04_interview-results} zu jedem Bereich Anforderungen ableitbar sind.

Für diese Experteninterviews, ist die Ermittlung quantitativer Ergebnisse nur ein sekundäres Ziel. In erster Linie stehen qualitative Aspekte im Vordergrund. Das Expertenwissen soll Aufschluss darüber geben, welche Fokuspunkte bei den Anforderungen an Development- und Deploymentumgebungen gesetzt werden. Deshalb ist eine kleine Stichprobe an Experten ausreichend.

Insgesamt befragt wurden vier Personen, davon alle mit abgeschlossenem Universitäts- oder Hochschulstudium der Informatik, zwei aus dem \Gls{am} und zwei aus dem \Gls{sdc}. Alle Befragten gaben an, aktuell im Bereich \Gls{development} tätig zu sein, 75 \% seien im \Gls{deployment} tätig und 25 \% sogar im Bereich Operations. Die \Glspl{ip} waren ausnahmslos männlichen Geschlechts, ihr Alter lag zwischen 26 Jahren und 51 Jahren, sowohl im Median als auch im Mittel waren sie 38 Jahre alt.

\subsection{Interviewfragen}
\label{subsec:04-01-03_interview-questions}

Eine vollständige Liste mit allen \Glspl{iq} ist im \autoref{ch:AA_expert-interviews} unter den \nameref{sec:AA-01_interview-questions} einsehbar.

Jede \acrfull{iq} soll einem bestimmten Zielbereich zugeordnet werden. Zielbereiche wiederum leiten sich aus dem eingangs beschriebenen \nameref{subsec:04-01-01_goal} (vgl. \autoref{subsec:04-01-01_goal}) ab. Erhoben werden sollen, abhängig vom Zielbereich

\begin{itemize}
    \item \texttt{(Eins)} die \textbf{Zusammensetzung} von (4 Fragen),
    \item \texttt{(Zwei)} \textbf{Vorgehensweisen} und \textbf{Interaktionswege} mit (1 Frage),
    \item \texttt{(Drei)} \textbf{Herausforderungen} im Umgang mit (3 Fragen),
    \item \texttt{(Vier)} ein \textbf{Ist-Stand} für Automatisierung und \Gls{developer-experience} in (3 Fragen), sowie
    \item \texttt{(Fünf)} Prioritäten bei \textbf{Anforderungen} an und \textbf{Zielen} für (4 Fragen)
\end{itemize}

Development- und Deploymentumgebungen (nachfolgend \q{Umgebungen} genannt).

\textbf{\nameref{subsec:AA-01-01_open-questions}} (\texttt{\hyperref[subsubsec:04-01-02-02_form]{(A)}} \textrightarrow \texttt{\hyperref[subsec:AA-01-01_open-questions]{IQ-Ax}}) zielen auf die Bereiche \texttt{(Eins)}, \texttt{(Zwei)} und \texttt{(Drei)} ab. Sie möchten explorativ und unvoreingenommen erste Erfahrungen des Experten beleuchten und sollen eine Vorstellung vom praktischen Umfeld erschaffen. Dazu erfragen sie die Idealvorstellung von Umgebungen (\texttt{\hyperref[subsec:AA-01-01_open-questions]{IQ-A0}}), die verwendeten Tools (\texttt{\hyperref[subsec:AA-01-01_open-questions]{IQ-A3}}), häufige Handgriffe (\texttt{\hyperref[subsec:AA-01-01_open-questions]{IQ-A2}}), auftretende Hindernisse (\texttt{\hyperref[subsec:AA-01-01_open-questions]{IQ-A1}}) und besondere Anforderungen an Umgebungen (\texttt{\hyperref[subsec:AA-01-01_open-questions]{IQ-A4}}).

\textbf{\nameref{subsec:AA-01-02_half-open-questions}} (\texttt{\hyperref[subsubsec:04-01-02-02_form]{(B)}} \textrightarrow \texttt{\hyperref[subsec:AA-01-02_half-open-questions]{IQ-Bx}}) beleuchten die Bereiche \texttt{(Vier)} und \texttt{(Fünf)}. Sie sollen das aktuelle Vorgehen bewertbar machen und rückbeziehen sich teilweise auf offene Fragen (\texttt{\hyperref[subsec:AA-01-01_open-questions]{IQ-Ax}}), um den Befragten bereits gegebene Antworten nochmal genauer reflektieren zu lassen. Dazu erfragen sie den momentanen Zeitbedarf für das Aufsetzen einer Umgebung (\texttt{\hyperref[subsec:AA-01-02_half-open-questions]{IQ-B0}}), die wichtigsten Schritte (\texttt{\hyperref[subsec:AA-01-02_half-open-questions]{IQ-B0}}) dabei, die unter ihnen, deren Optimierung den größten Mehrwert bringen würde (\texttt{\hyperref[subsec:AA-01-02_half-open-questions]{IQ-B2}}) und diejenigen, die explizit nicht automatisierbar (\texttt{\hyperref[subsec:AA-01-02_half-open-questions]{IQ-B3}}) sind. Zusätzlich sollen durch den Befragten drei gewichtete Ziele (\texttt{\hyperref[subsec:AA-01-02_half-open-questions]{IQ-B4}}) genannt werden, die er bei der Entwicklung einer eigenen neuen Toolchain verfolgen würde.

\textbf{\nameref{subsec:AA-01-03_closed-questions}} (\texttt{\hyperref[subsubsec:04-01-02-02_form]{(C)}} \textrightarrow \texttt{\hyperref[subsec:AA-01-03_closed-questions]{IQ-Cx}}) konzentrieren sich auf die Bereiche \texttt{(Eins)}, \texttt{(Drei)} und \texttt{(Vier)}. Sie sind eher quantitativ und sollen eine untergeordnete Rolle bei der Auswertung spielen, aber zur Entwicklung eines ungefähren numerischen Feldes des Ist-Stands beitragen. Dazu erfragen sie den prozentualen Anteil nicht automatisierbarer Schritte (\texttt{\hyperref[subsec:AA-01-03_closed-questions]{IQ-C0}}), die Flexibilität (\texttt{\hyperref[subsec:AA-01-03_closed-questions]{IQ-C1}}) und den Bedarf an Flexibilität (\texttt{\hyperref[subsec:AA-01-03_closed-questions]{IQ-C1}}, \texttt{\hyperref[subsec:AA-01-03_closed-questions]{IQ-C2}}, \texttt{\hyperref[subsec:AA-01-03_closed-questions]{IQ-C3}}) von Toolchains sowie die Mindestanforderungen an Umgebungen (\texttt{\hyperref[subsec:AA-01-03_closed-questions]{IQ-C4}}).

Der letzte Abschnitt des Interviews gibt dem Befragten eine \textbf{\nameref{subsec:AA-01-04_evaluation-requirements}} (\texttt{\hyperref[subsubsec:04-01-02-02_form]{(D)}} \textrightarrow \texttt{\hyperref[subsec:AA-01-04_evaluation-requirements]{IQ-Dx}}) vor, die in Form von Faktoren der \hyperref[sec:03-05_basic-idea-of-twelve-factor-app]{Twelve-Factor-App} genannt werden. Dieses Vorgehen soll dabei helfen, einzelne Elemente der geplanten \nameref{ch:05_toolchain-as-code} Strategie zu priorisieren und den richtigen Fokus für den größten Mehrwert zu setzen. Bewertet werden die Faktoren über die Vergabe von insgesamt sechs Punkten pro Befragtem, deren Verteilung frei entschieden werden kann. In \autoref{subsec:AA-01-04_evaluation-requirements} sind die genauen Faktoren und ihre Zuordnung zu Anforderungen aus \hyperref[ch:03_examination-of-existing-approaches]{bestehenden Ansätzen} aufgeführt.

\subsection{Interviewergebnisse}
\label{subsec:04-01-04_interview-results}

Die Ergebnisse der Experteninterviews sind im \autoref{ch:AA_expert-interviews} unter \nameref{sec:AA-03_interview-results} einsehbar.

Am Ende jeden Interviews gab es die Möglichkeit, freiwillig eine außerfachliche Rückmeldung zur Methodik und zum Interview selbst zu geben. Zwei Personen (\textbf{\hyperref[sec:AA-02_interview-persons]{IP-2}}, \textbf{\hyperref[sec:AA-02_interview-persons]{IP-3}}) haben diese Möglichkeit genutzt. Ergebnis war unter anderem, dass die Einleitung noch mehr Details zum Hintergrund der Arbeit hätte enthalten können und dass die Definitionen von Development- und Deploymentumgebungen noch deutlicher hätten hervorgehoben werden können. Sehr positiv wurde die Aufbereitung der \Glspl{iq} wahrgenommen, auch ihre Anordnung sei sehr angenehm gewesen. Ein Experte fand sich besonders gut eingebunden, weil einzelne Fragen aufeinander aufbauten und so ein Gesprächsfluss entstand.

Insbesondere die \textbf{\hyperref[subsec:AA-01-01_open-questions]{offenen Fragen}} (\texttt{\hyperref[subsec:AA-01-01_open-questions]{IQ-Ax}}) wurden sehr häufig genutzt, um zusätzliches Expertenwissen zu teilen, welches nicht unbedingt innerhalb des erwarteten Antwortraums lag. Die \textbf{\hyperref[subsec:AA-01-02_half-open-questions]{halb-offenen Fragen}} (\texttt{\hyperref[subsec:AA-01-02_half-open-questions]{IQ-Bx}}) konnten sehr gut zur Sortierung und Bewertung der Ergebnisse beitragen. Generell lieferten die nicht-geschlossenen Fragen (\texttt{\hyperref[subsec:AA-01-01_open-questions]{IQ-Ax}}, \texttt{\hyperref[subsec:AA-01-02_half-open-questions]{IQ-Bx}}) gute und ausführliche Ergebnisse. Die \textbf{\hyperref[subsec:AA-01-03_closed-questions]{geschlossenen Fragen}} (\texttt{\hyperref[subsec:AA-01-03_closed-questions]{IQ-Cx}}) waren etwas weniger ertragreich. Vielen \acrlong{ip} fiel es schwer, sich auf einen quantitativen Wert festzulegen. Zwar lassen sich in den Ergebnissen die grundlegenden Positionen auf einem Antwortspektrum ablesen, dennoch sind einige Ergebnisse zu gestreut, um für weitere Forschungen verwendbar zu sein.

Auffällig ist, dass die Experten zwar alle relevanten Bereiche (\Gls{development} und \Gls{deployment}) durch ihre Antworten abdecken, viele von ihnen sich allerdings eher dem \Gls{development} zuordnen lassen, was auf ein generell größeres Potential in diesem Bereich schließen lässt (siehe Abbildung 4.6 (t. b. d.)). Das \Gls{development} wurde fast immer zuerst, oft auch ausschließlich angesprochen, sofern immanente Fragen unberücksichtigt bleiben. Alle genannten Optionen zur Einsparung von Zeitressourcen (\texttt{\hyperref[subsec:AA-03-02_half-open-questions]{IQ-B2}}) liegen in diesem Bereich, ebenso treten die meisten Herausforderungen (\texttt{\hyperref[subsec:AA-03-01_open-questions]{IQ-A0}}) fast ausschließlich dort auf.

\setcounter{factorno}{-1}
\begin{longtable}{  |   >{\raggedleft\bfseries}p{0.0125\textwidth}              % Position (centered)
                    |   >{\raggedright\bfseries\small}p{0.2250\textwidth}       % Number, Factor (left-aligned)
                    |   >{\raggedright\itshape\small}p{0.6375\textwidth}        % Description (left-aligned)
                    |   >{}p{0.0250\textwidth}                                  % Points (centered)
                    | }
    \hline
          \upshape\normalsize
        & \upshape\normalsize\textbf{Faktor} 
        & \upshape\normalsize\textbf{Beschreibung} \cite{101:The-Twelve-Factor-App}
        & \upshape\normalsize\textbf{\acrshort{p}} \\
    \hline \hline
    \endhead
    \hline
    %   (I) Codebase
          0
        & %\setcounter{factorno}{01}\Roman{factorno}
          Codebase
        & One codebase tracked in revision control, with multiple deploys.
        & 04 \\
    \hline
    %   (II) Dependencies
          0
        & %\setcounter{factorno}{02}\Roman{factorno}
          Dependencies
        & Dependencies must be explicitly declared and isolated.
        & 04 \\
    \hline
    %   (V) Build, Release, Run
          0
        & %\setcounter{factorno}{05}\Roman{factorno}
          Build, Release, Run
        & Strictly separate build, release, and run stages.
        & 04 \\
    \hline
    %   (X) Dev / Prod Parity
          0
        & %\setcounter{factorno}{10}\Roman{factorno}
          Dev / Prod Parity
        & Keep development, staging, and production as similar as possible.
        & 04 \\
    \hline
    %   (III) Config
          1
        & %\setcounter{factorno}{03}\Roman{factorno}
          Config
        & Store configuration in the environment, not in the code.
        & 03 \\
    \hline
    %   (IV) Backing Services
          2
        & %\setcounter{factorno}{04}\Roman{factorno}
          Backing Services
        & Treat backing services (e.g., databases) as attached resources.
        & 02 \\
    \hline
    %   (XI) Logs
          2
        & %\setcounter{factorno}{11}\Roman{factorno}
          Logs
        & Treat logs as event streams.
        & 02 \\
    \hline
    %   (VIII) Concurrency
          3
        & %\setcounter{factorno}{08}\Roman{factorno}
          Concurrency
        & Scale out via the process model.
        & 01 \\
    \hline
    %   (VI) Processes
          -
        & %\setcounter{factorno}{06}\Roman{factorno}
          Processes
        & Execute the app as one or more stateless processes.
        & 00 \\
    \hline
    %   (VII) Port Binding
          -
        & %\setcounter{factorno}{07}\Roman{factorno}
          Port Binding
        & Export services via port binding.
        & 00 \\
    \hline
    %   (IX) Disposability
          -
        & %\setcounter{factorno}{09}\Roman{factorno}
          Disposability
        & Maximize robustness with fast startup and graceful shutdown.
        & 00 \\
    \hline
    %   (XII) Admin Processes
          -
        & %\setcounter{factorno}{12}\Roman{factorno}
          Admin Processes
        & Run admin or management tasks as one-off processes.
        & 00 \\
    \hline
    \caption{Interviewergebnisse zur Priorisierung der Faktoren der \q{Twelve-Factor-App}}
    \label{tab:interview-results-factors-priorities}
\end{longtable}
\vspace{1em}
\setcounter{factorno}{0}

Bei der Priorisierung von Faktoren der \hyperref[sec:03-05_basic-idea-of-twelve-factor-app]{Twelve-Factor-App} zur indirekten Bewertung von Anforderungen an Toolchains wurden insgesamt 24 \Glspl{p} vergeben. Trotz einer geringen Stichprobe ist eine klare Priorisierung erkennbar (siehe \autoref{tab:interview-results-factors-priorities} und \autoref{fig:g-10_interview-results-factors-priorities-total}).

\begin{figure}[h]
    \centering
    \includegraphics[width=0.95\textwidth]{g-10_interview-results-factors-priorities-total.png}
    \caption{Interviewergebnisse zur Ausprägung der Faktoren in Gesamtansicht}
    \label{fig:g-10_interview-results-factors-priorities-total}
\end{figure}

Etwa ein Drittel der Faktoren, nämlich \textbf{Codebase}, \textbf{Dependencies}, \textbf{Build, Release, Run} und \textbf{Dev / Prod Parity}, wurden von allen Befragten als gleichermaßen relevant eingestuft. Mit jeweils vier Punkten sind dies die wichtigsten Faktoren laut Expertenmeinung. Das untere Drittel erhielt insgesamt keinen einzigen der 24 vergebenen Punkte, was den Schluss zulässt, dass \textbf{Processes}, \textbf{Port Binding}, \textbf{Disposability} und \textbf{Admin Processes} mit Fokus auf Toolchains die vernachlässigbaren Faktoren sind. Die restlichen Faktoren, \textbf{Config}, \textbf{Backing Services}, \textbf{Logs} und \textbf{Concurrency}, befinden sich im mittleren Drittel und erhielten etwa 33 \% der Punkte. Die genaue Verteilung der Punkte auf die Faktoren kann \autoref{fig:g-09_interview-results-factors-priorities-per-persons} entnommen werden.

\begin{figure}[h]
    \centering
    \includegraphics[width=0.95\textwidth]{g-09_interview-results-factors-priorities-per-persons.png}
    \caption{Interviewergebnisse zur Priorisierung der Faktoren nach Interviewpersonen}
    \label{fig:g-09_interview-results-factors-priorities-per-persons}
\end{figure}

\section{Ableitung von Anforderungen}
\label{sec:04-02_derivation-of-requirements}

\subsection{Vorgehen bei der Auswertung}
\label{subsec:04-02-01_procedure-for-evaluation}

% place content here
Inhalt

\subsection{Toolchains im Bereich Development}
\label{subsec:04-02-02_toolchains-in-development}

% place content here
Inhalt

\subsection{Toolchains im Bereich Deployment}
\label{subsec:04-02-03_toolchains-in-deployment}

% place content here
Inhalt

\subsection{Durchgängigkeit von Toolchains}
\label{subsec:04-02-04_consistency-of-toolchains}

% place content here
Inhalt

\section{Allgemeines und Zusammenfassung der Erkenntnisse}
\label{sec:04-03_general-aspects-and-summery-of-findings}

Die Anforderungen \nameref{sec:04-02_derivation-of-requirements} erfolgte hauptsächlich über die Ergebnisse nicht-geschlossener Fragen (\texttt{\hyperref[subsec:AA-01-01_open-questions]{IQ-Ax}}, \texttt{\hyperref[subsec:AA-01-02_half-open-questions]{IQ-Bx}}). Die Ergebnisse der geschlossenen Fragen (\texttt{\hyperref[subsec:AA-01-03_closed-questions]{CQ-x}}) eignen sich hingegen dazu, einen quantitativen Ist-Stand abzubilden. Schlüsselerkenntnisse sind unter anderem, dass der Anteil nicht automatisierbarer Schritte (\texttt{\hyperref[subsec:AA-01-03_closed-questions]{IQ-C0}}) zwischen fünf und zwanzig Prozent liegt (\acrshort{vgl} \autoref{subsec:AA-03-03_closed-questions}) und dass die Flexibilität der Toolchains (\texttt{\hyperref[subsec:AA-01-03_closed-questions]{IQ-C1}}) je nach Abteilung der \Gls{ip} unterschiedlich ausfallen kann. So gaben \Glspl{ip} aus dem \Gls{am} an, eher weniger austauschbare Toolchains zu haben, wohingegen \Glspl{ip} aus dem \Gls{sdc} dort eher flexibler sind (\acrshort{vgl} \autoref{subsec:AA-03-03_closed-questions}). Anpassungen an der Umgebung eines Softwareprojekts (\texttt{\hyperref[subsec:AA-01-03_closed-questions]{IQ-C2}}) sind in der Regel wenig bis selten notwendig (\acrshort{vgl} \autoref{subsec:AA-03-03_closed-questions}) und der Tausch von Tools oder Bibliotheken (\texttt{\hyperref[subsec:AA-01-03_closed-questions]{IQ-C3}}) findet unterschiedlich häufig statt (\acrshort{vgl} \autoref{subsec:AA-03-03_closed-questions}).

Die Experteninterviews gaben deutlich Aufschluss über die wichtigsten Anforderungen in allen Bereichen, für das \Gls{development} (\texttt{DEV}), das \Gls{deployment} (\texttt{DEP}) und die Durchgängigkeit (\texttt{CNT}). Die Ergebnisse lassen außerdem eine klare Priorisierung der Anforderungen zu.

Im Folgenden werden die in \autoref{sec:04-02_derivation-of-requirements} ermittelten Anforderungen, unterschieden nach ihrem Bereich und sortiert nach ihrer Priorität, in klare Aussagen zusammengefasst.

\begin{table}[H]
    \begin{tabular}{ >{\bfseries\ttfamily}p{0.1\textwidth} >{}p{0.8\textwidth} }
        DEV-0   &   Eine Entwicklungsumgebung wird schnell und vollständig geladen. \\
        DEV-1   &   Die Applikation ist auf den lokalen Maschinen der Entwickler ausführbar. \\
        DEV-2   &   Konfiguration ist im \Gls{vcs} verfügbar. \\
        DEV-3   &   Alle nötigen Abhängigkeiten und \Glspl{sdk} \newline sind in der Umgebung bereitgestellt. \\
        DEV-4   &   Dokumentation ist aktuell und intuitiv. \\
        DEV-5   &   Skripte reduzieren komplexere imperative Aufgaben auf einen Befehl. \\
    \end{tabular}
    \caption{Anforderungen an Toolchains im Bereich Development}
    \label{tab:requirements-development}
\end{table}

\begin{table}[H]
    \begin{tabular}{ >{\bfseries\ttfamily}p{0.1\textwidth} >{}p{0.8\textwidth} }
        DEP-0   &   Mit Ausnahme einer händischen Bestätigung ist \newline das Deployment vollständig automatisiert. \\
        DEP-1   &   Konfiguration wird aus dem \Gls{vcs} bezogen. \\
        DEP-2   &   Umgebung und Applikation sind Container. \\
        DEP-3   &   Bei Problemen werden Entwickler sofort benachrichtigt. \\
    \end{tabular}
    \caption{Anforderungen an Toolchains im Bereich Deployment}
    \label{tab:requirements-deployment}
\end{table}

\begin{table}[H]
    \begin{tabular}{ >{\bfseries\ttfamily}p{0.1\textwidth} >{}p{0.8\textwidth} }
        CNT-0   &   Automatisierung erfolgt auf allen Ebenen über die gleichen Tools. \\
        CNT-1   &   Die lokale Umgebung entspricht möglichst vollständig der Produktivumgebung. \\
        CNT-2   &   Alle Umgebungen sind mit möglichst wenig Tooleinsatz \newline durch das Entwicklungsteam anpassbar. \\
        CNT-3   &   Die Umgebung sowie das Verhalten der Applikation in ihr \newline sind auf allen Ebenen reproduzierbar. \\
    \end{tabular}
    \caption{Anforderungen an Toolchains im Bereich Durchgängigkeit}
    \label{tab:requirements-continuity}
\end{table}

\begin{figure}[h]
    \centering
    \includegraphics[width=0.95\textwidth]{g-15_interview-results-requirements-total.png}
    \caption{Verteilung des Fokus der Anforderungen an Toolchains auf die Bereiche}
    \label{fig:g-15_interview-results-requirements-total}
\end{figure}

Nicht alle Interviewergebnisse sind jedoch einem spezifischen Bereich zuordbar. So liefern beispielsweise die definierten Ziele bei der Entwicklung von Toolchains (\texttt{\hyperref[subsec:AA-01-02_half-open-questions]{IQ-B4}}) (\acrshort{vgl} \autoref{subsec:AA-03-02_half-open-questions}) wichtige Erkenntnisse, die in allen Bereichen Relevanz haben.

Als für die \Glspl{ip} eher weniger wichtig erachtete Ziele wurden die Einhaltung von Standards (\texttt{\hyperref[sec:AA-02_interview-persons]{IP-2}}) ohne Einschränkung der Entwickler in ihren Freiheiten (\texttt{\hyperref[sec:AA-02_interview-persons]{IP-1}}) sowie die Stabilität der Umgebungen (\texttt{\hyperref[sec:AA-02_interview-persons]{IP-0}}, \texttt{{\hyperref[sec:AA-02_interview-persons]{IP-3}}}) genannt. Wichtiger waren hingegen die unkomplizierte Anpassbarkeit von Toolchains (\texttt{\hyperref[sec:AA-02_interview-persons]{IP-1}}, \texttt{\hyperref[sec:AA-02_interview-persons]{IP-2}}) und die Unterstützung von Fast Feedback beim \Gls{deployment} (\texttt{\hyperref[sec:AA-02_interview-persons]{IP-3}}). Die drei zentralsten Ziele, die in den Interviews genannt wurden, sind die Reduzierung des manuellen Aufwands (\texttt{\hyperref[sec:AA-02_interview-persons]{IP-1}}, \texttt{\hyperref[sec:AA-02_interview-persons]{IP-2}}) und die einfache Gestaltung der Toolchains (\texttt{\hyperref[sec:AA-02_interview-persons]{IP-3}}). Das beste Ergebnis sollte dabei über \nameref{sec:02-03_containerization} erreicht werden (\texttt{\hyperref[sec:AA-02_interview-persons]{IP-0}}).

Dies verdeutlicht, dass sehr viel Potential im Bereich \Gls{development} und bei der Reduzierung manueller Aufgaben liegt. Hier können andernfalls schnell individuelle und schlecht reproduzierbare Fehler auftreten. Dieses Ergebnis wird gestützt durch die priorisierten Faktoren der \hyperref[sec:03-05_basic-idea-of-twelve-factor-app]{Twelve-Factor-App} (\acrshort{vgl} \autoref{subsec:04-01-04_interview-results}). Als besonders wichtig bewertet wurden dort eine zentrale \Gls{codebase} für alle Bereiche (\textit{Codebase}), klare und transparent dokumentierte Abhängigkeiten (\textit{Dependencies}) sowie die Trennung einzelner Schritte (\textit{Build, Release, Run}), welche sich in den verschiedenen Umgebungen möglichst wenig unterscheiden sollten (\textit{Dev / Prod Parity}).

Dieses \autoref{ch:03_examination-of-existing-approaches} beantwortet die \acrlong{rq} \textbf{RQ-0} (siehe \autoref{sec:01-03_objectives-and-research-questions}).


	\addtocontents{toc}{\protect\newpage}
	\chapter{Toolchain-as-Code}
\label{ch:05_toolchain-as-code}

\section{Container als Basistechnologie}
\label{sec:05-01_containers-as-base-technology}

\subsection{Docker Container}
\label{subsec:05-01-01_docker-container}

Eine wichtige Anforderung im Bereich der \nameref{subsec:04-02-04_consistency-of-toolchains} ist die Reproduzierbarkeit von Umgebungen. Sie kann von vielen verschiedenen Faktoren und Komponenten eines Systems beeinflusst werden, unter ihnen Hardware und Firmware, Betriebssystem, Softwarebibliotheken, Compiler und Tools, Laufzeitumgebungen und Umgebungsvariablen. Container bieten ein Rahmenwerk, um diese Freiheitsgrade zu fixieren, indem exakte Zustände einer Softwareumgebung festgehalten werden können. Die gesamte Umgebung kann bei Bedarf verlässlich rekreiert werden, unabhängig von der zugrundeliegenden Infrastruktur. \cite{013:Role-of-Containers-in-Reproducibility} Damit leisten Container einen wichtigen Beitrag zu Reproduzierbarkeit und Konsistenz \cite{013:Role-of-Containers-in-Reproducibility,014:Managing-Container-based-Software-Development-Environments,023:Setting-up-CI-CD-Pipeline-in-the-Cloud-for-Web-Application,024:Investiugating-Impact-of-Containerization-on-Deployment-Process-in-DevOps}.

\nameref{sec:02-03_containerization} wurde bereits in \autoref{sec:02-03_containerization} motiviert. Die vorherrschende Technologie in diesem Bereich ist \textit{Docker} \cite{015:Containers-in-Software-Development,021:Docker-Security-Threat-Model-and-Best-Practices}. \textit{Docker} ist eine freue, quelloffene und relativ neue Software, die in der Lage ist, deutlich mehr virtuelle Umgebungen auf der gleichen Hardware zu deployen als konkurrierende Technologien. Sie kann zusammen mit Drittanbietertechnologien genutzt werden und ist kompatibel mit vielen \hyperref[sec:03-01_devops]{DevOps} Tools. \textit{Docker} ermöglicht das Bauen, Verteilen und Operieren von Anwendungen. \cite{021:Docker-Security-Threat-Model-and-Best-Practices} Das \Gls{deployment} von Containern ist dabei über mehrere Umgebungen und Clouds hinweg möglich \cite{023:Setting-up-CI-CD-Pipeline-in-the-Cloud-for-Web-Application,024:Investiugating-Impact-of-Containerization-on-Deployment-Process-in-DevOps}.

Bei Container-basierter Virtualisierung von Betriebssystemen wird eine Repräsentation des zugrundeliegenden Betriebssystems erzeugt, welche in einer isolierten Umgebung ausgeführt werden kann. Umgebungen eines Containers sind in sich geschlossen und reflektieren das in ihren Quelldaten beschriebene System ganz spezifisch. Kernprinzipien von \nameref{sec:02-03_containerization} sind die Isolierung und Virtualisierung von Prozessen, dementsprechend wird häufig von einer Virtualisierungsmethode auf Prozess- beziehungsweise auf Applikationsebene gesprochen. Damit dies möglich ist, enthalten Container alle zur Ausführung der paketierten Software notwendigen Bibliotheken und Abhängigkeiten. Üblicherweise soll ein Docker Container genau eine Applikation respektive eine einzelne Aufgabe beinhalten. \cite{014:Managing-Container-based-Software-Development-Environments}

\begin{figure}[h]
    \centering
    \includegraphics[width=0.95\textwidth]{g-16_architecture-of-docker.png}
    \caption{Architektur von Docker \acrshort{iAa} \citeauthor{021:Docker-Security-Threat-Model-and-Best-Practices}}
    \label{fig:g-16_architecture-of-docker}
\end{figure}

Die Architektur von \textit{Docker} ist leicht vereinfacht in \autoref{fig:g-16_architecture-of-docker} dargestellt. Grundlegender Bestandteil von \textit{Docker} ist die \textbf{Docker Engine}. Sie wurde von der \textit{Docker Inc.} entwickelt und beinhaltet den \textbf{Docker Daemon}, der für die Ausführung und Verwaltung von Containern verantwortlich ist. Nutzer können den Docker Daemon über eine \textbf{Client API} ansprechen, welche wiederum mit einer \textbf{\Gls{rest} API} kommuniziert. Hierüber ist die Interaktion mit Containern sowie mit \textit{Docker} Funktionalitäten wie \texttt{docker build} zum Erstellen, \texttt{docker pull} zum Beziehen beziehungsweise Herunterladen oder \texttt{docker run} zum Ausführen von Containern möglich. \textbf{Docker \Glspl{image}} sind fast immer die Basis von Containern. Sie bestehen aus mehreren aufeinander aufbauenden Schichten, \textbf{Layer} genannt. Die unterste dieser Layer ist das \textbf{Base \Gls{image}} Jede Modifikation an einem Container führt zu einem neuen Layer im resultierenden \Gls{image}. Um Dateien über die verschiedenen Layer hinweg handhaben zu können, verwendet \textit{Docker} das \textbf{Union File System}. \Glspl{image}, die nicht lokal verfügbar sind, müssen über ein \textbf{\Gls{image} Repository} wie \textit{Docker Hub} bezogen werden. Dieses ermöglicht die private oder öffentliche Ablage von \Glspl{image}, Nutzer können dort also ihre eigene Version eines Containers veröffentlichen. \cite{021:Docker-Security-Threat-Model-and-Best-Practices}

Der Ausgangspunkt eines Docker \Glspl{image} ist immer ein \textbf{Dockerfile}. Es enthält eine deklarative Beschreibung des Base \Glspl{image} inklusive Tags und einer konkreten Version sowie Angaben zur weiteren Konfiguration der Umgebung, die im späteren Container existieren soll. Dockerfiles sind im Format einer Textdatei abgelegt. Während des durch \texttt{docker build} ausgelösten \Gls{build} Prozesses generiert \textit{Docker} aus dieser Datei ein Docker \Gls{image}. \cite{020:Assessing-and-Improving-Quality-of-Docker-Artifacts} Ein einmal erstelltes \Gls{image} ist unveränderlich (englisch \q{immutable}). Dadurch ist sichergestellt, dass bei jedem Start eines \Glspl{image} immer die exakt gleiche Umgebung zur Verfügung steht. \cite{014:Managing-Container-based-Software-Development-Environments} Ein gebautes Docker Image kann auf \textit{DockerHub} zur Verfügung gestellt werden. Von dort aus kann es selbst als Base \Gls{image} in einem anderen Dockerfile referenziert werden.

\textit{Docker} ermöglicht die Beschleunigung von \Gls{cicd} Prozessen. Eine Pipeline lädt zunächst, das Base \Gls{image}, auf dessen Grundlage anschließend alle Anweisungen des Dockerfiles ausgeführt werden. Ist der \Gls{build} Prozess abgeschlossen, werden die ausgeführten Zeilen für zukünftige \Glspl{build} im Cache abgelegt, wodurch Layer wiederverwendet werden können. \cite{022:Automated-Cloud-Infrastructure-Continous-Integration-and-Continous-Delivery-using-Docker}

Ein weiteres wichtiges Tool ist \textit{Docker Compose}, welches den Mehr-Container Betrieb mit \textit{Docker} vereinfachen soll. Das Ziel ist die Vereinfachung der Kontrolle mehrerer Containerkomponenten wie Dienste, Netzwerke oder Volumen, indem sie für beliebig viele sogenannte \textbf{Services} in einer Konfigurationsdatei im \Gls{yaml} Format beschrieben werden. \textit{Docker Compose} bietet Kommandos zur Verwaltung ganzer Lebenszyklen einer Umgebung an und stellt eine weitere Effizienzsteigerung in Development- und Deploymentprozessen dar. \cite{308:Docker-Compose-Overview}

\subsection{Dev Container}
\label{subsec:05-01-02_dev-container}

Konsistenz und Einheitlichkeit sind auch im Entwicklungsprozess gewünschte Eigenschaften, was die Verwendung von \nameref{sec:02-03_containerization} Lösungen im \Gls{development} motiviert. Container können auf solche Weise genutzt werden, dass nur wichtige Komponenten wie Laufzeitbibliotheken oder Datenbankserver standardisiert werden (wie von \texttt{\hyperref[sec:AA-02_interview-persons]{IP-2}} gefordert, \acrshort{vgl} \autoref{subsec:AA-03-01_open-questions}), sodass die Developmentumgebung nach den persönlichen Bedürfnissen der Entwickler angepasst werden kann (wie von \texttt{\hyperref[sec:AA-02_interview-persons]{IP-1}} gefordert, \acrshort{vgl} \autoref{subsec:AA-03-01_open-questions}). \cite{014:Managing-Container-based-Software-Development-Environments}

Die grundlegende Idee hinter einer Container-basierten Entwicklungsumgebung ist ein Software Container, welcher zusätzlich Tools, Plugins, Software und Konfiguration für \Gls{development}, Test und \Gls{deployment} enthält, ähnlich wie \autoref{fig:g-17_architecture-of-dev-containers} es zeigt. Deren Ziel ist es, eine möglichst vollständige Entkopplung der Developmentumgebung von individuellen Maschinen zu erreichen. Ein solches Konzept lässt nach \citeauthor{014:Managing-Container-based-Software-Development-Environments} erwarten, dass sein Mehrwert sich proportional zur Komplexität der Organisation verhält, in der es eingesetzt wird. \cite{014:Managing-Container-based-Software-Development-Environments} Es verspricht das klassische \qit{It works on my Machine}-Problem zu lösen. \cite{204:Development-Containers-Simplified} Eine konkrete Implementierung des beschriebenen Ansatzes sollen Development Container, oder kurz \q{Dev Container} (siehe \autoref{fig:g-17_architecture-of-dev-containers}) liefern. Dev Container ermöglichen die Nutzung von Software Containern als vollständig ausgestattete Entwicklungsumgebungen \cite{303:Introduction-to-DevContainers,305:Using-DevContainers-in-JetBrains-IDEs,306:Development-Containers}. Vereinfacht handelt es sich um Docker Container, welche zusätzlich das Projekt, die \Gls{ide} und alles weitere enthalten \cite{305:Using-DevContainers-in-JetBrains-IDEs}.

\begin{figure}[h]
    \centering
    \includegraphics[width=0.95\textwidth]{g-17_architecture-of-dev-containers.png}
    \caption{Architektur von Dev Containern}
    \label{fig:g-17_architecture-of-dev-containers}
\end{figure}

In \autoref{fig:g-17_architecture-of-dev-containers} ist zu erkennen, wie die \nameref{fig:g-17_architecture-of-dev-containers} auf der \hyperref[fig:g-16_architecture-of-docker]{Architektur von Docker} (siehe \autoref{fig:g-16_architecture-of-docker}) aufbaut. Ein Docker Container enthält Binärdateien, Bibliotheken und die essentiellen Prozesse der Applikation, seine Funktionsweise unterscheidet sich nicht von der gewöhnlicher \nameref{subsec:05-01-01_docker-container}. Eine neue Komponente ist der \textbf{\Gls{ide} Server} inklusive wichtiger Einstellungen, Tools und Plugins, die für die Entwicklung an dem Projekt notwendig oder hilfreich sind. Er kann mit einer aus mehreren aktuell unterstützten \Glspl{ide} verbunden werden, welche direkt auf dem \Gls{hostsystem} laufen und lediglich ein \Gls{gui} bereitstellen, über das wie gewohnt mit dem Projekt interagiert werden kann.

Jeder Dev Container verfügt über ein eigenes \textbf{File System}, welches das Projekt mit seinen Dateien und Verzeichnissen enthält. Dieses File System kann entweder über einen \textbf{Volume Mount} direkt mit dem \Gls{hostsystem} synchronisiert werden oder der Klon eines \textit{\Gls{git}} Repositories sein. Die einzigen Installationen, die auf dem lokalen System des Entwicklers notwendig sind, um Dev Container zu nutzen, sind \textit{Docker} und eine \Gls{ide} beziehungsweise ein Editor, welcher die Verwendung von Dev Containern unterstützt.

Die \nameref{sec:02-03_containerization} von Applikationen und Entwicklungsumgebungen bietet vielfältige Möglichkeiten für Entwickler. Sie unterstützt bei der Trennung von Tools, Bibliotheken und Laufzeitumgebungen, kann Applikationen ausführen und eignet sich für den Einsatz in \acrfull{ci} und beim Testen. Genau wie Docker Container können Dev Container lokal oder in privaten und hybriden Clouds betrieben werden. \cite{306:Development-Containers} Sie können dazu beitragen, drei Prinzipien zu erfüllen, die die Produktivität im \Gls{development} steigern, indem sie die Entwicklungsumgebung vom \Gls{hostsystem} isolieren, zu ihrer Konsistenz im gesamten Entwicklungsteam beitragen und gleichzeitig ausreichend flexibel gegenüber Anpassungen an die spezifischen Bedürfnisse von Entwicklern und Projekten sind. Dies hat positive Auswirkungen auf das Onboarding neuer Entwickler, die Geschwindigkeit beim Aufsetzen des Workspace, sowie auf Cross-Plattform Kompatibilität. Außerdem können Dev Container die Zusammenarbeit fördern und Remote \Gls{development} auf entfernten Maschinen erleichtern. \cite{200:Dev-Containers-Future-of-Development-Environments,202:Maximizing-Efficiency-with-Dev-Containers,305:Using-DevContainers-in-JetBrains-IDEs}.

Zusammengefasst erweitern Dev Container also die Anwendungsfälle von Containern vom \Gls{deployment} auf das \Gls{development}. Unter anderem im Bereich des \nameref{sec:02-01_web-development} besteht ein großer Bedarf an solchen Lösungen. \citeauthor{202:Maximizing-Efficiency-with-Dev-Containers} identifiziert zwei zentrale Anwendungsfälle in diesem Gebiet. Die Nutzung als primäre Entwicklungsumgebung und gleichzeitig in der Integration wurde bereits angerissen. Bei der Identifizierung und Behebung von Fehlern in laufenden Systemen können Dev Container ebenfalls hilfreich sein. Durch eine intuitivere Interaktion mit laufenden Containern reduzieren sie hier die weiteren Hindernisse für Entwickler. Trotz vieler Vorteile sind Dev Container jedoch nicht für jeden Anwendungsfall die passende Lösung. Ihre Abhängigkeit zu \textit{Docker}, Hardwarebeschränkungen und Beeinträchtigungen der Performanz durch eine zusätzliche Abstraktionsschicht sollten bei der Entscheidung berücksichtigt werden. \cite{202:Maximizing-Efficiency-with-Dev-Containers}

Im Folgenden soll ein Einblick in die technische Funktionsweise von Dev Containern und deren Einsatz in der Praxis gegeben werden. Das zentrale Element eines Dev Containers ist eine Konfigurationsdatei auf Projektebene, welche die deklarative Beschreibung seines Aufbaus enthält, die \texttt{devcontainer.json} \cite{202:Maximizing-Efficiency-with-Dev-Containers}. Ihr Format ist entspricht \textit{\Gls{json} with Comments} und sie wird standardmäßig an drei Orten im Verzeichnisbaum eines Projekts gesucht: im Wurzelverzeichnis des Projekts (\texttt{\$PROJECT\_ROOT\$/.devcontainer.json}), in einem \texttt{.devcontainer} Verzeichnis unmittelbar eine Ebene unterhalb des Wurzelverzeichnisses (\texttt{\$PROJECT\_ROOT\$/.devcontainer/devcontainer.json}) oder in maximal einem weiteren Subverzeichnis dieses Pfades (\texttt{\$PROJECT\_ROOT\$/.devcontainer/\$FOLDER\$/.devcontainer.json}) \cite{204:Development-Containers-Simplified,303:Introduction-to-DevContainers,305:Using-DevContainers-in-JetBrains-IDEs}. Dadurch passt sich die Konfiguration eines Projekts beliebigen Komplexitätsebenen an, indem eine möglichst sprechende Platzierung unterstützt und die Verwendung mehrerer Konfigurationen ermöglicht wird. Zusätzliche Individualisierungen sind möglich, indem beispielsweise ein eigenes \texttt{Dockerfile} als Base Image für den Dev Container definiert wird. Auch die Verwendung von \textit{Docker Compose} ist möglich, um mehrere Container zu orchestrieren. \cite{202:Maximizing-Efficiency-with-Dev-Containers}

Werden Dev Container zusammen mit dem Editor \textit{Visual Studio Code} von \textit{Microsoft} genutzt, werden Änderungen an der \texttt{devcontainer.json} Datei automatisch erkannt und dem Nutzer wird proaktiv ein Rebuild des Containers vorgeschlagen. Auch ist dann die Angabe von spezifischen Erweiterungen und deren Einstellungen möglich. Diese können deklarativ angegeben werden und integrieren dann automatisch beim Erstellen des Dev Containers bestimmte Softwarekomponenten direkt im Container. \cite{201:How-to-develop-with-Dev-Containers} \textit{Microsoft} unternimmt mit dem \citetitle{306:Development-Containers} Standard außerdem einen Versuch zur Standardisierung von Dev Containern. Er beinhaltet neben der technischen Spezifikation eine \Gls{json} Referenz, eine Liste aus Tools, in denen die Verwendung von Dev Containern aktuell unterstützt wird, sowie eine Sammlung sogenannter Features, deren Zweck es ist, besonders einfach über die Konfiguration in Dev Containern installierbar zu sein. Die Spezifikation enthält einen Überblick über allgemeine Konfigurationen wie den Namen des Dev Containers, Szenario-bezogene Eigenschaften für Dockerfiles, Docker Images oder \textit{Docker Compose}, Tool-spezifische Einstellungen, Optionen für Skripte in bestimmten Lebenszyklusphasen des Containers, Angaben zu Mindestanforderungen an die Hardware und Einstellungen für die Portfreigabe. Außerdem finden sich dort Angaben zu Datentypen und deren Formatierungen sowie eine Beschreibung zur Verwendung von Variablen. \cite{306:Development-Containers}

Durch die zahlreichen Möglichkeiten ist im Grunde genommen eine beliebig komplexe und individuelle Konfiguration eines Dev Containers möglich. Der überwiegende Teil der Attribute ist jedoch optional. Eine minimale Konfiguration für eine voll funktionsfähige Umgebung muss den Namen des Dev Containers (\texttt{name}), das Base Image (\texttt{image}) und die wichtigsten Abhängigkeiten (\texttt{features}) enthalten. Empfehlenswert ist außerdem die Verwendung von Lifecycle Scripts (\texttt{initializeCommand}, \texttt{postCreateCommand}, \texttt{postStartCommand} und \texttt{postAttachCommand}). \cite{204:Development-Containers-Simplified}

Der Blogger \citeauthor{203:Dev-Environment-as-a-Code-with-DevContainers-Dotfiles-and-GitHub-Codespaces} beschreibt in dem Artikel \citetitle{203:Dev-Environment-as-a-Code-with-DevContainers-Dotfiles-and-GitHub-Codespaces} einen Ansatz für Entwicklungsumgebungen, der auf Dev Containern basiert. Er zeigt dort unter anderem die typische Anatomie einer Entwicklungsumgebung, bestehend aus persönlichen und projektbezogenen Konfigurationen und Tools. Im Kontext der Individualisierung greift er unter anderem auf die Idee von \hyperref[sec:03-04_dotfiles]{Dotfiles} zurück, mit deren Hilfe er seine Entwicklungsumgebung personalisiert. \cite{203:Dev-Environment-as-a-Code-with-DevContainers-Dotfiles-and-GitHub-Codespaces}

\textit{GitHub Codespaces}, ein Produkt von \textit{GitHub}, ermöglicht die Nutzung von Dev Containern in der Cloud und besitzt die Fähigkeit, solche Dotfile Repositories einzubinden. Folgt die Installationsdatei einer der in \autoref{sec:03-04_dotfiles} beschriebenen Namenskonventionen, kann sie automatisch aus einem persönlichen Dotfile Repository geladen und zusammen mit den restlichen Konfigurationsdateien in den Dev Container kopiert werden. \cite{304:Personalizing-GitHub-Codespaces-for-your-Account} Dies ermöglicht jedem Entwickler die Angabe eines eigenen Dotfile Repositories mit einer individuellen Konfiguration seiner Developmentumgebung, ohne diese jedoch in den zentralen Ressourcen des Projekts ablegen zu müssen. \textit{Visual Studio Code} unterstützt die Verwendung von Dotfiles ebenfalls, sofern diese in den Nutzereinstellungen konfiguriert ist \cite{201:How-to-develop-with-Dev-Containers}. Alternativ besteht die Möglichkeit, die Einstellungen des Editors über ein \textit{GitHub} Konto mit der Cloud zu synchronisieren \cite{304:Personalizing-GitHub-Codespaces-for-your-Account}.

\section{Strategie für Toolchains}
\label{sec:05-02_strategy-for-toolchains}

\subsection{Teilkonzepte zur Realisierung}
\label{subsec:05-02-01_sub-concepts-for-implementation}

Die \nameref{ch:05_toolchain-as-code} Strategie soll sich maßgeblich an denen in \autoref{ch:04_requirements-for-development-and-deployment-environments} ermittelten Anforderungen der Entwickler orientieren. Dabei soll für jede Anforderung ein Teilkonzept formuliert werden, welches auf Erkenntnissen aus vorherigen Untersuchungen der Arbeit basiert, größtenteils auf der \nameref{ch:03_examination-of-existing-approaches} (\autoref{ch:03_examination-of-existing-approaches}). Insgesamt konnte jedem der ursprünglichen Anforderungen ein Teilkonzept zugeordnet werden.

Entsprechend diesem Vorgehen werden die Teilkonzepte, genau wie die Anforderungen, in die drei Bereiche \Gls{dev}, \Gls{dep} und \Gls{cnt} unterteilt. Ihre Sortierung entspricht der ursprünglichen Priorisierung der Anforderungen, wie in \autoref{subsec:04-02-01_procedure-for-evaluation} erklärt.

In \autoref{tab:concepts-development} sind die \nameref{tab:concepts-development} formuliert, die die \nameref{tab:requirements-development} (siehe \autoref{tab:requirements-development}) erfüllen.

\begin{table}[H]
    \begin{tabular}{ >{\bfseries\ttfamily}p{0.10\textwidth} >{}p{0.30\textwidth} | >{}p{0.50\textwidth} }
        DEV-0   &   Eine Entwicklungsumgebung wird schnell und vollständig geladen. &
        \nameref{subsec:05-01-02_dev-container} stellen eine Umgebung inklusive ihrer Konfiguration bereit, die gemeinsam gestartet werden können. \\
        \hline
        DEV-1   &   Die Applikation ist auf den lokalen Maschinen der Entwickler ausführbar. &
        Die Applikation ist in einem eigenständigen \nameref{subsec:05-01-01_docker-container} gekapselt, der lokal ausführbar ist und kann beim Entwickler gestartet werden. \\
        \hline
        DEV-2   &   Konfiguration ist im \Gls{vcs} verfügbar. &
        Gemäß dem \hyperref[sec:03-03_gitops]{GitOps} Ansatz liegen alle Konfigurationen als Dateien in den Repositories vor. \\
        \hline
        DEV-3   &   Alle nötigen Abhängigkeiten und \Glspl{sdk} sind in der Umgebung bereitgestellt. &
        \nameref{subsec:05-01-01_docker-container} enthalten das Basisbetriebssystem, \Glspl{sdk}, Bibliotheken, \Gls{build} Tools und weitere Softwarekomponenten. \\
        \hline
        DEV-4   &   Dokumentation ist aktuell und intuitiv. &
        Deklarative Konfigurationen sind selbst-dokumentierend und \textit{Markdown} Dateien liefern einen kompakten Überblick inklusive Anleitung zum Starten des Projekts. \\
        \hline
        DEV-5   &   Skripte reduzieren komplexere imperative Aufgaben auf einen Befehl. &
        \textit{Task-Files} stehen zur Verfügung, um aufwändige Befehlskompositionen über kurze und zentrale Kommandos zu starten. \\
    \end{tabular}
    \caption{Teilkonzepte für Toolchains im Bereich Development}
    \label{tab:concepts-development}
\end{table}

In \autoref{tab:concepts-deployment} sind die \nameref{tab:concepts-deployment} formuliert, die die \nameref{tab:requirements-deployment} (siehe \autoref{tab:requirements-deployment}) erfüllen.

\begin{table}[H]
    \begin{tabular}{ >{\bfseries\ttfamily}p{0.10\textwidth} >{}p{0.30\textwidth} | >{}p{0.50\textwidth} }
        DEP-0   &   Mit Ausnahme einer händischen Bestätigung ist das \Gls{deployment} vollständig automatisiert. &
        Beim \Gls{deployment} der Umgebung in ein \Gls{image} Repository ist in der \Gls{cicd} Konfiguration eine manuelle Bestätigung erforderlich. \\
        \hline
        DEP-1   &   Konfiguration wird aus dem \Gls{vcs} bezogen. &
        Gemäß dem \hyperref[sec:03-03_gitops]{GitOps} Ansatz verfügt jedes Repository über eine \Gls{cicd} Konfiguration, sofern dies sinnvoll oder erforderlich ist. \\
        \hline
        DEP-2   &   Umgebung und Applikation sind Container. &
        \textit{Docker} packt die Applikation inklusive aller Abhängigkeiten in einem Container. \\
        \hline
        DEP-3   &   Bei Problemen werden Entwickler sofort benachrichtigt. &
        \Gls{github-actions} Workflows können  so konfiguriert werden, dass sie Benachrichtigungen bei Fehlern in einer Pipeline per E-Mail versenden. \\
    \end{tabular}
    \caption{Teilkonzepte für Toolchains im Bereich Deployment}
    \label{tab:concepts-deployment}
\end{table}

In \autoref{tab:concepts-continuity} sind die \nameref{tab:concepts-continuity} formuliert, die die \nameref{tab:requirements-continuity} (siehe \autoref{tab:requirements-continuity}) erfüllen.

\begin{table}[H]
    \begin{tabular}{ >{\bfseries\ttfamily}p{0.10\textwidth} >{}p{0.30\textwidth} | >{}p{0.50\textwidth} }
        CNT-0   &   Automatisierung erfolgt auf allen Ebenen über die gleichen Tools. &
        Durch Kombination von \Gls{cicd} aus dem \hyperref[sec:03-03_gitops]{GitOps} Ansatz und \nameref{sec:02-03_containerization} ist gewährleistet, dass Integrations- und Deploymentkonfigurationen genauso wie Abhängigkeiten und Werkzeuge immer von der gleichen Quelle bezogen werden. \\
        \hline
        CNT-1   &   Die lokale Umgebung entspricht möglichst vollständig der Produktivumgebung. &
        Sowohl im \Gls{development} als auch im \Gls{deployment} werden die gleichen Container und Konfigurationen verwendet. Beim \Gls{development} steht außerdem eine Datenbank mit dem gleichen Schema und mindestens Beispieldaten zur Verfügung. \\
        \hline
        CNT-2   &   Alle Umgebungen sind mit möglichst wenig Tooleinsatz durch das Entwicklungsteam anpassbar. &
        Da gemäß \hyperref[sec:03-03_gitops]{GitOps} alle Konfigurationen im Code vorliegen, kann jeder Entwickler mit grundlegendem Verständnis von \Gls{vcs} wie \textit{Git} diese lesen und anpassen. \\
        \hline
        CNT-3   &   Die Umgebung sowie das Verhalten der Applikation in ihr sind auf allen Ebenen reproduzierbar. &
        \nameref{subsec:05-01-01_docker-container} und \nameref{subsec:05-01-02_dev-container} kapseln Applikation und Umgebung in einem in sich geschlossenem und reproduzierbarem System. \\
    \end{tabular}
    \caption{Teilkonzepte für Toolchains im Bereich Durchgängigkeit}
    \label{tab:concepts-continuity}
\end{table}

In \autoref{sec:03-05_concept-of-twelve-factor-app} wurde die \hyperref[sec:03-05_concept-of-twelve-factor-app]{Twelve-Factor-App} vorgestellt. Sie besteht aus zwölf Faktoren, die \textit{Software as a Service} erfüllen sollte. Im Rahmen der \hyperref[sec:04-01_collection-of-requirements-using-expert-interviews]{Erhebung von Anforderungen an Umgebungen} wurden diese Faktoren durch Experten priorisiert (\acrshort{vgl}\autoref{tab:interview-results-factors-priorities}). Der als am wichtigsten bewerteten oberen Hälfte dieser Faktoren sollen daher \nameref{tab:concepts-factors} zugeordnet werden. Diese sind in \autoref{tab:concepts-factors} dargestellt.

\setcounter{factorno}{-1}
\begin{table}[H]
    \begin{tabular}{ | >{\bfseries}p{0.0125\textwidth} | >{\raggedright\bfseries}p{0.2500\textwidth} | >{}p{0.6375\textwidth} | }
        \hline
              
            & \textbf{Faktor} 
            & \textbf{Teilkonzept} \\
        \hline \hline
        %   (I) Codebase
            0
            & %\setcounter{factorno}{01}\Roman{factorno}
            Codebase
            & Die \Gls{codebase} der Umgebung kann für Integration und Entwicklung genutzt werden, die \Gls{codebase} der Applikation ist lokal und als Deployment verwendbar. \\
        \hline
        %   (II) Dependencies
            0
            & %\setcounter{factorno}{02}\Roman{factorno}
            Dependencies
            & Jedes Repository verfügt über Konfigurationsdateien, in denen Abhängigkeiten deklarativ aufgeführt sind. \\
        \hline
        %   (V) Build, Release, Run
            0
            & %\setcounter{factorno}{05}\Roman{factorno}
            Build, Release, Run
            & GitOps lässt die Konfiguration des \Gls{cicd} Konzepts für mehrere Ebenen und verschiedene Pipelines zu. \\
        \hline
        %   (X) Dev / Prod Parity
            0
            & %\setcounter{factorno}{10}\Roman{factorno}
            Dev / Prod Parity
            & Development- und Deploymentumgebungen für die Applikation greifen auf die gleiche containerisierte Umgebung zurück. \\
        \hline
        %   (III) Config
            1
            & %\setcounter{factorno}{03}\Roman{factorno}
            Config
            & Variable Konfigurationen wie Datenbankverbindungen werden außerhalb der Revision in Umgebungsvariablen abgelegt. \\
        \hline
        %   (IV) Backing Services
            2
            & %\setcounter{factorno}{04}\Roman{factorno}
            Backing Services
            & Auf Services wie Datenbanken kann bei Bedarf lokal zugegriffen werden, sie sind jedoch nicht im Umfang des Deployments der Applikation enthalten. \\
        \hline
        %   (XI) Logs
            2
            & %\setcounter{factorno}{11}\Roman{factorno}
            Logs
            & \acrshort{na} \\
        \hline
        %   (VIII) Concurrency
            3
            & %\setcounter{factorno}{08}\Roman{factorno}
            Concurrency
            & \acrshort{na} \\
        \hline
        %   (VI) Processes
            -
            & %\setcounter{factorno}{06}\Roman{factorno}
            Processes
            & \acrshort{na} \\
        \hline
        %   (VII) Port Binding
            -
            & %\setcounter{factorno}{07}\Roman{factorno}
            Port Binding
            & \acrshort{na} \\
        \hline
        %   (IX) Disposability
            -
            & %\setcounter{factorno}{09}\Roman{factorno}
            Disposability
            & \acrshort{na} \\
        \hline
        %   (XII) Admin Processes
            -
            & %\setcounter{factorno}{12}\Roman{factorno}
            Admin Processes
            & \acrshort{na} \\
        \hline
    \end{tabular}
    \caption{Teilkonzepte für Toolchains auf Basis der \q{Twelve-Factor-App}}
    \label{tab:concepts-factors}
\end{table}
\setcounter{factorno}{0}

\subsection{Architektur im Toolchain-as-Code Ansatz}
\label{subsec:05-02-02_architecture-in-the-toolchain-as-code-approach}

\subsubsection{Datenhaltung in Ablageorten}
\label{subsubsec:05-02-02-01_data-storage-in-repositories}

Die Datenhaltung der \nameref{ch:05_toolchain-as-code} Strategie basiert auf den Prinzipien von \hyperref[sec:03-03_gitops]{GitOps} und dem \hyperref[sec:03-04_dotfiles]{Dotfiles} Konzept. Genau wie bei \hyperref[sec:03-03_gitops]{GitOps} gibt es zwei Repositories.

Das \textbf{Environment Repository} enthält die Konfiguration der Container für Integration Environment und Creation Environment. Diese Komponenten werden im kommenden \autoref{subsubsec:05-02-02-02_sub-components-in-environments} genauer beschrieben. Die Basis dieses Repositorys besteht aus zwei aufeinander aufbauenden Dockerfiles, jedes von ihnen für entsprechend eines der Environments. Das Dockerfile des Integration Environments ist minimal und enthält nur die zum \Gls{build} der Applikation erforderlichen Komponenten. Aufbauend auf diesem Dockerfile sind im Creation Environment zusätzlich die für die Entwicklung notwendigen Softwarekomponenten enthalten.

Das \textbf{Application Repository} enthält den Quellcode der Applikation selbst sowie die Konfiguration für ihren Container. Letztere besteht aus einem Dockerfile mit allen für den Betrieb der Applikation erforderlichen Komponenten sowie optional einem weiteren Dockerfile, welches einen Backing Service für den lokalen Betrieb der Applikation bereitstellt. Zur bequemeren Verwaltung mehrerer Services kann bei Bedarf \textit{Docker Compose} mit einer entsprechenden Konfigurationsdatei hinzugezogen werden. Zusätzlich enthält dieses Repository auch die Konfiguration für einen \nameref{subsec:05-01-02_dev-container}, welcher die Entwicklungsumgebung für die Applikation bereitstellt. \linebreak[4]
Dafür werden in einem gesonderten Verzeichnis die deklarative Beschreibung der Entwicklungsumgebung in einer \texttt{devcontainer.json} Datei sowie ein weiteres Unterverzeichnis mit möglichen Ressourcen zur Konfiguration der Entwicklungsumgebung abgelegt. Die \texttt{devcontainer.json} Konfiguration kann auch auf einer optionalen \textit{Docker Compose} Konfiguration basieren, welche neben dem \nameref{subsec:05-01-02_dev-container} auch den \nameref{subsec:05-01-01_docker-container} der Applikation einbinden kann.

Nicht notwendig, jedoch nützlich kann das Einbinden eines dritten Repositories sein, welches \hyperref[sec:03-04_dotfiles]{Dotfiles} zur Individualisierung der Entwicklungsumgebung enthält. Es besteht dann aus Konfigurationsdateien sowie einem Installationsdatei zur automatischen Einrichtung. Bei diesem Repository kann es sich um ein zentrales handeln, aus welchem die Konfigurationen für alle Entwickler einheitlich gezogen werden, oder um die persönlichen \hyperref[sec:03-04_dotfiles]{Dotfiles} jedes Entwicklers, welche über ein privates Repository verwaltet werden.

Alle Artefakt-generierenden Repositories, also das Environment Repository und das Application Repository, enthalten zusätzlich eine \Gls{cicd} Konfiguration, welche für \nameref{subsec:05-02-03_workflows-and-continuity-in-the-toolchain-as-code-approach} verantwortlich ist und in \autoref{subsec:05-02-03_workflows-and-continuity-in-the-toolchain-as-code-approach} beschrieben wird.

Eine vollständige Übersicht über die Struktur der Repositories, ihre Dateien und Verzeichnisse sowie die Beziehungen untereinander kann \autoref{fig:g-18_repositories-in-toolchain-as-code-strategy} entnommen werden.

\begin{figure}[htp]
    \centering
    \includegraphics[width=1.00\textwidth]{g-18_repositories-in-toolchain-as-code-strategy.png}
    \caption{Repositories in der Toolchain-as-Code Strategie}
    \label{fig:g-18_repositories-in-toolchain-as-code-strategy}
\end{figure}

\subsubsection{Teilkomponenten in Umgebungen}
\label{subsubsec:05-02-02-02_sub-components-in-environments}

\nameref{sec:02-03_containerization} ist ein wichtiger Bestandteil im Kontext dieser Arbeit und alle Artefakte, welche durch die \Gls{cicd} Konfigurationen der Repositories generiert werden, sind Container. Insgesamt entstehen so vier verschiedene Softwareumgebungen in Form von Container \Glspl{image}, die teilweise aufeinander aufbauen oder voneinander abhängen.

Das \textbf{Integration Environment} basiert auf dem Integration Dockerfile des Environment Repositories (\acrshort{vgl} \autoref{subsubsec:05-02-02-01_data-storage-in-repositories}). Es enthält das Betriebssystem, grundlegende Abhängigkeiten wie \Glspl{sdk} sowie Tools zum \Gls{build} der Applikation. Es stellt also alle Komponenten bereit, die für die Integration der Applikation von Relevanz sind. Erweitert wird das Integration Environment durch das \textbf{Creation Environment}, dessen Dockerfile auf dem Integration Dockerfile aufbaut. Die Erweiterung erfolgt dabei um Komponenten, welche für die Entwicklung der Applikation notwendig sind, also Entwickler bei der Arbeit unterstützen.

Das \textbf{Development Environment} wird durch einen \nameref{subsec:05-01-02_dev-container} auf Grundlage des Creation Environments gebaut. Es stellt das Backend einer ausgestatteten \Gls{ide} bereit, inklusive Plugins und Einstellungen. Es ist außerdem personalisiert durch die \hyperref[sec:03-04_dotfiles]{Dotfiles} des Entwicklungsteams.

Das tatsächliche Produkt befindet sich im \textbf{Application Image}. Dieses basiert auf dem Integration Environment und enthält nur die wichtigsten Komponenten, die absolut notwendig sind, um die Applikation auszuführen. Es handelt sich also um die produktive Umgebung der Applikation. Durch sogenannte Multi-Stage Builds (siehe \nameref{subsec:05-03-01_performance} in \autoref{sec:05-03_best-practices-with-docker-and-docker-compose}) können beispielsweise \Gls{build} Tools nach der Integration entfernt werden, um die Größe des \Glspl{image} zu reduzieren. Das \Gls{image} besteht dann nur noch aus dem Betriebssystem, der Laufzeitumgebung und wichtigen Abhängigkeiten. Außerdem enthält es die ausführbare Version der Applikation und kann theoretisch an den Kunden ausgeliefert werden.

Die genauen Bestandteile der einzelnen Umgebungen sind in \autoref{fig:g-19_environments-in-toolchain-as-code-strategy} dargestellt.

\begin{figure}[htp]
    \centering
    \includegraphics[width=1.00\textwidth]{g-19_environments-in-toolchain-as-code-strategy.png}
    \caption{Umgebungen in der Toolchain-as-Code Strategie}
    \label{fig:g-19_environments-in-toolchain-as-code-strategy}
\end{figure}

\subsection{Abläufe und Kontinuität im Toolchain-as-Code Ansatz}
\label{subsec:05-02-03_workflows-and-continuity-in-the-toolchain-as-code-approach}

\autoref{fig:g-20_ci-cd-concept-of-toolchain-as-code-strategy} zeigt das \nameref{fig:g-20_ci-cd-concept-of-toolchain-as-code-strategy} basierend auf Konzepten aus \hyperref[sec:03-01_devops]{DevOps} und \hyperref[sec:03-03_gitops]{GitOps}. Zur Reduzierung der Komplexität wird ein push-basierter Ansatz (siehe \autoref{fig:g-07_gitops-push-based-deployment}) gewählt. Anders als bei \hyperref[sec:03-03_gitops]{GitOps} liegt der Fokus der \nameref{ch:05_toolchain-as-code} Strategie jedoch mehr auf der Umgebung und auf der Toolchain als auf der Applikation selbst, weshalb die Pipeline der Applikation auf dem \Gls{image} ihrer Umgebung basiert, jedoch nicht umgekehrt. Auch müssen die einzelnen Pipelines sich nicht gegenseitig in einer Kette auslösen, sondern können unabhängig voneinander arbeiten. Wenn überhaupt muss jedoch nur die Pipeline der Umgebung nach Erfolg einen Rebuild der Applikation auslösen, welche auf ihr basiert, während die Applikation immer auf der zuletzt bereitgestellten Umgebung lauffähig sein sollte.

\begin{figure}[h]
    \centering
    \includegraphics[width=1.00\textwidth]{g-20_ci-cd-concept-of-toolchain-as-code-strategy.png}
    \caption{CI/CD Konzept der Toolchain-as-Code Strategie}
    \label{fig:g-20_ci-cd-concept-of-toolchain-as-code-strategy}
\end{figure}

App Container und \nameref{subsec:05-01-02_dev-container} \circled{0} ziehen sich zunächst das Environment Image, je nachdem entweder als Integration Image oder als Creation Image, aus der entsprechenden \Gls{container-registry} für die Umgebung. Die \nameref{subsec:05-01-02_dev-container} Konfiguration erstellt daraufhin die Developmentumgebung und \circled{1} klont das Application Repository. Ein \circled{2} Commit und Push in das Application Repository, unabhängig davon, ob sie aus einem \nameref{subsec:05-01-02_dev-container} oder aus einem lokalen Arbeitsverzeichnis erfolgen, stoßen \circled{3} die Integration, also einen Rebuild der Applikation, sowie \circled{4} das \Gls{deployment}, also das Hochladen des \Glspl{image} in ein entsprechendes \Gls{container-registry}, an. Das \Gls{image} der Applikation könnte von dort aus zusätzlich noch im Rahmen von \Gls{delivery} auf Stages oder für den Kunden zur Verfügung gestellt werden, dies ist jedoch kein expliziter Teil der \nameref{ch:05_toolchain-as-code} Strategie. Analog dazu stoßen \circled{5} Commit und Push in das Environment Repository \circled{6} die Integration und \circled{7} das \Gls{deployment} der Umgebung an.

Nach erfolgreichem Bereitstellen der Umgebung als \Gls{image} könnte als weitere Automatisierung ein Rebuild der Applikation auf Basis der neuen Umgebung angestoßen werden. Änderungen an der Umgebung sind generell kritischer als Änderungen an der Applikation (wie von \texttt{\hyperref[sec:AA-02_interview-persons]{IP-3}} ausgesagt, \acrshort{vgl} \autoref{subsec:AA-03-01_open-questions}), da diese immer beide Images betreffen. Hochladen eines neuen Images der Umgebung könnte daher noch eine manuelle Bestätigung enthalten. 

\subsection{Zusammenfassung der Strategie}
\label{subsec:05-02-04_summary-of-strategy}

Drei Repositories sind für Applikation und Umgebung verantwortlich, zwei von ihnen generieren zusammen drei Artefakte mit jeweils unterschiedlichen Einsatzzielen. Die Artefakte basieren größtenteils aufeinander und sind daher möglichst konsistent.

Die in \autoref{ch:03_examination-of-existing-approaches} beschriebenen Ansätze tragen maßgeblich zur \nameref{ch:05_toolchain-as-code} Strategie bei. \hyperref[sec:03-01_devops]{DevOps} unterstützt bei der Automatisierung der Abläufe mittels \Gls{cicd} und \hyperref[sec:03-03_gitops]{GitOps} zentralisiert Aufgaben, indem es das Repository als einzige Quelle definiert und deklarative Konfigurationen fordert. Personalisierung und Individualisierung der Developmentumgebung erfolgen mittels \hyperref[sec:03-04_dotfiles]{Dotfiles}. Dabei werden die grundlegenden Konzepte der \hyperref[sec:03-05_concept-of-twelve-factor-app]{Twelve-Factor-App} berücksichtigt. Insgesamt erfüllt die Strategie die in \autoref{ch:04_requirements-for-development-and-deployment-environments} ermittelten Anforderungen der Entwickler und verfügt somit über gute Voraussetzungen für eine erfolgreiche Implementierung.

Dabei ist der Umfang der Strategie nicht starr, sondern kann den Bedingungen in der Praxis angepasst werden. So kann das \hyperref[sec:03-04_dotfiles]{Dotfiles} Repository zentral oder individuell sein, aber auch vollständig weggelassen werden. \hyperref[sec:03-01_devops]{DevOps} Pipelines könnten sich, wie in \hyperref[sec:03-03_gitops]{GitOps} vorgegeben, gegenseitig auslösen, sofern sich der zusätzliche Komplexitätsgrad im konkreten Anwendungsfall lohnt. Handelt es sich um eine besonders sicherheitskritische Infrastuktur, könnte ein pull-basierter Deployment Ansatz (siehe \autoref{fig:g-08_gitops-pull-based-deployment}) das \Gls{deployment} stärker absichern. Manuelle Schritte sind in der Strategie vollständig eliminiert, können jedoch optional eingebaut werden, sofern dies aus organisatorischen Gründen erforderlich sein sollte.

Auch die bei der Implementierung der Strategie verwendeten Tools sind nicht vorgeschrieben.

\begin{itemize}
    \item Das konkrete \textbf{\Gls{vcs}} ist frei wählbar.
    \item Die \textbf{Code Hosting Plattform} kann frei gewählt werden.
    \item Ein \textbf{Integrationssystem} kann ebenfalls selbst bestimmt werden.
    \item Über das verwendete \textbf{Containerization Tool} kann frei entschieden werden.
    \item Viele \textbf{Entwicklungsumgebungen} und \textbf{Editoren} funktionieren mit der Strategie.
    \item \textbf{Cloud Development} ist ebenfalls möglich.
    \item Das \textbf{Image Repository} kann frei gewählt werden.
    \item Selbstverständlich können auch die \textbf{Programmiersprachen}, \textbf{\Glspl{sdk}} und \textbf{\Gls{build} Tools} für die Applikation durch das Entwicklungsteam festgelegt werden.
    \item \textbf{Base Images} müssen abhängig von den Anforderungen des Projekts gewählt werden.
\end{itemize}

Einige Tools sind oft sinnvoller als andere und nicht jede Technologie ist für jedes Projekt geeignet. Einen Vorschlag für eine konkrete Technologieauswahl im \hyperref[ch:02_technological-environment]{technologischen Umfeld} dieser Arbeit macht \autoref{sec:06-01_technologies-and-tools} im Rahmen der \hyperref[ch:06_prototypical-implementation-of-the-concept]{prototypischen Umsetzung des Konzepts}.

Dieser \autoref{sec:05-02_strategy-for-toolchains} beantwortet die \acrlong{rq} \textbf{RQ-2} (siehe \autoref{sec:01-03_objectives-and-research-questions}).

\section{Best Practices mit Docker und Docker Compose}
\label{sec:05-03_best-practices-with-docker-and-docker-compose}

\subsection{Performanz}
\label{subsec:05-03-01_performance}

% place content here
Inhalt

\subsection{Sicherheit}
\label{subsec:05-03-02_security}

% place content here
Inhalt

\subsection{Wartbarkeit}
\label{subsec:05-03-03_maintainability}

% place content here
Inhalt

\subsection{Weiteres}
\label{subsec:05-03-04_further}

% place content here
Inhalt


	\chapter{Prototypische Umsetzung des Konzepts}
\label{ch:06_prototypical-implementation-of-the-concept}

\section{Technologien und Werkzeuge}
\label{sec:06-01_technologies-and-tools}

In der \nameref{subsec:05-02-04_summary-of-strategy} wurden Freiheitsgrade bei der Auswahl der konkreten Technologien und Werkzeuge für eine Implementierung genannt. In diesem Kapitel soll ein prototypischer \Gls{poc} entwickelt werden, der von möglichst vielen Entwicklern verstanden und eventuell sogar als Projektvorlage genutzt werden kann. Die konkrete Technologieauswahl soll daher möglichst verbreitete Tools enthalten, welche gut miteinander interagieren beziehungsweise eventuell sogar vom gleichen Herausgeber stammen.

Für den \Gls{poc} soll die folgende Toolchain verwendet werden:

\begin{itemize}
    \item \textit{\Gls{git}} als \textbf{\Gls{vcs}} mit \textit{GitHub} als \textbf{\textit{Code Hosting Plattform}},
    \item \textit{GitHub Actions} als \textbf{\textit{Integration System}},
    \item \textit{Docker} als \textbf{\textit{Containerization Tool}},
    \item \textit{Visual Studio Code} als \textbf{\Gls{ide}} und \textit{GitHub Codespaces} für \textbf{\textit{Cloud Development}},
    \item \textit{Docker Hub} als \textbf{\textit{Image Repository}},
    \item \textit{TypeScript} als \textbf{\textit{Programmiersprache}},
    \item \textit{Deno 2.0} als \textbf{\textit{Laufzeitumgebung}}, sowie
    \item \textit{PostgreSQL} als \textbf{\textit{Datenbank}}.
\end{itemize}

\textit{\Gls{git}} ist Namensgeber für \hyperref[sec:03-03_gitops]{GitOps} und außerdem der Standard unter den Versionsverwaltungssystemen in der Softwareentwicklung. Fast 90 \% der Entwickler nutzen es regulär \cite{207:Developer-Ecosystem} und sind vertraut mit dem Tool. Daher wird im Prototypen \textit{\Gls{git}} als \textbf{\Gls{vcs}} verwendet. Eine naheliegende und bekannte \textbf{\textit{Code Hosting Plattform}} ist \textit{GitHub}. Die Plattform bietet vollständige Unterstützung von \textit{Git} Aktionen und dient als Ablageort für zentrale Remote Repositories. \cite{301:About-GitHub-and-Git} \textit{GitHub} ist genauso verbreitet wie \textit{Git} selbst, 87 \% der Entwickler nutzen es. Sowohl im privaten als auch im unternehmerischen Kontext liegt es damit vor Alternativen wie \textit{GitLab} oder \textit{Bitbucket} \cite{207:Developer-Ecosystem}. Ebenfalls von \textit{GitHub} stammt das \textbf{Integrationssystem} \textit{GitHub Actions}, welches eine Plattform für \Gls{cicd} ist und die Automatisierung von \Gls{build}, Test und \Gls{deployment} Pipelines ermöglicht \cite{302:Understanding-GitHub-Actions}. Etwa die Hälfte der Entwickler nutzen \textit{GitHub Actions}, wodurch es auf Platz Zwei der meist verwendeten solcher Systeme liegt \cite{207:Developer-Ecosystem}. Workflows in \textit{GitHub Actions} können Benachrichtigungen an Entwickler senden, beispielsweise bei Erfolg oder Fehlschlagen einer Pipeline \cite{307:Notifications-for-Workflow-Runs}. Ein wichtiger Vorteil für die Wahl von \textit{GitHub Actions} ist die direkte Integration in \textit{GitHub}, welches bereits als Bestandteil der Toolchain ausgewählt wurde.

In der Literatur zu \nameref{sec:02-03_containerization} kommt \textit{Docker} am häufigsten vor und ist dort mit über 50 \% das mit Abstand verbreitetste \textbf{Containerization Tool} \cite{015:Containers-in-Software-Development}. Auch in der Praxis ist \textit{Docker} sehr verbreitet, etwa 54 \% der Entwickler nutzen es \cite{206:Developer-Survey-2024,207:Developer-Ecosystem}. Für den Betrieb von Containern nutzen etwa 60 \% \textit{Docker Compose} und 44 \% \textit{Docker Run} \cite{207:Developer-Ecosystem}. Daher wird auch der Prototyp \textit{Docker} verwenden. \nameref{subsec:05-01-02_dev-container} werden aktuell laut offizieller Spezifikation nur von \textit{Microsoft Visual Studio Code}, \textit{Microsoft Visual Studio} sowie \textit{JetBrains IntelliJ IDEA} unterstützt \cite{306:Development-Containers}, wobei die Integration in \textit{Visual Studio Code} am ausgereiftesten ist \cite{204:Development-Containers-Simplified}, wohingegen sich das Feature bei \textit{JetBrains} noch in der Entwicklungsphase befindet \cite{306:Development-Containers}. Außerdem ist \textit{Visual Studio Code} unter Entwicklern dreifach so verbreitet wie \textit{Visual Studio} oder \textit{IntelliJ IDEA} und etwa 74 \% von ihnen entwickeln ihren Code in dieser \textbf{\Gls{ide}}. Auch im Remote Development liegt \textit{Visual Studio Code} vorne. Während etwa 40 \% der Entwickler mit dem Editor auf entfernte Maschinen zugreifen, nutzen nur 23 \% \Glspl{ide} von \textit{JetBrains} dazu \cite{207:Developer-Ecosystem}.

\textbf{Cloud Development} war in der \nameref{subsec:05-02-04_summary-of-strategy} ebenfalls ein möglicher Freiheitsgrad. \nameref{subsec:05-01-02_dev-container} Konfigurationen für \textit{Visual Studio Code} werden auch von \textit{GitHub Codespaces} akzeptiert \cite{306:Development-Containers}, welches auf \textit{Visual Studio Code} als Basis setzt. Bei einem \textit{Codespace} handelt es sich um eine Developmentumgebung in der Cloud, die wiederholbar durch Konfigurationsdateien im \Gls{vcs} erstellt werden kann \cite{310:GitHub-Codespaces-Overview}. Technisch betrachtet, betreibt \textit{GitHub.com} dabei verschiedene Rechner auf Basis virtueller Maschinen, die in unterschiedlichen Leistungsstufen zwischen zwei und 32 Prozessorkernen genutzt werden können \cite{306:Development-Containers}. Die gesamte Entwicklungsumgebung läuft dabei in einem Webbrowser \cite{310:GitHub-Codespaces-Overview}. Solche browserbasierten \Glspl{ide} stellen eine moderne Alternative zu lokalen Entwicklungsumgebungen dar \cite{004:Continous-Integration-and-Development-Tool-Setup-and-Pipeline-Evolution}. Neben dem Browser können auch einige Desktopanwendungen sich mit einem \textit{Codespace} verbinden \cite{310:GitHub-Codespaces-Overview}. Etwa 42 \% der Entwickler nutzen \textit{GitHub Codespaces} für Cloud Development \cite{207:Developer-Ecosystem}. Die verbreitetsten \textbf{Image Repositories} sind \textit{Docker Hub} und \textit{GitHub Container Registry}, wobei \textit{Docker Hub} mit insgesamt 19 \% vorne liegt \cite{207:Developer-Ecosystem}. Da \textit{Docker} bereits in die Toolchain integriert wurde, wird auch \textit{Docker Hub} als Image Repository verwendet. Genau wie \textit{Docker} stammt es von der \textit{Docker Inc}.

Der Prototyp wird einen \hyperref[sec:02-02_microservices]{Microservice} mit einer trivialen Schnittstelle bereitstellen. Dafür werden eine Programmiersprache aus dem Bereich \nameref{sec:02-01_web-development}, eine Laufzeitumgebung sowie eine Datenbank als Backing Service benötigt. Zum Zeitpunkt dieser Arbeit ist \textit{JavaScript} die am meisten verwendete Programmiersprache. Etwa zwei Drittel entwickeln Software mit \textit{JavaScript}, etwa die Hälfte weniger nutzen \textit{TypeScript} \cite{206:Developer-Survey-2024,207:Developer-Ecosystem}. Allerdings könnten sich fast 50 \% der Nutzer von \textit{JavaScript} vorstellen, zu \textit{TypeScript} zu wechseln \cite{206:Developer-Survey-2024}, weshalb die \textbf{Programmiersprache} \textit{TypeScript} für den Prototypen ausgewählt wird. Als \textbf{Laufzeitumgebung} soll \textit{Deno 2.0} verwendet werden. Es handelt sich dabei um eine relativ junge Laufzeitumgebung für \textit{JavaScript} und moderne Webanwendungen, die quelloffen ist und aktuell über 200 Tausend aktive Nutzer hat. \textit{Deno} kommt mit integrierten Features wie Code Linter, Code Formatter und Test Runner. Von Haus aus kann \textit{JavaScript} sogar zu ausführbaren Anwendungen kompiliert werden. Viele moderne Sicherheitsfeatures werden ebenfalls mitgeliefert. Ein wichtiger Vorteil von \textit{Deno 2.0} ist seine Abwärtskompatibilität zu \textit{Node.js}, gegenüber welchem es jedoch deutlich performanter ist. \cite{309:Deno} Als \textbf{Datenbank} wird \textit{PostgreSQL} verwendet. Es hat \textit{MySQL} mittlerweile bei den beliebtesten Datenbanken überholt und wird von fast der Hälfte der Entwickler genutzt. \cite{206:Developer-Survey-2024}

\section{Umsetzung einer Toolchain-as-Code Konfiguration}
\label{sec:06-02_implementation-of-a-toolchain-as-code-configuration}

Der Prototyp befindet sich als quelloffenes Projekt in mehreren Repositories einer \textit{GitHub Organization} und kann über \url{https://github.com/Toolchain-as-Code/} aufgerufen werden. Dabei besteht das Projekt aus drei Hauptkomponenten, abgelegt in je einem dedizierten Repository (siehe \autoref{fig:s-00_overview-repositories-github-organization}), deren Zwecke sich an der \nameref{ch:05_toolchain-as-code} Strategie ausrichten.

\begin{table}[H]
    \centering
    \begin{tabular}{ >{\bfseries\raggedright}p{0.3\textwidth} >{}p{0.7\textwidth} }
        Application Repository &
        \url{https://github.com/Toolchain-as-Code/tac-application} \\
        Environment Repository &
        \url{https://github.com/Toolchain-as-Code/tac-environment} \\
        Dotfiles Repository &
        \url{https://github.com/Toolchain-as-Code/tac-dotfiles} \\
    \end{tabular}
\end{table}

\begin{figure}[h]
    \centering
    \includegraphics[width=0.95\textwidth]{s-00_overview-repositories-github-organization.png}
    \caption{Übersicht über Repositories in \textit{GitHub} Organisation}
    \label{fig:s-00_overview-repositories-github-organization}
\end{figure}

\begin{wrapfigure}{r}{0.4\textwidth}
    \vspace{-20pt}
    \centering
    \includegraphics[width=0.39\textwidth]{s-01_using-template-repository.png}
    \caption{Verwendung eines Template Repositories}
    \label{fig:s-01_using-template-repository}
    \vspace{-00pt}
\end{wrapfigure}

Jedes Repository ist als sogenanntes \q{Template Repository} konfiguriert, sodass es von grundsätzlich jedem \textit{GitHub} Nutzer als Vorlage bei der Erstellung eines neuen Repositorys verwendet werden kann. In dem Fall werden alle Dateien und Verzeichnisse des Template Repositorys in das neue Repository kopiert. \autoref{fig:s-01_using-template-repository} zeigt, wie direkt in \textit{GitHub} mittels weniger Klicks ein neues Repository oder ein \textit{GitHub Codespace} auf Basis des Template Repositorys erstellt werden kann. Dank einer \textit{MIT License} kann das Projekt von jedem Nutzer frei verwendet, modifiziert und weitergegeben werden.

Jedes Repository enthält standardmäßig vier Dateien, die teilweise zur Konfiguration und teilweise zur Dokumentation dienen. Eine \texttt{README.md} Datei enthält eine kurze Beschreibung des Repositorys und eine Anleitung zu dessen Verwendung, während eine \texttt{LICENSE} Datei die Lizenz des Projekts festlegt. Da das \Gls{vcs} \textit{\Gls{git}} verwendet wird, ist eine \texttt{.gitignore} Datei enthalten, mit deren Hilfe bestimmte Dateien und Verzeichnisse von der Versionskontrolle ausgeschlossen werden können. Spezifischere Konfigurationen sind in einer \texttt{.gitattributes} Datei angegeben, welche beispielsweise den Typ der Zeilenumbrüche für verschiedene Betriebssysteme verwaltet.

Alle weiteren Dateien unterscheiden sich abhängig von der Art des Repositories. In den folgenden Abschnitten werden daher Aufbau und Inhalt der einzelnen Repositories genauer beschrieben. Dabei wird auch auf Ausschnitte aus dem Quellcode des Projekts eingegangen. Der vollständige Quellcode der relevantesten Dateien befindet sich in \autoref{ch:BB_source-code-of-prototype} dieser Arbeit.

\subsection{Dotfiles Repository}
\label{subsec:06-02-01_dotfiles-repository}

Die Struktur der Verzeichnisse sowie der Quellcode ausgewälter Dateien des \nameref{sec:BB-03_dotfiles-repository} sind in \autoref{sec:BB-03_dotfiles-repository} von \autoref{ch:BB_source-code-of-prototype} beigefügt und werden hier teilweise aufgegriffen. Grundlagen zum Dotfiles Repositorys sind in \autoref{subsubsec:05-02-02-01_data-storage-in-repositories} beschrieben.

Das wichtigste Element eines Dotfile Repositorys sind die Konfigurationsdateien, also die Dotfiles. Sie liegen unter \texttt{configs/} ab und sind so angeordnet, dass sie ausgehend von dort das \texttt{\$HOME} Verzeichnis eines Nutzers abbilden. Im minimalen Repository aus dem Prototypen sind zwei solcher Dateien enthalten, wovon sich eine in einem Unterordner befindet. Die Datei \texttt{.zshrc} richtet die in der Umgebung verfügbare \textit{ZSH} Shell ein, während \texttt{.config/starship.toml} die Konfiguration einer individuellen Schnittstelle zu dieser Shell definiert.

Ein ebenfalls vorhandenes Skript \nameref{subsec:BB-03-01_install-sh} dient der automatischen Installation dieser Konfigurationsdateien. Es lädt dazu das Dotfiles Repository als Archiv herunter, entpackt es lokal und kopiert alle enthaltenen Dateien in ihre gespiegelten Pfade im \texttt{\$HOME} Verzeichnis. Das Skript kann direkt von \textit{GitHub} aus ausgeführt werden, indem folgender Befehl verwendet wird:

\begin{codebox}{Bash}{Befehl zum Ausführen des Installationsskripts der Dotfiles}
    bash -c $(curl -#fL https://raw.githubusercontent.com/Toolchain-as-Code/ tac-dotfiles/refs/heads/main/install.sh)
\end{codebox}

\subsection{Environment Repository}
\label{subsec:06-02-02_environment-repository}

Die Struktur der Verzeichnisse sowie der Quellcode ausgewälter Dateien des \nameref{sec:BB-02_environment-repository} sind in \autoref{sec:BB-02_environment-repository} von \autoref{ch:BB_source-code-of-prototype} beigefügt und werden hier teilweise aufgegriffen. Grundlagen zum Environment Repositorys sind in \autoref{subsubsec:05-02-02-01_data-storage-in-repositories} beschrieben.

Im Application Repository sind unter \texttt{.envcontainer/} insgesamt drei Dockerfiles enthalten: \nameref{subsec:BB-02-02_envcontainer-X-app-dockerfile} für das spätere Integration Environment mit minimalen Abhängigkeiten, \nameref{subsec:BB-02-03_envcontainer-X-dev-dockerfile} für das erweiterte Creation Environment und \nameref{subsec:BB-02-01_envcontainer-X-base-dockerfile} als gemeinsame Basis für beide Umgebungen.

\begin{codebox}{Dockerfile}{\texttt{FROM}-Instruktion und Argumente der Dockerfiles}
ARG BASE_REPOSITORY=docker.io/library
ARG BASE_IMAGE=tac-environment
ARG BASE_ENVIRONMENT=base
ARG BASE_RELEASE=latest

FROM ${BASE_REPOSITORY}/${BASE_IMAGE}:${BASE_ENVIRONMENT}-${BASE_RELEASE} AS base
\end{codebox}

Die \texttt{FROM}-Instruktionen zur Angabe des Base \Glspl{image} sind in allen Dockerfiles identisch. Sie verwenden sogenannte \q{\Gls{build} Args}, welche den Ort und die Version des Base \Glspl{image} angeben. Diese Werte sind bereits so vorbelegt, dass ein lokaler \Gls{build} des Docker \Glspl{image} möglich ist, andere Umgebungen sie jedoch überschreiben können.

Das \nameref{subsec:BB-02-01_envcontainer-X-base-dockerfile} selbst basiert wiederum auf einem speziell auf die Softwareentwicklung mit \textit{Deno} zugeschnittenen \Gls{image}: \texttt{denoland/deno:debian-2.1.0}. Dieses enthält die \textit{Deno} Laufzeitumgebung in der Version 2.1.0 und basiert auf der \textit{Debian} \textit{Linux} Distribution. Im Rahmen des Prototypen stellt das \nameref{subsec:BB-02-02_envcontainer-X-app-dockerfile} keine reale Erweiterung dar, wurde zur semantischen Vollständigkeit und besseren Anpassbarkeit jedoch hinzugefügt. Das \nameref{subsec:BB-02-03_envcontainer-X-dev-dockerfile} erweitert das Base \Gls{image} um verschiedene Werkzeuge, die für die Entwicklung nützlich sind. Neben den Paketen \textit{curl}, \textit{git}, \textit{postgresql-client}, \textit{unzip} und \textit{zip} installiert es außerdem die \textit{ZSH} Shell inklusive zusätzlicher Erweiterungen. Deren Konfiguration wird durch die Installation der Dotfiles abgeschlossen.

\begin{codebox}{Dockerfile}{Auszug aus .envcontainer/Dev.Dockerfile}
RUN apt update && apt update && apt install -y \
    curl git postgresql-client unzip zip \
    && apt clean

RUN bash -c "$(curl -#fL https://raw.githubusercontent.com/ Toolchain-as-Code/tac-dotfiles/refs/heads/main/install.sh)"

RUN apt update && \
    apt install -y zsh fzf && \
    chsh -s /usr/bin/zsh && \
    apt clean
RUN curl -sS https://starship.rs/install.sh | sh -s -- -y -v latest
RUN curl -sS https://raw.githubusercontent.com/ zdharma-continuum/zinit/HEAD/scripts/install.sh | sh
RUN zsh -il
\end{codebox}

Damit für alle Umgebungen \Gls{cicd} Pipelines automatisch die entsprechenden \Glspl{image} bauen und in eine \Gls{container-registry} hochladen können, enthält das Verzeichnis \texttt{.github/workflows/} jeweils eine Datei pro Dockerfile, welche einen \textit{\Gls{github-actions}} Workflow konfiguriert (siehe \autoref{subsec:BB-02-04_github-workflows-X-base-environment-build-and-push-yaml}, \autoref{subsec:BB-02-05_github-workflows-X-integration-environment-build-and-push-yaml} und \autoref{subsec:BB-02-06_github-workflows-X-creation-environment-build-and-push-yaml}). Wird durch das \textit{\Gls{git}} Repository beispielsweise ein \texttt{push} Ereignis ausgelöst, startet dies automatisch den Workflow, der das Base \Gls{image} baut und in ein Repository auf \textit{Docker Hub} hochlädt. Der Abschluss dieses Workflows löst wiederum die Pipelines aus, die für den \Gls{build} und das \Gls{deployment} von Integration und Creation Environment zuständig sind.

\begin{codebox}{yaml}{Auslöser der Pipelines für die Umgebungen}
on:
    workflow_run:
        workflows: [ "Base Environment" ]
        types:
            - completed
\end{codebox}

\subsection{Application Repository}
\label{subsec:06-02-03_application-repository}

Die Struktur der Verzeichnisse sowie der Quellcode ausgewälter Dateien des \nameref{sec:BB-01_application-repository} sind in \autoref{sec:BB-01_application-repository} von \autoref{ch:BB_source-code-of-prototype} beigefügt und werden hier teilweise aufgegriffen. Grundlagen zum Application Repositorys sind in \autoref{subsubsec:05-02-02-01_data-storage-in-repositories} beschrieben.

Das Application Repository ist aufgeteilt in \texttt{.appcontainer/}, \texttt{.devcontainer/} und \texttt{app/}, wobei \texttt{app/} den Quellcode der \textit{Deno} Anwendung in \textit{TypeScript} enthält. Die beiden anderen Verzeichnisse bestehen jeweils auf einem Dockerfile, das die Anwendung in einer spezifischen Umgebung ausführt. Das \nameref{subsec:BB-01-03_appcontainer-X-dockerfile} basiert auf dem Integration Environment und wird nur durch die \Gls{cicd} Pipeline \nameref{subsec:BB-01-06_github-X-workflows-X-application-build-and-push-yaml} gebaut. Das \nameref{subsec:BB-01-05_devcontainer-X-dockerfile} soll für die Entwicklung verwendet werden können und basiert daher auf dem Creation Environment. Die Dockerfiles ziehen die \Glspl{image} aus der \textit{GitHub} \Gls{container-registry} oder nutzen lokale \Glspl{build}.

Damit die Anwendung lokal ausführbar ist, steht eine \nameref{subsec:BB-01-02_docker-compose-yaml} Datei zur Verfügung, welche neben dem bereits bestehenden Dockerfile auch eine \textit{PostgreSQL} Datenbank definiert, die mit Daten aus \texttt{.devcontainer/init\_db.sql} initialisiert wird.

\begin{codebox}{yaml}{Auszug aus docker-compose.yaml}
services:
    db:
        image: postgres:17
        restart: unless-stopped
    app:
        build:
            context: .
            dockerfile: .devcontainer/Dockerfile
\end{codebox}

Der Zugang zur Datenbank muss über Umgebungsvariablen für \texttt{PGHOST}, \texttt{PGPORT}, \texttt{PGDATABASE}, \texttt{PGUSER} und \texttt{PGPASSWORD} eingerichtet werden. Dafür steht eine \texttt{example.env} Datei bereit, die als Vorlage für eine nicht im \Gls{vcs} gespeicherte \texttt{.env} Datei dient.

Eine zentrale Komponente des Application Repositorys ist der \nameref{subsec:05-01-02_dev-container}, welcher über die \nameref{subsec:BB-01-04_devcontainer-X-devcontainer-json} Datei konfiguriert wird. Der \nameref{subsec:05-01-02_dev-container} basiert auf der bereits bestehenden Infrastruktur, indem er den \textit{Docker Compose Stack} in dem Editor \textit{Visual Studio Code} startet, welcher die angegebenen Erweiterungen installiert.

\begin{codebox}{json}{Auszug aus devcontainer.json}
{
    "dockerComposeFile": "../docker-compose.yaml",
    "service": "app",
    "remoteUser": "root"
}
\end{codebox}

\section{Praxiseinsatz des Toolchain-as-Code Prototypen}
\label{sec:06-03_practical-use-of-the-toolchain-as-code-prototype}

\subsection{Toolchain-as-Code im Development}
\label{subsec:06-03-01_toolchain-as-code-in-development}

% place content here
Inhalt

\subsection{Toolchain-as-Code im Deployment}
\label{subsec:06-03-02_toolchain-as-code-in-deployment}

% place content here
Inhalt

\section{Kurzbewertung der Strategie anhand des Prototypen}
\label{sec:06-03_brief-evaluation-of-strategy-based-on-prototype}

% place content here
Inhalt


	\chapter{Fazit und Ausblick}
\label{ch:07_conclusion-and-outlook}

\section{Zusammenfassung der Forschungsergebnisse}
\label{sec:07-01_summary-of-research-results}

% place content here
Inhalt

\section{Weiterer Forschungsbedarf}
\label{sec:07-02_further-research-needs}

Trotz der Beantwortung aller \hyperref[sec:01-03_objectives-and-research-questions]{Forschungsfragen} und der Erfüllung vieler \hyperref[sec:04-03_general-aspects-and-summery-of-findings]{Anforderungen}, blieben einige Aspekte unberücksichtigt oder wurden nicht erschöpfend behandelt, sodass das Ergebnis dieser Arbeit als Ausgangspunkt für weitere Forschungen genutzt werden kann.

Die entwickelte Toolchain-as-Code Strategie konzentriert sich hauptsächlich auf Neuentwicklungen mit Technologien aus den Bereichen \nameref{sec:02-01_web-development}, \nameref{sec:02-02_microservices} und \nameref{sec:02-03_containerization}. \Glspl{legacy-system} stellen Entwicklungsteams oft vor viel größere Herausforderungen, da ihre Toolchains selten modernen Anforderungen und Standards entsprechen. Dateien und Verzeichnisstrukturen folgen überholten Konventionen und nicht selten haben sich über die Zeit überraschend langlebige Provisorien durchgesetzt. Wie die Architektur gemäß dem Konzept von Toolchain-as-Code in solchen Projekten eingeführt werden kann und ob dies überhaupt möglich beziehungsweise sinnvoll ist, wurde in dieser Arbeit nicht diskutiert.

Wird Software in größeren Organisationen entwickelt, bestehen oft zusätzliche Herausforderungen wie \Glspl{vpn}, Proxies, Mirrors oder Certificates. Insbesondere bei der Konfiguration von \Gls{cicd} und \nameref{sec:02-03_containerization} Tools kann dies zu deutlich aufwändigeren Einrichtungsprozessen führen. Eine konkrete Lösung hierfür liefert diese Arbeit nicht.

Ebenfalls nur untergeordnet thematisiert wurde die \Gls{delivery} von Software, also ihre Bereitstellung für den Kunden. Im Verbund mit \nameref{sec:02-03_containerization} ist hier oft die Konfiguration komplexer Bereitstellungs- und Orchestrierungswerkzeuge notwendig. Es kann sich schnell um äußerst anspruchsvolle und kostenintensive Aufgaben handeln, eine solche Infrastruktur zu betreiben, weshalb die Toolchain-as-Code Strategie sich auf das \Gls{deployment} von Software beschränkt.

Eine vergleichbare Komplexität liegt in den Themen Sicherheit, Wartbarkeit und Performanz von Software. Jedes einzelne von ihnen würde einer eigenen Arbeit oder sogar mehreren bedürfen, um es erschöpfend behandeln zu können. Zwar wurden im Kontext von \nameref{sec:02-03_containerization} und der vierten \acrlong{rq} einige Best Practices vorgestellt, die eine grundlegende Orientierung bieten können, jedoch handelt es sich dabei nur um die Spitze des Eisbergs.

Auch in der Literatur kann Inspiration für weitere Forschungsbedarfe im Kontext dieser Arbeit gefunden werden. So stellen \citeauthor{000:CI-CD-Deployment-in-DevOps-reduce-Gap-Developer-Operation} fest, dass \hyperref[sec:03-01_introduction-to-devops]{DevOps} Geschwindigkeit über Sicherheit stellt \cite{010:Efficient-Application-Deployment-GitOps-for-Faster-and-Secure-CI-CD-Cycles}. Im Sicherheitsbereich könnten daher, wie bereits angesprochen, konkretere Lösungen einen wichtigen Beitrag leisten. \Gls{mlops} zur Implementierung und Administration von Modellen für \Gls{ki} ist ein aufkommendes Thema mit vielen Chancen \cite{010:Efficient-Application-Deployment-GitOps-for-Faster-and-Secure-CI-CD-Cycles}. Vorlagen für komplette Pipelines und Projekte sowie deren Beitrag zur Reduzierung des Aufwands bei der Nutzung von \hyperref[sec:03-01_introduction-to-devops]{DevOps} mit \nameref{sec:02-02_microservices} sind ein Thema in dem Paper \citetitle{019:Advanced-DevOps-Environment-for-Microservices-based-Applications} \cite{019:Advanced-DevOps-Environment-for-Microservices-based-Applications}. Bereits angesprochen wurde die Abgrenzung der Toolchain-as-Code Strategie gegenüber Multi-Repository Projekten, ähnlich wie in \hyperref[sec:03-03_gitops-as-further-evolution]{GitOps} \cite{109:GitOps}. Die falsche Verwendung von \hyperref[sec:03-04_idea-of-dotfiles]{Dotfiles} kann sicherheitskritisch sein und wurde unter diesem Aspekt von \citeauthor{029:Connecting-the-Dotfiles} genauer untersucht \cite{029:Connecting-the-Dotfiles}. Spannend insbesondere für große Organisationen wäre außerdem, wie Entwicklungsumgebungen als \nameref{subsec:05-01-02_dev-container} on-premise bereitgestellt werden können \cite{014:Managing-Container-based-Software-Development-Environments}.



	%---------------------------------------------------------------------------
	% BIBLIOGRAPHY
	%---------------------------------------------------------------------------

	\clearpage
	\renewcommand{\bibname}{Literaturverzeichnis}		% set bibliography title, default: Literatur
	\addcontentsline{toc}{chapter}{Literaturverzeichnis}
	\printbibliography

	\listoffigures
	\listoftables
	
	\clearpage

	%---------------------------------------------------------------------------
	% APPENDIX
	%---------------------------------------------------------------------------

	\appendix
	\pagenumbering{Roman}
	\chapter{Experteninterviews}
\label{ch:AA_expert-interviews}

\section{Interviewfragen}
\label{sec:AA-01_interview-questions}

Die eindeutige Identifizierung der Interviewfragen ist über die folgende Kombination möglich:

\begin{quote}
    \begin{verbatim}
        IQ-<Fragenbereich><Fragennummer> (<Fokusbereich>)
    \end{verbatim}
\end{quote}

Der \texttt{Fragenbereich} ist ein Buchstabe von \texttt{A} bis \texttt{C} und gibt an, ob es sich um eine (\texttt{A}) \textbf{offene} Frage, (\texttt{B}) \textbf{halb-offene} Frage oder (\texttt{C}) \textbf{geschlossene} Frage handelt. Die \texttt{Fragennummer} ist eine fortlaufende Nummer beginnend bei \texttt{0}, die die Position der Frage innerhalb ihres Bereichs angibt. Der \texttt{Fokusbereich} ist als Zahl von \texttt{Eins} bis \texttt{Fünf} definiert und ordnet die Frage einem von fünf Fragezielen zu, darunter (\texttt{Eins}) die \textbf{Zusammensetzung} von, (\texttt{Zwei}) die \textbf{Vorgehensweisen} und \textbf{Interaktionswege} mit, (\texttt{Drei}) \textbf{Herausforderungen} im Umgang mit, (\texttt{Vier}) Ist-Standards für \textbf{Automatisierungsgrad} und \textbf{Developer-Experience} in sowie (\texttt{Fünf}) die Prioritäten bei \textbf{Anforderungen} an und \textbf{Zielen} für Development- und Deployment-Umgebungen.

\clearpage

\subsection{Offene Fragen}
\label{subsec:AA-01-01_open-questions}

\begin{table}[H]
    \centering
    \begin{tabular}{ >{\raggedright\bfseries}p{0.2\textwidth} p{0.7\textwidth} }
        IQ-A0 (Eins) & 
        Wenn Sie sich eine ideale Development- beziehungsweise Deploymentumgebung vorstellen, was wären die wichtigsten Merkmale, die sie auszeichnen würde? \\
        \hline
        IQ-A1 (Drei) &
        Welche Herausforderungen oder Hindernisse treten typischerweise bei der Einrichtung und Nutzung Ihrer Entwicklungs- oder Deploymentumgebungen auf? \\
        \hline
        IQ-A2 (Zwei) &
        Bezogen auf die einzelnen Schritte, wie gehen Sie üblicherweise bei der Konfiguration der einzelnen Werkzeuge Ihrer Entwicklungs- beziehungsweise Deploymentumgebungen vor? \\
        \hline
        IQ-A3 (Eins) &
        Was sind typische Komponenten, Werkzeuge oder andere Elemente Ihrer Entwicklungs- oder Deploymentumgebungen? \\
        \hline
        IQ-A4 (Fünf) &
        Gibt es spezifische Anforderungen oder Standards, die Sie bei der Nutzung Ihrer Entwicklungs- und Deploymentumgebungen erfüllen müssen? \\
    \end{tabular}
\end{table}

\clearpage

\subsection{Halb-offene Fragen}
\label{subsec:AA-01-02_half-open-questions}

\begin{table}[H]
    \centering
    \begin{tabular}{ >{\raggedright\bfseries}p{0.2\textwidth} p{0.7\textwidth} }
        IQ-B0 (Fünf) &
        Welcher der Schritte, die Sie bei der Konfiguration Ihrer Umgebung (IQ-A2) durchlaufen, ist Ihrer Meinung nach der wichtigste für den Erfolg eines Projekts? \newline
        [Schritte aus IQ-A2] \\
        \hline
        IQ-B1 (Vier) &
        Wie viele Zeitressourcen benötigen Sie typischerweise für das Onboarding neuer Entwickler im Hinblick auf Ihre Toolchains? \newline
        [Ressourcen in h] \\
        \hline
        IQ-B2 (Fünf) &
        Welcher der Schritte (IQ-A2), die Sie genannt haben, würde durch eine Optimierung oder Automatisierung die meisten Zeitressourcen einsparen? \newline
        [Schritte aus IQ-A2] \\
        \hline
        IQ-B3 (Vier) &
        Gibt es manuelle Schritte in Ihrer Development- oder Deploymentumgebung, deren Automatisierung nicht möglich oder nicht sinnvoll ist beziehungsweise die eine Automatisierung ab einem bestimmten Punkt blockieren? Falls ja, welche? \newline
        [ja (mit Ergänzung) oder nein] \\
        \hline
        IQ-B4 (Fünf) &
        Welche drei Ziele würden Sie verfolgen, wenn Sie eine Toolchain für ein neues Projekt im eingangs beschriebenen Projektkontext entwerfen müssten? \newline
        [Liste aus Zielen] \\
    \end{tabular}
\end{table}

\clearpage

\subsection{Geschlossene Fragen}
\label{subsec:AA-01-03_closed-questions}

\begin{table}[H]
    \centering
    \begin{tabular}{ >{\raggedright\bfseries}p{0.2\textwidth} p{0.7\textwidth} }
        IQ-C0 (Eins) &
        Wie groß ist Ihrer Meinung nach der Anteil an manuellen Schritten (IQ-B3), deren Automatisierung nicht möglich oder nicht sinnvoll ist? \newline
        [Wert zwischen 0 \% und 100 \%] \\
        \hline
        IQ-C1 (Eins) &
        Was denken Sie, wie gut könnten einzelne Komponenten Ihrer Toolchains (wie beispielsweise ein Paketmanager oder ein Pipeline-Tool) ausgetauscht werden, also wie flexibel sind die Toolchains? \newline
        [Skala von 1 (sehr wenig austauschbar) bis 5 (sehr flexibel)] \\
        \hline
        IQ-C2 (Drei) &
        Wie häufig müssen Sie Anpassungen an Ihrer Development- oder Deploymentumgebung vornehmen, um neuen Projektanforderungen gerecht zu werden? \newline
        [Skala von 1 (sehr selten) bis 5 (sehr häufig)] \\
        \hline
        IQ-C3 (Drei) &
        Wie regelmäßig aktualisieren Sie die verwendeten Tools und Bibliotheken in Ihrer Umgebung? \newline
        [Skala von 1 (selten oder nie) bis 5 (sehr regelmäßig)] \\
        \hline
        IQ-C4 (Vier) &
        An welcher Stelle wird Ihre Software in der Regel nach Anpassungen das erste Mal ausgeführt? \newline
        [Auswahl zwischen \q{lokal}, \q{Pipeline} und \q{X-Stage}] \\
    \end{tabular}
\end{table}

\clearpage

\subsection{Bewertungsmöglichkeit für Anforderungen}
\label{subsec:AA-01-04_evaluation-requirements}

\textbf{(Fünf)}

Einzelne Faktoren des Konzepts der \q{Twelve-Factor-App} erlauben Rückschlüsse auf weitere im Rahmen dieser Arbeit vorgestellte Konzepte. Es lassen sich Anforderungen an Development- und Deploymentumgebungen von ihnen ableiten. Die Fragestellung hinter diesem Fragenbereich liegt darin, die abgeleiteten Anforderungen zu bewerten. 

Dafür stehen insgesamt 6 Punkte zur Verfügung, die frei auf die einzelnen Faktoren verteilt werden können (also beispielsweise 6x1, 1x3 + 1x2 + 1x1, 2x3, 1x6, ...).

\newcounter{factornoappendix}
\setcounter{factornoappendix}{-1}
\newcommand{\factornumberappendix}{\stepcounter{factornoappendix}\Roman{factornoappendix}}
\begin{longtable}{  |   >{\raggedleft\factornumberappendix}p{0.025\textwidth}   % Number (centered)
                        >{\raggedright\bfseries}p{0.175\textwidth}              % Factor (left-aligned)
                    |   >{\raggedright}p{0.600\textwidth}                       % Requirements and linking Concepts (left-aligned)
                    |   >{}p{0.100\textwidth}                                   % Points (centered)
                    | }
    \hline
        & \upshape\textbf{Faktor} 
        & \upshape\textbf{Anforderungen und verknüpfte Konzepte}
        & \upshape\textbf{Punkte} \\
    \hline \hline
    \endhead
    \hline
    %   (I) Codebase
        & Codebase
        & \textit{GitOps} \textrightarrow Repository als \q{Single Source of Truth}
        & ~ \\
    \hline
    %   (II) Dependencies
        & Dependencies
        & \textit{GitOps} \textrightarrow Dependencies deklarativ in Konfigurationsdateien
        & ~ \\
    \hline
    %   (III) Config
        & Config
        & \textit{Dotfiles} \textrightarrow Konfiguration in \texttt{.env} files außerhalb des Repositories
        & ~ \\
    \hline
    %   (IV) Backing Services
        & Backing Services
        & \textit{Dotfiles} \textrightarrow Zugang zu Diensten über Config
        & ~ \\
    \hline
    %   (V) Build, Release, Run
        & Build, Release, Run
        & \textit{DevOps} \textrightarrow CI/CD Pipelines, \newline
          \textit{Dotfiles} \textrightarrow Config in Release Stage
        & ~ \\
    \hline
    %   (VI) Processes
        & Processes
        & -/-
        & ~ \\
    \hline
    %   (VII) Port Binding
        & Port Binding
        & \textit{GitOps} \textrightarrow Dependency Declaration für eingebundene Webserver, \newline
          \textit{Dotfiles} \textrightarrow Konfiguration von Ports
        & ~ \\
    \hline
    %   (VIII) Concurrency
        & Concurrency
        & -/-
        & ~ \\
    \hline
    %   (IX) Disposability
        & Disposability
        & \textit{Best Containerization Practices} \textrightarrow kompakte Gestaltung von Docker Images
        & ~ \\
    \hline
    %   (X) Dev / Prod Parity
        & Dev / Prod Parity
        & Toolchain-as-Code Strategie
        & ~ \\
    \hline
    %   (XI) Logs
        & Logs
        & \textit{Best Containerization Practices} \textrightarrow Prozesse loggen nach \texttt{stdout}
        & ~ \\
    \hline
    %   (XII) Admin Processes
        & Admin Processes
        & -/-
        & ~ \\
    \hline
\end{longtable}
\vspace{1em}
\setcounter{factornoappendix}{0}
  \clearpage
\section{Interviewpersonen}
\label{sec:AA-02_interview-persons}

Die eindutige Identifizierung der Interviewpersonen ist über die folgende Kombination möglich:

\begin{quote}
    \begin{verbatim}
        IP-<Personennummer>
    \end{verbatim}
\end{quote}

Die \texttt{Personennummer} ist eine fortlaufende Nummer beginnend bei \texttt{0}, die die Position der Person entsprechend der Reihenfolge der Interviews angibt.

\begin{longtable}{  |   >{\bfseries}p{0.100\textwidth}                  % Identifiert
                        >{\raggedright\bfseries}p{0.225\textwidth}      % Name
                    |   >{\raggedright}p{0.325\textwidth}               % Department
                    |    p{0.250\textwidth}                             % Field(s) of Activity
                    | }
    \hline
        & \upshape\textbf{Name} 
        & \upshape\textbf{Abteilung} 
        & \upshape\textbf{Tätigkeitsfeld(er)} \\
    \hline \hline
    \endhead
    \hline
        IP-0
        & Interviewperson 0
        & Software Development Center
        & 
        \begin{itemize}
            \item Development
            \item Deployment
            \item Operations
        \end{itemize} \\
    \hline
        IP-1
        & Interviewperson 1
        & Application Management
        & 
        \begin{itemize}
            \item Development (eher)
            \item Deployment
            \item Operations
            \item Support
        \end{itemize} \\
    \hline
        IP-2
        & Interviewperson 2
        & Application Management
        & 
        \begin{itemize}
            \item Development
            \item Deployment
        \end{itemize} \\
    \hline
        IP-3
        & Interviewperson 3
        & Software Development Center
        & 
        \begin{itemize}
            \item Development
        \end{itemize} \\
    \hline
\end{longtable}
    \clearpage
\section{Interviewergebnisse}
\label{sec:AA-03_interview-results}

Die Dokumentation der Interviewergebnisse nutzt die gleichen Identifikationen für die \nameref{sec:AA-01_interview-questions} und die \nameref{sec:AA-02_interview-persons} wie bereits in den vorherigen Abschnitten definiert. Die Reihenfolge der Interviewfragen ist unverändert, die Antworten der Interviewpersonen sind in den jeweiligen Abschnitten zusammengefasst. Dabei ist jede Antwort protokolliert, die von einer Interviewperson zu einer Interviewfrage gegeben wurde. Wurde eine Antwort in gleicher oder ähnlicher Form von mehreren Interviewpersonen gegeben, so wurde sie zusammengefasst und mit den entsprechenden Identifikationen versehen. Die Zuordnung einer Antwort erfolgt dabei über den folgenden Zusatz:

\begin{quote}
    \begin{verbatim}
        {<Anzahl>: <Interviewperson-x>, <Interviewperson-y>, ...}
    \end{verbatim}
\end{quote}

Dabei gibt die \texttt{Anzahl} an, wie viele Interviewpersonen die Antwort in gleicher oder ähnlicher Form gegeben haben. Hinter dem Doppelpunkt folgen anschließend die Identifikationen der Interviewpersonen, die die Antwort gegeben haben.

\clearpage

\subsection{Offene Fragen}
\label{subsec:AA-03-01_open-questions}

\begin{longtable}{ >{\raggedright\bfseries}p{0.2\textwidth} p{0.7\textwidth} }
    IQ-A0 (Eins) & 
    Wenn Sie sich eine ideale Development- beziehungsweise Deploymentumgebung vorstellen, was wären die wichtigsten Merkmale, die sie auszeichnen würde? \\
    \nopagebreak
    \multicolumn{2}{ >{\raggedright}p{0.9\textwidth} }{
        \begin{itemize}
            \item Dependencies sollten einfach einsehbar und verwaltbar sein \mbox{\textbf{\{1: IP-0\}}}
            \item Konfigurationsaufwand sollte gering sein \mbox{\textbf{\{1: IP-1\}}}
            \item Konfiguration erfolgt über versionierte Dateien \mbox{\textbf{\{2: IP-2, IP-3\}}}
            \item Entwicklungsumgebung sollte möglichst \linebreak[1] vollständig geladen werden \mbox{\textbf{\{2: IP-1, IP-3\}}}
            \item Shell-Skripte erledigen bestimmte Aufgaben \mbox{\textbf{\{1: IP-3\}}}
            \item Tools sind einheitlich vorgegeben \mbox{\textbf{\{1: IP-2\}}}
            \item lokale Umgebung ist möglichst nah an der Produktivumgebung \mbox{\textbf{\{1: IP-3\}}}
            \item Deployment sollte bestenfalls per Container oder serverless erfolgen \mbox{\textbf{\{1: IP-1\}}}
            \item Deployment ist maximal automatisiert \mbox{\textbf{\{1: IP-2\}}}
            \item Fast Feedback und Alerting ermöglichen \linebreak[1] schnelle Reaktion der Entwickler \mbox{\textbf{\{1: IP-3\}}}
            \item Umgebung sollte durch das Entwicklungsteam anpassbar sein \mbox{\textbf{\{1: IP-3\}}}
        \end{itemize}
    } \\
    \hline
    IQ-A1 (Drei) &
    Welche Herausforderungen oder Hindernisse treten typischerweise bei der Einrichtung und Nutzung Ihrer Entwicklungs- oder Deploymentumgebungen auf? \\
    \nopagebreak
    \multicolumn{2}{ >{\raggedright}p{0.9\textwidth} }{
        \begin{itemize}
            \item fehlende Reproduzierbarkeit \mbox{\textbf{\{1: IP-2\}}}
            \item teilweise hoher Konfigurationsbedarf \mbox{\textbf{\{1: IP-2\}}}
            \item unterschiedliche Versionen von SDKs in verschiedenen Projekten \mbox{\textbf{\{1: IP-0\}}}
            \item Einrichtung von CA-Zertifikaten \mbox{\textbf{\{1: IP-0\}}}
            \item fehlende Administratorrechte \mbox{\textbf{\{1: IP-0\}}}
            \item veraltete Dokumentation \mbox{\textbf{\{1: IP-2\}}}
            \item veraltete Skripte \mbox{\textbf{\{1: IP-3\}}}
        \end{itemize}
    } \\
    \hline
    IQ-A2 (Zwei) &
    Bezogen auf die einzelnen Schritte, wie gehen Sie üblicherweise bei der Konfiguration der einzelnen Werkzeuge Ihrer Entwicklungs- beziehungsweise Deploymentumgebungen vor? \\
    \nopagebreak
    \multicolumn{2}{ >{\raggedright}p{0.9\textwidth} }{
        \begin{itemize}
            \item Lesen und Befolgen der Dokumentation \mbox{\textbf{\{2: IP-2, IP-3\}}}
            \item Installation lokaler Tools wie Git und Docker \mbox{\textbf{\{3: IP-0, IP-1, IP-3\}}}
            \item Setzen der Umgebungsvariablen \mbox{\textbf{\{1: IP-1\}}}
            \item Konfiguration der integrierten Entwicklungsumgebung / des Editors \linebreak[1] über ein entsprechendes Konfigurationsverzeichnis im VCS \mbox{\textbf{\{1: IP-1\}}}
            \item Installation der Abhängigkeiten über Paketmanager \mbox{\textbf{\{1: IP-1\}}}
            \item Einrichtung und Start des Projekts lokal \mbox{\textbf{\{2: IP-2, IP-3\}}}
            \item Einrichtung und Start der Tests lokal \mbox{\textbf{\{1: IP-3\}}}
            \item Evolution eines eigenen Artefakts auf die nächste Stage \mbox{\textbf{\{1: IP-2\}}}
        \end{itemize}
    } \\
    \hline
    IQ-A3 (Eins) &
    Was sind typische Komponenten, Werkzeuge oder andere Elemente Ihrer Entwicklungs- oder Deploymentumgebungen? \\
    \nopagebreak
    \multicolumn{2}{ >{\raggedright}p{0.9\textwidth} }{
        \begin{itemize}
            \item Task-Files im Repository \mbox{\textbf{\{1: IP-0\}}}
            \item Bash-Skripte im Repository \mbox{\textbf{\{2: IP-0, IP-1\}}}
            \item \textit{Unix}-Tools \mbox{\textbf{\{1: IP-1\}}}
            \item VCS (\textit{Git}) \mbox{\textbf{\{1: IP-2\}}}
            \item Build Tools (\textit{Maven}) \mbox{\textbf{\{2: IP-2, IP-3\}}}
            \item Plugins für Editor oder integrierte Entwicklungsumgebung \mbox{\textbf{\{2: IP-1, IP-2\}}}
            \begin{itemize}
                \item \textit{SonarLint} \mbox{\textbf{\{1: IP-2\}}}
            \end{itemize}
            \item Pipeline Tools (\textit{GitHub Actions}) \mbox{\textbf{\{1: IP-3\}}}
            \item Deployment Tools (\textit{Argo CD}) \mbox{\textbf{\{1: IP-2\}}}
            \item Container Orchestration Tools (\textit{Kubernetes}) \mbox{\textbf{\{1: IP-2\}}}
            \item Infrastructure-as-Code (IaC) für Umgebungen und Pipelines \mbox{\textbf{\{1: IP-3\}}}
        \end{itemize}
    } \\
    \hline
    IQ-A4 (Fünf) &
    Gibt es spezifische Anforderungen oder Standards, die Sie bei der Nutzung Ihrer Entwicklungs- und Deploymentumgebungen erfüllen müssen? \\
    \nopagebreak
    \multicolumn{2}{ >{\raggedright}p{0.9\textwidth} }{
        \begin{itemize}
            \item Tools sollten plattformübergreifend funktionieren \mbox{\textbf{\{1: IP-0\}}}
            \item Standards und Konventionen sollten eingehalten werden \mbox{\textbf{\{2: IP-2, IP-3\}}}
            \begin{itemize}
                \item \textit{Git Workflow} \mbox{\textbf{\{1: IP-2\}}}
                \item \textit{Google Java Styleguide} \mbox{\textbf{\{1: IP-2\}}}
            \end{itemize}
        \end{itemize}
    } \\
\end{longtable}

\clearpage

\subsection{Halb-offene Fragen}
\label{subsec:AA-03-02_half-open-questions}

\begin{longtable}{ >{\raggedright\bfseries}p{0.2\textwidth} p{0.7\textwidth} }
    IQ-B0 (Fünf) &
    Welcher der Schritte, die Sie bei der Konfiguration Ihrer Umgebung (IQ-A2) durchlaufen, ist Ihrer Meinung nach der wichtigste für den Erfolg eines Projekts? \newline
    [Schritte aus IQ-A2] \\
    \nopagebreak
    \multicolumn{2}{ >{\raggedright}p{0.9\textwidth} }{
        \begin{itemize}
            \item Vorhandensein einer eindeutigen Config \mbox{\textbf{\{1: IP-0\}}}
            \item Existenz eins einzigen Commands zum Starten der Infrastruktur \mbox{\textbf{\{1: IP-0\}}}
            \item Aufsetzen der Entwicklungsumgebung \mbox{\textbf{\{1: IP-1\}}}
            \item Evolution eines eigenen Artefakts auf die nächste Stage \mbox{\textbf{\{1: IP-2\}}}
            \item Einrichtung eines einfachen und verständlichen Setups \mbox{\textbf{\{1: IP-3\}}}
            \item Vermeidung unnötiger Komplexität von Beginn an \mbox{\textbf{\{1: IP-3\}}}
        \end{itemize}
    } \\
    \hline
    IQ-B1 (Vier) &
    Wie viele Zeitressourcen benötigen Sie typischerweise für das Onboarding neuer Entwickler im Hinblick auf Ihre Toolchains? \newline
    [Ressourcen in h] \\
    \nopagebreak
    \multicolumn{2}{ >{\raggedright}p{0.9\textwidth} }{
        \begin{itemize}
            \item $ < 01 h $ in neuem Projekt \mbox{\textbf{\{1: IP-0\}}}
            \item $ \sim 08 h $ (1 Tag) in neuem Projekt \mbox{\textbf{\{2: IP-1, IP-3\}}}
            \item $ \sim 16 h - 24 h $ (2 - 3 Tage) in neuem Projekt \mbox{\textbf{\{1: IP-2\}}}
        \end{itemize}
    } \\
    \hline
    IQ-B2 (Fünf) &
    Welcher der Schritte (IQ-A2), die Sie genannt haben, würde durch eine Optimierung oder Automatisierung die meisten Zeitressourcen einsparen? \newline
    [Schritte aus IQ-A2] \\
    \nopagebreak
    \multicolumn{2}{ >{\raggedright}p{0.9\textwidth} }{
        \begin{itemize}
            \item Installation der Entwicklungsumgebung \mbox{\textbf{\{1: IP-3\}}}
            \item Installation von SDKs \mbox{\textbf{\{1: IP-1\}}}
            \item Installation von Dependencies \mbox{\textbf{\{2: IP-0, IP-3\}}}
            \item Einrichtung des Projekts lokal mit erster Ausführung \mbox{\textbf{\{1: IP-2\}}}
        \end{itemize}
    } \\
    \hline
    IQ-B3 (Vier) &
    Gibt es manuelle Schritte in Ihrer Development- oder Deploymentumgebung, deren Automatisierung nicht möglich oder nicht sinnvoll ist beziehungsweise die eine Automatisierung ab einem bestimmten Punkt blockieren? \newline
    Falls ja, welche? \\
    \nopagebreak
    \multicolumn{2}{ >{\raggedright}p{0.9\textwidth} }{
        \begin{itemize}
            \item Hinzufügen von Umgebungsvariablen aus der Config \mbox{\textbf{\{1: IP-2\}}}
            \item Auswahl von Tests beim Starten der Software \mbox{\textbf{\{1: IP-0\}}}
            \item Bestätigung beim Deployment auf \linebreak[1] Produktivumgebungen oder Stages \mbox{\textbf{\{3: IP-0, IP-1, IP-3\}}}
        \end{itemize}
    } \\
    \hline
    IQ-B4 (Fünf) &
    Welche drei Ziele würden Sie verfolgen, wenn Sie eine Toolchain für ein neues Projekt im eingangs beschriebenen Projektkontext entwerfen müssten? \newline
    [Liste aus Zielen] \\
    \nopagebreak
    \multicolumn{2}{ >{\raggedright}p{0.9\textwidth} }{
        \begin{itemize}
            \item \mbox{\textbf{\{1: IP-0\}}}
            \begin{enumerate}
                \item Containerisierung \mbox{\textbf{\{1: IP-0\}}}
                \item 12-Faktor-Prinzipien \mbox{\textbf{\{1: IP-0\}}}
                \item Verwendung einer Sprache mit Typprüfung und minimalen externen Abhängigkeiten \mbox{\textbf{\{1: IP-0\}}}
            \end{enumerate}
            \item \mbox{\textbf{\{1: IP-1\}}}
            \begin{enumerate}
                \item Reduzierung von manuellem Aufwand \mbox{\textbf{\{1: IP-1\}}}
                \item Senkung der Einstiegskurve \mbox{\textbf{\{1: IP-1\}}}
                \item Einräumen von Freiheiten für neue Entwickler \mbox{\textbf{\{1: IP-1\}}}
            \end{enumerate}
            \item \mbox{\textbf{\{1: IP-2\}}}
            \begin{enumerate}
                \item Reduzierung des Arbeitsaufwands \mbox{\textbf{\{1: IP-2\}}}
                \item starke Anpassbarkeit \mbox{\textbf{\{1: IP-2\}}}
                \item Toolchain entspricht einem (Unternehmens-)Standard \mbox{\textbf{\{1: IP-2\}}}
            \end{enumerate}
            \item \mbox{\textbf{\{1: IP-3\}}}
            \begin{enumerate}
                \item Einfachheit \mbox{\textbf{\{1: IP-3\}}}
                \item Unterstützung von Fast Feedback \mbox{\textbf{\{1: IP-3\}}}
                \item Stabilität \mbox{\textbf{\{1: IP-3\}}}
            \end{enumerate}
        \end{itemize}
    } \\
\end{longtable}

\clearpage

\subsection{Geschlossene Fragen}
\label{subsec:AA-03-03_closed-questions}

\begin{longtable}{ >{\raggedright\bfseries}p{0.2\textwidth} p{0.7\textwidth} }
    IQ-C0 (Eins) &
    Wie groß ist Ihrer Meinung nach der Anteil an manuellen Schritten (IQ-B3), deren Automatisierung nicht möglich oder nicht sinnvoll ist? \newline
    [Wert zwischen 0 \% und 100 \%] \\
    \nopagebreak
    \multicolumn{2}{ >{\raggedright}p{0.9\textwidth} }{
        10 \% bis 20 \% \mbox{\textbf{\{3: IP-0, IP-1, IP-2\}}} \\
        05 \% \mbox{\textbf{\{1: IP-3\}}}
    } \\
    \hline
    IQ-C1 (Eins) &
    Was denken Sie, wie gut könnten einzelne Komponenten Ihrer Toolchains (wie beispielsweise ein Paketmanager oder ein Pipeline-Tool) ausgetauscht werden, also wie flexibel sind die Toolchains? \newline
    [Skala von 1 (sehr wenig austauschbar) bis 5 (sehr flexibel)] \\
    \nopagebreak
    \multicolumn{2}{ >{\raggedright}p{0.9\textwidth} }{
        2 \mbox{\textbf{\{1: IP-2\}}} \\
        3 \mbox{\textbf{\{1: IP-1\}}} \\
        4 bis 5 \mbox{\textbf{\{2: IP-0, IP-3\}}}
    } \\
    \hline
    IQ-C2 (Drei) &
    Wie häufig müssen Sie Anpassungen an Ihrer Development- oder Deploymentumgebung vornehmen, um neuen Projektanforderungen gerecht zu werden? \newline
    [Skala von 1 (sehr selten) bis 5 (sehr häufig)] \\
    \nopagebreak
    \multicolumn{2}{ >{\raggedright}p{0.9\textwidth} }{
        2 bis 3 \mbox{\textbf{\{4: IP-0, IP-1, IP-2, IP-3\}}}
    } \\
    \hline
    IQ-C3 (Drei) &
    Wie regelmäßig aktualisieren Sie die verwendeten Tools und Bibliotheken in Ihrer Umgebung? \newline
    [Skala von 1 (selten oder nie) bis 5 (sehr regelmäßig)] \\
    \nopagebreak
    \multicolumn{2}{ >{\raggedright}p{0.9\textwidth} }{
        1 bis 2 \mbox{\textbf{\{1: IP-0\}}} \\
        3 \mbox{\textbf{\{1: IP-2\}}} \\
        4 \mbox{\textbf{\{1: IP-3\}}} \\
        5 \mbox{\textbf{\{1: IP-1\}}}
    } \\
    \hline
    IQ-C4 (Vier) &
    An welcher Stelle wird Ihre Software in der Regel nach Anpassungen das erste Mal ausgeführt? \newline
    [Auswahl zwischen \q{lokal}, \q{Pipeline} und \q{X-Stage}] \\
    \nopagebreak
    \multicolumn{2}{ >{\raggedright}p{0.9\textwidth} }{
        lokal \mbox{\textbf{\{4: IP-0, IP-1, IP-2, IP-3\}}}
    } \\
\end{longtable}

\clearpage

\subsection{Bewertungsmöglichkeit für Anforderungen}
\label{subsec:AA-03-04_evaluation-requirements}

\setcounter{factornoappendix}{-1}
\begin{longtable}{  |   >{\raggedleft\factornumberappendix}p{0.025\textwidth}   % Number (centered)
                        >{\raggedright\bfseries}p{0.175\textwidth}              % Factor (left-aligned)
                    |   >{\raggedright}p{0.600\textwidth}                       % Requirements and linking Concepts (left-aligned)
                    |   >{}p{0.100\textwidth}                                   % Points (centered)
                    | }
    \hline
        & \upshape\textbf{Faktor} 
        & \upshape\textbf{Anforderungen und verknüpfte Konzepte}
        & \upshape\textbf{Punkte} \\
    \hline \hline
    \endhead
    \hline
    %   (I) Codebase
        & Codebase
        & \textit{GitOps} \textrightarrow Repository als \q{Single Source of Truth}
        & 04 \\
    \hline
    %   (II) Dependencies
        & Dependencies
        & \textit{GitOps} \textrightarrow Dependencies deklarativ in Konfigurationsdateien
        & 04 \\
    \hline
    %   (III) Config
        & Config
        & \textit{Dotfiles} \textrightarrow Konfiguration in \texttt{.env} files außerhalb des Repositories
        & 03 \\
    \hline
    %   (IV) Backing Services
        & Backing Services
        & \textit{Dotfiles} \textrightarrow Zugang zu Diensten über Config
        & 02 \\
    \hline
    %   (V) Build, Release, Run
        & Build, Release, Run
        & \textit{DevOps} \textrightarrow CI/CD Pipelines, \newline
          \textit{Dotfiles} \textrightarrow Config in Release Stage
        & 04 \\
    \hline
    %   (VI) Processes
        & Processes
        & -/-
        & 00 \\
    \hline
    %   (VII) Port Binding
        & Port Binding
        & \textit{GitOps} \textrightarrow Dependency Declaration für eingebundene Webserver, \newline
          \textit{Dotfiles} \textrightarrow Konfiguration von Ports
        & 00 \\
    \hline
    %   (VIII) Concurrency
        & Concurrency
        & -/-
        & 01 \\
    \hline
    %   (IX) Disposability
        & Disposability
        & \textit{Best Containerization Practices} \textrightarrow kompakte Gestaltung von Docker Images
        & 00 \\
    \hline
    %   (X) Dev / Prod Parity
        & Dev / Prod Parity
        & Toolchain-as-Code Strategie
        & 04 \\
    \hline
    %   (XI) Logs
        & Logs
        & \textit{Best Containerization Practices} \textrightarrow Prozesse loggen nach \texttt{stdout}
        & 02 \\
    \hline
    %   (XII) Admin Processes
        & Admin Processes
        & -/-
        & 00 \\
    \hline
\end{longtable}
\vspace{1em}
\setcounter{factornoappendix}{0}
    \clearpage
\section{Interviewauswertung}
\label{sec:AA-04_interview-evaluation}

Die Auswertung der Interviews erfolgt anhand der in den vorherigen Abschnitten dokumentierten Ergebnisse. Ziel ist die Zuordnung der Aussagen zu genau einer von vier möglichen Kategorien:

\begin{itemize}
    \item \hyperref[subsec:AA-04-01_requirements-development]{Anforderungen an Toolchains im Bereich \textbf{Development}},
    \item \hyperref[subsec:AA-04-02_requirements-deployment]{Anforderungen an Toolchains im Bereich \textbf{Deployment}},
    \item \hyperref[subsec:AA-04-03_requirements-continuity]{Anforderungen an die \textbf{Durchgängigkeit} von Toolchains}, oder
    \item \hyperref[subsec:AA-04-04_requirements-general]{\textbf{Allgemeine} Anforderungen}.
\end{itemize}

Dabei behalten sie den im vorherigen Abschnitt zu den \nameref{sec:AA-03_interview-results} vergebenen Zuordnungszusatz bei. Weiterhin werden alle Aussagen innerhalb ihres Anforderungebsereichs in Anforderungsgruppen zusammengefasst. Über diese Gruppen soll ermöglicht werden, aus dem Querschnitt aller in ihr enthaltenen Aussagen eine möglichst umfassende Anforderung an Toolchains abzuleiten. Diese Anforderungsgruppen werden anschließend gewichtet und absteigend nach ihrer Priorität sortiert. Als Indikator für die Gewichtung einer Anforderungsgruppe dient die Anzahl der in ihr enthaltenen Aussagen, wobei mehrfach getroffene Aussagen auch entsprechend mehrfach einfließen. Zusätzlich wird jeder Anforderungsgruppe, sofern passend, einer oder mehrere Faktoren aus dem Konzept der \q{Twelve-Factor App} zugeordnet, der als Referenz bei der methodischen Auswertung unterstützen kann.

\subsection{Anforderungen an Toolchains im Bereich Development}
\label{subsec:AA-04-01_requirements-development}

\subsubsection{Entwicklungsumgebung}
\label{subsubsec:AA-04-01-01_req-dev-development-environment}

\vspace{0.5em}
\begin{tabular}{ll@{}ll@{}}
    \textbf{Gewichtung:}    &   09 Gewichtungspunkte    \\
    \textbf{Faktor:}        &   X: Dev / Prod Parity    \\
\end{tabular}

\begin{flushleft}
    \begin{itemize}
        \item Aufsetzen der Entwicklungsumgebung ist sehr wichtig \mbox{\textbf{\{1: IP-1\}}}
        \item Installation der Entwicklungsumgebung hat Optimierungspotential \mbox{\textbf{\{1: IP-3\}}}
        \item Entwicklungsumgebung sollte möglichst vollständig geladen werden \mbox{\textbf{\{2: IP-1, IP-3\}}}
        \item Tools sind einheitlich vorgegeben \mbox{\textbf{\{1: IP-2\}}}
        \item fehlende Administratorrechte sind eine Herausforderung \mbox{\textbf{\{1: IP-0\}}}
        \item verwendete Tools sind
        \begin{itemize}
            \item \textit{Git} \mbox{\textbf{\{4: IP-0, IP-1, IP-2 IP-3\}}}
            \item \textit{Docker} \mbox{\textbf{\{3: IP-0, IP-1, IP-3\}}}
            \item \textit{Unix}-Tools \mbox{\textbf{\{1: IP-1\}}}
            \item VCS (\textit{Git}) \mbox{\textbf{\{1: IP-2\}}}
        \end{itemize}
        \item Plugins für Editor oder integrierte Entwicklungsumgebung \linebreak[2] sind installiert \mbox{\textbf{\{2: IP-1, IP-2\}}}
    \end{itemize}
\end{flushleft}

\subsubsection{Ausführbarkeit}
\label{subsubsec:AA-04-01-02_req-dev-executability}

\vspace{0.5em}
\begin{tabular}{ll@{}ll@{}}
    \textbf{Gewichtung:}    &   08 Gewichtungspunkte    \\
    \textbf{Faktor:}        &   I: Codebase             \\
\end{tabular}

\begin{flushleft}
    \begin{itemize}
        \item Software wird in der Regel nach Anpassungen \linebreak[4] das erste Mal lokal ausgeführt \mbox{\textbf{\{4: IP-0, IP-1, IP-2, IP-3\}}}
        \item Einrichtung des Projekts lokal mit \linebreak[4] erster Ausführung hat Optimierungspotential \mbox{\textbf{\{1: IP-2\}}}
        \item Einrichtung und Start des Projekts lokal ist Schritt bei der Einrichtung \mbox{\textbf{\{2: IP-2, IP-3\}}}
        \item Einrichtung und Start der Tests lokal ist Schritt bei der Einrichtung \mbox{\textbf{\{1: IP-3\}}}
    \end{itemize}
\end{flushleft}

\subsubsection{Konfiguration}
\label{subsubsec:AA-04-01-03_req-dev-configuration}

\vspace{0.5em}
\begin{tabular}{ll@{}ll@{}}
    \textbf{Gewichtung:}    &   07 Gewichtungspunkte    \\
    \textbf{Faktor:}        &   III: Config             \\
\end{tabular}

\begin{flushleft}
    \begin{itemize}
        \item teilweise hoher Konfigurationsbedarf ist eine Herausforderung \mbox{\textbf{\{1: IP-2\}}}
        \item Konfigurationsaufwand sollte gering sein \mbox{\textbf{\{1: IP-1\}}}
        \item Konfiguration erfolgt über versionierte Dateien \mbox{\textbf{\{2: IP-2, IP-3\}}}
        \item Konfiguration der integrierten Entwicklungsumgebung beziehungsweise des Editors \linebreak[1] erfolgt über ein entsprechendes Konfigurationsverzeichnis im VCS \mbox{\textbf{\{1: IP-1\}}}
        \item Setzen der Umgebungsvariablen ist Schritt bei der Einrichtung \mbox{\textbf{\{1: IP-1\}}}
        \item Hinzufügen von Umgebungsvariablen aus der Config ist nicht automatisierbar \mbox{\textbf{\{1: IP-2\}}}
    \end{itemize}
\end{flushleft}

\subsubsection{Abhängigkeiten}
\label{subsubsec:AA-04-01-04_req-dev-dependencies}

\vspace{0.5em}
\begin{tabular}{ll@{}ll@{}}
    \textbf{Gewichtung:}    &   06 Gewichtungspunkte    \\
    \textbf{Faktor:}        &   II: Dependencies        \\
\end{tabular}

\begin{flushleft}
    \begin{itemize}
        \item Installation von SDKs hat Optimierungspotential \mbox{\textbf{\{1: IP-1\}}}
        \item Installation von Dependencies hat Optimierungspotential \mbox{\textbf{\{2: IP-0, IP-3\}}}
        \item Dependencies sollten einfach einsehbar und verwaltbar sein \mbox{\textbf{\{1: IP-0\}}}
        \item unterschiedliche Versionen von SDKs \linebreak[1] in verschiedenen Projekten sind eine Herausforderung \mbox{\textbf{\{1: IP-0\}}}
        \item Installation der Abhängigkeiten sollte über Paketmanager erfolgen \mbox{\textbf{\{1: IP-1\}}}
    \end{itemize}
\end{flushleft}

\subsubsection{Dokumentation}
\label{subsubsec:AA-04-01-05_req-dev-documentation}

\vspace{0.5em}
\begin{tabular}{ll@{}ll@{}}
    \textbf{Gewichtung:}    &   04 Gewichtungspunkte    \\
    \textbf{Faktor:}        &   -/-                     \\
\end{tabular}

\begin{flushleft}
    \begin{itemize}
        \item Einrichtung eines einfachen und verständlichen Setups ist sehr wichtig \mbox{\textbf{\{1: IP-3\}}}
        \item veraltete Dokumentation ist eine Herausforderung \mbox{\textbf{\{1: IP-2\}}}
        \item Lesen und Befolgen der Dokumentation ist Schritt bei der Einrichtung \mbox{\textbf{\{2: IP-2, IP-3\}}}
    \end{itemize}
\end{flushleft}

\subsubsection{Skripte}
\label{subsubsec:AA-04-01-06_req-dev-scripts}

\vspace{0.5em}
\begin{tabular}{ll@{}ll@{}}
    \textbf{Gewichtung:}    &   02 Gewichtungspunkte    \\
    \textbf{Faktor:}        &   V: Build, Release, Run  \\
\end{tabular}

\begin{flushleft}
    \begin{itemize}
        \item Shell-Skripte erledigen bestimmte Aufgaben \mbox{\textbf{\{1: IP-3\}}}
        \item veraltete Skripte sind eine Herausforderung \mbox{\textbf{\{1: IP-3\}}}
    \end{itemize}
\end{flushleft}

\subsubsection{Sonstiges}
\label{subsubsec:AA-04-01-07_req-dev-miscellaneous}

\begin{flushleft}
    \begin{itemize}
        \item Einrichtung von CA-Zertifikaten ist Schritt bei der Einrichtung \mbox{\textbf{\{1: IP-0\}}}
        \item Standards und Konventionen sollten eingehalten werden \mbox{\textbf{\{2: IP-2, IP-3\}}}
        \begin{itemize}
            \item \textit{Git Workflow} \mbox{\textbf{\{1: IP-2\}}}
            \item \textit{Google Java Styleguide} \mbox{\textbf{\{1: IP-2\}}}
        \end{itemize}
        \item Onboarding neuer Entwickler im Hinblick auf Toolchains dauert
        \begin{itemize}
            \item $ < 01 h $ in neuem Projekt \mbox{\textbf{\{1: IP-0\}}}
            \item $ \sim 08 h $ (1 Tag) in neuem Projekt \mbox{\textbf{\{2: IP-1, IP-3\}}}
            \item $ \sim 16 h - 24 h $ (2 - 3 Tage) in neuem Projekt \mbox{\textbf{\{1: IP-2\}}}
        \end{itemize}
    \end{itemize}
\end{flushleft}

\subsection{Anforderungen an Toolchains im Bereich Deployment}
\label{subsec:AA-04-02_requirements-deployment}

\subsubsection{Automatisierung}
\label{subsubsec:AA-04-02-01_req-dep-automation}

\vspace{0.5em}
\begin{tabular}{ll@{}ll@{}}
    \textbf{Gewichtung:}    &   06 Gewichtungspunkte    \\
    \textbf{Faktor:}        &   V: Build, Release, Run  \\
\end{tabular}

\begin{flushleft}
    \begin{itemize}
        \item Deployment ist maximal automatisiert \mbox{\textbf{\{1: IP-2\}}}
        \item Evolution eines eigenen Artefakts auf die nächste Stage \linebreak[1] ist Schritt bei der Einrichtung \mbox{\textbf{\{1: IP-2\}}}
        \item verwendete Tools sind
        \begin{itemize}
            \item Pipeline Tools (\textit{GitHub Actions}) \mbox{\textbf{\{1: IP-3\}}}
            \item Deployment Tools (\textit{Argo CD}) \mbox{\textbf{\{1: IP-2\}}}
            \item Container Orchestration Tools (\textit{Kubernetes}) \mbox{\textbf{\{1: IP-2\}}}
            \item Infrastructure-as-Code (IaC) für Umgebungen und Pipelines \mbox{\textbf{\{1: IP-3\}}}
        \end{itemize}
        \item Bestätigung beim Deployment auf Produktivumgebungen oder Stages \linebreak[1] ist nicht automatisierbar \mbox{\textbf{\{3: IP-0, IP-1, IP-3\}}}
    \end{itemize}
\end{flushleft}

\subsubsection{Konfiguration}
\label{subsubsec:AA-04-02-02_req-dep-configuration}

\vspace{0.5em}
\begin{tabular}{ll@{}ll@{}}
    \textbf{Gewichtung:}    &   02 Gewichtungspunkte    \\
    \textbf{Faktor:}        &   III: Config             \\
\end{tabular}

\begin{flushleft}
    \begin{itemize}
        \item Konfiguration erfolgt über versionierte Dateien \mbox{\textbf{\{2: IP-2, IP-3\}}}
    \end{itemize}
\end{flushleft}

\subsubsection{Containerization}
\label{subsubsec:AA-04-02-03_req-dep-containerization}

\vspace{0.5em}
\begin{tabular}{ll@{}ll@{}}
    \textbf{Gewichtung:}    &   01 Gewichtungspunkt     \\
    \textbf{Faktor:}        &   -/-                     \\
\end{tabular}

\begin{flushleft}
    \begin{itemize}
        \item Deployment sollte bestenfalls per Container oder serverless erfolgen \mbox{\textbf{\{1: IP-1\}}}
    \end{itemize}
\end{flushleft}

\subsubsection{Qualität und Sicherheit}
\label{subsubsec:AA-04-02-04_req-dep-quality-security}

\vspace{0.5em}
\begin{tabular}{ll@{}ll@{}}
    \textbf{Gewichtung:}    &   01 Gewichtungspunkt     \\
    \textbf{Faktor:}        &   -/-                     \\
\end{tabular}

\begin{flushleft}
    \begin{itemize}
        \item Fast Feedback und Alerting ermöglichen \linebreak[1] schnelle Reaktion der Entwickler \mbox{\textbf{\{1: IP-3\}}}
    \end{itemize}
\end{flushleft}

\subsection{Anforderungen an die Durchgängigkeit von Toolchains}
\label{subsec:AA-04-03_requirements-continuity}

\subsubsection{Tools und Skripte}
\label{subsubsec:AA-04-03-01_req-cnt-tools-scripts}

\vspace{0.5em}
\begin{tabular}{ll@{}ll@{}}
    \textbf{Gewichtung:}    &   05 Gewichtungspunkte    \\
    \textbf{Faktor:}        &   V: Build, Release, Run  \\
\end{tabular}

\begin{flushleft}
    \begin{itemize}
        \item Task-Files im Repository \mbox{\textbf{\{1: IP-0\}}}
        \item Bash-Skripte im Repository \mbox{\textbf{\{2: IP-0, IP-1\}}}
        \item Build Tools (\textit{Maven}) \mbox{\textbf{\{2: IP-2, IP-3\}}}
    \end{itemize}
\end{flushleft}

\subsubsection{Parität}
\label{subsubsec:AA-04-03-02_req-cnt-parity}

\vspace{0.5em}
\begin{tabular}{ll@{}ll@{}}
    \textbf{Gewichtung:}    &   03 Gewichtungspunkte    \\
    \textbf{Faktor:}        &   X: Dev / Prod Parity    \\
\end{tabular}

\begin{flushleft}
    \begin{itemize}
        \item lokale Umgebung ist möglichst nah an der Produktivumgebung \mbox{\textbf{\{1: IP-3\}}}
        \item Tools sollten plattformübergreifend funktionieren \mbox{\textbf{\{1: IP-0\}}}
        \item Vorhandensein einer eindeutigen Config ist gegeben \mbox{\textbf{\{1: IP-0\}}}
    \end{itemize}
\end{flushleft}

\subsubsection{Anpassbarkeit}
\label{subsubsec:AA-04-03-03_req-cnt-adaptability}

\vspace{0.5em}
\begin{tabular}{ll@{}ll@{}}
    \textbf{Gewichtung:}    &   02 Gewichtungspunkte    \\
    \textbf{Faktor:}        &   III: Config             \\
\end{tabular}

\begin{flushleft}
    \begin{itemize}
        \item Umgebung sollte durch das Entwicklungsteam anpassbar sein \mbox{\textbf{\{1: IP-3\}}}
        \item Vermeidung unnötiger Komplexität von Beginn an ist ein Ziel \mbox{\textbf{\{1: IP-3\}}}
    \end{itemize}
\end{flushleft}

\subsubsection{Reproduzierbarkeit}
\label{subsubsec:AA-04-03-04_req-cnt-reproducibility}

\vspace{0.5em}
\begin{tabular}{ll@{}ll@{}}
    \textbf{Gewichtung:}    &   02 Gewichtungspunkte    \\
    \textbf{Faktor:}        &   X: Dev / Prod Parity    \\
\end{tabular}

\begin{flushleft}
    \begin{itemize}
        \item fehlende Reproduzierbarkeit ist eine Herausforderung \mbox{\textbf{\{1: IP-2\}}}
        \item Existenz eines einzigen Commands \linebreak[1] zum Starten der Infrastruktur ist wichtig \mbox{\textbf{\{1: IP-0\}}}
    \end{itemize}
\end{flushleft}

\subsection{Allgemeine Anforderungen}
\label{subsec:AA-04-04_requirements-general}

\subsubsection{Ziele}
\label{subsubsec:AA-04-04-01_req-gen-goals}

\begin{flushleft}
    \begin{itemize}
        \item Ziele mit großer Priorität
        \begin{itemize}
            \item Reduzierung von manuellem Aufwand \mbox{\textbf{\{1: IP-1\}}}
            \item Reduzierung des Arbeitsaufwands \mbox{\textbf{\{1: IP-2\}}}
            \item Einfachheit \mbox{\textbf{\{1: IP-3\}}}
            \item Containerisierung \mbox{\textbf{\{1: IP-0\}}}
        \end{itemize}
        \item Ziele mit mittlerer Priorität
        \begin{itemize}
            \item Senkung der Einstiegskurve \mbox{\textbf{\{1: IP-1\}}}
            \item starke Anpassbarkeit \mbox{\textbf{\{1: IP-2\}}}
            \item 12-Faktor-Prinzipien \mbox{\textbf{\{1: IP-0\}}}
            \item Unterstützung von Fast Feedback \mbox{\textbf{\{1: IP-3\}}}
        \end{itemize}
        \item Ziele mit geringer Priorität
        \begin{itemize}
            \item Einräumen von Freiheiten für neue Entwickler \mbox{\textbf{\{1: IP-1\}}}
            \item Toolchain entspricht einem (Unternehmens-)Standard \mbox{\textbf{\{1: IP-2\}}}
            \item Stabilität \mbox{\textbf{\{1: IP-3\}}}
            \item Verwendung einer Sprache mit Typprüfung \linebreak[1] und minimalen externen Abhängigkeiten \mbox{\textbf{\{1: IP-0\}}}
        \end{itemize}
    \end{itemize}
\end{flushleft}

\subsubsection{Sonstiges}
\label{subsubsec:AA-04-04-02_req-gen-miscellaneous}

\begin{flushleft}
    \begin{itemize}
        \item Anteil nicht automatisierbarer Schritte liegt zwischen 5 \% und 20 \%
        \item Toolchains sind je nach Umfeld unterschiedlich flexibel
        \begin{itemize}
            \item im Application Management eher weniger austauschbar
            \item im Software Development Center eher flexibler
        \end{itemize}
        \item Anpassungen an der Umgebung sind wenig bis selten notwendig
        \item Tools und Bibliotheken werden unterschiedlich häufig ausgetauscht
    \end{itemize}
\end{flushleft}
 \clearpage

	\chapter{Quellcode des Prototypen}
\label{ch:BB-source-code-of-prototype}

% replace text with content here
Anhang

\end{document}
