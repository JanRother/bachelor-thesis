\section{Strategie für Toolchains}
\label{sec:05-02_strategy-for-toolchains}

\subsection{Teilkonzepte zur Realisierung}
\label{subsec:05-02-01_sub-concepts-for-implementation}

Die \nameref{ch:05_toolchain-as-code} Strategie soll sich maßgeblich an denen in \autoref{ch:04_requirements-for-development-and-deployment-environments} ermittelten Anforderungen der Entwickler orientieren. Dabei soll für jede Anforderung ein Teilkonzept formuliert werden, welches auf Erkenntnissen aus vorherigen Untersuchungen der Arbeit basiert, größtenteils auf der \nameref{ch:03_examination-of-existing-approaches} (\autoref{ch:03_examination-of-existing-approaches}). Insgesamt konnte jedem der ursprünglichen Anforderungen ein Teilkonzept zugeordnet werden.

Entsprechend diesem Vorgehen werden die Teilkonzepte, genau wie die Anforderungen, in die drei Bereiche \Gls{dev}, \Gls{dep} und \Gls{cnt} unterteilt. Ihre Sortierung entspricht der ursprünglichen Priorisierung der Anforderungen, wie in \autoref{subsec:04-02-01_procedure-for-evaluation} erklärt.

In \autoref{tab:concepts-development} sind die \nameref{tab:concepts-development} formuliert, die die \nameref{tab:requirements-development} (siehe \autoref{tab:requirements-development}) erfüllen.

\begin{table}[H]
    \begin{tabular}{ >{\bfseries\ttfamily}p{0.10\textwidth} >{}p{0.30\textwidth} | >{}p{0.50\textwidth} }
        DEV-0   &   Eine Entwicklungsumgebung wird schnell und vollständig geladen. &
        \nameref{subsec:05-01-02_dev-container} stellen eine Umgebung inklusive ihrer Konfiguration bereit, die gemeinsam gestartet werden können. \\
        \hline
        DEV-1   &   Die Applikation ist auf den lokalen Maschinen der Entwickler ausführbar. &
        Die Applikation ist in einem eigenständigen \nameref{subsec:05-01-01_docker-container} gekapselt, der lokal ausführbar ist und kann beim Entwickler gestartet werden. \\
        \hline
        DEV-2   &   Konfiguration ist im \Gls{vcs} verfügbar. &
        Gemäß dem \hyperref[sec:03-03_gitops]{GitOps} Ansatz liegen alle Konfigurationen als Dateien in den Repositories vor. \\
        \hline
        DEV-3   &   Alle nötigen Abhängigkeiten und \Glspl{sdk} sind in der Umgebung bereitgestellt. &
        \nameref{subsec:05-01-01_docker-container} enthalten das Basisbetriebssystem, \Glspl{sdk}, Bibliotheken, \Gls{build} Tools und weitere Softwarekomponenten. \\
        \hline
        DEV-4   &   Dokumentation ist aktuell und intuitiv. &
        Deklarative Konfigurationen sind selbst-dokumentierend und \textit{Markdown} Dateien liefern einen kompakten Überblick inklusive Anleitung zum Starten des Projekts. \\
        \hline
        DEV-5   &   Skripte reduzieren komplexere imperative Aufgaben auf einen Befehl. &
        \textit{Task-Files} stehen zur Verfügung, um aufwändige Befehlskompositionen über kurze und zentrale Kommandos zu starten. \\
    \end{tabular}
    \caption{Teilkonzepte für Toolchains im Bereich Development}
    \label{tab:concepts-development}
\end{table}

In \autoref{tab:concepts-deployment} sind die \nameref{tab:concepts-deployment} formuliert, die die \nameref{tab:requirements-deployment} (siehe \autoref{tab:requirements-deployment}) erfüllen.

\begin{table}[H]
    \begin{tabular}{ >{\bfseries\ttfamily}p{0.10\textwidth} >{}p{0.30\textwidth} | >{}p{0.50\textwidth} }
        DEP-0   &   Mit Ausnahme einer händischen Bestätigung ist das \Gls{deployment} vollständig automatisiert. &
        Beim \Gls{deployment} der Umgebung in ein \Gls{image} Repository ist in der \Gls{cicd} Konfiguration eine manuelle Bestätigung erforderlich. \\
        \hline
        DEP-1   &   Konfiguration wird aus dem \Gls{vcs} bezogen. &
        Gemäß dem \hyperref[sec:03-03_gitops]{GitOps} Ansatz verfügt jedes Repository über eine \Gls{cicd} Konfiguration, sofern dies sinnvoll oder erforderlich ist. \\
        \hline
        DEP-2   &   Umgebung und Applikation sind Container. &
        \textit{Docker} packt die Applikation inklusive aller Abhängigkeiten in einem Container. \\
        \hline
        DEP-3   &   Bei Problemen werden Entwickler sofort benachrichtigt. &
        \Gls{github-actions} Workflows können  so konfiguriert werden, dass sie Benachrichtigungen bei Fehlern in einer Pipeline per E-Mail versenden. \\
    \end{tabular}
    \caption{Teilkonzepte für Toolchains im Bereich Deployment}
    \label{tab:concepts-deployment}
\end{table}

In \autoref{tab:concepts-continuity} sind die \nameref{tab:concepts-continuity} formuliert, die die \nameref{tab:requirements-continuity} (siehe \autoref{tab:requirements-continuity}) erfüllen.

\begin{table}[H]
    \begin{tabular}{ >{\bfseries\ttfamily}p{0.10\textwidth} >{}p{0.30\textwidth} | >{}p{0.50\textwidth} }
        CNT-0   &   Automatisierung erfolgt auf allen Ebenen über die gleichen Tools. &
        Durch Kombination von \Gls{cicd} aus dem \hyperref[sec:03-03_gitops]{GitOps} Ansatz und \nameref{sec:02-03_containerization} ist gewährleistet, dass Integrations- und Deploymentkonfigurationen genauso wie Abhängigkeiten und Werkzeuge immer von der gleichen Quelle bezogen werden. \\
        \hline
        CNT-1   &   Die lokale Umgebung entspricht möglichst vollständig der Produktivumgebung. &
        Sowohl im \Gls{development} als auch im \Gls{deployment} werden die gleichen Container und Konfigurationen verwendet. Beim \Gls{development} steht außerdem eine Datenbank mit dem gleichen Schema und mindestens Beispieldaten zur Verfügung. \\
        \hline
        CNT-2   &   Alle Umgebungen sind mit möglichst wenig Tooleinsatz durch das Entwicklungsteam anpassbar. &
        Da gemäß \hyperref[sec:03-03_gitops]{GitOps} alle Konfigurationen im Code vorliegen, kann jeder Entwickler mit grundlegendem Verständnis von \Gls{vcs} wie \textit{Git} diese lesen und anpassen. \\
        \hline
        CNT-3   &   Die Umgebung sowie das Verhalten der Applikation in ihr sind auf allen Ebenen reproduzierbar. &
        \nameref{subsec:05-01-01_docker-container} und \nameref{subsec:05-01-02_dev-container} kapseln Applikation und Umgebung in einem in sich geschlossenem und reproduzierbarem System. \\
    \end{tabular}
    \caption{Teilkonzepte für Toolchains im Bereich Durchgängigkeit}
    \label{tab:concepts-continuity}
\end{table}

In \autoref{sec:03-05_concept-of-twelve-factor-app} wurde die \hyperref[sec:03-05_concept-of-twelve-factor-app]{Twelve-Factor-App} vorgestellt. Sie besteht aus zwölf Faktoren, die \textit{Software as a Service} erfüllen sollte. Im Rahmen der \hyperref[sec:04-01_collection-of-requirements-using-expert-interviews]{Erhebung von Anforderungen an Umgebungen} wurden diese Faktoren durch Experten priorisiert (\acrshort{vgl}\autoref{tab:interview-results-factors-priorities}). Der als am wichtigsten bewerteten oberen Hälfte dieser Faktoren sollen daher \nameref{tab:concepts-factors} zugeordnet werden. Diese sind in \autoref{tab:concepts-factors} dargestellt.

\setcounter{factorno}{-1}
\begin{table}[H]
    \begin{tabular}{ | >{\bfseries}p{0.0125\textwidth} | >{\raggedright\bfseries}p{0.2500\textwidth} | >{}p{0.6375\textwidth} | }
        \hline
              
            & \textbf{Faktor} 
            & \textbf{Teilkonzept} \\
        \hline \hline
        %   (I) Codebase
            0
            & %\setcounter{factorno}{01}\Roman{factorno}
            Codebase
            & Die \Gls{codebase} der Umgebung kann für Integration und Entwicklung genutzt werden, die \Gls{codebase} der Applikation ist lokal und als Deployment verwendbar. \\
        \hline
        %   (II) Dependencies
            0
            & %\setcounter{factorno}{02}\Roman{factorno}
            Dependencies
            & Jedes Repository verfügt über Konfigurationsdateien, in denen Abhängigkeiten deklarativ aufgeführt sind. \\
        \hline
        %   (V) Build, Release, Run
            0
            & %\setcounter{factorno}{05}\Roman{factorno}
            Build, Release, Run
            & GitOps lässt die Konfiguration des \Gls{cicd} Konzepts für mehrere Ebenen und verschiedene Pipelines zu. \\
        \hline
        %   (X) Dev / Prod Parity
            0
            & %\setcounter{factorno}{10}\Roman{factorno}
            Dev / Prod Parity
            & Development- und Deploymentumgebungen für die Applikation greifen auf die gleiche containerisierte Umgebung zurück. \\
        \hline
        %   (III) Config
            1
            & %\setcounter{factorno}{03}\Roman{factorno}
            Config
            & Variable Konfigurationen wie Datenbankverbindungen werden außerhalb der Revision in Umgebungsvariablen abgelegt. \\
        \hline
        %   (IV) Backing Services
            2
            & %\setcounter{factorno}{04}\Roman{factorno}
            Backing Services
            & Auf Services wie Datenbanken kann bei Bedarf lokal zugegriffen werden, sie sind jedoch nicht im Umfang des Deployments der Applikation enthalten. \\
        \hline
        %   (XI) Logs
            2
            & %\setcounter{factorno}{11}\Roman{factorno}
            Logs
            & \acrshort{na} \\
        \hline
        %   (VIII) Concurrency
            3
            & %\setcounter{factorno}{08}\Roman{factorno}
            Concurrency
            & \acrshort{na} \\
        \hline
        %   (VI) Processes
            -
            & %\setcounter{factorno}{06}\Roman{factorno}
            Processes
            & \acrshort{na} \\
        \hline
        %   (VII) Port Binding
            -
            & %\setcounter{factorno}{07}\Roman{factorno}
            Port Binding
            & \acrshort{na} \\
        \hline
        %   (IX) Disposability
            -
            & %\setcounter{factorno}{09}\Roman{factorno}
            Disposability
            & \acrshort{na} \\
        \hline
        %   (XII) Admin Processes
            -
            & %\setcounter{factorno}{12}\Roman{factorno}
            Admin Processes
            & \acrshort{na} \\
        \hline
    \end{tabular}
    \caption{Teilkonzepte für Toolchains auf Basis der \q{Twelve-Factor-App}}
    \label{tab:concepts-factors}
\end{table}
\setcounter{factorno}{0}

\subsection{Architektur im Toolchain-as-Code Ansatz}
\label{subsec:05-02-02_architecture-in-the-toolchain-as-code-approach}

\subsubsection{Datenhaltung in Ablageorten}
\label{subsubsec:05-02-02-01_data-storage-in-repositories}

Die Datenhaltung der \nameref{ch:05_toolchain-as-code} Strategie basiert auf den Prinzipien von \hyperref[sec:03-03_gitops]{GitOps} und dem \hyperref[sec:03-04_dotfiles]{Dotfiles} Konzept. Genau wie bei \hyperref[sec:03-03_gitops]{GitOps} gibt es zwei Repositories.

Das \textbf{Environment Repository} enthält die Konfiguration der Container für Integration Environment und Creation Environment. Diese Komponenten werden im kommenden \autoref{subsubsec:05-02-02-02_sub-components-in-environments} genauer beschrieben. Die Basis dieses Repositorys besteht aus zwei aufeinander aufbauenden Dockerfiles, jedes von ihnen für entsprechend eines der Environments. Das Dockerfile des Integration Environments ist minimal und enthält nur die zum \Gls{build} der Applikation erforderlichen Komponenten. Aufbauend auf diesem Dockerfile sind im Creation Environment zusätzlich die für die Entwicklung notwendigen Softwarekomponenten enthalten.

Das \textbf{Application Repository} enthält den Quellcode der Applikation selbst sowie die Konfiguration für ihren Container. Letztere besteht aus einem Dockerfile mit allen für den Betrieb der Applikation erforderlichen Komponenten sowie optional einem weiteren Dockerfile, welches einen Backing Service für den lokalen Betrieb der Applikation bereitstellt. Zur bequemeren Verwaltung mehrerer Services kann bei Bedarf \textit{Docker Compose} mit einer entsprechenden Konfigurationsdatei hinzugezogen werden. Zusätzlich enthält dieses Repository auch die Konfiguration für einen \nameref{subsec:05-01-02_dev-container}, welcher die Entwicklungsumgebung für die Applikation bereitstellt. \linebreak[4]
Dafür werden in einem gesonderten Verzeichnis eine \texttt{devcontainer.json} Datei mit der deklarativen Beschreibung der Entwicklungsumgebung sowie ein weiteres Unterverzeichnis mit möglichen Ressourcen zur Konfiguration der Entwicklungsumgebung abgelegt. Die \texttt{devcontainer.json} Konfiguration kann auch auf einer optionalen \textit{Docker Compose} Konfiguration basieren, welche neben dem \nameref{subsec:05-01-02_dev-container} auch den \nameref{subsec:05-01-01_docker-container} der Applikation einbinden kann.

Nicht notwendig, jedoch nützlich kann das Einbinden eines dritten Repositories sein, welches \hyperref[sec:03-04_dotfiles]{Dotfiles} zur Individualisierung der Entwicklungsumgebung enthält. Es besteht dann aus Konfigurationsdateien sowie einem Installationsdatei zur automatischen Einrichtung. Bei diesem Repository kann es sich um ein zentrales handeln, aus welchem die Konfigurationen für alle Entwickler einheitlich gezogen werden, oder um die persönlichen \hyperref[sec:03-04_dotfiles]{Dotfiles} jedes Entwicklers, welche über ein privates Repository verwaltet werden.

Alle Artefakt-generierenden Repositories, also das Environment Repository und das Application Repository, enthalten zusätzlich eine \Gls{cicd} Konfiguration, welche für \nameref{subsec:05-02-03_workflows-and-continuity-in-the-toolchain-as-code-approach} verantwortlich ist und in \autoref{subsec:05-02-03_workflows-and-continuity-in-the-toolchain-as-code-approach} beschrieben wird.

Eine vollständige Übersicht über die Struktur der Repositories, ihre Dateien und Verzeichnisse sowie die Beziehungen untereinander kann \autoref{fig:g-18_repositories-in-toolchain-as-code-strategy} entnommen werden.

\begin{figure}[htp]
    \centering
    \includegraphics[width=1.00\textwidth]{g-18_repositories-in-toolchain-as-code-strategy.png}
    \caption{Repositories in der Toolchain-as-Code Strategie}
    \label{fig:g-18_repositories-in-toolchain-as-code-strategy}
\end{figure}

\subsubsection{Teilkomponenten in Umgebungen}
\label{subsubsec:05-02-02-02_sub-components-in-environments}

\nameref{sec:02-03_containerization} ist ein wichtiger Bestandteil im Kontext dieser Arbeit und alle Artefakte, welche durch die \Gls{cicd} Konfigurationen der Repositories generiert werden, sind Container. Insgesamt entstehen so vier verschiedene Softwareumgebungen in Form von Container \Glspl{image}, die teilweise aufeinander aufbauen oder voneinander abhängen.

Das \textbf{Integration Environment} basiert auf dem Integration Dockerfile des Environment Repositories (\acrshort{vgl} \autoref{subsubsec:05-02-02-01_data-storage-in-repositories}). Es enthält das Betriebssystem, grundlegende Abhängigkeiten wie \Glspl{sdk} sowie Tools zum \Gls{build} der Applikation. Es stellt also alle Komponenten bereit, die für die Integration der Applikation von Relevanz sind. Erweitert wird das Integration Environment durch das \textbf{Creation Environment}, dessen Dockerfile auf dem Integration Dockerfile aufbaut. Die Erweiterung erfolgt dabei um Komponenten, welche für die Entwicklung der Applikation notwendig sind, also Entwickler bei der Arbeit unterstützen.

Das \textbf{Development Environment} wird durch einen \nameref{subsec:05-01-02_dev-container} auf Grundlage des Creation Environments gebaut. Es stellt das Backend einer ausgestatteten \Gls{ide} bereit, inklusive Plugins und Einstellungen. Es ist außerdem personalisiert durch die \hyperref[sec:03-04_dotfiles]{Dotfiles} des Entwicklungsteams.

Das tatsächliche Produkt befindet sich im \textbf{Application Image}. Dieses basiert auf dem Integration Environment und enthält nur die wichtigsten Komponenten, die absolut notwendig sind, um die Applikation auszuführen. Es handelt sich also um die produktive Umgebung der Applikation. Durch sogenannte Multi-Stage Builds (siehe \nameref{subsec:05-03-01_performance} in \autoref{sec:05-03_best-practices-with-docker-and-docker-compose}) können beispielsweise \Gls{build} Tools nach der Integration entfernt werden, um die Größe des \Glspl{image} zu reduzieren. Das \Gls{image} besteht dann nur noch aus dem Betriebssystem, der Laufzeitumgebung und wichtigen Abhängigkeiten. Außerdem enthält es die ausführbare Version der Applikation und kann theoretisch an den Kunden ausgeliefert werden.

Die genauen Bestandteile der einzelnen Umgebungen sind in \autoref{fig:g-19_environments-in-toolchain-as-code-strategy} dargestellt.

\begin{figure}[htp]
    \centering
    \includegraphics[width=1.00\textwidth]{g-19_environments-in-toolchain-as-code-strategy.png}
    \caption{Umgebungen in der Toolchain-as-Code Strategie}
    \label{fig:g-19_environments-in-toolchain-as-code-strategy}
\end{figure}

\subsection{Abläufe und Kontinuität im Toolchain-as-Code Ansatz}
\label{subsec:05-02-03_workflows-and-continuity-in-the-toolchain-as-code-approach}

\autoref{fig:g-20_ci-cd-concept-of-toolchain-as-code-strategy} zeigt das \nameref{fig:g-20_ci-cd-concept-of-toolchain-as-code-strategy} basierend auf Konzepten aus \hyperref[sec:03-01_devops]{DevOps} und \hyperref[sec:03-03_gitops]{GitOps}. Zur Reduzierung der Komplexität wird ein push-basierter Ansatz (siehe \autoref{fig:g-07_gitops-push-based-deployment}) gewählt. Anders als bei \hyperref[sec:03-03_gitops]{GitOps} liegt der Fokus der \nameref{ch:05_toolchain-as-code} Strategie jedoch mehr auf der Umgebung und auf der Toolchain als auf der Applikation selbst, weshalb die Pipeline der Applikation auf dem \Gls{image} ihrer Umgebung basiert, jedoch nicht umgekehrt. Auch müssen die einzelnen Pipelines sich nicht gegenseitig in einer Kette auslösen, sondern können unabhängig voneinander arbeiten. Wenn überhaupt muss jedoch nur die Pipeline der Umgebung nach Erfolg einen Rebuild der Applikation auslösen, welche auf ihr basiert, während die Applikation immer auf der zuletzt bereitgestellten Umgebung lauffähig sein sollte.

\begin{figure}[h]
    \centering
    \includegraphics[width=1.00\textwidth]{g-20_ci-cd-concept-of-toolchain-as-code-strategy.png}
    \caption{CI/CD Konzept der Toolchain-as-Code Strategie}
    \label{fig:g-20_ci-cd-concept-of-toolchain-as-code-strategy}
\end{figure}

App Container und \nameref{subsec:05-01-02_dev-container} \circled{0} ziehen sich zunächst das Environment Image, je nachdem entweder als Integration Image oder als Creation Image, aus der entsprechenden \Gls{container-registry} für die Umgebung. Die \nameref{subsec:05-01-02_dev-container} Konfiguration erstellt daraufhin die Developmentumgebung und \circled{1} klont das Application Repository. Ein \circled{2} Commit und Push in das Application Repository, unabhängig davon, ob sie aus einem \nameref{subsec:05-01-02_dev-container} oder aus einem lokalen Arbeitsverzeichnis erfolgen, stoßen \circled{3} die Integration, also einen Rebuild der Applikation, sowie \circled{4} das \Gls{deployment}, also das Hochladen des \Glspl{image} in ein entsprechendes \Gls{container-registry}, an. Das \Gls{image} der Applikation könnte von dort aus zusätzlich noch im Rahmen von \Gls{delivery} auf Stages oder für den Kunden zur Verfügung gestellt werden, dies ist jedoch kein expliziter Teil der \nameref{ch:05_toolchain-as-code} Strategie. Analog dazu stoßen \circled{5} Commit und Push in das Environment Repository \circled{6} die Integration und \circled{7} das \Gls{deployment} der Umgebung an.

Nach erfolgreichem Bereitstellen der Umgebung als \Gls{image} könnte als weitere Automatisierung ein Rebuild der Applikation auf Basis der neuen Umgebung angestoßen werden. Änderungen an der Umgebung sind generell kritischer als Änderungen an der Applikation (wie von \texttt{\hyperref[sec:AA-02_interview-persons]{IP-3}} ausgesagt, \acrshort{vgl} \autoref{subsec:AA-03-01_open-questions}), da diese immer beide Images betreffen. Hochladen eines neuen Images der Umgebung könnte daher noch eine manuelle Bestätigung enthalten. 

\subsection{Zusammenfassung der Strategie}
\label{subsec:05-02-04_summary-of-strategy}

Drei Repositories sind für Applikation und Umgebung verantwortlich, zwei von ihnen generieren zusammen drei Artefakte mit jeweils unterschiedlichen Einsatzzielen. Die Artefakte basieren größtenteils aufeinander und sind daher möglichst konsistent.

Die in \autoref{ch:03_examination-of-existing-approaches} beschriebenen Ansätze tragen maßgeblich zur \nameref{ch:05_toolchain-as-code} Strategie bei. \hyperref[sec:03-01_devops]{DevOps} unterstützt bei der Automatisierung der Abläufe mittels \Gls{cicd} und \hyperref[sec:03-03_gitops]{GitOps} zentralisiert Aufgaben, indem es das Repository als einzige Quelle definiert und deklarative Konfigurationen fordert. Personalisierung und Individualisierung der Developmentumgebung erfolgen mittels \hyperref[sec:03-04_dotfiles]{Dotfiles}. Dabei werden die grundlegenden Konzepte der \hyperref[sec:03-05_concept-of-twelve-factor-app]{Twelve-Factor-App} berücksichtigt. Insgesamt erfüllt die Strategie die in \autoref{ch:04_requirements-for-development-and-deployment-environments} ermittelten Anforderungen der Entwickler und verfügt somit über gute Voraussetzungen für eine erfolgreiche Implementierung.

Dabei ist der Umfang der Strategie nicht starr, sondern kann den Bedingungen in der Praxis angepasst werden. So kann das \hyperref[sec:03-04_dotfiles]{Dotfiles} Repository zentral oder individuell sein, aber auch vollständig weggelassen werden. \hyperref[sec:03-01_devops]{DevOps} Pipelines könnten sich, wie in \hyperref[sec:03-03_gitops]{GitOps} vorgegeben, gegenseitig auslösen, sofern sich der zusätzliche Komplexitätsgrad im konkreten Anwendungsfall lohnt. Handelt es sich um eine besonders sicherheitskritische Infrastuktur, könnte ein pull-basierter Deployment Ansatz (siehe \autoref{fig:g-08_gitops-pull-based-deployment}) das \Gls{deployment} stärker absichern. Manuelle Schritte sind in der Strategie vollständig eliminiert, können jedoch optional eingebaut werden, sofern dies aus organisatorischen Gründen erforderlich sein sollte.

Auch die bei der Implementierung der Strategie verwendeten Tools sind nicht vorgeschrieben.

\begin{itemize}
    \item Das konkrete \textbf{\Gls{vcs}} ist frei wählbar.
    \item Die \textbf{Code Hosting Plattform} kann frei gewählt werden.
    \item Ein \textbf{Integrationssystem} kann ebenfalls selbst bestimmt werden.
    \item Über das verwendete \textbf{Containerization Tool} kann frei entschieden werden.
    \item Viele \textbf{Entwicklungsumgebungen} und \textbf{Editoren} funktionieren mit der Strategie.
    \item \textbf{Cloud Development} ist ebenfalls möglich.
    \item Das \textbf{Image Repository} kann frei gewählt werden.
    \item Selbstverständlich können auch die \textbf{Programmiersprachen}, \textbf{\Glspl{sdk}} und \textbf{\Gls{build} Tools} für die Applikation durch das Entwicklungsteam festgelegt werden.
    \item \textbf{Base Images} müssen abhängig von den Anforderungen des Projekts gewählt werden.
\end{itemize}

Einige Tools sind oft sinnvoller als andere und nicht jede Technologie ist für jedes Projekt geeignet. Einen Vorschlag für eine konkrete Technologieauswahl im \hyperref[ch:02_technological-environment]{technologischen Umfeld} dieser Arbeit macht \autoref{sec:06-01_technologies-and-tools} im Rahmen der \hyperref[ch:06_prototypical-implementation-of-the-concept]{prototypischen Umsetzung des Konzepts}.

Dieser \autoref{sec:05-02_strategy-for-toolchains} beantwortet die \acrlong{rq} \textbf{RQ-2} (siehe \autoref{sec:01-03_objectives-and-research-questions}).
