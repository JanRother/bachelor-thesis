\chapter{Quellcode des Prototypen}
\label{ch:BB_source-code-of-prototype}

Der Prototyp befindet sich als quelloffenes Projekt in mehreren Repositories einer \textit{GitHub Organization} und kann über \url{https://github.com/Toolchain-as-Code/} aufgerufen werden. Dabei besteht das Projekt aus drei Hauptkomponenten, abgelegt in je einem dedizierten Repository.

\begin{table}[H]
    \centering
    \begin{tabular}{ >{\bfseries\raggedright}p{0.3\textwidth} >{}p{0.7\textwidth} }
        Application Repository &
        \url{https://github.com/Toolchain-as-Code/tac-application} \\
        Environment Repository &
        \url{https://github.com/Toolchain-as-Code/tac-environment} \\
        Dotfiles Repository &
        \url{https://github.com/Toolchain-as-Code/tac-dotfiles} \\
    \end{tabular}
\end{table}

Die einzelnen Repositories hängen dabei zusammen wie in \autoref{fig:p-00_structure-of-project} dargestellt.

\begin{figure}[H]
    \begin{tikzpicture}[
        grow via three points={one child at (0.5,-0.7) and
        two children at (0.5,-0.7) and (0.5,-1.4)},
        edge from parent path={(\tikzparentnode.south) |- (\tikzchildnode.west)}]

        \node [root] {Toolchain-as-Code}
            child { node [unselected] {\hyperref[sec:BB-01_application-repository]{tac-application}}}
            child { node [unselected] {\hyperref[sec:BB-02_environment-repository]{tac-environment}}}
            child { node [unselected] {\hyperref[sec:BB-03_dotfiles-repository]{tac-dotfiles}}};
    \end{tikzpicture}
    \caption{Struktur des Projekts}
    \label{fig:p-00_structure-of-project}
\end{figure}

In den folgenden Abschnitten ist jedes Repository mit seiner vollständigen Verzeichnisstruktur aufgeführt. Außerdem wird der Quellcode der für die Umsetzung des Prototypen relevantesten Dateien dokumentiert. Irrelevante Zeilen können dabei entfernt oder gekürzt sein.

\section{Application Repository}
\label{sec:BB-01_application-repository}

\begin{figure}[H]
    \begin{tikzpicture}[
        grow via three points={one child at (0.5,-0.7) and
        two children at (0.5,-0.7) and (0.5,-1.4)},
        edge from parent path={(\tikzparentnode.south) |- (\tikzchildnode.west)}]

        \node [root] {Toolchain-as-Code}
            child { node [selected] {\hyperref[sec:BB-01_application-repository]{tac-application}}
                child { node [folder] {.appcontainer}
                    child { node {\hyperref[subsec:BB-01-03_appcontainer-X-dockerfile]{Dockerfile}}}
                }
                child [missing] {}
                child { node [folder] {.devcontainer}
                    child { node {db\_init.sql}}
                    child { node {\hyperref[subsec:BB-01-04_devcontainer-X-devcontainer-json]{devcontainer.json}}}
                    child { node {\hyperref[subsec:BB-01-05_devcontainer-X-dockerfile]{Dockerfile}}}
                }
                child [missing] {}
                child [missing] {}
                child [missing] {}
                child { node [folder] {.github}
                    child { node [folder] {workflows}
                        child { node {\hyperref[subsec:BB-01-06_github-X-workflows-X-application-build-and-push-yaml]{application-build-and-push.yaml}}}
                    }
                }
                child [missing] {}
                child { node [folder] {app}
                    child { node {.dockerignore}}
                    child { node {\hyperref[subsec:BB-01-07_app-X-server-ts]{server.ts}}}
                }
                child [missing] {}
                child [missing] {}
                child { node [ignored] {.env}}
                child { node {.gitattributes}}
                child { node {.gitignore}}
                child { node {\hyperref[subsec:BB-01-02_docker-compose-yaml]{docker-compose.yaml}}}
                child { node {\hyperref[subsec:BB-01-01_example-env]{example.env}}}
                child { node {LICENSE}}
                child { node {README.md}}
            }
            child [missing] {}
            child [missing] {}
            child [missing] {}
            child [missing] {}
            child [missing] {}
            child [missing] {}
            child [missing] {}
            child [missing] {}
            child [missing] {}
            child [missing] {}
            child [missing] {}
            child [missing] {}
            child [missing] {}
            child [missing] {}
            child [missing] {}
            child [missing] {}
            child [missing] {}
            child [missing] {}
            child { node [unselected] {\hyperref[sec:BB-02_environment-repository]{tac-environment}}}
            child { node [unselected] {\hyperref[sec:BB-03_dotfiles-repository]{tac-dotfiles}}};
    \end{tikzpicture}
    \caption{Verzeichnisstruktur des Application Repositories}
    \label{fig:p-01_directory-structure-of-application-repository}
\end{figure}

\vspace{3em}

\subsection{\texttt{example.env}}
\label{subsec:BB-01-01_example-env}

\begin{codebox}{text}{example.env}
PGHOST=db
PGPORT=5432
PGDATABASE=test
PGUSER=root
PGPASSWORD=root
\end{codebox}

\subsection{\texttt{docker-compose.yaml}}
\label{subsec:BB-01-02_docker-compose-yaml}

\begin{codebox}{yaml}{docker-compose.yaml}
    services:
    db:
        image: postgres:17
        restart: unless-stopped
        environment:
            - DATABASE_HOST=${PGHOST}
            - POSTGRES_DB=${PGDATABASE}
            - POSTGRES_USER=${PGUSER}
            - POSTGRES_PASSWORD=${PGPASSWORD}
        ports:
            - "5432:5432"
        volumes: 
            - ./.devcontainer/db_init.sql :/docker-entrypoint-initdb.d/db_init.sql
    app:
        build:
            context: .
            dockerfile: .devcontainer/Dockerfile
            args:
                  # overwrites default BASE_X values from Dockerfile
                - BASE_REPOSITORY=janrother
                - BASE_IMAGE=tac-environment
                - BASE_ENVIRONMENT=creation
                - BASE_RELEASE=latest
        environment:
            - PGHOST=${PGHOST}
            - PGPORT=${PGPORT}
            - PGDATABASE=${PGDATABASE}
            - PGUSER=${PGUSER}
            - PGPASSWORD=${PGPASSWORD}
        ports:
            - "8000:8000"
\end{codebox}

\subsection{\texttt{.appcontainer/Dockerfile}}
\label{subsec:BB-01-03_appcontainer-X-dockerfile}

\begin{codebox}{Dockerfile}{.appcontainer/Dockerfile}
ARG BASE_REPOSITORY=janrother
ARG BASE_IMAGE=tac-environment
ARG BASE_ENVIRONMENT=integration
ARG BASE_RELEASE=latest

FROM ${BASE_REPOSITORY}/${BASE_IMAGE}:${BASE_ENVIRONMENT}-${BASE_RELEASE} AS integration

WORKDIR /app

USER deno

COPY ./app/server.ts .
RUN deno install --entrypoint server.ts
RUN deno cache server.ts

EXPOSE 8000

CMD ["run", "--allow-net", "--allow-read", "--allow-env", "server.ts"]
\end{codebox}

\subsection{\texttt{.devcontainer/devcontainer.json}}
\label{subsec:BB-01-04_devcontainer-X-devcontainer-json}

\begin{codebox}{json}{.devcontainer/devcontainer.json}
{
    "name": "TaC DevContainer",
    "dockerComposeFile": "../docker-compose.yaml",
    "service": "app",
    "remoteUser": "root",
    "workspaceFolder": "/workspace/",

    "onCreateCommand": "",
    "updateContentCommand": "apt update && apt upgrade -y",
    "postCreateCommand": "",
    "postStartCommand": "",
    "postAttachCommand": "",
  
    "customizations": {
    "vscode": {
        "settings": {
            "terminal.integrated.defaultProfile.linux": "zsh",
            "terminal.integrated.profiles.linux": {
                "zsh": {
                "path": "/bin/zsh"
                }
            },
            "editor.wordWrap": "on",
            "editor.tabSize": 4
        },
        "extensions": [
            "denoland.vscode-deno",
            "ms-vscode-remote.remote-containers",
            "github.codespaces",
            "isudox.vscode-jetbrains-keybindings",
            "github.copilot",
            "github.copilot-chat"
            // ...
        ]
    }
  }
}
\end{codebox}

\subsection{\texttt{.devcontainer/Dockerfile}}
\label{subsec:BB-01-05_devcontainer-X-dockerfile}

\begin{codebox}{Dockerfile}{.devcontainer/Dockerfile}
ARG BASE_REPOSITORY=janrother
ARG BASE_IMAGE=tac-environment
ARG BASE_ENVIRONMENT=creation
ARG BASE_RELEASE=latest

FROM ${BASE_REPOSITORY}/${BASE_IMAGE}:${BASE_ENVIRONMENT}-${BASE_RELEASE} AS creation
USER deno

COPY . /workspace/.

WORKDIR /workspace

RUN deno install --entrypoint ./app/server.ts
RUN deno cache ./app/server.ts

EXPOSE 8000

CMD ["run", "--allow-net", "--allow-read", "--allow-env", "./app/server.ts"]
\end{codebox}

\subsection{\texttt{.github/workflows/application-build-and-push.yaml}}
\label{subsec:BB-01-06_github-X-workflows-X-application-build-and-push-yaml}

\begin{codebox}{yaml}{.github/workflows/application-build-and-push.yaml (1/2)}
name: Application

on:
    push:
    branches:
        - main
    tags:
        - v*
    pull_request:
    branches:
        - main

##### NOTE:
##### Most steps are provided as pseudo code with only their names.

jobs:
    build:
    name: Build Application
    runs-on: ubuntu-latest
    permissions:
        contents: read
    steps:
        - name: Checkout Repository
        - name: Create Artifact Directory
        - name: Setup Docker Buildx
        - name: Debug Build Arguments
        - name: Build Image
          id: build
          uses: docker/build-push-action@v6
          with:
              push: false
              context: .
              file: ./.appcontainer/Dockerfile
              tags: ${{ vars.REPOSITORY }}/${{ vars.IMAGE }}:latest
              build-args: |
              # Arguments from GitHub Actions vars.X or as fixed values.
              outputs: type=docker,dest=${{ runner.temp }}/artifacts/application.tar
        - name: Upload Image as Artifact
\end{codebox}

\begin{codebox}{yaml}{.github/workflows/application-build-and-push.yaml [2/2]}
    push:
    name: Push Application
    runs-on: ubuntu-latest
    needs: build
    permissions:
        contents: read
    steps:
        - name: Download Image as Artifact
        - name: Load Image from Artifact
        - name: Login to Docker Hub
          uses: docker/login-action@v3
          with:
              username: ${{ secrets.DOCKER_USERNAME }}
              password: ${{ secrets.DOCKER_TOKEN }}
        - name: Push Image to Docker Hub  
\end{codebox}

\subsection{\texttt{app/server.ts}}
\label{subsec:BB-01-07_app-X-server-ts}

\begin{codebox}{typescript}{app/server.ts}
import postgres from "https://deno.land/x/postgresjs/mod.js";

const sql = postgres();

Deno.serve(async (_req) => {
    const result = await sql`SELECT * FROM bachelor_thesis`;
    return new Response(JSON.stringify(result));
});    
\end{codebox}
     \clearpage
\section{Environment Repository}
\label{sec:BB-02_environment-repository}

\begin{figure}[H]
    \begin{tikzpicture}[
        grow via three points={one child at (0.5,-0.7) and
        two children at (0.5,-0.7) and (0.5,-1.4)},
        edge from parent path={(\tikzparentnode.south) |- (\tikzchildnode.west)}]

        \node [root] {Toolchain-as-Code}
            child { node [unselected] {\hyperref[sec:BB-01_application-repository]{tac-application}}}
            child { node [selected] {\hyperref[sec:BB-02_environment-repository]{tac-environment}}
                child { node [folder] {.envcontainer}
                    child { node {\hyperref[subsec:BB-02-02_envcontainer-X-app-dockerfile]{App.Dockerfile}}}
                    child { node {\hyperref[subsec:BB-02-01_envcontainer-X-base-dockerfile]{Base.Dockerfile}}}
                    child { node {\hyperref[subsec:BB-02-03_envcontainer-X-dev-dockerfile]{Dev.Dockerfile}}}
                }
                child [missing] {}
                child [missing] {}
                child [missing] {}
                child { node [folder] {.github}
                    child { node [folder] {workflows}
                        child { node {\hyperref[subsec:BB-02-04_github-workflows-X-base-environment-build-and-push-yaml]{base-environment-build-and-push.yaml}}}
                        child { node {\hyperref[subsec:BB-02-06_github-workflows-X-creation-environment-build-and-push-yaml]{creation-environment-build-and-push.yaml}}}
                        child { node {\hyperref[subsec:BB-02-05_github-workflows-X-integration-environment-build-and-push-yaml]{integration-environment-build-and-push.yaml}}}
                    }
                }
                child [missing] {}
                child [missing] {}
                child [missing] {}
                child [missing] {}
                child { node {.gitattributes}}
                child { node {.gitignore}}
                child { node {LICENSE}}
                child { node {README.md}}
            }
            child [missing] {}
            child [missing] {}
            child [missing] {}
            child [missing] {}
            child [missing] {}
            child [missing] {}
            child [missing] {}
            child [missing] {}
            child [missing] {}
            child [missing] {}
            child [missing] {}
            child [missing] {}
            child [missing] {}
            child { node [unselected] {\hyperref[sec:BB-03_dotfiles-repository]{tac-dotfiles}}};
    \end{tikzpicture}
    \caption{Verzeichnisstruktur des Environment Repositories}
    \label{fig:p-02_directory-structure-of-environment-repository}
\end{figure}

\vspace{3em}

\subsection{\texttt{.envcontainer/Base.Dockerfile}}
\label{subsec:BB-02-01_envcontainer-X-base-dockerfile}

\begin{codebox}{Dockerfile}{.envcontainer/Base.Dockerfile}
ARG DENO_VERSION=2.1.0

ARG BASE_REPOSITORY=denoland
ARG BASE_IMAGE=deno
ARG BASE_ENVIRONMENT=debian
ARG BASE_RELEASE=${DENO_VERSION}

##### NOTE:
##### Because of a current issue with the Base Image of Deno
##### for the Alpine Linux, which seems to be caused by some
##### incompatibilities with the musl libc, the DevContainer
##### has to be based on Debian Linux for now.

FROM ${BASE_REPOSITORY}/${BASE_IMAGE}:${BASE_ENVIRONMENT}-${BASE_RELEASE} AS builder

ARG AUTHOR="Jan Rother"
ARG DATE="2024-12"
ARG VERSION="1.0"

LABEL org.opencontainers.image.author=$AUTHOR \
      org.opencontainers.image.date=$DATE \
      org.opencontainers.image.version=$VERSION \
      org.opencontainers.image.title="Deno Application" \
      org.opencontainers.image.description="Docker Image"
\end{codebox}

\subsection{\texttt{.envcontainer/App.Dockerfile}}
\label{subsec:BB-02-02_envcontainer-X-app-dockerfile}

\begin{codebox}{Dockerfile}{.envcontainer/App.Dockerfile}
ARG BASE_REPOSITORY=janrother
ARG BASE_IMAGE=tac-environment
ARG BASE_ENVIRONMENT=base
ARG BASE_RELEASE=latest

FROM ${BASE_REPOSITORY}/${BASE_IMAGE}:${BASE_ENVIRONMENT}-${BASE_RELEASE} AS base
\end{codebox}

\subsection{\texttt{.envcontainer/Dev.Dockerfile}}
\label{subsec:BB-02-03_envcontainer-X-dev-dockerfile}

\begin{codebox}{Dockerfile}{.envcontainer/Dev.Dockerfile}
ARG BASE_REPOSITORY=docker.io/library
ARG BASE_IMAGE=tac-environment
ARG BASE_ENVIRONMENT=base
ARG BASE_RELEASE=latest

FROM ${BASE_REPOSITORY}/${BASE_IMAGE}:${BASE_ENVIRONMENT}-${BASE_RELEASE} AS base

RUN apt update && apt update && apt install -y \
    curl \
    git \
    postgresql-client \
    unzip \
    zip \
    && apt clean

# Dotfiles
RUN bash -c  \
    "$(curl -#fL https://raw.githubusercontent.com/ Toolchain-as-Code/tac-dotfiles/refs/heads/main/install.sh)"

# Terminal: ZSH
RUN apt update && \
    apt install -y zsh fzf && \
    chsh -s /usr/bin/zsh && \
    apt clean

# Terminal: Starship Prompt
RUN curl -sS https://starship.rs/install.sh | sh -s -- -y -v latest
# Terminal: Zinit
RUN curl -sS https://raw.githubusercontent.com/ zdharma-continuum/zinit/HEAD/scripts/install.sh | sh
# Terminal: Zinit Plugins
RUN zsh -il
\end{codebox}

\subsection[\texttt{.github/workflows/base-environment-build-and-push.yaml}]{\texttt{.github/workflows/base-environment-...-push.yaml}}
\label{subsec:BB-02-04_github-workflows-X-base-environment-build-and-push-yaml}

\begin{codebox}{yaml}{.github/workflows/base-environment-build-and-push.yaml}
name: Base Environment

on:
    push:
        branches:
            - main
        tags:
            - v*

##### NOTE:
##### Most steps are provided as pseudo code with only their names.

jobs:
    build:
        name: Build Base Environment
        runs-on: ubuntu-latest
        permissions:
            contents: read
        steps:
            -   name: Checkout Repository
            -   name: Create Artifact Directory
            -   name: Setup Docker Buildx
            -   name: Debug Build Arguments
            -   name: Build Image
            -   name: Upload Image as Artifact
    
    push:
        name: Push Base Environment
        runs-on: ubuntu-latest
        needs: build
        permissions:
            contents: read
        steps:
            -   name: Download Image as Artifact
            -   name: Load Image from Artifact
            -   name: Login to Docker Hub
            -   name: Push Image to Docker Hub
\end{codebox}

\subsection[\texttt{.github/workflows/integration-environment-build-and-push.yaml}]{\texttt{.github/workflows/integration-environment-...-push.yaml}}
\label{subsec:BB-02-05_github-workflows-X-integration-environment-build-and-push-yaml}

\begin{codebox}{yaml}{.github/workflows/integration-environment-build-and-push.yaml}
name: Creation Environment

on:
    workflow_run:
        workflows: [ "Base Environment" ]
        types:
            - completed

##### NOTE:
##### Most steps are provided as pseudo code with only their names.

jobs:
    build:
        name: Build Creation Environment
        runs-on: ubuntu-latest
        if: ${{ github.event.workflow_run.conclusion == 'success' }}
        permissions:
            contents: read
        steps:
            -   name: Checkout Repository
            -   name: Setup Docker Buildx
            -   name: Create Artifact Directory
            -   name: Debug Build Arguments
            -   name: Build Image
            -   name: Upload Image as Artifact
    
    push:
        name: Push Creation Environment
        runs-on: ubuntu-latest
        needs: build
        permissions:
            contents: read
        steps:
            -   name: Download Image as Artifact
            -   name: Load Image from Artifact
            -   name: Login to Docker Hub
                uses: docker/login-action@v3
            -   name: Push Image to Docker Hub
\end{codebox}

\subsection[\texttt{.github/workflows/creation-environment-build-and-push.yaml}]{\texttt{.github/workflows/creation-environment-...-push.yaml}}
\label{subsec:BB-02-06_github-workflows-X-creation-environment-build-and-push-yaml}

\begin{codebox}{yaml}{.github/workflows/creation-environment-build-and-push.yaml}
name: Integration Environment

on:
    workflow_run:
        workflows: [ "Base Environment" ]
        types:
            - completed

##### NOTE:
##### Most steps are provided as pseudo code with only their names.

jobs:
    build:
        name: Build Integration Environment
        runs-on: ubuntu-latest
        if: ${{ github.event.workflow_run.conclusion == 'success' }}
        permissions:
            contents: read
        steps:
            -   name: Checkout Repository
            -   name: Setup Docker Buildx
            -   name: Create Artifact Directory
            -   name: Debug Build Arguments
            -   name: Build Image
            -   name: Upload Image as Artifact
    
    push:
        name: Push Integration Environment
        runs-on: ubuntu-latest
        needs: build
        permissions:
            contents: read
        steps:
            -   name: Download Image as Artifact
            -   name: Load Image from Artifact
            -   name: Login to Docker Hub
            -   name: Push Image to Docker Hub
\end{codebox}
     \clearpage
\section{Dotfiles Repository}
\label{sec:BB-03_dotfiles-repository}

\begin{figure}[H]
    \begin{tikzpicture}[
        grow via three points={one child at (0.5,-0.7) and
        two children at (0.5,-0.7) and (0.5,-1.4)},
        edge from parent path={(\tikzparentnode.south) |- (\tikzchildnode.west)}]
        
        \node [root] {Toolchain-as-Code}
            child { node [unselected] {\hyperref[sec:BB-01_application-repository]{tac-application}}}
            child { node [unselected] {\hyperref[sec:BB-02_environment-repository]{tac-environment}}}
            child { node [selected] {\hyperref[sec:BB-03_dotfiles-repository]{tac-dotfiles}}
                child { node [folder] {configs}
                    child { node {.zshrc}}
                    child { node [folder] {.config}
                        child { node {starship.toml}}
                    }
                }
                child [missing] {}
                child [missing] {}
                child [missing] {}
                child { node {.gitattributes}}
                child { node {.gitignore}}
                child { node {\hyperref[subsec:BB-03-01_install-sh]{install.sh}}}
                child { node {LICENSE}}
                child { node {README.md}}
            };
    \end{tikzpicture}
    \caption{Verzeichnisstruktur des Dotfiles Repositories}
    \label{fig:p-03_directory-structure-of-dotfiles-repository}
\end{figure}

\vspace{3em}

\subsection{\texttt{install.sh}}
\label{subsec:BB-03-01_install-sh}

\begin{codebox}{Bash}{install.sh (1/3)}
#!/usr/bin/env bash
set -e

GITHUB_USER=toolchain-as-code
GITHUB_REPO=tac-dotfiles

DOTFILES_ARCHIVE=/tmp/${GITHUB_REPO}.tar.gz
DOTFILES_DIR=/tmp/${GITHUB_REPO}
TARGET_DIR=${HOME}
\end{codebox}

\begin{codebox}{Bash}{install.sh (2/3)}
get_dotfiles() {
    echo "      - Creating temporary directory:"
    mkdir -p "${DOTFILES_DIR}" || { exit 1 }
    echo "      - Downloading dotfiles."
    curl -#fLo "${DOTFILES_ARCHIVE}" "https://github.com/${GITHUB_USER}/${GITHUB_REPO}/tarball/main" || { exit 1 }
    echo "      - Extracting dotfiles."
    tar -xzf "${DOTFILES_ARCHIVE}" --strip-components 1 -C "${DOTFILES_DIR}" || { exit 1 }
    echo "      - Removing dotfiles archive."
    rm -f "${DOTFILES_ARCHIVE}" || { exit 1 }
}

install_dotfiles() {
    echo "      - Changing to dotfiles directory."
    cd "${DOTFILES_DIR}/configs" || { exit 1 }

    echo "      - Installing dotfiles:"
    find . -type f | while IFS= read -r file; do
    target="${TARGET_DIR}/${file#./}"
    mkdir -p "$(dirname "${target}")" || { exit 1 }
    if cp "${file}" "${target}" 2>/dev/null; then
        echo "          '${file}' -> '${target}'"
    else
        exit 1
    fi
    done
}

remove_temporary_files() {
    echo "      - Removing temporary files:"
    rm -rf ${DOTFILES_DIR}
    echo "          ${DOTFILES_DIR}"
    echo "      - Changing back to the original directory."
    cd -
}
\end{codebox}

\begin{codebox}{Bash}{install.sh (3/3)}
install() {
    echo " "
    echo "+++++++ INSTALLING DOTFILES +++++++"
    echo "from ${GITHUB_USER}/${GITHUB_REPO}"
    echo "-----------------------------------"
    echo " "

    echo "(0/3) GETTING DOTFILES"
    get_dotfiles
    echo "(1/3) INSTALLING DOTFILES"
    install_dotfiles
    echo "(2/3) CLEANING UP"
    remove_temporary_files
    echo "(3/3) DONE"
    echo " "
}

install
\end{codebox}
        \clearpage
