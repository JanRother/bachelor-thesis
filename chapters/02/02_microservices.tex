\section{Microservices}
\label{sec:02-02_microservices}

Der Begriff Microservices beschreibt einen Architekturstil in der Softwareentwicklung. Ein System folgt dieser Architektur, wenn es sich aus mehreren Komponenten zusammensetzt, welche als Services bezeichnet werden. Jeder dieser Services ist verantwortlich für genau eine Aufgabe oder einen kleinen Funktionsblock. Solche Systeme nutzen leichtgewichtige Kommunikationsprotokolle zum Austausch von Daten untereinander. \cite{027:Containerized-Microservices-Deployment-Approach} Alle Services sind unabhängig voneinander. Dies hat Auswirkungen auf \Gls{development}, \Gls{deployment} und die Skalierbarkeit von Systemen. \cite{028:Analyzing-Microservices-and-Monolithic-Systems}

Das bereits in \autoref{ch:01_introduction-and-motivation} angesprochene Softwareunternehmen \textit{JetBrains} entwickelt Tools für Softwareentwickler weltweit. Zum Zeitpunkt dieser Arbeit verzeichnet es mehr als 11 Millionen Nutzer ihrer \Glspl{ide}, unter ihnen 287 Tausend Geschäftskunden. Entwickelt werden Produkte von \textit{JetBrains} durch über zweitausend Mitarbeitende an 13 internationalen Standorten. \cite{210:Jet-Brains-Company} Die \textit{JetBrains State of Developer Ecosystem} Umfrage erreichte im Jahr 2023 knapp über 23 Tausend Entwickler. Für den Bereich Microservices ergab sich, dass ein Drittel der Teilnehmenden an ihnen beteiligt sind. Fast alle von ihnen gaben an, auch einen entsprechenden Designansatz zu verwenden. Drei Viertel der Befragten organisieren die Kommunikation der einzelnen Services über \textit{\Gls{rest}} oder \textit{\Gls{rpc}}. Insgesamt greifen nur 20 \% der Teilnehmenden noch auf ein monolithisches Backend zurück. \cite{207:Developer-Ecosystem}

Bei monolithischen Architekturen befinden sich alle Funktionalitäten einer Applikation in einer einzelnen \Gls{codebase}. Eine \Gls{codebase} \glsdesc{codebase}. Das wichtigste Merkmal von Monolithen ist die enge Kopplung ihrer einzelnen Systemkomponenten. Eine steil anwachsende Komplexität solcher Systeme wird schnell zu einer Herausforderung und entschleunigt die Evolution einer Applikation. \cite{028:Analyzing-Microservices-and-Monolithic-Systems} Monolithen führen zwangsläufig zu einem sehr aufwändigen \Gls{deployment} und sind aufgrund ihrer Größe oft unflexibel. Diese Nachteile motivieren den Einsatz von Microservices als Alternative. Sie ermöglichen beispielsweise einen deutlich effizienteren Einsatz von Pipelines für \Gls{cicd}. \cite{019:Advanced-DevOps-Environment-for-Microservices-based-Applications} Weitere Vorteile von Microservices sind eine schnellere Bereitstellung von Software sowie eine deutlich stärkere Autonomie der Komponenten \cite{019:Advanced-DevOps-Environment-for-Microservices-based-Applications,027:Containerized-Microservices-Deployment-Approach}. Sie folgen einem Domain-Driven Design \cite{019:Advanced-DevOps-Environment-for-Microservices-based-Applications} und sind vor allem skalierbarer als monolithische Ansätze \cite{019:Advanced-DevOps-Environment-for-Microservices-based-Applications,027:Containerized-Microservices-Deployment-Approach,028:Analyzing-Microservices-and-Monolithic-Systems}. Die genannten Faktoren haben einen positiven Einfluss auf die Wartbarkeit von Microservices. Das bedeutet jedoch nicht, dass bei solchen verteilten Systemen keine zusätzliche Komplexität entsteht. Auch bei diesem Ansatz kann sie schnell zu einem operationalen Mehraufwand führen. Daher erfordert eine erfolgreiche Implementierung von Microservices Architekturen gute Softwareteams, die in der Lage sind, diese Komplexität zu managen.

Wirft man den Blick auf große Unternehmen, so fällt auf, dass einige weiterhin monolithische Ansätze verfolgen. In Bezug auf Datendurchsatz und Mehrläufigkeit werden Microservices mitunter von Monolithen übertroffen. Unter anderem \textit{Stack Overflow} oder \textit{Amazon} machen sich diese bessere Performanz zunutze. Dennoch vertrauen viele Konzerne, unter ihnen \textit{Amazon}, \textit{Netflix} und \textit{Spotify} auf Microservices und etablieren diese Architekturen für ihre Software. \cite{028:Analyzing-Microservices-and-Monolithic-Systems}

Noch besser einsetzbar sind Microservices in Verbindung mit \nameref{sec:02-03_containerization}.
