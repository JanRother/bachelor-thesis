\section{Containerization}
\label{sec:02-03_containerization}

\qit{Container sind leichtgewichtige, isolierte Umgebungen, welche eine Applikation inklusive ihrer Abhängigkeiten paketieren und es ihr dadurch ermöglichen, konsistent einsetzbar zu sein, auch über verschiedene Umgebungen hinweg.} \cite{023:Setting-up-CI-CD-Pipeline-in-the-Cloud-for-Web-Application}

Historisch war ein Grund für die Entwicklung solcher Technologien ein Anstieg an Komplexität in den Ökosystemen von Softwareentwicklern. Zunächst wurden virtuelle Maschinen genutzt, um dieses Problem anzugehen, jedoch bieten Container Technologien eine deutlich schlankere Alternative \cite{014:Managing-Container-based-Software-Development-Environments}. Mittlerweile finden sie daher eine starke Verwendung in \nameref{sec:02-02_microservices} Architekturen \cite{014:Managing-Container-based-Software-Development-Environments} und Cloudumgebungen \cite{025:Exploring-Solutions-for-Container-Image-Security}.

Die \textit{JetBrains State of Developer Ecosystem} deckte 2023 auch den Bereich \hyperref[sec:03-01_devops]{DevOps} ab und fand heraus, dass mit 54 \% mehr als die Hälfte der teilnehmenden Entwickler \textit{Docker} in der Softwareentwicklung nutzen \cite{207:Developer-Ecosystem}. \textit{Docker} ist eine Software, die Containerization ermöglicht und sich in diesem Bereich weitestgehend durchgesetzt hat. Knapp ein Drittel, nämlich 60 \% der Entwickler, nutzen \textit{Docker Compose}, 19 \% legen ihre damit gebauten Artefakte in \textit{Docker Hub} ab \cite{207:Developer-Ecosystem}. Dabei handelt es sich um eine \Gls{container-registry}, dies \glsdesc{container-registry}. Mit 18 \% auch häufig genutzt ist die \textit{GitHub Container Registry}.

Auch in der Literatur zeigt sich ein deutlicher Trend in Richtung von \textit{Docker}. Eine systematische Übersichtsarbeit zu Containern in der Softwareentwicklung von \citeauthor{015:Containers-in-Software-Development} lieferte ähnliche Ergebnisse. 57 \% der dort identifizierten Paper thematisieren \textit{Docker}. Die Motivation hinter dieser Studie war kontinuierliche Softwareentwicklung. \cite{015:Containers-in-Software-Development}

\textit{Docker} selbst beschreibt in diesem Rahmen einen \q{Container-First-Approach}, bei welchem Container jeden einzelnen Kontaktpunkt mit Software beeinflussen: Infrastruktur, Anwendungen, \hyperref[sec:03-01_devops]{DevOps}, Sicherheit, Plattform und viele mehr werden genannt. Zentrale Vorteile dieses Ansatzes sind Standardisierung, Isolation, Wiederholbarkeit und Konsistenz von Softwareentwicklung. Entwicklungsumgebungen sind schneller einsatzbereit und Entwicklern bleibt mehr Zeit für Innovation. Dies hat eine Einsparung von Zeitressourcen für Wartung und Infrastruktur zur Folge. \cite{016:Effectively-managing-all-of-those-Applications} Die Leichtgewichtigkeit von Containern, nicht nur im \Gls{development}, sondern auch im \Gls{deployment} machen sie prädestiniert für den Einsatz in Cloudumgebungen \cite{015:Containers-in-Software-Development,024:Investiugating-Impact-of-Containerization-on-Deployment-Process-in-DevOps,025:Exploring-Solutions-for-Container-Image-Security}. Sie ermöglichen reproduzierbare Umgebungen für Software über alle Ebenen hinweg \cite{013:Role-of-Containers-in-Reproducibility,024:Investiugating-Impact-of-Containerization-on-Deployment-Process-in-DevOps}. Container besitzen die Fähigkeit, konsistente Produktiv- und Testumgebungen bereitzustellen, sie sind modular, portabel und sie ermöglichen Automatisierung und Abstraktion \cite{014:Managing-Container-based-Software-Development-Environments}.

Die Summe aller genannten Vorteile überwiegt Nachteilen wie der Herausforderung, zunächst in die benötigten Kompetenzen zu investieren und überzeugt daher auch Unternehmen und Organisationen, wo Containerization immer stärker Einzug halten wird. Noch 2020 lag der Anteil an Containerization in Produktivumgebungen bei etwa 30 \%, bis 2025 prognostizieren \citeauthor{020:Assessing-and-Improving-Quality-of-Docker-Artifacts} einen Anstieg auf 85 \% \cite{020:Assessing-and-Improving-Quality-of-Docker-Artifacts}. Nachteile und Risiken dieser Technologie sowie der richtige Umgang mit ihnen werden in \autoref{sec:05-03_best-practices-with-docker-and-docker-compose} näher beleuchtet.
