\section{Web-Development}
\label{sec:02-01_web-development}

Einen wichtigen Fokus dieser Arbeit stellt das Web-Development dar, auf welches sich die \nameref{ch:05_toolchain-as-code} Strategie fokussieren wird. Grund dafür ist unter anderem der eingangs umrissene \nameref{sec:01-02_project-context}.

Web-Development bezeichnet das Erstellen, das Aufbauen und die Pflege von Webanwendungen. Typische Artefakte aus diesem Feld sind Applikationen, die über das Internet kommunizieren. Sehr häufig entstehen Webseiten \cite{209:Web-Development}. Viele Unternehmen setzen mittlerweile auf diese Webanwendungen, deren zentraler Vorteil der Entfall jeglicher kundenseitiger Installationsbedarfe ist. Bereits ein Webbrowser reicht aus, auf sie zugreifen zu können \cite{002:Optimizing-Cloud-Applications-with-DevOps}. Web-Development ist neben dem \textit{World Wide Web}, \textit{Web Browsern}, \textit{Web Servern} und \textit{Web Pages} einer von fünf Bereichen der sogenannten Web-Technologien, welche einen dem Web-Development übergeordneten Begriff bezeichnen. In diesem Feld ermöglichen Tools und Technologien die Kommunikation zwischen Geräten verschiedener Art über das Internet. Unterschieden wird dabei meist zwischen zwei Tätigkeits- und Technologiefeldern: Das \textbf{Frontend-Development} ist für die Entwicklung der Client Side einer Applikation zuständig. Umgesetzt wird dies häufig über Programmier- und Auszeichnungssprachen wie \textit{\Gls{html}}, \textit{\Gls{css}}, \textit{JavaScript} beziehungsweise \textit{TypeScript}, oft im Verbund mit Paketmanagern wie \textit{npm} oder \textit{yarn}, \textit{JavaScript} Frameworks wie \textit{Angular JS} oder \textit{Vue.js} und weiteren \textit{JavaScript} Bibliotheken. Das \textbf{Backend-Development} hingegen ist für die Server Side einer Applikation zuständig. Typische Programmiersprachen sind hier \textit{Java} oder \textit{C\#} und Frameworks wie \textit{Spring} oder \textit{.NET} \cite{208:Web-Technology}.

Die \textit{Stack Overflow Developer Survey 2024} liefert hilfreiche Erkenntnisse zu Trends bei Technologien. Es zeigt sich, dass sich unter den 65 Tausend teilnehmenden Entwicklern Web-Technologien wie \textit{JavaScript} mit 62 \%, \textit{TypeScript} mit 38 \% sowie \textit{\Gls{html}} und \textit{\Gls{css}} mit 52 \% unter den fünf meistgenannten und -genutzten Technologien insgesamt befinden. Unter professionellen Entwicklern sind die einzelnen Relativwerte sogar noch größer, wenn auch nur im einstelligen Prozentbereich \cite{206:Developer-Survey-2024}. Wird in diesem Kontext von \Gls{development} gesprochen, so ist der Aufbau einer kompletten Applikation von Grund auf gemeint.

Ein weiterer wichtiger Trend bei den Web-Technologien ist \Gls{cloud-native}. \Gls{cloud-native} beschreibt eine Menge von Technologien, die es Organisationen ermöglichen, skalierbare Applikationen in neuen und dynamischen Umgebungen zu bauen und zu betreuen. Diese Umgebungen sind häufig öffentliche, private oder hybride Clouds. Beispiele für \gls{cloud-native} Technologien sind \nameref{sec:02-02_microservices} und \nameref{sec:02-03_containerization}, auf welche im Folgenden näher eingegangen wird.
