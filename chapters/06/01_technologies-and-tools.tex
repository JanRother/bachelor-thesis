\section{Technologien und Werkzeuge}
\label{sec:06-01_technologies-and-tools}

In der \nameref{subsec:05-02-04_summary-of-strategy} wurden Freiheitsgrade bei der Auswahl der konkreten Technologien und Werkzeuge für eine Implementierung genannt. In diesem Kapitel soll ein prototypischer \Gls{poc} entwickelt werden, der von möglichst vielen Entwicklern verstanden und eventuell sogar als Projektvorlage genutzt werden kann. Die konkrete Technologieauswahl soll daher möglichst verbreitete Tools enthalten, welche gut miteinander interagieren beziehungsweise eventuell sogar vom gleichen Herausgeber stammen.

\textit{\Gls{git}} ist Namensgeber für \hyperref[sec:03-03_gitops-as-further-evolution]{GitOps} und außerdem der Standard unter den Versionsverwaltungssystemen in der Softwareentwicklung. Fast 90 \% der Entwickler nutzen es regulär \cite{207:Developer-Ecosystem} und sind vertraut mit dem Tool. Daher wird im Prototypen \textit{\Gls{git}} als \textbf{\Gls{vcs}} verwendet. Eine naheliegende und bekannte \textbf{\textit{Code Hosting Plattform}} ist \textit{GitHub}. Die Plattform bietet vollständige Unterstützung von \textit{Git} Aktionen und dient als Ablageort für zentrale Remote Repositories. Tools wie \textit{GitHub Desktop} stehen zusätzlich zur Verfügung und vereinfachen die Arbeit mit \textit{Git}. \cite{301:About-GitHub-and-Git} \textit{GitHub} ist genauso verbreitet wie \textit{Git} selbst, 87 \% der Entwickler nutzen es. Sowohl im privaten als auch im unternehmerischen Kontext liegt es damit vor Alternativen wie \textit{GitLab} oder \textit{Bitbucket} \cite{207:Developer-Ecosystem}. Ebenfalls von \textit{GitHub} stammt das \textbf{Integrationssystem} \textit{GitHub Actions}, welches eine Plattform für \Gls{cicd} ist und die Automatisierung von \Gls{build}, Test und \Gls{deployment} Pipelines ermöglicht \cite{302:Understanding-GitHub-Actions}. Etwa die Hälfte der Entwickler nutzen \textit{GitHub Actions}, wodurch es auf Platz Zwei der meist verwendeten solcher Systeme liegt \cite{207:Developer-Ecosystem}. Workflows in \textit{GitHub Actions} können Benachrichtigungen an Entwickler senden, beispielsweise bei Erfolg oder Fehlschlagen einer Pipeline \cite{307:Notifications-for-Workflow-Runs}. Ein wichtiger Vorteil für die Wahl von \textit{GitHub Actions} ist die direkte Integration in \textit{GitHub}, welches bereits als Bestandteil der Toolchain ausgewählt wurde.

In der Literatur zu \nameref{sec:02-03_containerization} kommt \textit{Docker} am häufigsten vor und ist dort mit über 50 \% das mit Abstand verbreitetste \textbf{Containerization Tool} \cite{015:Containers-in-Software-Development}. Auch in der Praxis ist \textit{Docker} sehr verbreitet, etwa 54 \% der Entwickler nutzen es \cite{206:Developer-Survey-2024,207:Developer-Ecosystem}. Für den Betrieb von Containern nutzen etwa 60 \% \textit{Docker Compose} und 44 \% \textit{Docker Run} \cite{207:Developer-Ecosystem}. Daher wird auch der Prototyp \textit{Docker} verwenden. \nameref{subsec:05-01-02_dev-container} werden aktuell laut offizieller Spezifikation nur von \textit{Microsoft Visual Studio Code}, \textit{Microsoft Visual Studio} sowie \textit{JetBrains IntelliJ IDEA} unterstützt \cite{306:Development-Containers}, wobei die Integration in \textit{Visual Studio Code} am ausgereiftesten ist \cite{204:Development-Containers-Simplified}, wohingegen sich das Feature bei \textit{JetBrains} noch in der Entwicklungsphase befindet \cite{306:Development-Containers}. Außerdem ist \textit{Visual Studio Code} unter Entwicklern dreifach so verbreitet wie \textit{Visual Studio} oder \textit{IntelliJ IDEA} und etwa 74 \% von ihnen entwickeln ihren Code in dieser \textbf{\Gls{ide}}. Auch im Remote Development liegt \textit{Visual Studio Code} vorne. Während etwa 40 \% der Entwickler mit dem Editor auf entfernte Maschinen zugreifen, nutzen nur 23 \% \Glspl{ide} von \textit{JetBrains} dazu \cite{207:Developer-Ecosystem}.

\textbf{Cloud Development} war in der \nameref{subsec:05-02-04_summary-of-strategy} ebenfalls ein möglicher Freiheitsgrad. \nameref{subsec:05-01-02_dev-container} Konfigurationen für \textit{Visual Studio Code} werden auch von \textit{GitHub Codespaces} akzeptiert \cite{306:Development-Containers}, welches auf \textit{Visual Studio Code} als Basis setzt. Bei einem \textit{Codespace} handelt es sich um eine Developmentumgebung in der Cloud, die wiederholbar durch Konfigurationsdateien im \Gls{vcs} erstellt werden kann \cite{310:GitHub-Codespaces-Overview}. Technisch betrachtet, betreibt \textit{GitHub.com} dabei verschiedene Rechner auf Basis virtueller Maschinen, die in unterschiedlichen Leistungsstufen zwischen zwei und 32 Prozessorkernen genutzt werden können \cite{306:Development-Containers}. Die gesamte Entwicklungsumgebung läuft dabei in einem Webbrowser \cite{310:GitHub-Codespaces-Overview}. Solche browserbasierten \Glspl{ide} stellen eine moderne Alternative zu lokalen Entwicklungsumgebungen dar \cite{004:Continous-Integration-and-Development-Tool-Setup-and-Pipeline-Evolution}. Neben dem Browser können auch einige Desktopanwendungen sich mit einem \textit{Codespace} verbinden \cite{310:GitHub-Codespaces-Overview}. Etwa 42 \% der Entwickler nutzen \textit{GitHub Codespaces} für Cloud Development \cite{207:Developer-Ecosystem}. Die verbreitetsten \textbf{Image Repositories} sind \textit{Docker Hub} und \textit{GitHub Container Registry}, wobei \textit{Docker Hub} mit insgesamt 19 \% vorne liegt \cite{207:Developer-Ecosystem}. Da \textit{Docker} bereits in die Toolchain integriert wurde, wird auch \textit{Docker Hub} als Image Repository verwendet. Genau wie \textit{Docker} stammt es von der \textit{Docker Inc}.

Der Prototyp wird einen \hyperref[sec:02-02_microservices]{Microservice} mit einer trivialen Schnittstelle bereitstellen. Dafür werden eine Programmiersprache aus dem Bereich \nameref{sec:02-01_web-development}, eine Laufzeitumgebung sowie eine Datenbank als Backing Service benötigt. Zum Zeitpunkt dieser Arbeit ist \textit{JavaScript} die am meisten verwendete Programmiersprache. Etwa zwei Drittel entwickeln Software mit \textit{JavaScript}, etwa die Hälfte weniger nutzen \textit{TypeScript} \cite{206:Developer-Survey-2024,207:Developer-Ecosystem}. Allerdings könnten sich fast 50 \% der Nutzer von \textit{JavaScript} vorstellen, zu \textit{TypeScript} zu wechseln \cite{206:Developer-Survey-2024}, weshalb die \textbf{Programmiersprache} \textit{TypeScript} für den Prototypen ausgewählt wird. Als \textbf{Laufzeitumgebung} soll \textit{Deno 2.0} verwendet werden. Es handelt sich dabei um eine relativ junge Laufzeitumgebung für \textit{JavaScript} und moderne Webanwendungen, die quelloffen ist und aktuell über 200 Tausend aktive Nutzer hat. \textit{Deno} kommt mit integrierten Features wie Code Linter, Code Formatter und Test Runner. Von Haus aus kann \textit{JavaScript} sogar zu ausführbaren Anwendungen kompiliert werden. Viele moderne Sicherheitsfeatures werden ebenfalls mitgeliefert. Ein wichtiger Vorteil von \textit{Deno 2.0} ist seine Abwärtskompatibilität zu \textit{Node.js}, gegenüber welchem es jedoch deutlich performanter ist. \cite{309:Deno} Als \textbf{Datenbank} wird \textit{PostgreSQL} verwendet. Es hat \textit{MySQL} mittlerweile bei den beliebtesten Datenbanken überholt und wird von fast der Hälfte der Entwickler genutzt. \cite{206:Developer-Survey-2024}

Zusammenfassend wird für den \Gls{poc} die folgende Toolchain verwendet:

\begin{itemize}
    \item \textit{\Gls{git}} als \textbf{\Gls{vcs}} mit \textit{GitHub} als \textbf{\textit{Code Hosting Plattform}},
    \item \textit{GitHub Actions} als \textbf{\textit{Integration System}},
    \item \textit{Docker} als \textbf{\textit{Containerization Tool}},
    \item \textit{Visual Studio Code} als \textbf{\Gls{ide}} und \textit{GitHub Codespaces} für \textbf{\textit{Cloud Development}},
    \item \textit{Docker Hub} als \textbf{\textit{Image Repository}},
    \item \textit{TypeScript} als \textbf{\textit{Programmiersprache}},
    \item \textit{Deno 2.0} als \textbf{\textit{Laufzeitumgebung}}, sowie
    \item \textit{PostgreSQL} als \textbf{\textit{Datenbank}}.
\end{itemize}
