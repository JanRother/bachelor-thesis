\section{Praxiseinsatz des Toolchain-as-Code Prototypen}
\label{sec:06-03_practical-use-of-the-toolchain-as-code-prototype}

\subsection{Toolchain-as-Code im Development}
\label{subsec:06-03-01_toolchain-as-code-in-development}

Durch eine abgestimmte Konfiguration kann die entwickelte Toolchain vollständig für das \Gls{development} an der Applikation eingesetzt werden. Dieser Abschnitt wird einen Einblick in die Verwendung der \nameref{ch:05_toolchain-as-code} Lösung im Bereich der Developmentumgebungen geben.

\begin{wrapfigure}{l}{0.7\textwidth}
    \vspace{-10pt}
    \centering
    \includegraphics[width=0.69\textwidth]{s-02_run-dotfiles-install-script.png}
    \caption{Ausführen des Installationsskripts für Dotfiles}
    \label{fig:s-02_run-dotfiles-install-script}
    \vspace{-05pt}
\end{wrapfigure}

Die Einrichtung einer Entwicklungsumgebung erfolgt dank der \hyperref[sec:03-04_dotfiles]{Dotfiles} vollkommen automatisiert (siehe \autoref{fig:s-02_run-dotfiles-install-script}) schon bei der Erstellung des Creation Environments wie in \autoref{subsec:06-02-02_environment-repository} beschrieben. Dadurch steht dem Entwickler in der Developmentumgebung die bereits vorkonfigurierte \textit{ZSH} Shell (siehe \autoref{fig:s-06_extensions-dev-container}) zur Verfügung. Es wäre auch möglich, weitere Konfigurationen zu installieren und so theoretisch das gesamte System anzupassen.

\begin{figure}[h]
    \centering
    \begin{minipage}[b]{0.49\textwidth}
        \centering
        \includegraphics[width=\textwidth]{s-07_detect-dev-container-vscode.png}
        \caption{Erkennung eines Dev Container Projekts durch Visual Studio Code}
        \label{fig:s-07_detect-dev-container-vscode}
    \end{minipage}
    \hfill
    \begin{minipage}[b]{0.49\textwidth}
        \centering
        \includegraphics[width=\textwidth]{s-08_rebuild-dev-container-vscode.png}
        \caption{Rebuild eines Dev Containers bei Änderungen durch Visual Studio Code}
        \label{fig:s-08_rebuild-dev-container-vscode}
    \end{minipage}
\end{figure}

Wenn das Projekt auf der lokalen Maschine des Entwicklers in \textit{Visual Studio Code} geöffnet wird und die \textit{Dev Containers} Erweiterung von \textit{Microsoft} installiert ist, erkennt die \Gls{ide} automatisch das \texttt{.devcontainer/} Verzeichnis mit der Konfiguration für den \nameref{subsec:05-01-02_dev-container} (siehe \autoref{fig:s-07_detect-dev-container-vscode}). Dadurch kann statt der lokalen Umgebung sofort die vorkonfigurierte Developmentumgebung inklusive des Creation Environments gestartet werden. Ändert sich die \texttt{devcontainer.json} Datei, in welcher die Developmentumgebung spezifiziert ist, bietet \textit{Visual Studio Code} dem Entwickler unmittelbar einen Rebuild des Containers an (siehe \autoref{fig:s-08_rebuild-dev-container-vscode}).

\begin{figure}[h]
    \centering
    \includegraphics[width=0.95\textwidth]{s-10_access-github-codespace.png}
    \caption{Betrieb eines GitHub Codespaces}
    \label{fig:s-10_access-github-codespace}
\end{figure}

\begin{wrapfigure}{r}{0.49\textwidth}
    \vspace{-10pt}
    \centering
    \includegraphics[width=0.48\textwidth]{s-05_shell-dev-container.png}
    \caption{Shell im Dev Container}
    \label{fig:s-05_shell-dev-container}

    \vspace{05pt}
    
    \includegraphics[width=0.48\textwidth]{s-06_extensions-dev-container.png}
    \caption{Extensions im Dev Container}
    \label{fig:s-06_extensions-dev-container}
    \vspace{-05pt}
\end{wrapfigure}

\begin{figure}
    \begin{minipage}[t]{0.49\textwidth}
        \centering
        \includegraphics[width=0.9\textwidth]{s-09_start-github-codespace.png}
        \caption{Start eines GitHub Codespaces}
        \label{fig:s-09_start-github-codespace}
    \end{minipage}
\end{figure}

Auf der Hauptseite des Repositorys auf \textit{GitHub} kann über das Menü \textit{Code} neben einem lokalen Klon auch ein \textit{Codespace} eingerichtet werden (siehe \autoref{fig:s-09_start-github-codespace}). Der Zugriff auf einen \textit{Codespace} erfolgt entweder über die \Gls{gui} einer lokalen \Gls{ide} oder direkt über den Browser (siehe \autoref{fig:s-10_access-github-codespace}). In beiden Fällen sind ebenfalls sowohl die angegebenen Extensions (siehe \autoref{fig:s-06_extensions-dev-container}) als auch die personalisierte Shell (siehe \autoref{fig:s-05_shell-dev-container}) verfügbar.

\clearpage

\begin{figure}[H]
    \centering
    \includegraphics[width=0.95\textwidth]{s-03_local-database-docker-compose.png}
    \caption{Betrieb der lokalen Datenbank in Docker Compose}
    \label{fig:s-03_local-database-docker-compose}
\end{figure}

\begin{figure}[H]
    \centering
    \includegraphics[width=0.95\textwidth]{s-04_local-app-docker-compose.png}
    \caption{Zugriff auf die lokale Applikation in Docker Compose}
    \label{fig:s-04_local-app-docker-compose}
\end{figure}

Durch die \texttt{docker-compose.yml} Datei im \texttt{.devcontainer/} Verzeichnis stehen lokal sowohl die Applikation als auch eine Datenbank als Backing Service bereit, wodurch die entwickelte Software vollständig lokal betrieben werden kann. \autoref{fig:s-09_start-github-codespace} zeigt eine Datenbankanfrage direkt über \textit{PostgreSQL}, welche problemlos möglich ist. Dadurch kann auch die Applikation selbst, also der Dienst der in \texttt{app/server.ts} definiert ist, lokal über seinen Port angesprochen werden. In \autoref{fig:s-04_local-app-docker-compose} ist dargestellt, wie eine Anfrage an \texttt{localhost:8000} gestellt wird, welches der durch den Container bereitgestellte Port ist. Das Ergebnis erhält die korrekte Antwort, die in \texttt{app/server.ts} programmiert ist.

\subsection{Toolchain-as-Code im Deployment}
\label{subsec:06-03-02_toolchain-as-code-in-deployment}

Durch die \Gls{cicd} Konfiguration, die in jedem Artefakt-generierenden Repository vorhanden ist, können alle Umgebungen sowie die Applikation automatisisert durch \textit{\Gls{github-actions}} integriert und deployt werden. Dieser Abschnitt wird einen Einblick in die Verwendung der \nameref{ch:05_toolchain-as-code} Lösung im Bereich der Deploymentumgebungen geben.

\begin{wrapfigure}{l}{0.4\textwidth}
    \vspace{-10pt}
    \centering
    \includegraphics[width=0.39\textwidth]{s-11_push-tag-environment-repository.png}
    \caption{Push eines Tags in das Environment Repository}
    \label{fig:s-11_push-tag-environment-repository}
    \vspace{-05pt}
\end{wrapfigure}

Auslöser der Pipelines beider Repositories ist eine \texttt{push} Aktion, entweder auf den \texttt{main} Branch oder mit einem bestimmten Tag. In \autoref{fig:s-11_push-tag-environment-repository} ist dargestellt, wie ein Tag mit einer Versionsangabe in das Environment Repository gepusht wird. Der Prozess funktioniert wie in \autoref{subsec:06-02-02_environment-repository} beschrieben. Aus \autoref{fig:s-12_overview-cicd-pipelines} ist ersichtlich, wie zunächst die Pipeline für das Base Environment durchläuft, gefolgt von den Pipelines für Integration und Creation Environment. Die Abbildungen \ref{fig:s-13_cicd-pipeline-base-environment}, \ref{fig:s-14_cicd-pipeline-integration-environment} und \ref{fig:s-15_cicd-pipeline-creation-environment} zeigen die einzelnen Schritte dieser Pipelines, welche jeweils in zwei Jobs unterteilt sind. Der erste Job übernimmt \Gls{ci} Aufgaben und ist für den \Gls{build} des \Glspl{image} verantwortlich, während \Gls{cd} Aufgaben im zweiten Job durchgeführt werden, um das \Gls{image} in die \Gls{container-registry} zu pushen und die Umgebung zu aktualisieren. Sind alle Pipelines erfolgreich, befinden sich die neuesten Versionen der Umgebungen in \textit{\Gls{github-actions}} (siehe \autoref{fig:s-16_environment-images-docker-hub}) und können von dort aus mittels \texttt{docker pull} verwendet werden.

\begin{figure}[H]
    \centering
    \includegraphics[width=0.75\textwidth]{s-12_overview-cicd-pipelines.png}
    \caption{Übersicht über CI/CD Pipelines nach Push eines Tags}
    \label{fig:s-12_overview-cicd-pipelines}
\end{figure}

\begin{figure}[H]
    \centering
    \includegraphics[width=0.95\textwidth]{s-17_environment-images-size.png}
    \caption{Vergleich der Image Größe von Integration und Creation Environment}
    \label{fig:s-17_environment-images-size}
\end{figure}

Der erhebliche Vorteil einer Trennung von Integration und Creation Enironment wird in \autoref{fig:s-17_environment-images-size} deutlich. Die Größe der über \texttt{docker pull} bezogenen \Glspl{image} unterscheidet sich gravierend zwischen den beiden Umgebungen. Während die vollständige Umgebung für das Integration Environment nur 221 MB groß ist, benötigt das Creation Environment mit 406 MB fast doppelt so viel. Diese Faktoren können einen erheblichen Einfluss auf die Performanz und die Kosten der Infrastruktur haben.

\begin{figure}[hp]
    \centering
    \includegraphics[width=0.95\textwidth]{s-13_cicd-pipeline-base-environment.png}
    \caption{CI/CD Pipeline für Base Environment}
    \label{fig:s-13_cicd-pipeline-base-environment}

    \vfil
    
    \includegraphics[width=0.95\textwidth]{s-14_cicd-pipeline-integration-environment.png}
    \caption{CI/CD Pipeline für Integration Environment}
    \label{fig:s-14_cicd-pipeline-integration-environment}

    \vfil

    \includegraphics[width=0.95\textwidth]{s-15_cicd-pipeline-creation-environment.png}
    \caption{CI/CD Pipeline für Creation Environment}
    \label{fig:s-15_cicd-pipeline-creation-environment}
\end{figure}

\begin{figure}[hp]
    \centering
    \includegraphics[width=0.95\textwidth]{s-16_environment-images-docker-hub.png}
    \caption{Environment Images auf Docker Hub}
    \label{fig:s-16_environment-images-docker-hub}
\end{figure}
