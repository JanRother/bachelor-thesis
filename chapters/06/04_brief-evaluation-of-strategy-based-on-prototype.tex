\section{Kurzbewertung der Strategie anhand des Prototypen}
\label{sec:06-04_brief-evaluation-of-strategy-based-on-prototype}

\autoref{sec:06-04_brief-evaluation-of-strategy-based-on-prototype} zeigt, dass jedes der in \autoref{sec:06-02_implementation-of-a-toolchain-as-code-configuration} entwickelten Repositories seinen Zweck entlang der \nameref{ch:05_toolchain-as-code} Strategie erfüllt. Das \nameref{subsec:06-02-03_application-repository} hält durch seine Aufteilung in Unterverzeichnisse seine Komplexität in Grenzen. Das \nameref{subsec:06-02-02_environment-repository} erreicht ebenfalls eine sehr große Kompaktheit. Die Umgebungen haben trotz ihrer unterschiedlichen Aufgaben und Fähigkeiten eine einheitliche Grundlage. Auch das \nameref{subsec:06-02-01_dotfiles-repository} integriert sich gut in die Architektur und kann durch weitere Konfigurationen ausgebaut werden. Soll ausgehend vom entwickelten Prototypen ein neues Projekt aufgesetzt werden, genügt es, neue Repositories auf Basis der Template Repositories anzulegen, die Konfigurationen entsprechend den Anforderungen des neuen Projekts anzupassen und Umgebungsvariablen zu setzen.

Insgesamt zeigt sich, dass die entwickelte \nameref{ch:05_toolchain-as-code} Strategie umsetzbar und praxistauglich ist. Die \hyperref[subsubsec:05-02-02-01_data-storage-in-repositories]{Repositories} wechselwirken miteinander wie geplant, die einzelnen \hyperref[subsubsec:05-02-02-02_sub-components-in-environments]{Umgebungen} grenzen sich sinnvoll voneinander ab und die \hyperref[subsec:05-02-03_workflows-and-continuity-in-the-toolchain-as-code-approach]{CI/CD Flows} sind technisch realisierbar. Durch die Einhaltung dieser Architektur werden beim Einrichten einer Development- oder Deploymentumgebung in bestehenden Projekten Zeitressourcen eingespart. Zwar muss diese Zeit zunächst für das initiale Aufsetzen der Architektur investiert werden, was insbesondere für Neueinsteiger herausfordernd sein kann, jedoch ist dies nur einmalig notwendig und amortisiert sich mit jedem Entwickler, der das bestehende Projekt unkompliziert nutzen kann.

Dank bewährter Konzepte funktioniert die Strategie im Bereich \Gls{deployment} problemlos. Auch im \Gls{development} ist sie gut anwendbar. Nicht ausschließlich, aber überwiegend dort kann es trotzdem noch zu technischen Einschränkungen kommen. Im Rahmen der Entwicklung des Prototypen wurde beispielsweise festgestellt, dass nicht immer alle Erweiterungen in \textit{Visual Studio Code} korrekt geladen oder überhaupt installiert werden können. Ebenfalls Probleme machen noch einige \textit{Alpine} Docker \Glspl{image}, konkret das von \textit{Deno}. Fehlende \textit{C} Bibliotheken schienen hier zu Inkompatibilitäten mit \textit{Visual Studio Code} zu führen, weshalb auf ein anderes Base \Gls{image} zurückgegriffen werden musste. Sollen andere \Glspl{ide} als \textit{Visual Studio Code} verwendet werden, sind fehlerhafte oder eingeschränkte Funktionalitäten ebenfalls nicht ausgeschlossen, da sich \nameref{subsec:05-01-02_dev-container} aktuell noch in der Entwicklungsphase befinden. Auch in \textit{GitHub Codespaces} bestehen noch ein paar Einschränkungen. So ist die Verwendug von \textit{Docker Compose} zwar möglich, jedoch gibt es noch Probleme bei der Verwendung von Umgebungsvariablen aus \textit{.env} Dateien, welche vor dem \Gls{build} des Containers nicht im Repository angelegt werden können, wie es lokal möglich wäre. Dies führt im Prototypen beispielsweise dazu, dass die Datenbank nicht korrekt initialisiert wird. Im \Gls{cicd} Bereich kann es ebenfalls zu Herausforderungen kommen, wenn Aktionen von \textit{\Gls{github-actions}} für gewisse Anwendungsfälle keine Lösung bereitstellen, was im Prototypen bei der Verwendug lokal gebauter \Glspl{image} als Basis zunächst ein Problem war.

Insgesamt sollte erwähnt werden, dass in der Softwareentwicklung jedes Tool Herausforderungen mit sich bringen kann. Was die \nameref{ch:05_toolchain-as-code} Strategie jedoch erreicht, ist die Vermeidung lokaler und schlecht reproduzierbarer Fehler. Jedes ihrer Teilkonzepte ließ sich im Rahmen des Prototypen gut umsetzen und eine Praxistauglichkeit auch bei größeren Projekten ist erwartbar. Spätestens die Weiterentwicklung der \nameref{subsec:05-01-02_dev-container} wird noch zu deutlich mehr Reife führen.
