\section{Ableitung von Anforderungen}
\label{sec:04-02_derivation-of-requirements}

\subsection{Vorgehen bei der Auswertung}
\label{subsec:04-02-01_procedure-for-evaluation}

Auf Basis der im vorherigen \autoref{sec:04-01_collection-of-requirements-using-expert-interviews} erhobenen Aussagen zu Toolchains im Bereich von Development- und Deploymentumgebungen sollen in diesem Abschnitt konkrete Anforderungen an diese Toolchains abgeleitet werden. Um dies möglichst strukturiert und nachvollziehbar zu gestalten, folgt der Ableitungsprozess einem vordefinierten Vorgehen bestehend aus sechs Phasen. Dieses Vorgehen ist in \autoref{fig:g-11_process-for-derivation-of-requirements} dargestellt und wird im Folgenden erläutert.

\begin{figure}[h]
    \centering
    \includegraphics[width=0.95\textwidth]{g-11_process-for-derivation-of-requirements.png}
    \caption{Vorgehen zur Ableitung von Anforderungen aus den Interviewergebnissen}
    \label{fig:g-11_process-for-derivation-of-requirements}
\end{figure}

Die Ableitung von Anforderungen beginnt mit der \textbf{Protokollierung der Einzelergebnisse} aus jedem Experteninterview. Dies erfolgt in Stichpunkten und transparent für den Interviewpartner, welcher das Protokoll im Anschluss an das Interview zur Einsicht erhält. Der nächste Schritt ist die \textbf{Zusammenfassung der Einzelergebnisse} in einem konsolidierenden Interviewprotokoll. Dieses enthält alle relevanten Aussagen der \Glspl{ip} inklusive einer Markierung, die angibt, welche \Gls{ip} die Aussage getroffen hat. Gleiche oder ähnliche Aussagen werden dabei zusammengefasst und erhalten entsprechend mehrere Markierungen. Anschließend erfolgt eine \textbf{Sortierung nach Relevanzbereich}, also nach \Gls{development} (\texttt{DEV}), \Gls{deployment} (\texttt{DEP}), Durchgängigkeit (\texttt{CNT} für \textit{Continuity}) oder \textit{Allgemeines} (\texttt{GEN} für \textit{General}). In der vierten Phase erfolgt die \textbf{Gruppierung der Aussagen in Anforderungsbereiche}. Dabei werden zunächst thematisch zusammengehörige Aussagen identifiziert und anschließend mit einem passenden Oberbegriff versehen. Jeder Aspekt kann dabei genau einmal zugeordnet werden. Eine sehr wichtige Phase ist die \textbf{Gewichtung der Anforderungen nach Punktewert}, welcher über die Anzahl der zu einem Anforderungsbereich gehörenden Aussagen bestimmt wird. Für $ n $ Aussagen ergibt sich der $ Punktewert = \sum_{i=1}^{n} (Aussage_i \times Häufigkeit_i) $, sodass mehrfach getroffene Aussagen von mehreren \Glspl{ip} auch zu einem höheren Gewicht führen. Der letzte Schritt ist die \textbf{Ableitung konkreter Anforderungen} mit Prioritäten, indem die sortierten, gruppierten und gewichteten Aussagen nun absteigend nach Punktewert geordnet werden.

Das Ergebnis dieses Vorgehens ist im \autoref{ch:AA_expert-interviews} als \nameref{sec:AA-04_interview-evaluation} genau dokumentiert.

\subsection{Toolchains im Bereich Development}
\label{subsec:04-02-02_toolchains-in-development}

Das vollständige Ergebnis der Auswertung für Anforderungen an Toolchains im Bereich \Gls{development} befindet sich in Listenform im \autoref{subsec:AA-04-01_requirements-development} des \autoref{ch:AA_expert-interviews}.

Besonders wichtig und häufig im Bereich \Gls{development} ist mit neun Gewichtungspunkten die \textbf{\nameref{subsubsec:AA-04-01-01_req-dev-development-environment}} (\texttt{DEV-0}, \acrshort{vgl} \autoref{subsubsec:AA-04-01-01_req-dev-development-environment}). Hier geht es darum, dass sie möglichst intuitiv, schnell und umfänglich zugänglich ist. In den Aussagen finden sich außerdem Parallelen zum Faktor \textit{Dev / Prod Parity} aus dem Konzept der \hyperref[sec:03-05_concept-of-twelve-factor-app]{Twevle-Factor-App}. Die \textbf{\nameref{subsubsec:AA-04-01-02_req-dev-executability}} (\texttt{DEV-1}, \acrshort{vgl} \autoref{subsubsec:AA-04-01-02_req-dev-executability}) ist mit acht Gewichtungspunkten ebenfalls sehr relevant. Entwickler möchten die entwickelte Software lokal ausführen können, Parallelen finden sich beim Faktor \textit{Codebase} der \hyperref[sec:03-05_concept-of-twelve-factor-app]{Twevle-Factor-App}. \textbf{\nameref{subsubsec:AA-04-01-03_req-dev-configuration}} (\texttt{DEV-2}, \acrshort{vgl} \autoref{subsubsec:AA-04-01-03_req-dev-configuration}) folgt und hängt mit dem Faktor \textit{Config} zusammen. Sieben Gewichtungspunkte fordern eine einfache und versionierte Konfiguration. Ebenfalls eine Rolle spielen mit sechs Gewichtungspunkten \textbf{\nameref{subsubsec:AA-04-01-04_req-dev-dependencies}} (\texttt{DEV-3}, \acrshort{vgl} \autoref{subsubsec:AA-04-01-04_req-dev-dependencies}), welche bestenfalls für jedes Projekt isoliert und transparent verfügbar sein sollten, ähnlich dem Faktor \textit{Dependencies}. Keinem Faktor zuordnen lassen sich die Anforderungen an die \textbf{\nameref{subsubsec:AA-04-01-05_req-dev-documentation}} (\texttt{DEV-4}, \acrshort{vgl} \autoref{subsubsec:AA-04-01-05_req-dev-documentation}) eines Projekts, die vier Gewichtungspunkte erhält. Sie sollte aktuell und verständlich sein. Der Anforderungsbereich mit den wenigsten, nämlich zwei Gewichtungspunkten, sind \textbf{\nameref{subsubsec:AA-04-01-06_req-dev-scripts}} (\texttt{DEV-5}, \acrshort{vgl} \autoref{subsubsec:AA-04-01-06_req-dev-scripts}), welche für bestimmte Aufgaben bereitstehen sollen. Der Faktor \textit{Build, Release, Run} hängt mit diesem Bereich zusammen. Weitere getroffene Aussagen betrafen die Einrichtung von Certificates und die Einhaltung von Standards und Konventionen. Die benötigte Zeit für das Onboarding neuer Entwickler in einem Projekt variiert sehr stark zwischen weniger als einer Stunde und mehreren Tagen. Die meisten \Glspl{ip} gaben an, dass dies etwa einen Tag in Anspruch nimmt.

Die genaue Verteilung der Gewichtungspunkte ist in \autoref{fig:g-12_interview-results-requirements-development} dargestellt.

\begin{figure}[h]
    \centering
    \includegraphics[width=0.95\textwidth]{g-12_interview-results-requirements-development.png}
    \caption{Anforderungen im Bereich Development und deren Priorisierung}
    \label{fig:g-12_interview-results-requirements-development}
\end{figure}

\subsection{Toolchains im Bereich Deployment}
\label{subsec:04-02-03_toolchains-in-deployment}

Das vollständige Ergebnis der Auswertung für Anforderungen an Toolchains im Bereich \Gls{deployment} befindet sich in Listenform im \autoref{subsec:AA-04-02_requirements-deployment} des \autoref{ch:AA_expert-interviews}.

Dieser Bereich enthält deutlich weniger Anforderungsbereiche, wie bereits im Abschnitt \nameref{subsec:04-01-04_interview-results} festgestellt. Mit Abstand die meisten Gewichtungspunkte erhält \textbf{\nameref{subsubsec:AA-04-02-01_req-dep-automation}} (\texttt{DEP-0}, \acrshort{vgl} \autoref{subsubsec:AA-04-02-01_req-dep-automation}). Insgesamt sechs Punkte lassen darauf hindeuten, dass hier das meiste Potential steckt. Automatisierung ist auch im Faktor \textit{Build, Release, Run} der \hyperref[sec:03-05_concept-of-twelve-factor-app]{Twevle-Factor-App} ein wichtiger Aspekt. Gefordert wird ein maximal automatisiertes \Gls{deployment}, wobei für Produktivumgebungen oftmals noch eine manuelle Bestätigung notwendig ist. \textbf{\nameref{subsubsec:AA-04-02-02_req-dep-configuration}} (\texttt{DEP-1}, \acrshort{vgl} \autoref{subsubsec:AA-04-02-02_req-dep-configuration}) erhält beim \Gls{deployment} nur zwei Gewichtungspunkte. Auch hier lässt es sich dem Faktor \textit{Config} zuordnen. Konfigurationen sollten über versionierte Dateien erfolgen. Eine zentrale Erkenntnis, wenn auch nur mit einem Gewichtungspunkt, ist die Forderung nach \textbf{\nameref{subsubsec:AA-04-02-03_req-dep-containerization}} (\texttt{DEP-2}, \acrshort{vgl} \autoref{subsubsec:AA-04-02-03_req-dep-containerization}). Sie kommt in der \hyperref[sec:03-05_concept-of-twelve-factor-app]{Twevle-Factor-App} nicht vor. Auch keinem Faktor zuordnen lässt sich der Aspekt \textbf{\nameref{subsubsec:AA-04-02-04_req-dep-quality-security}} (\texttt{DEP-3}, \acrshort{vgl} \autoref{subsubsec:AA-04-02-04_req-dep-quality-security}), der ebenfalls nur einen Gewichtungspunkt erhält. Schnelles Feedback ist hier ein wichtiger Aspekt.

Die genaue Verteilung der Gewichtungspunkte ist in \autoref{fig:g-13_interview-results-requirements-deployment} dargestellt.

\begin{figure}[h]
    \centering
    \includegraphics[width=0.95\textwidth]{g-13_interview-results-requirements-deployment.png}
    \caption{Anforderungen im Bereich Deployment und deren Priorisierung}
    \label{fig:g-13_interview-results-requirements-deployment}
\end{figure}

\subsection{Durchgängigkeit von Toolchains}
\label{subsec:04-02-04_consistency-of-toolchains}

Das vollständige Ergebnis der Auswertung für Anforderungen an die Durchgängigkeit von Toolchains befindet sich in Listenform im \autoref{subsec:AA-04-03_requirements-continuity} des \autoref{ch:AA_expert-interviews}.

\textbf{\nameref{subsubsec:AA-04-03-01_req-cnt-tools-scripts}} (\texttt{CNT-0}, \acrshort{vgl} \autoref{subsubsec:AA-04-03-01_req-cnt-tools-scripts}) haben einen großen Einfluss auf die Durchgängigkeit von Toolchains. Sie erhalten fünf Gewichtungspunkte und lassen sich dem Faktor \textit{Build, Release, Run} der \hyperref[sec:03-05_concept-of-twelve-factor-app]{Twevle-Factor-App} zuordnen. Genannt wurden beispielsweise \textit{Task-Files} und \textit{Bash-Skripte} im Repository zur Automatisierung von Aufgaben auf allen Ebenen. Drei Gewichtungspunkte erhält der Bereich \textbf{\nameref{subsubsec:AA-04-03-02_req-cnt-parity}} (\texttt{CNT-1}, \acrshort{vgl} \autoref{subsubsec:AA-04-03-02_req-cnt-parity}). Er bezieht sich auf die Parität zwischen Entwicklungs- und Produktivumgebungen, welche auch laut dem Faktor \textit{Dev / Prod Parity} der \hyperref[sec:03-05_concept-of-twelve-factor-app]{Twevle-Factor-App} möglichst nah beieinander liegen sollten. Ebenfalls eine Rolle spielt die \textbf{\nameref{subsubsec:AA-04-03-03_req-cnt-adaptability}} (\texttt{CNT-2}, \acrshort{vgl} \autoref{subsubsec:AA-04-03-03_req-cnt-adaptability}) von Toolchains mit zwei Gewichtungspunkten. Sie sollte durch das Entwicklungsteam möglich sein, ohne ein unnötig komplexes Vorgehen zu erfordern. Der letzte wichtige Anforderungsbereich mit ebenfalls zwei Gewichtungspunkten ist \textbf{\nameref{subsubsec:AA-04-03-04_req-cnt-reproducibility}} (\texttt{CNT-3}, \acrshort{vgl} \autoref{subsubsec:AA-04-03-04_req-cnt-reproducibility}). Hierbei geht es um die Wiederholbarkeit von Umgebungen und Prozessen, die auch im Faktor \textit{Dev / Prod Parity} eine Rolle spielt.

Die genaue Verteilung der Gewichtungspunkte ist in \autoref{fig:g-14_interview-results-requirements-contiuity} dargestellt.

\begin{figure}[h]
    \centering
    \includegraphics[width=0.95\textwidth]{g-14_interview-results-requirements-continuity.png}
    \caption{Anforderungen im Bereich Durchgängigkeit und deren Priorisierung}
    \label{fig:g-14_interview-results-requirements-contiuity}
\end{figure}
