\section{Zusammenfassung der Forschungsergebnisse}
\label{sec:07-01_summary-of-research-results}

In \autoref{ch:01_introduction-and-motivation} wurden vier zentrale \hyperref[sec:01-03_objectives-and-research-questions]{Forschungsfragen} formuliert, die im Rahmen dieser Arbeit beantwortet werden sollten. Die Ergebnisse werden im Folgenden zusammengefasst.

\vspace{1em}
\begin{table}[H]
    \centering
    \begin{tabular}{p{0.1\textwidth} p{0.8\textwidth}}
        \textbf{RQ-0} & Welche Anforderungen an Development- und Deploymentumgebungen haben Entwickler und Administratoren, die diese Bereiche im Rahmen von Softwareprojekten mit Microservices Architektur verantworten? \\
    \end{tabular}
\end{table}

Die \acrlong{rq} \textbf{RQ-0} wurde in \autoref{ch:04_requirements-for-development-and-deployment-environments} beantwortet.

Im Bezug auf Toolchains liegt ein besonderer Fokus der Entwickler auf den Faktoren \textit{Codebase}, \textit{Dependencies}, \textit{Build, Release Run} und \textit{Dev / Prod Parity} aus dem Konzept der \hyperref[sec:03-05_basic-idea-of-twelve-factor-app]{Twelve-Factor-App}. Daraus ergeben sich die Ziele einer zentralen Ablage des Quellcodes für alle Deployments, der deklarativen Angabe von Abhängigkeiten, dem Trennen verschiedener Phasen der Artefaktgenerierung sowie der möglichst vollständigen Vereinheitlichung von Development- und Deploymentumgebung. Zusammenfassend erwarten Entwickler von \textbf{\hyperref[subsec:04-02-02_toolchains-in-development]{Toolchains im Bereich Development}}, dass sie schnell verfügbar sind. Die entwickelte Software sollte lokal ausführbar und ihre Konfiguration im \Gls{vcs} versioniert sein. Abhängigkeiten sollten gut zugänglich und transparent angegeben sein. Wichtig sind außerdem eine aktuelle und intuitive Dokumentation sowie automatisierende Skripte. \textbf{\hyperref[subsec:04-02-03_toolchains-in-deployment]{Toolchains im Bereich Deployment}} sollten fast vollständig automatisiert sein. Auch hier erwarten Entwickler, dass Konfiguration versioniert ist. Sowohl die Applikation als auch ihre Umgebung sollten containerisiert sein. Von \Gls{cicd} Pipelines wird gefordert, dass sie in der Lage sind, schnelles Feedback an die Entwickler zu übermitteln. In Bezug auf die \textbf{\hyperref[subsec:04-02-04_consistency-of-toolchains]{Durchgängigkeit von Toolchains}} wollen Entwickler bestenfalls die gleichen Tools auf allen Ebenen nutzen. Bestenfalls steht die gleiche Umgebung sowohl lokal als auch produktiv zur Verfügung. Tools sollten schnell und unkompliziert durch das Entwicklungsteam anpassbar sein. Von Applikation und Umgebung wird erwartet, dass ihr Verhalten reproduzierbar ist.

\vspace{1em}
\begin{table}[H]
    \centering
    \begin{tabular}{p{0.1\textwidth} p{0.8\textwidth}}
        \textbf{RQ-1} & Welche Ansätze für das Application Management bestehen bereits, um Software sicher, wartbar, performant und in kurzen Abständen zu entwickeln und auszuliefern? Welche Grenzen haben diese Ansätze gegebenenfalls im Kontext des Ziels dieser Arbeit? \\
    \end{tabular}
\end{table}

Die \acrlong{rq} \textbf{RQ-1} wurde in \autoref{ch:03_examination-of-existing-approaches} beantwortet.

Insgesamt wurden vier Ansätze und Methoden vorgestellt, die gemeinsam das Potential haben, die Anforderungen an Toolchains zu einem großen Teil zu erfüllen. \textbf{\hyperref[sec:03-01_introduction-to-devops]{DevOps}} ist ein eher organisatorischer Ansatz. Er enthält keine feste Methodik, sondern eher eine Vielzahl an Praktiken. Ziel von \hyperref[sec:03-01_introduction-to-devops]{DevOps} ist es, die Bereiche \Gls{development} und Operations näher zusammenzubringen. Ein sehr wichtiges Konzept dabei ist \acrfull{cicd}, welches dazu geeignet ist, Bereitstellungszyklen von Software zu verkürzen. Für Toolchains bietet \hyperref[sec:03-01_introduction-to-devops]{DevOps} keine oder kaum Lösungen an. \textbf{\hyperref[sec:03-03_gitops-as-further-evolution]{GitOps}} kann \hyperref[sec:03-01_introduction-to-devops]{DevOps} hier mit einer technischeren Herangehensweise ergänzen. Es beinhaltet einen deklarativen Ansatz für die Konfiguration von Deploymentumgebungen und basiert auf dem \acrfull{vcs} \Gls{git} als \q{Single Source of Truth}. Besonders eignet sich diese Methodik für \gls{cloud-native} Applikationen. Sie schlägt eine Trennung von Umgebung und Applikation in mindestens zwei Repositories vor und fokussiert sich deutlich mehr auf Entwickler. Bei \textbf{\hyperref[sec:03-04_idea-of-dotfiles]{Dotfiles}} handelt es sich um ein vergleichsweise kompaktes Konzept, welches ursprünglich aus dem Bereich von \textit{Linux} und \textit{Unix} stammt. Zentrales Element sind Dateien mit persönlichen Konfigurationen einer Entwicklungsumgebung, welche in einem dedizierten Repository abgelegt werden. Im Idealfall stehen sie zusammen mit einem Installationsskript bereit und reduzieren so die Zeit für das Einrichten einer Umgebung dramatisch. Die \textbf{\q{\hyperref[sec:03-05_basic-idea-of-twelve-factor-app]{Twelve-Factor-App}}} ist ein Manifest für Software-as-a-Service Applikationen. Sie listet zwölf elementare Eigenschaften guter Applikationen auf. Diese Faktoren lassen sich an vielen Stellen auf Probleme im Bereich von Toolchains anwenden und nutzen häufig auch das deklarative Prinzip zur Automatisierung von Softwareentwicklung. Die \hyperref[sec:03-05_basic-idea-of-twelve-factor-app]{Twelve-Factor-App} legt Wert auf Portabilität, Deploybarkeit und Skalierung. Sie hilft dabei, die Differenzen zwischen \Gls{development} und \Gls{deployment} zu reduzieren.

\vspace{1em}
\begin{table}[H]
    \centering
    \begin{tabular}{p{0.1\textwidth} p{0.8\textwidth}}
        \textbf{RQ-2} & Wie kann eine Toolchain-as-Code Strategie bestehende Ansätze kombinieren, um die Lücke zwischen \Gls{development} und \Gls{deployment} zu schließen, manuelle Schritte im Rahmen der Toolchain zu reduzieren und dabei gleichermaßen eine effiziente Konfiguration von Softwareprojekten zu ermöglichen?
    \end{tabular}
\end{table}

Die \acrlong{rq} \textbf{RQ-2} wurde in \autoref{sec:05-02_strategy-for-toolchains} von \autoref{ch:05_toolchain-as-code} beantwortet.

Die \nameref{ch:05_toolchain-as-code} Strategie erfüllt die ermittelten Anforderungen auf Basis der erarbeiteten \textbf{Teilkonzepte}. Pipelines aus dem \hyperref[sec:03-01_introduction-to-devops]{DevOps} Konzept automatisieren Integration ud \Gls{deployment}. Umgebung und Applikation sind gemäß \hyperref[sec:03-03_gitops-as-further-evolution]{GitOps} in voneinander getrennten \textit{\Gls{git}} Repositories. Die Personalisierung von Umgebungen erfolgt optional über ein \hyperref[sec:03-04_idea-of-dotfiles]{Dotfiles} Repository und die Toolchain-bezogenen Faktoren der \q{\hyperref[sec:03-05_basic-idea-of-twelve-factor-app]{Twelve-Factor-App}} tragen zu Zuverlässigkeit und Stabilität der Strategie bei, während sie die Lücke zwischen \Gls{development} und \Gls{deployment} schließen. Die gesamte Strategie basiert auf \textbf{\nameref{sec:02-03_containerization}} mit \hyperref[subsec:05-01-01_docker-container]{Docker Containern} und \hyperref[subsec:05-01-02_dev-container]{Dev Containern}. Als \textbf{zentrale Elemente} macht die Strategie Vorschläge und Vorgaben in Bezug auf die \hyperref[subsubsec:05-02-02-02_sub-components-in-environments]{Komponenten einzelner Container und Repositories}, die \hyperref[subsubsec:05-02-02-01_data-storage-in-repositories]{Beziehungen zwischen Repositories und Images} sowie die \hyperref[subsec:05-02-03_workflows-and-continuity-in-the-toolchain-as-code-approach]{Prozesse hinter Entwicklung, Integration und Deployment}. Sie kann in unterschiedlichem Umfang angewendet werden und ist flexibel bei den konkreten Tools, mit denen sie umgesetzt werden kann.

\vspace{1em}
\begin{table}[H]
    \centering
    \begin{tabular}{p{0.1\textwidth} p{0.8\textwidth}}
        \textbf{RQ-3} & Was sind Best Practices beim Einrichten von Development- und Deploymentumgebungen, die Sicherheit, Wartbarkeit und Performanz begünstigen? \\
    \end{tabular}
\end{table}

Die \acrlong{rq} \textbf{RQ-3} wurde in \autoref{sec:05-03_best-practices-with-docker-and-docker-compose} von \autoref{ch:05_toolchain-as-code} beantwortet.

Der Aufbau von Containern und \Glspl{image} kann einen Einfluss auf die Sicherheit, Wartbarkeit und Performanz von Softwareprojekten haben. Best Practices im Bereich \textbf{\nameref{subsec:05-03-02_security}} sind die sorgfältige Auswahl von Base \Glspl{image} und der regelmäßige Rebuild von \Glspl{image}. Außerdem sollte darauf geachtet werden, Rechte in Containern so wenig wie möglich zu vergeben. Im Bereich \textbf{\nameref{subsec:05-03-03_maintainability}} ist es wichtig, mehrzeilige Befehle sortiert und lesbar aufzubauen, ein Arbeitsverzeichnis für den Container anzugeben und ihn mit Hilfe von Labels möglichst genau zu beschreiben. Die \textbf{\nameref{subsec:05-03-01_performance}} von Containern kann insbesondere durch die Verwendung von Multi-Stage \Glspl{build} verbessert werden, indem diese zu deutlich kompakteren \Glspl{image} führen. Auch durch die Verwendung von Docker Ignore und der Vermeidung unnötiger Paketinstallationen kann die Größe von \Glspl{image} positiv beeinflusst werden. Dateien, die nur für einen einzigen Schritt beim \Gls{build} des Containers benötigt werden, sollten dort per Mounting hinzugefügt werden, anstatt sie in das \Gls{image} zu kopieren. \textbf{\hyperref[subsec:05-03-04_further]{Weitere Faktoren}} sind ebenfalls zu berücksichtigen, darunter Reproduzierbarkeit und Konventionen bei der Portfreigabe. Insgesamt besteht beim Umgang mit Versionen von \Glspl{image} und Abhängigkeiten viel Optimierungspotential.

\vspace{2em}

Alle vier zu Beginn der Arbeit eingeführten und motivierten \hyperref[sec:01-03_objectives-and-research-questions]{Forschungsfragen} konnten abschließend beantwortet werden. Jedes Kapitel entwickelt dabei eine Grundlage, eine Vorbedingung oder ein Teilkonzept, welche schlussendlich zu einer Strategie verbunden werden.

Diese Arbeit konnte zeigen, wie viele Faktoren Einfluss auf die Effizienz und Produktivität von Entwicklern haben. Insbesondere die Vielzahl an Bereichen und die Menge an Tools, die Teil des modernen Software Engineering sind, stellen vielfältige und herausfordernde Anforderungen an Entwicklungsteams. Nur mit einer systematischen Herangehensweise kann es gelingen, Toolchains vom Beginn eines Projekts an so zu gestalten, dass sie möglichst wenig Zeit in der Entwicklung kosten und Entwicklern somit wertvoller Raum für Innovation gegeben wird. Zwar bestehen bereits einige Lösungen in Form organisatorischer Vorgehensmodelle und technischer Grundkonzepte, diese müssen jedoch intelligent kombiniert werden.

Der im Rahmen dieser Arbeit entwickelten \q{Toolchain-as-Code} Strategie gelingt dies auf eine Weise, die spezifisch genug ist, um einen eindeutigen Leitfaden zur Implementierung ableiten zu können, und gleichzeitig ausreichend Flexibilität bietet, um auf verschiedene Projekte mit unterschiedlichen Technologien anwendbar zu sein. Ihre Erweiterbarkeit prädestiniert sie für den Einsatz in Organisationen beliebiger Komplexität und mit bereits bekannten Konzepten als Grundlage kann sie dank einer flachen Lernkurve schnell von Entwicklern umgesetzt werden.

Toolchain-as-Code hat das Potential, Software Engineering nachhaltig für die Zukunft aufzustellen und technologische Wertschöpfungsketten zu optimieren.
