\section{Weiterer Forschungsbedarf}
\label{sec:07-02_further-research-needs}

Trotz der Beantwortung aller \hyperref[sec:01-03_objectives-and-research-questions]{Forschungsfragen} und der Erfüllung vieler \hyperref[sec:04-03_general-aspects-and-summery-of-findings]{Anforderungen}, blieben einige Aspekte unberücksichtigt oder wurden nicht erschöpfend behandelt, sodass das Ergebnis dieser Arbeit als Ausgangspunkt für weitere Forschungen genutzt werden kann.

Die entwickelte Toolchain-as-Code Strategie konzentriert sich hauptsächlich auf Neuentwicklungen mit Technologien aus den Bereichen \nameref{sec:02-01_web-development}, \nameref{sec:02-02_microservices} und \nameref{sec:02-03_containerization}. \Glspl{legacy-system} stellen Entwicklungsteams oft vor viel größere Herausforderungen, da ihre Toolchains selten modernen Anforderungen und Standards entsprechen. Dateien und Verzeichnisstrukturen folgen überholten Konventionen und nicht selten haben sich über die Zeit überraschend langlebige Provisorien durchgesetzt. Wie die Architektur gemäß dem Konzept von Toolchain-as-Code in solchen Projekten eingeführt werden kann und ob dies überhaupt möglich beziehungsweise sinnvoll ist, wurde in dieser Arbeit nicht diskutiert.

Wird Software in größeren Organisationen entwickelt, bestehen oft zusätzliche Herausforderungen wie \Glspl{vpn}, Proxies, Mirrors oder Certificates. Insbesondere bei der Konfiguration von \Gls{cicd} und \nameref{sec:02-03_containerization} Tools kann dies zu deutlich aufwändigeren Einrichtungsprozessen führen. Eine konkrete Lösung hierfür liefert diese Arbeit nicht.

Ebenfalls nur untergeordnet thematisiert wurde die \Gls{delivery} von Software, also ihre Bereitstellung für den Kunden. Im Verbund mit \nameref{sec:02-03_containerization} ist hier oft die Konfiguration komplexer Bereitstellungs- und Orchestrierungswerkzeuge notwendig. Es kann sich schnell um äußerst anspruchsvolle und kostenintensive Aufgaben handeln, eine solche Infrastruktur zu betreiben, weshalb die Toolchain-as-Code Strategie sich auf das \Gls{deployment} von Software beschränkt.

Eine vergleichbare Komplexität liegt in den Themen Sicherheit, Wartbarkeit und Performanz von Software. Jedes einzelne von ihnen würde einer eigenen Arbeit oder sogar mehreren bedürfen, um es erschöpfend behandeln zu können. Zwar wurden im Kontext von \nameref{sec:02-03_containerization} und der vierten \acrlong{rq} einige Best Practices vorgestellt, die eine grundlegende Orientierung bieten können, jedoch handelt es sich dabei nur um die Spitze des Eisbergs.

Auch in der Literatur kann Inspiration für weitere Forschungsbedarfe im Kontext dieser Arbeit gefunden werden. So stellen \citeauthor{000:CI-CD-Deployment-in-DevOps-reduce-Gap-Developer-Operation} fest, dass \hyperref[sec:03-01_introduction-to-devops]{DevOps} Geschwindigkeit über Sicherheit stellt \cite{010:Efficient-Application-Deployment-GitOps-for-Faster-and-Secure-CI-CD-Cycles}. Im Sicherheitsbereich könnten daher, wie bereits angesprochen, konkretere Lösungen einen wichtigen Beitrag leisten. \Gls{mlops} zur Implementierung und Administration von Modellen für \Gls{ki} ist ein aufkommendes Thema mit vielen Chancen \cite{010:Efficient-Application-Deployment-GitOps-for-Faster-and-Secure-CI-CD-Cycles}. Vorlagen für komplette Pipelines und Projekte sowie deren Beitrag zur Reduzierung des Aufwands bei der Nutzung von \hyperref[sec:03-01_introduction-to-devops]{DevOps} mit \nameref{sec:02-02_microservices} sind ein Thema in dem Paper \citetitle{019:Advanced-DevOps-Environment-for-Microservices-based-Applications} \cite{019:Advanced-DevOps-Environment-for-Microservices-based-Applications}. Bereits angesprochen wurde die Abgrenzung der Toolchain-as-Code Strategie gegenüber Multi-Repository Projekten, ähnlich wie in \hyperref[sec:03-03_gitops-as-further-evolution]{GitOps} \cite{109:GitOps}. Die falsche Verwendung von \hyperref[sec:03-04_idea-of-dotfiles]{Dotfiles} kann sicherheitskritisch sein und wurde unter diesem Aspekt von \citeauthor{029:Connecting-the-Dotfiles} genauer untersucht \cite{029:Connecting-the-Dotfiles}. Spannend insbesondere für große Organisationen wäre außerdem, wie Entwicklungsumgebungen als \nameref{subsec:05-01-02_dev-container} on-premise bereitgestellt werden können \cite{014:Managing-Container-based-Software-Development-Environments}.
