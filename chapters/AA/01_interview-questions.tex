\section{Interviewfragen}
\label{sec:AA-01_interview-questions}

Die eindeutige Identifizierung der Interviewfragen ist über die folgende Kombination möglich:

\begin{quote}
    \begin{verbatim}
        IQ-<Fragenbereich><Fragennummer> (<Fokusbereich>)
    \end{verbatim}
\end{quote}

Der \texttt{Fragenbereich} ist ein Buchstabe von \texttt{A} bis \texttt{C} und gibt an, ob es sich um eine (\texttt{A}) \textbf{offene} Frage, (\texttt{B}) \textbf{halb-offene} Frage oder (\texttt{C}) \textbf{geschlossene} Frage handelt. Die \texttt{Fragennummer} ist eine fortlaufende Nummer beginnend bei \texttt{0}, die die Position der Frage innerhalb ihres Bereichs angibt. Der \texttt{Fokusbereich} ist als Zahl von \texttt{Eins} bis \texttt{Fünf} definiert und ordnet die Frage einem von fünf Fragezielen zu, darunter (\texttt{Eins}) die \textbf{Zusammensetzung} von, (\texttt{Zwei}) die \textbf{Vorgehensweisen} und \textbf{Interaktionswege} mit, (\texttt{Drei}) \textbf{Herausforderungen} im Umgang mit, (\texttt{Vier}) Ist-Standards für \textbf{Automatisierungsgrad} und \textbf{Developer-Experience} in sowie (\texttt{Fünf}) die Prioritäten bei \textbf{Anforderungen} an und \textbf{Zielen} für Development- und Deployment-Umgebungen.

\clearpage

\subsection{Offene Fragen}
\label{subsec:AA-01-01_open-questions}

\begin{table}[H]
    \centering
    \begin{tabular}{ >{\raggedright\bfseries}p{0.2\textwidth} p{0.7\textwidth} }
        IQ-A0 (Eins) & 
        Wenn Sie sich eine ideale Development- beziehungsweise Deploymentumgebung vorstellen, was wären die wichtigsten Merkmale, die sie auszeichnen würde? \\
        \hline
        IQ-A1 (Drei) &
        Welche Herausforderungen oder Hindernisse treten typischerweise bei der Einrichtung und Nutzung Ihrer Entwicklungs- oder Deploymentumgebungen auf? \\
        \hline
        IQ-A2 (Zwei) &
        Bezogen auf die einzelnen Schritte, wie gehen Sie üblicherweise bei der Konfiguration der einzelnen Werkzeuge Ihrer Entwicklungs- beziehungsweise Deploymentumgebungen vor? \\
        \hline
        IQ-A3 (Eins) &
        Was sind typische Komponenten, Werkzeuge oder andere Elemente Ihrer Entwicklungs- oder Deploymentumgebungen? \\
        \hline
        IQ-A4 (Fünf) &
        Gibt es spezifische Anforderungen oder Standards, die Sie bei der Nutzung Ihrer Entwicklungs- und Deploymentumgebungen erfüllen müssen? \\
    \end{tabular}
\end{table}

\subsection{Halb-offene Fragen}
\label{subsec:AA-01-02_half-open-questions}

\begin{table}[H]
    \centering
    \begin{tabular}{ >{\raggedright\bfseries}p{0.2\textwidth} p{0.7\textwidth} }
        IQ-B0 (Fünf) &
        Welcher der Schritte, die Sie bei der Konfiguration Ihrer Umgebung (IQ-A2) durchlaufen, ist Ihrer Meinung nach der wichtigste für den Erfolg eines Projekts? \newline
        [Schritte aus IQ-A2] \\
        \hline
        IQ-B1 (Vier) &
        Wie viele Zeitressourcen benötigen Sie typischerweise für das Onboarding neuer Entwickler im Hinblick auf Ihre Toolchains? \newline
        [Ressourcen in h] \\
        \hline
        IQ-B2 (Fünf) &
        Welcher der Schritte (IQ-A2), die Sie genannt haben, würde durch eine Optimierung oder Automatisierung die meisten Zeitressourcen einsparen? \newline
        [Schritte aus IQ-A2] \\
        \hline
        IQ-B3 (Vier) &
        Gibt es manuelle Schritte in Ihrer Development- oder Deploymentumgebung, deren Automatisierung nicht möglich oder nicht sinnvoll ist beziehungsweise die eine Automatisierung ab einem bestimmten Punkt blockieren? Falls ja, welche? \newline
        [ja (mit Ergänzung) oder nein] \\
        \hline
        IQ-B4 (Fünf) &
        Welche drei Ziele würden Sie verfolgen, wenn Sie eine Toolchain für ein neues Projekt im eingangs beschriebenen Projektkontext entwerfen müssten? \newline
        [Liste aus Zielen] \\
    \end{tabular}
\end{table}

\clearpage

\subsection{Geschlossene Fragen}
\label{subsec:AA-01-03_closed-questions}

\begin{table}[H]
    \centering
    \begin{tabular}{ >{\raggedright\bfseries}p{0.2\textwidth} p{0.7\textwidth} }
        IQ-C0 (Eins) &
        Wie groß ist Ihrer Meinung nach der Anteil an manuellen Schritten (IQ-B3), deren Automatisierung nicht möglich oder nicht sinnvoll ist? \newline
        [Wert zwischen 0 \% und 100 \%] \\
        \hline
        IQ-C1 (Eins) &
        Was denken Sie, wie gut könnten einzelne Komponenten Ihrer Toolchains (wie beispielsweise ein Paketmanager oder ein Pipeline-Tool) ausgetauscht werden, also wie flexibel sind die Toolchains? \newline
        [Skala von 1 (sehr wenig austauschbar) bis 5 (sehr flexibel)] \\
        \hline
        IQ-C2 (Drei) &
        Wie häufig müssen Sie Anpassungen an Ihrer Development- oder Deploymentumgebung vornehmen, um neuen Projektanforderungen gerecht zu werden? \newline
        [Skala von 1 (sehr selten) bis 5 (sehr häufig)] \\
        \hline
        IQ-C3 (Drei) &
        Wie regelmäßig aktualisieren Sie die verwendeten Tools und Bibliotheken in Ihrer Umgebung? \newline
        [Skala von 1 (selten oder nie) bis 5 (sehr regelmäßig)] \\
        \hline
        IQ-C4 (Vier) &
        An welcher Stelle wird Ihre Software in der Regel nach Anpassungen das erste Mal ausgeführt? \newline
        [Auswahl zwischen \q{lokal}, \q{Pipeline} und \q{X-Stage}] \\
    \end{tabular}
\end{table}

\clearpage

\subsection{Bewertungsmöglichkeit für Anforderungen}
\label{subsec:AA-01-04_evaluation-requirements}

\textbf{(Fünf)}

Einzelne Faktoren des Konzepts der \q{Twelve-Factor-App} erlauben Rückschlüsse auf weitere im Rahmen dieser Arbeit vorgestellte Konzepte. Es lassen sich Anforderungen an Development- und Deploymentumgebungen von ihnen ableiten. Die Fragestellung hinter diesem Fragenbereich liegt darin, die abgeleiteten Anforderungen zu bewerten. 

Dafür stehen insgesamt 6 Punkte zur Verfügung, die frei auf die einzelnen Faktoren verteilt werden können (also beispielsweise 6x1, 1x3 + 1x2 + 1x1, 2x3, 1x6, ...).

\newcounter{factornoappendix}
\setcounter{factornoappendix}{-1}
\newcommand{\factornumberappendix}{\stepcounter{factornoappendix}\Roman{factornoappendix}}
\begin{longtable}{  |   >{\raggedleft\factornumberappendix}p{0.025\textwidth}   % Number (centered)
                        >{\raggedright\bfseries}p{0.175\textwidth}              % Factor (left-aligned)
                    |   >{\raggedright}p{0.600\textwidth}                       % Requirements and linking Concepts (left-aligned)
                    |   >{}p{0.100\textwidth}                                   % Points (centered)
                    | }
    \hline
        & \upshape\textbf{Faktor} 
        & \upshape\textbf{Anforderungen und verknüpfte Konzepte}
        & \upshape\textbf{Punkte} \\
    \hline \hline
    \endhead
    \hline
    %   (I) Codebase
        & Codebase
        & \textit{GitOps} \textrightarrow Repository als \q{Single Source of Truth}
        & ~ \\
    \hline
    %   (II) Dependencies
        & Dependencies
        & \textit{GitOps} \textrightarrow Dependencies deklarativ in Konfigurationsdateien
        & ~ \\
    \hline
    %   (III) Config
        & Config
        & \textit{Dotfiles} \textrightarrow Konfiguration in \texttt{.env} files außerhalb des Repositories
        & ~ \\
    \hline
    %   (IV) Backing Services
        & Backing Services
        & \textit{Dotfiles} \textrightarrow Zugang zu Diensten über Config
        & ~ \\
    \hline
    %   (V) Build, Release, Run
        & Build, Release, Run
        & \textit{DevOps} \textrightarrow CI/CD Pipelines, \newline
          \textit{Dotfiles} \textrightarrow Config in Release Stage
        & ~ \\
    \hline
    %   (VI) Processes
        & Processes
        & -/-
        & ~ \\
    \hline
    %   (VII) Port Binding
        & Port Binding
        & \textit{GitOps} \textrightarrow Dependency Declaration für eingebundene Webserver, \newline
          \textit{Dotfiles} \textrightarrow Konfiguration von Ports
        & ~ \\
    \hline
    %   (VIII) Concurrency
        & Concurrency
        & -/-
        & ~ \\
    \hline
    %   (IX) Disposability
        & Disposability
        & \textit{Best Containerization Practices} \textrightarrow kompakte Gestaltung von Docker Images
        & ~ \\
    \hline
    %   (X) Dev / Prod Parity
        & Dev / Prod Parity
        & Toolchain-as-Code Strategie
        & ~ \\
    \hline
    %   (XI) Logs
        & Logs
        & \textit{Best Containerization Practices} \textrightarrow Prozesse loggen nach \texttt{stdout}
        & ~ \\
    \hline
    %   (XII) Admin Processes
        & Admin Processes
        & -/-
        & ~ \\
    \hline
\end{longtable}
\vspace{1em}
\setcounter{factornoappendix}{0}
