\section{Interviewpersonen}
\label{sec:AA-02_interview-persons}

Die eindutige Identifizierung der Interviewpersonen ist über die folgende Kombination möglich:

\begin{quote}
    \begin{verbatim}
        IP-<Personennummer>
    \end{verbatim}
\end{quote}

Die \texttt{Personennummer} ist eine fortlaufende Nummer beginnend bei \texttt{0}, die die Position der Person entsprechend der Reihenfolge der Interviews angibt.

\begin{longtable}{  |   >{\bfseries}p{0.100\textwidth}                  % Identifiert
                        >{\raggedright\bfseries}p{0.225\textwidth}      % Name
                    |   >{\raggedright}p{0.325\textwidth}               % Department
                    |    p{0.250\textwidth}                             % Field(s) of Activity
                    | }
    \hline
        & \upshape\textbf{Name} 
        & \upshape\textbf{Abteilung} 
        & \upshape\textbf{Tätigkeitsfeld(er)} \\
    \hline \hline
    \endhead
    \hline
        IP-0
        & Interviewperson 0
        & Software Development Center
        & 
        \begin{itemize}
            \item Development
            \item Deployment
            \item Operations
        \end{itemize} \\
    \hline
        IP-1
        & Interviewperson 1
        & Application Management
        & 
        \begin{itemize}
            \item Development (eher)
            \item Deployment
            \item Operations
            \item Support
        \end{itemize} \\
    \hline
        IP-2
        & Interviewperson 2
        & Application Management
        & 
        \begin{itemize}
            \item Development
            \item Deployment
        \end{itemize} \\
    \hline
        IP-3
        & Interviewperson 3
        & Software Development Center
        & 
        \begin{itemize}
            \item Development
        \end{itemize} \\
    \hline
\end{longtable}
