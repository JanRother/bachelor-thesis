\section{Überblick und Zusammenfassung der Ansätze}
\label{sec:03-06_overview-and-summary-of-approaches}

Das in \autoref{ch:02_technological-environment} beschriebene \hyperref[ch:02_technological-environment]{technische Umfeld} unterliegt einer stetigen Weiterentwicklung. Ziel im Software Engineering ist es, die neuen Möglichkeiten möglichst tief in bestehende Zusammenarbeits"=, Entwicklungs"= und Architekturmodelle zu integrieren. Viele aufkommende Konzepte liefern Vorschläge zur Beschleunigung von Softwareevolution und ihrer Auslieferung, decken jedoch unterschiedliche Bereiche ab und verfolgen verschiedene Ansätze.

\textbf{\hyperref[sec:03-01_devops]{DevOps}} soll die Lücke zwischen \Gls{development} und Operations schließen. Es handelt sich dabei jedoch um kein spezifisches Vorgehensmodell oder Rahmenwerk, sondern eher um eine Sammlung aus Methodiken. Eine sehr verbreitete dieser Methodiken ist \acrfull{cicd}, welches in vielen großen Unternehmen eingesetzt wird. DevOps wird teilweise unterschiedlich ausgelegt und praktiziert. Leider liefert es für den Bereich der Toolchains noch keine einheitlichen Lösungen.

\textbf{\hyperref[sec:03-03_gitops]{GitOps}} kann zwar als Weiterentwicklung von \hyperref[sec:03-01_devops]{DevOps} betrachtet werden, ergänzt es jedoch eher als es zu ersetzen. Das Konzept stellt einen deklarativen Ansatz vor, um die Herausforderungen im Toolchain Bereich zu lösen und hat eine vollständige Automatisierung der Aufgaben im \Gls{development} und im \Gls{deployment} von sowohl der Applikation selbst als auch insbesondere ihrer Zielumgebung als Idealziel. Wie der Name bereits vermuten lässt, nutzt GitOps \textit{\Gls{git}} als zentrales Tool in allen Bereichen. Es liefert auf Basis dieses \Gls{vcs} konkrete Ansätze zu Architekturen der Toolchain, fokussiert sich jedoch hauptsächlich auf Deploymentumgebungen und das \Gls{deployment}. Für Developmentumgebungen stellt es keinen Ansatz bereit.

\textbf{\hyperref[sec:03-04_dotfiles]{Dotfiles}} beschreibt ein Konzept aus dem stark individualisierten Raum von Entwicklern. Es zielt hauptsächlich auf den Bereich \Gls{development} ab und kann dabei helfen, eine weitestgehende Reproduzierbarkeit und Wiederverwendbarkeit von Developmentumgebungen zu erreichen. Meist wird darunter ein vollständiges und dediziertes Repository aus Konfigurationsdateien verstanden, das Konzept lässt sich allerdings auch in andere Ansätze integrieren.

Die \textbf{\hyperref[sec:03-05_concept-of-twelve-factor-app]{Twelve-Factor-App}} ist eine Methodik zur Entwicklung moderner und skalierbarer Softwareanwendungen. Sie definiert zwölf spezifische Prinzipien, Faktoren genannt, die Anwendungen für die Cloud optimieren sollen. Neun dieser Prinzipien stützen sich ganz oder in Teilen auf den Bereich Toolchain. Die Faktoren konzentrieren sich auf die Entwicklung wartbarer, skalierbarer und portierbarer Anwendungen und weisen an einigen Stellen Schnittmengen mit Elementen anderer in dieser Arbeit vorgestellter Konzepte auf.

Trotz vieler guter Lösungsansätze sind einige Probleme offen geblieben und es bestehen weitere Verbesserungspotentiale. Im Rahmen von \nameref{ch:05_toolchain-as-code} soll eine \nameref{sec:05-02_strategy-for-toolchains} entwickelt werden, die auch diese auf das \Gls{development} bezogenen Herausforderungen betrachtet. Zu berücksichtigende Aspekte bei der Verwendung von \nameref{sec:02-03_containerization} sollen in Form von \nameref{sec:05-03_best-practices-with-docker-and-docker-compose} ebenfalls noch untersucht werden. Diese Punkte werden aufbauend auf weitere Untersuchungen zu einem späteren Zeitpunkt in dieser Arbeit nochmals aufgegriffen.

Dieses \autoref{ch:03_examination-of-existing-approaches} beantwortet die \acrlong{rq} \textbf{RQ-1} (siehe \autoref{sec:01-03_objectives-and-research-questions}).
