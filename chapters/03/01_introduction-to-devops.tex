\section{Einführung in DevOps}
\label{sec:03-01_introduction-to-devops}

Zusammen über die Hälfte der Entwickler gaben 2023 an, in in die Entwicklung von Infrastruktur (43 \%) und in DevOps (12 \%) involviert zu sein \cite{207:Developer-Ecosystem}. DevOps ist eine Methodik im Software Engineering, deren Ziel es ist, die Lücke zwischen den beiden Bereichen \Gls{development}, also der Entwicklung von Quellcode, und Operations, also dem Betrieb der entwickelten Software, zu schließen. Dabei legt sie klare Schwerpunkte auf Kommunikation und Zusammenarbeit, \acrlong{ci}, Qualitätssicherung und automatisiertes Deployment. DevOps selbst ist jedoch keine spezifische Methodik, sondern viel mehr ein Verbund aus verschiedenen Praktiken. \cite{001:DevOps-Adoption-in-Software-Development}

DevOps betrachtet stets den gesamten Prozess der Erschaffung von Software, zentrale Strategien bauen fast immer auf agilen Konzepten auf \cite{001:DevOps-Adoption-in-Software-Development}. Die zentrale Mission ist die Auslieferung von Software in kürzeren Abständen und in weniger Zeit \cite{006:Prevalence-of-GitOps-DevOps-in-Fast-CI-CD-Cycles}. Dies soll gelingen, obwohl \Gls{development} und Operations oft gegenläufige Ziele haben: Das \Gls{development} möchte möglichst schnell neue Funktionalitäten und Änderungen in der Software für den Kunden bereitstellen, während Operations Wert auf zuverlässige und sichere Software legen, welche jedoch durch regelmäßige Änderungen vulnerabler wird. Unter DevOps werden die Aufgaben und Verantwortungen für Software innerhalb eines Teams geteilt. Das betrifft alle Bereiche, von \Gls{development} bis \Gls{deployment}, wodurch auch \Gls{development} und Operations enger zusammenarbeiten sollen. \cite{000:CI-CD-Deployment-in-DevOps-reduce-Gap-Developer-Operation}

Die Integration zwischen \Gls{development} und operationellem \Gls{deployment} muss kontinuierlich sein. Ein Paper von \citeauthor{005:Continous-Software-Engineering-and-Beyond} identifiziert drei Bereiche, in denen sogenannte \q{Continuous}-Aktivitäten vorkommen: Business Strategy, \Gls{development} und Operations -- also von der Planung, über die Entwicklung bis hin zum Betrieb sollte Kontinuität gegeben sein. \cite{005:Continous-Software-Engineering-and-Beyond} Hier zeigt sich, dass Continuous-Aktivitäten eine wichtige Komponente von DevOps sind \cite{000:CI-CD-Deployment-in-DevOps-reduce-Gap-Developer-Operation}. Eine Aktivität ist \q{continuous}, wenn sie ein gleiches Muster konsistent und systematisch wiederholt, in DevOps typischerweise einzelne Schritte der Entwicklung und Implementierung, des Betriebs oder der Qualitätssicherung \cite{007:Analysis-of-Declarative-and-Pull-based-Deployment-Models-on-GitOps}. Eine der bekanntesten Aktivitäten, die diese Kriterien erfüllt, ist \acrfull{ci}. Sie umfasst zusammenhängende Schritte wie die Kompilierung von Quellcode, die Ausführung von Tests, die Prüfung auf Einhaltung von Standards und das Bauen von Paketen für den \Gls{deployment}-Bereich. Eine große Rolle spielt die Regelmäßigkeit der Integration. Je frequenter integriert wird, desto schneller erhalten Entwickler Feedback zu getätigten Änderungen. Schlägt sie fehl, so sollten Artefakte wie zum Beispiel Logs möglichst transparent und übersichtlich bereitgestellt werden. Dies unterstützt Entwickler dabei, in kurzer Zeit Lösungen für die Ursachen des Problems zu finden und es zu beheben. Häufigere \Glspl{release}, eine gesteigerte Vorhersagbarkeit und eine verbesserte Kommunikation in produktiveren Entwicklungsteams sind nur ein paar der Vorteile, die DevOps im Software Engineering hat. \acrfull{cd} und \acrfull{cde} bauen auf \Gls{ci} auf. \acrshort{cd} stellt die gebaute und validierte Software automatisch in einer für sie vorkonfigurierten Umgebung bereit und \acrshort{cde} macht sie zusätzlich für den Kunden verfügbar. Dabei ist \Gls{deployment} eine zwingende Vorbedingung für \Gls{delivery}, aber nicht notwendigerweise vice versa. Beide sind abhängig von den Artefakten der (erfolgreichen) Integration. \cite{005:Continous-Software-Engineering-and-Beyond} Die Kombination aus \Gls{ci} und \Gls{cd} ist \acrfull{cicd}. \Gls{cicd} ist ein sehr weit verbreitetes Konzept und zählt zu den Best Practices von DevOps. Eine typische Organisation von Softwareaktivitäten entlang eines DevOps Prozesses kann \autoref{fig:g-05_devops-workflow} entnommen werden.

\begin{figure}[h]
    \centering
    \includegraphics[width=0.95\textwidth]{g-05_devops-workflow.png}
    \caption{DevOps Prozessablauf (vereinfacht) \acrshort{iAa} \citeauthor{008:GitOps-Approach-to-Cloud-Cluster-System-Deployment}}
    \label{fig:g-05_devops-workflow}
\end{figure}

Der DevOps Ansatz folgt einigen ihm zugrundeliegenden Prinzipien, an denen sich bei der Implementierung orientiert werden kann. Softwareentwicklung mit DevOps findet in Iterationen und mit Inkrementen statt. Wiederkehrende Aufgaben werden automatisiert und Teams werden befähigt, alle notwendigen Aufgaben entlang des Lebenszyklus der Software selbst auszuführen, was eine Alternative zur Verteilung der Verantwortung darstellt und breit aufgestellte Entwickler hervorbringt. Kollaboration ist ein weiteres wichtiges Prinzip. \cite{009:GitOps-Evolution-of-DevOps} Folgen Teams diesen Prinzipien, kann eine Implementierung des Ansatzes gelingen und eine Reihe von Vorteilen mit sich bringen, von denen sowohl Entwickler als auch Kunden profitieren. DevOps erhöht die Frequenz mit der Software ausgeliefert werden kann, ohne zu Einschnitten in der Qualität des Produkts zu führen. Tatsächlich kann diese sich bei einem konsequenten \Gls{ci} Konzept sogar steigern. Insbesondere bei Administration und Wartung kann mit einer nicht unwesentlichen Zeitersparnis gerechnet werden, wodurch sich auch die Kosteneffizienz verbessert. Kommt es zu unerwarteten Fehlern, ist außerdem das Zurückrollen der Software auf eine vorherige, stabile Version deutlich einfacher. Die Einbindung der Continous-Praktiken ist für die Erreichung aller Vorteile unumgänglich. Sie können zum Beispiel durch Automatisierung von Pipelines umgesetzt werden. Dabei sind \acrfull{ci}, \acrfull{cd} und \acrfull{cde}, in dieser Reihenfolge, auf den ersten drei Plätzen bezüglich ihrer Verbreitung. \cite{001:DevOps-Adoption-in-Software-Development}

Das Konzept \Gls{iac} kann ebenfalls Teil von DevOps sein und kommt häufig in der Literatur vor \cite{001:DevOps-Adoption-in-Software-Development}. Es handelt sich dabei um ein Best Practice von DevOps, welches die Verwaltung und Provisionierung der Infrastruktur mit Hilfe von Code ermöglicht. Vorteile von \Gls{iac} sind unter anderem konsistente und skalierbare Infrastrukturlösungen, Versionierbarkeit involvierter Konfigurationen und Reproduzierbarkeit von Ergebnissen. \cite{012:Compare-and-Contrast-various-Software-Development-Methodologies}

Neben diesen Konzepten gibt es weitere Best Practices bei der Verwendung von DevOps, darunter auch Continuous Monitoring, welches sich unter den fünf meist erwähnten in der Literatur befindet. Es beschreibt vor allem Feedbackmechanismen. Entwicklungsteams werden dadurch befähigt, kontinuierlich zu iterieren und Verbesserungen der Applikation einzupflegen. Continous Monitoring hat dadurch einen positiven Einfluss auf die Robustheit von Systemen. \cite{012:Compare-and-Contrast-various-Software-Development-Methodologies}

DevOps ist mittlerweile eine Methodik, die in den meisten modernen Unternehmen zum Einsatz kommt \cite{020:Assessing-and-Improving-Quality-of-Docker-Artifacts}. Konzerne wie \textit{Google}, \textit{Apple} oder \textit{Amazon} haben die beschriebenen Praktiken erfolgreich implementiert und profitieren von ihnen \cite{001:DevOps-Adoption-in-Software-Development}.
