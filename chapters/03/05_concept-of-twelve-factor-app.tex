\section{Konzept der Twelve-Factor-App}
\label{sec:03-05_concept-of-twelve-factor-app}

Die \q{Twelve-Factor-App} ist eine Methodik für Webanwendungen, so genannte Software-as-a-Service Applikationen. Sie ist unabhängig von der verwendeten Programmiersprache und auch für Applikationen bestehend aus mehreren Services anwendbar. \cite{101:The-Twelve-Factor-App}

Eine Twelve-Factor-App erfüllt die folgenden von \citeauthor{103:Creating-Cloud-native-applications-12-Factor-Applications} beschriebenen Eigenschaften \cite{103:Creating-Cloud-native-applications-12-Factor-Applications}:

\begin{itemize}
    \item Zur Automatisierung von Softwareprozessen verwendet sie ein \textbf{deklaratives Paradigma}.
    \item Sie ermöglicht die \textbf{Portabilität} der Software über verschiedene Umgebungen hinweg.
    \item \textbf{Deploybarkeit} in modernen Cloud Plattformen ist gegeben.
    \item Sie \textbf{reduziert Divergenzen} zwischen Entwicklungs- und Produktivumgebungen.
    \item Ohne größere Anpassungsbedarfe kann sie beliebige \textbf{Skalierbarkeit} erreichen.
\end{itemize}

Um diese Eigenschaften zu erfüllen, werden zwölf namensgebende Faktoren beschrieben, die in der Applikation umgesetzt sein müssen \cite{101:The-Twelve-Factor-App,102:Twelve-Factor-App-Revisited}.

\vspace{1em}
\newcounter{factorno}
\setcounter{factorno}{-1}
\newcommand{\factornumber}{\stepcounter{factorno}\Roman{factorno}}
\begin{longtable}{  |   >{\raggedleft\factornumber}p{0.025\textwidth}   % Number (centered)
                        >{\raggedright\bfseries}p{0.175\textwidth}      % Factor (left-aligned)
                    |   >{\raggedright\itshape}p{0.150\textwidth}       % Description (left-aligned)
                    |    p{0.550\textwidth}                             % Basics (block)
                    | }
    \hline
        & \upshape\textbf{Faktor} 
        & \upshape\textbf{Beschreibung} 
        & \upshape\textbf{Grundprinzipien} \\
    \hline \hline
    \endhead
    \hline
    %   (I) Codebase
        & Codebase
        & \q{One codebase tracked in revision control, with multiple deploys.}
        & Eine \Gls{codebase} \glsdesc{codebase}. Ein Deploy beschreibt die laufende Instanz einer Applikation in einer Produktivumgebung, einer Stage oder der lokalen Entwicklungsumgebung. Die Applikation sollte genau eine Codebase haben, aus der beliebig viele Deploys hervorgehen. \\
    \hline
    %   (II) Dependencies
        & Dependencies
        & \q{Dependencies must be explicitly declared and isolated.}
        & Für Bezug und Installation von Programmierbibliotheken sollten Online Repositories verwendet werden. Abhängigkeiten werden deklarativ in einem Manifest angegeben und isoliert vom umgebenden System installiert. \\
    \hline
    %   (III) Config
        & Config
        & \q{Store configuration in the environment, not in the code.}
        & Die Konfiguration kann sich zwischen verschiedenen Deploys unterscheiden, beispielsweise bei Credentials oder Variablen wie dem Namen des Hosts. Die Speicherung solcher flexiblen Konfigurationen erfolgen nicht im Code direkt, sondern in \q{Config}-Dateien außerhalb der Revision, beispielsweise in Umgebungsvariablen und \texttt{.env}-Dateien. \\
    \hline
    %   (IV) Backing Services
        & Backing Services
        & \q{Treat backing services as attached resources.}
        & Ein Backing Service ist ein Dienst, der durch die Applikation über das Netzwerk konsumiert wird, wobei nicht zwischen lokalen Diensten und solchen von Drittanbietern unterschieden werden sollte. Die Austauschbarkeit dieser Dienste muss zu jedem Zeitpunkt gewährleistet sein und ihr Zugang darf nicht in der Config gespeichert sein. \\
    \hline
    %   (V) Build, Release, Run
        & Build, Release, Run
        & \q{Strictly separate build, release, and run stages.}
        & Stages für \Gls{build}, \Gls{release} und \Gls{run} sollten getrennt werden. \Gls{build} \glsdesc{build}. \Gls{release} \glsdesc{release}. \Gls{run} \glsdesc{run}. Rollbacks der Software sind nur über vorherige \Glspl{release} möglich, aber nicht über Änderungen am Quellcode in Produktivumgebungen. \Glspl{release} sind eindeutig identifizierbar. \\
    \hline
    %   (VI) Processes
        & Processes
        & \q{Execute the app as one or more stateless processes.}
        & Die Applikation läuft als einer oder mehrere zustandslose Prozesse, die einer Share-Nothing-Architektur folgen. Zustände sind in den Backing Services gespeichert. \\
    \hline
    %   (VII) Port Binding
        & Port Binding
        & \q{Export services via port binding.}
        & Erreichbar sollte die Applikation beispielsweise über \Gls{http} oder \Gls{https} auf einem freigegebenen Port sein, auf dem sie auf Anfragen wartet. Lokale Entwicklungsumgebungen nutzen den \texttt{localhost}. Die Applikation kann dadurch als Backing Service von anderen Applikationen konsumiert werden. \\
    \hline
    %   (VIII) Concurrency
        & Concurrency
        & \q{Scale out via the process model.}
        & Die Architektur der Applikation orientiert sich am \textit{Unix Process Model}. Skalierung erfolgt über die Anzahl laufender Prozesse, Verwaltung des Workloads über Prozesstypen. \\
    \hline
    %   (IX) Disposability
        & Disposability
        & \q{Maximize robustness with fast startup and graceful shutdown.}
        & Prozesse können zu jedem Zeitpunkt gestartet oder gestoppt werden. Dies führt zu mehr Skalierbarkeit, schnellerem \Gls{deployment} von Änderungen an Code oder Config sowie einem robusten \Gls{deployment} in Produktivumgebungen. Alle Prozesse sollten eine minimale Zeit zum Hochfahren benötigen und gegen unerwartetes Beenden abgesichert sein. \\
    \hline
    %   (X) Dev / Prod Parity
        & Dev / Prod Parity
        & \q{Keep development, staging, and production as similar as possible.}
        & Die Applikation strebt eine Reduzierung der Lücken zwischen \Gls{development} und \Gls{deployment} an, um die vergehenden Zeit zwischen ihnen zu verringern. Verantwortlich für beide Bereiche sollen dabei möglichst ähnliche Personen oder Personengruppen sein. Development- und Deploymentumgebungen sollten dementsprechend möglichst gleich sein. Damit ist auch die Gleichheit der Backing Services in den verschiedenen Stages gemeint. \textit{Docker} als virtuelle Umgebung kann hierbei unterstützen. \\
    \hline
    %   (XI) Logs
        & Logs
        & \q{Treat logs as event streams.}
        & Logs sollten drei Eigenschaften erfüllen: Aggregation, Chronologie und das Auftreten in Textform. Eine Zeile sollte dabei ein Ereignis beschreiben. Die Applikation selbst sollte keine Logs schreiben, dies ist Aufgabe der einzelnen Prozesse. Sie protokollieren alle Ereignisse in \texttt{stdout}, sodass Zielspeicher und Ausgabeorte durch die Deploymentumgebung definiert werden können. \\
    \hline
    %   (XII) Admin Processes
        & Admin Processes
        & \q{Run admin or management tasks as one-off processes.}
        & Administrative Prozesse dienen der Verwaltung der Applikation. Sie sollten in der gleichen Umgebung ausgeführt werden wie alle anderen Aufgaben. \\
    \hline
    \caption{Faktoren der \q{Twelve-Factor-App} mit Kurzbeschreibung}
    \label{tab:ftwelve-factor-app-factors}
\end{longtable}
\vspace{1em}
\setcounter{factorno}{0}

Die Technologien im Cloud Bereich erfahren eine stetige Weiterentwicklung. Die Methodik der Twelve-Factor-App ist weiterhin noch anwendbar, wurde jedoch ausgeweitet, sodass mittlerweile das Konzept einer \q{15-Factor-App} existiert. Änderungen an den bestehenden Faktoren wurden dort nicht vorgenommen, jedoch werden drei zusätzliche Faktoren vorgestellt: \textbf{API First}, \textbf{Telemetry} sowie \textbf{Authentication and Authorization}. \cite{104:15-Factor-Cloud-native-Java-Applications} Diese neuen Faktoren haben einen Einfluss auf die Architektur der Applikation selbst, jedoch nicht auf deren Toolchain. Daher wird die 15-Factor-App im Rahmen dieser Arbeit nicht weiter beleuchtet.
