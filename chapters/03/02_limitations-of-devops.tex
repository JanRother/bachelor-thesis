\section{Grenzen von DevOps}
\label{sec:03-02_limitations-of-devops}

Trotz aller Vorteile bietet DevOps keine Lösung für alle Herausforderungen des Software Engineering und kann an Grenzen stoßen. Wissenschaftlich ist der Ansatz noch nicht stark recherchiert, unter anderem aufgrund seiner unklaren Definition \cite{009:GitOps-Evolution-of-DevOps}. Seine Anwendung auf bestehende Systeme ist nicht trivial, oft muss mit Einbußen bei Sicherheit oder Performanz gerechnet werden \cite{000:CI-CD-Deployment-in-DevOps-reduce-Gap-Developer-Operation}. Auch für \Glspl{legacy-system} kann DevOps nicht ohne Anpassungen funktionieren. Eine besondere Schwierigkeit besteht darin, Personen zu finden, die gleichermaßen effizient im \Gls{development} und im \Gls{deployment} eingesetzt werden können. DevOps setzt Teams und Verantwortliche voraus, die Fähigkeiten in diesen beiden beteiligten Aufgabenbereichen aufweisen. \cite{001:DevOps-Adoption-in-Software-Development} Bei Betrachtung der organisatorischen Ebene stellen historisch gewachsene Kulturen und Strukturen in Unternehmen eine Hürde dar. Sie können den effektiven Einsatz von DevOps hemmen. \cite{000:CI-CD-Deployment-in-DevOps-reduce-Gap-Developer-Operation}

Die Implementierung von DevOps ist ein nicht-trivialer Prozess, bestehend aus der Bestimmung existierender Aufgaben, der Umwandlung manueller in automatische Prozesse und der Auswahl beziehungsweise Anpassung von Tools zur Erledigung dieser Aufgaben. Herausforderungen liegen neben Architekturen, neuen Methoden und Sicherheit von Pipelines insbesondere im Bereich der Toolchains. \cite{007:Analysis-of-Declarative-and-Pull-based-Deployment-Models-on-GitOps} Hier bietet DevOps schlichtweg keine konkreten Lösungen an. Eine systematische Übersichtsarbeit von \citeauthor{001:DevOps-Adoption-in-Software-Development} zu Hindernissen bei der Implementierung von DevOps ergab, dass komplexe Technologieumgebungen eine Herausforderung darstellen. Das zentrale Problem solcher Umgebungen ist deren eingeschränkte Wiederholbarkeit. Speziell Unterschiede zwischen verschiedenen Produktivumgebungen können demzufolge die Fehlerbehebung erschweren. \cite{001:DevOps-Adoption-in-Software-Development} Insgesamt lässt sich feststellen, dass das Thema Toolchain die wohl größte Herausforderung von DevOps darstellt. Die Standardisierung und Automatisierung von Tools und Prozessen ist eine schwierige Aufgabe, ganz besonders, da Komplexität naturgemäß mit der Größe der Organisation wächst, in der ein System umgesetzt werden soll. Dass sich dort viele Plattformen und noch mehr Anwendungen im Bestand befinden, erschwert diese Aufgabe zusätzlich. \cite{000:CI-CD-Deployment-in-DevOps-reduce-Gap-Developer-Operation}

Weitere Herausforderungen bei der Einführung von DevOps bestehen, sind jedoch, anders als Technologieumgebungen oder \Glspl{legacy-system} keiner technischen Natur, sondern haben eher organisatorische oder kulturelle Hintergründe. \cite{001:DevOps-Adoption-in-Software-Development} Die vorliegende Arbeit fokussiert sich bewusst auf Ansätze und Strategien für die Neuentwicklung von Software und liefert keine Ergebnisse für \Glspl{legacy-system}. Daher besteht eines der größten Potentiale in der Reduzierung der auf die Toolchain bezogenen Herausforderungen.
