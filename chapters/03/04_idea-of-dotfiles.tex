\section{Idee von Dotfiles}
\label{sec:03-04_idea-of-dotfiles}

Verglichen mit \hyperref[sec:03-01_introduction-to-devops]{DevOps} oder \hyperref[sec:03-03_gitops-as-further-evolution]{GitOps} beschreiben Dotfiles ein relativ kompaktes Konzept. Der Grundgedanke soll an dieser Stelle dennoch eingeführt werden, da er eine wichtige Rolle bei der Konfiguration von Entwicklungsumgebungen spielt.

Das Konzept stammt ursprünglich aus Betriebssystemen basierend auf \textit{Unix} oder \textit{Linux}, wo Dateinamen einen Punkt (\texttt{.}, englisch \q{Dot}) als Suffix enthalten können. Die eigentliche Funktion dieses Suffix ist, dass Dateien mit führendem Punkt im Dateinamen standardmäßig aus der Ausgabe des Befehls \texttt{ls} zum Anzeigen der Dateien und Ordner eines bestimmten Verzeichnisses ausgeschlossen werden. Dabei hat sich durchgesetzt, dass diese Art von Dateien oftmals dazu verwendet wird, um die Umgebung eines Entwicklers oder Nutzers zu konfigurieren. \cite{029:Connecting-the-Dotfiles}

Der Schwerpunkt von Dotfiles liegt also auf Individualisierungsaspekten, ihr Ziel ist die Verbesserung der \Gls{developer-experience}. Sie bestehen in der Regel aus typischen Nutzerindividualisierungen, Konfigurationen für Dienstprogramme und plattformbezogenen Einstellungen. Im Verbund mit einem Installationsskript kann eine komplette Umgebung durch das Ausführen einer Befehlszeile aufgesetzt werden. Dabei enthalten Dotfiles die eigentlichen Konfigurationen, während das Installationsskript die Befehle zum Kopieren der Dotfiles an ihre Zielorte im Dateisystem beinhaltet. \cite{203:Dev-Environment-as-a-Code-with-DevContainers-Dotfiles-and-GitHub-Codespaces} Üblicherweise haben diese Installationsskripte einen der folgenden konventionellen Dateinamen: \texttt{install.sh}, \texttt{install}, \texttt{bootstrap.sh}, \texttt{bootstrap}, \texttt{script/bootstrap}, \texttt{setup.sh}, \texttt{setup}, \texttt{script/setup}. Sind Dotfiles in einem Repository abgelegt, können sie beispielsweise auf \Gls{github} gespeichert und verwaltet werden. \Gls{github} ist in der Lage, automatisch einen dieser Dateinamen zu erkennen und das Installationsskript selbst auszuführen, wenn es in ein von \Gls{github} bereitgestelltes Computing Environment injiziert wird. \cite{304:Personalizing-GitHub-Codespaces-for-your-Account} Einer der größten Vorteile von Dotfiles ist die Reproduzierbarkeit von Systemen und deren Konfiguration \cite{029:Connecting-the-Dotfiles}.

Eine Untersuchung zu den am häufigsten vorkommenden Dateinamen in öffentlichen Dotfile Repositories von \citeauthor{029:Connecting-the-Dotfiles} ergab, dass die häufigsten \Gls{mime} Typen \texttt{text/plain} und \texttt{image/x} sind, wobei \texttt{x} für die verschiedensten Subtypen steht. Neben \texttt{README.md} für die Dokumentation der Repositories sind \texttt{.gitignore} zur Angabe von Dateien in einem \Gls{git} Repository, welche vom \Gls{vcs} ignoriert werden sollen, \texttt{.vimrc} zur Konfiguration eines Dateieditors und \texttt{.zshrc} zur Konfiguration eines Befehlszeileninterpreters die verbreitetsten Dateinamen. Generell stellen \citeauthor{029:Connecting-the-Dotfiles} fest, dass die häufigsten Dateien \texttt{.*ignore}-, \texttt{.*rc}- oder \texttt{.*conf*}-Dateien sind. \cite{029:Connecting-the-Dotfiles}

Eingerichtet werden können mit Hilfe von Dotfiles beispielsweise \textit{Unix}- und \textit{Linux}-Systeme, das \Gls{wsl} oder \nameref{subsec:05-01-02_dev-container} \cite{203:Dev-Environment-as-a-Code-with-DevContainers-Dotfiles-and-GitHub-Codespaces}, welche in \autoref{subsec:05-01-02_dev-container} genauer vorgestellt werden.

Befragte der oben genannten Studie von \citeauthor{029:Connecting-the-Dotfiles} gaben an, dass sie Dotfile Repositories zu 53 \% nutzen würden, um schnell neue Maschinen aufzusetzen oder bestehende zu synchronisieren. Das gilt für physische genauso wie für virtuelle Maschinen. Ebenfalls etwa die Hälfte nutzten hauptsächlich \Gls{git} als Tool zur Verwaltung von Dotfiles. Die Umfrage richtete sich an insgesamt 1.650 Autoren öffentlicher Dotfile Repositories auf \Gls{github}. \cite{029:Connecting-the-Dotfiles}
