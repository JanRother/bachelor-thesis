\section{Methodik und Aufbau}
\label{sec:01-05_methodology-and-structure}

\subsection{Wissenschaftlicher Ansatz}
\label{subsec:01-05-01_scientific-approach}

Die Forschungen in dieser Arbeit sollen sich auf drei zentrale methodische Elemente stützen:

\begin{itemize}
    \begin{minipage}[t]{0.75\textwidth}
        \item Eine \textbf{Literaturzusammenstellung} als initialer Berührungspunkt mit dem Forschungsgebiet und zur Ermittlung einer Ausgangssituation.
        \item \textbf{Experteninterviews} zur Herleitung von Anforderungen an die durch die spätere Strategie abzudeckenden Bereiche.
        \item Die Entwicklung eines \textbf{Prototypen} zur Validierung des Ansatzes und um einen ersten Leitfaden zur Implementierung der Strategie zu liefern.
    \end{minipage}
\end{itemize}

\begin{comment}
    (A) Einführung -> 01, 02
    (B) DevOps und GitOps -> 03; RQ-1
    (C) Dev Container und Developer Experience -> 03; RQ-2
    (D) Anforderungen an Development und Deployment Umgebungen -> 04; RQ-0
    (E) Experten-Interview -> 04
    (F) Sicherheit, Wartbarkeit und Performanz von Docker Images -> 05; RQ-3

    (0) Paper -> 30 Stk.
    (1) Artikel -> 12 Stk.
    (2) Journale oder Blogs -> 14 Stk.
    (3) Dokumentationen -> 11 Stk.
    (4) Bücher -> 02 Stk.

    - Durchlauf eines Peer-Reviews
    - Veröffentlichung in einem renommierten Journal oder einer renommierten Konferenz
    - Aktualität
    - Methodik
    - Transparenz der Daten und Quellen
    - Anzahl an Zitationen in anderer Literatur
\end{comment}

Die \textbf{Literaturzusammenstellung} soll insbesondere die Erkenntnisse von \autoref{ch:02_technological-environment} und \autoref{ch:03_examination-of-existing-approaches} stützen. Die Methodik wird an dieser Stelle vorgegriffen. Als Hauptquelle dient die Literaturdatenbank \textit{IEEE Xplore} (\url{https://ieeexplore.ieee.org/}). Bei zu wenigen Ergebnissen über diese Quelle, wird zusätzlich \textit{Google Scholar} (\url{https://scholar.google.com}) herangezogen. Technische Themen, insbesondere solche, die sehr praxisnah oder sehr jung sind, werden häufig besser über \textit{Google} (\url{https://www.google.com/}) gefunden. Die Suchanfragen werden dabei entlang zentraler Stichworte der Arbeit aufgebaut und zuvor entlang dem \nameref{subsec:01-05-02_structure-of-the-thesis} sowie unter Berücksichtigung von \nameref{sec:01-03_objectives-and-research-questions} gruppiert. Die Gruppe \textsc{(A) Einführung} stützt die Abschnitte zu \nameref{ch:01_introduction-and-motivation} (\autoref{ch:01_introduction-and-motivation}) und \nameref{ch:02_technological-environment} (\autoref{ch:02_technological-environment}). Die Gruppen \textsc{(B) DevOps und GitOps} sowie \textsc{(C) Dev Container und Developer Experience} liefern Ergebnisse für die \nameref{ch:03_examination-of-existing-approaches} (\autoref{ch:03_examination-of-existing-approaches}) und zur Beantwortung der Forschungsfragen \textsc{(B)} \textbf{RQ-1} beziehungsweise \textsc{(C)} \textbf{RQ-2}. Zur Ermittlung von \nameref{ch:04_requirements-for-development-and-deployment-environments} (\autoref{ch:04_requirements-for-development-and-deployment-environments}) soll Gruppe \textsc{(D) Anforderungen an Development und Deployment Umgebungen} mit Gruppe \textsc{(E) Experten-Interview} als wissenschaftliche Grundlage der methodischen Unterstützung dienen. \textsc{(F) Sicherheit, Wartbarkeit und Performanz von Docker Images} soll einen Beitrag zur \nameref{ch:05_toolchain-as-code} Strategie (\autoref{ch:05_toolchain-as-code}) und zur Beantwortung der Forschungsfrage \textsc{(F)} \textbf{RQ-3} leisten. Auch die Ergebnisse der Literaturzusammenstellung wurden während der Forschung gruppiert, sodass folgende Aussage getroffen werden kann: Insgesamt wurden 69 wissenschaftliche Quellen für die Literaturzusammenstellung herangezogen, unter ihnen \textsc{(0)} 30 \textsc{Paper}, \textsc{(1)} 12 \textsc{Artikel}, \textsc{(2)} 14 \textsc{Journale oder Blogs}, \textsc{(3)} 11 \textsc{Dokumentationen} sowie \textsc{(4)} 02 \textsc{Bücher}. Der Anteil ihrer Verwendung in der Arbeit konnte dabei über ein Schulnotensystems (von 1 bis 6) erfolgen, wobei Quellen mit der Note 6 nicht eingeflossen sind. Faktoren wie das Durchlaufen eines Peer-Reviews, die Veröffentlichung in einem renommierten Journal oder einer renommierten Konferenz, die Aktualität, die Methodik, die Transparenz der Daten und Quellen sowie die Anzahl an Zitationen in anderer Literatur wurden dabei berücksichtigt.

Für die \textbf{Experteninterviews} wird ein leitfadengestütztes Interview vorbereitet. Mit insgesamt vier Experten soll dieses Interview anschließend durchgeführt werden, wobei ein Protokoll in Stichpunkten angelegt werden soll. Die konsolidierten Ergebnisse sollen anschließend bei der Herleitung von Anforderungen an die \nameref{ch:05_toolchain-as-code} Strategie dienen. Die genaue Methodik wird zuvor im Abschnitt \nameref{subsec:04-01-02_methodology} (\autoref{subsec:04-01-02_methodology}) erarbeitet werden.

Ziel des \textbf{Prototypen} ist die Erstellung eines minimalen Projekts auf Basis der erhobenen Anforderungen und der entwickelten Strategie. Dem vorgelagert wird eine begründete Auswahl von \nameref{sec:06-01_technologies-and-tools} (\autoref{sec:06-01_technologies-and-tools}) stehen. Abschließend soll es möglich sein, das entwickelte Konzept anhand des Prototypen zu bewerten.

Zwei Herangehensweisen an wissenschaftliche Arbeiten sind unter anderem möglich: ein deduktiver und ein induktiver Ansatz (siehe \autoref{fig:g-03_methodology-of-inductive-approach}). Im \textbf{deduktiven Ansatz} ist das Ziel die Aufstellung von Theorien und Hypothesen. Typische Bestandteile sind häufig Motivation, Abgrenzung, Grundlagenaufbereitung und Hypothesenbildung. Wird der \textbf{induktive Ansatz} gewählt, sind die aufgestellten Theorien und Hypothesen anhand einer praktischen Umsetzung kritisch zu reflektieren. Nach Ausdetaillierung des entworfenen Konzepts folgen dessen Umsetzung und Anwendung. So kann das erhaltene Produkt vor dem Fazit bewertet werden. \cite{400:Beitrag-Produktrepraesentation-fuer-Bedarfs-und-Kapazitaetsmanagement-digitalisierter-Fahrzeuge}

\begin{figure}[h]
    \centering
    \includegraphics[width=0.95\textwidth]{g-03_methodology-of-inductive-approach.png}
    \caption{Methodisches Vorgehen nach dem induktiven Ansatz \acrshort{iAa} \citeauthor{400:Beitrag-Produktrepraesentation-fuer-Bedarfs-und-Kapazitaetsmanagement-digitalisierter-Fahrzeuge}}
    \label{fig:g-03_methodology-of-inductive-approach}
\end{figure}

Die Umsetzbarkeit einer \nameref{ch:05_toolchain-as-code} Strategie soll möglichst deutlich und reproduzierbar dargestellt werden. Auch Praktikabilität und Nutzen sollen aus ersten Erfahrungen abgeleitet werden. Daher wurde unter anderem die Methodik des Prototypen als ein Bestandteil der Arbeit gewählt und deshalb wird diese Arbeit einem induktiven Ansatz wie in \autoref{fig:g-03_methodology-of-inductive-approach} folgen.

\subsection{Aufbau der Arbeit}
\label{subsec:01-05-02_structure-of-the-thesis}

Sowohl dieser wissenschaftliche Ansatz als auch die gewählten Methodiken finden sich deutlich im Aufbau der Arbeit und in der Kapitelstruktur wieder, wie \autoref{fig:g-04_structure-of-thesis} zeigt.

\begin{figure}[h]
    \centering
    \includegraphics[width=0.95\textwidth]{g-04_structure-of-thesis.png}
    \caption{Aufbau und Struktur der Arbeit}
    \label{fig:g-04_structure-of-thesis}
\end{figure}

Das aktuelle \autoref{ch:01_introduction-and-motivation} enthält die \nameref{ch:01_introduction-and-motivation} zu dieser Arbeit. Dazu führt es zunächst in die Problemstellung ein, nennt und motiviert erste Teilthemen und erarbeitet aus den identifizierten Problemräumen vier zentrale Forschungsfragen. Außerdem enthält es Erläuterungen zum methodischen und wissenschaftlichen Vorgehen.

\pagebreak[4]

\autoref{ch:02_technological-environment} skizziert daraufhin ein \nameref{ch:02_technological-environment} und stützt sich stark auf die \textbf{Literaturzusammenstellung}. Wichtig in diesem Abschnitt sind eine wissenschaftliche Betrachtung des Projektumfelds sowie die Begründungen hinter der Wahl einzelner Teiltechnologien. Es enthält Zusammenfassungen der Bereiche \nameref{sec:02-01_web-development}, \nameref{sec:02-02_microservices} und \nameref{sec:02-03_containerization}.

Das \autoref{ch:03_examination-of-existing-approaches} widmet sich der \nameref{ch:03_examination-of-existing-approaches}. Auch hier basieren die zusammengestellten Inhalte auf der \textbf{Literaturzusammenstellung}. Es geht um eine kritische Untersuchung erster bereits existierender Ansätze, welche Lösungen für die Problemstellungen der Arbeit bieten. Betrachtet werden die Vor- und Nachteile von \hyperref[sec:03-01_introduction-to-devops]{DevOps} und \hyperref[sec:03-03_gitops-as-further-evolution]{GitOps}. Außerdem gibt es eine Einführung in das Konzept der \hyperref[sec:03-04_idea-of-dotfiles]{Dotfiles} und in die \hyperref[sec:03-05_basic-idea-of-twelve-factor-app]{\q{Twelve-Factor-App}}.

Auf den Erkenntnissen aufbauend werden \nameref{ch:04_requirements-for-development-and-deployment-environments} in \autoref{ch:04_requirements-for-development-and-deployment-environments} ermittelt. Als Grundlage hierfür sollen \textbf{Experteninterviews} dienen. Herausgefunden werden soll, welche Anforderungen Entwickler an Toolchains in den Bereichen \Gls{development} und \Gls{deployment} sowie an die Durchgängigkeit dieser Toolchains stellen. Dazu werden Interviewfragen ausgearbeitet und die Ergebnisse strukturiert ausgewertet.

Das wichtigste Kapitel dieser Arbeit ist \autoref{ch:05_toolchain-as-code}. In \nameref{ch:05_toolchain-as-code} entsteht das Hauptergebnis der Forschungen (siehe \autoref{fig:g-03_methodology-of-inductive-approach}). Dazu werden zunächst bisherige Ergebnisse zusammengestellt, welche anschließend durch zusätzliche Recherchen zu konkreten Technologien für \nameref{sec:02-03_containerization} ergänzt werden. Ergebnis des Kapitels ist eine ausgearbeitete \nameref{ch:05_toolchain-as-code} Strategie inklusive zugehöriger Architektur und Prozessen.

Diese Strategie soll als \textbf{Prototyp} in \autoref{ch:06_prototypical-implementation-of-the-concept} umgesetzt werden. Auf eine Festlegung von Basistechnologien für die Realisierung einer minimalen Toolchain auf Basis der entworfenen Ansätze folgt dort die \nameref{ch:06_prototypical-implementation-of-the-concept}. Auf Grundlage der hierbei getroffenen Erkenntnisse soll eine Kurzbewertung der \nameref{ch:05_toolchain-as-code} Strategie ermöglicht werden.

\autoref{ch:07_conclusion-and-outlook} ist das letzte Kapitel und enthält \nameref{ch:07_conclusion-and-outlook}. Hier werden die Forschungsfragen aus \autoref{sec:01-03_objectives-and-research-questions} beantwortet. Außerdem sollen im Ausblick nicht betrachtete Aspekte des Forschungsgebiets benannt werden und es wird eine Einschätzung zu weiteren Entwicklungen auf dem Forschungsgebiet getroffen.
