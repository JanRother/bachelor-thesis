\section{Projektkontext}
\label{sec:01-02_project-context}

Der Ursprung dieser Arbeit befindet sich im \Gls{am} der \Gls{it} eines großen Automobilherstellers. Diese Abteilung ist verantwortlich für die Entwicklung, den Betrieb und die Administration von Anwendungen und Systemen, welche unter anderem für die \Gls{te} bereitgestellt werden.

Viele Softwareprojekte in diesem Bereich, insbesondere die Neuentwicklungen, folgen immer häufiger dem \nameref{sec:02-02_microservices} Architekturmuster. Ein solches Projekt soll auf Basis von Werkzeugen wie \Gls{git}, \hyperref[sec:02-03_containerization]{Docker} und \hyperref[sec:02-01_web-development]{Web-Technologien} realisiert werden. Ein zentrales Ziel ist die nachhaltige, sichere, wartbare und performante Auswahl und Aufstellung der verwendeten Technologien.

Um dies möglichst fundiert leisten zu können, sollen mit dieser Arbeit ein Konzept und ein Prototyp entwickelt werden. Dieses Konzept fokussiert sich, wie im vorherigen Absatz motiviert, primär auf Neuentwicklungen von Applikationen.

Trotz enger fachlicher Verbindungen wird diese Arbeit organisatorisch unabhängig vom Projektkontext durchgeführt. Die Erkenntnisse sind daher universell einsetzbar.
