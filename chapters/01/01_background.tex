\section{Hintergrund}
\label{sec:01-01_background}

Der Begriff \Gls{developer-experience} ist definiert als die Menge aller Erfahrungen, die Entwickler von Software mit verschiedenen Arten von Artefakten und Aktivitäten machen, die während der Entwicklung von Software durchlaufen werden \cite{017:Developer-Experience-Concept-and-Definition,100:Developer-Experience-Glueckliche-Entwickler-schreiben-besseren-Code}. \q{Experience} als Teilbegriff bezieht sich dabei nicht auf die Erfahrenheit der Beteiligten, sondern vielmehr darauf, wie ihre subjektive Wahrnehmung der Arbeit an einem Projekt ist. Darunter fallen Bereiche wie die Entwicklungsinfrastruktur, aber auch Zusammenarbeit und Wertschätzung \cite{017:Developer-Experience-Concept-and-Definition}. Wird die Infrastruktur betrachtet, so spielen viele Faktoren eine Rolle, unter ihnen Entwicklungs- und Verwaltungswerkzeuge, Programmiersprachen, Programmierbibliotheken, Plattformen und Methoden \cite{100:Developer-Experience-Glueckliche-Entwickler-schreiben-besseren-Code}. Dabei umfasst die kognitive Dimension von \Gls{developer-experience} die Wahrnehmung dieser Entwicklungsinfrastruktur. Hierzu zählen beispielsweise die Interaktionen mit Entwicklungswerkzeugen oder die Ausführung von Softwareprozessen \cite{017:Developer-Experience-Concept-and-Definition}. Ein Indikator, der in gewissem Maß als Metrik für \Gls{developer-experience} dienen kann, ist die Zeit, die benötigt wird, um Änderungen am Quellcode in der daraus resultierenden Software widerzuspiegeln. Dieser Wert hat einen Einfluss auf die Geschwindigkeit, mit der Entwickler Fehler identifizieren und beheben können \cite{100:Developer-Experience-Glueckliche-Entwickler-schreiben-besseren-Code}.

Bei Softwareentwicklung handelt es sich um eine komplexe und kognitiv herausfordernde Aktivität bestehend aus vielen verschiedenen Phasen \cite{014:Managing-Container-based-Software-Development-Environments}. In den letzten Jahren lassen sich immer wieder rasante Entwicklungen in dieser Disziplin beobachten. Das Feld beschreibt nicht mehr nur die Entwicklung von Quellcode, sondern vielmehr die Erstellung ganzer Produkte. Hinter dem Begriff Software Engineering steht eine komplexe Wertschöpfungskette. Insbesondere das \nameref{sec:02-01_web-development} entwickelt sich in einer rasanten Geschwindigkeit, deutlich steiler als das Software Engineering selbst \cite{026:Some-Trends-in-Web-Application-Development}. Die starken Entwicklungen digitaler Informationstechnologien bleiben jedoch nicht folgenlos, sondern tragen maßgeblich zu einem Anstieg der Komplexität in der Softwareentwicklung bei. Software selbst wird mittlerweile als das komplexeste Artefakt moderner Computertechnik bezeichnet. Dies hat zur Konsequenz, dass Entwicklungsinfrastrukturen permanent angepasst werden und Softwareumgebungen nur noch selten ausreifen \cite{018:Software-Development-Productivity}.

Die Softwarekonzerne \textit{GitKraken} und \textit{JetBrains} haben eine Untersuchung zur Zusammenarbeit in Entwicklungsteams durchgeführt (siehe \autoref{fig:g-00_git-collaboration-report}). Ergebnis sind unter anderem signifikante Hürden, die die Produktivität von Entwicklern negativ beeinflussen. Über ein Drittel der Entwickler benannte zu viele Kontextwechsel als größte Herausforderung \cite{213:2024-State-of-Git-Collaboration}. Um welche konkrete Art von Kontext es sich handelt, geht aus der Umfrage nicht hervor. Sowohl der Wechsel organisatorischen als auch technischen Kontexts kann also ein Hindernis sein. Diesbezüglich zeigt \autoref{fig:g-00_git-collaboration-report} außerdem, dass Probleme mit Infrastruktur oder den verwendeten Technologien als produktivitätshämmende Faktoren aufgeführt wurden \cite{213:2024-State-of-Git-Collaboration}.

\pagebreak[4]

\begin{figure}[h]
    \centering
    \begin{minipage}[b]{0.39\textwidth}
        \centering
        \includegraphics[width=\textwidth]{g-00_git-collaboration-report.png}
        \caption{Ergebnis des Git Collaboration Report 2024 zu Hindernissen in der Softwareentwicklung \cite{213:2024-State-of-Git-Collaboration}}
        \label{fig:g-00_git-collaboration-report}
    \end{minipage}
    \hfill
    \begin{minipage}[b]{0.59\textwidth}
        \centering
        \includegraphics[width=\textwidth]{g-01_stack-overflow-developer-survey-2024.png}
        \caption{Ergebnis der Stack Overflow Developer Survey 2024 zu typischen Problempunkten bei der Softwareentwicklung \cite{212:Developer-Survey}}
        \label{fig:g-01_stack-overflow-developer-survey-2024}
    \end{minipage}
\end{figure}

Die größte weltweite Gemeinschaft von privaten und professionellen Softwareentwicklern, \textit{Stack Overflow}, führt jährlich eine Umfrage zu vielen Bereichen des Software Engineering durch. Alleine 2024 haben über 65 Tausend Entwickler teilgenommen und Fragen aus insgesamt sieben Bereichen beantwortet. Dass ein eigener Bereich für \Gls{developer-experience} erst 2022 neu eingeführt wurde, zeigt deutlich den Anstieg an Relevanz in diesem Thema \cite{212:Developer-Survey}. In der Umfrage aus 2024 (siehe \autoref{fig:g-01_stack-overflow-developer-survey-2024}) gaben jeweils etwa 33 \% der Entwickler an, dass die Komplexität der Technologien für den \Gls{build} von Software und jene für das \Gls{deployment} zu Frustration bei der Entwicklung führten. Ähnlich viele empfänden Frustration bei der Verlässlichkeit von Werkzeugen und Systemen und etwa 23 \% geben die alleinige Anzahl der verwendeten Softwarewerkzeuge als Problem an \cite{206:Developer-Survey-2024}. Diese Werkzeuge spielen eine elementare Rolle in der Softwareentwicklung, wo sie Entwickler bei der Schaffung qualitativ hochwertiger Software unterstützen. Zu ihnen zählen unter anderem \glspl{ide} und \glspl{vcs} \cite{014:Managing-Container-based-Software-Development-Environments}. Verfügbar sein müssen solche Werkzeuge teilweise auf den verschiedenen Ebenen der Softwareentwicklung, wie Entwicklungs"=, Test"= oder, seltener, auch Produktivumgebungen. Dass sie dort jeweils in unterschiedlichem Umfang und unterschiedlicher Konfiguration eingesetzt werden, führt zu einem Neu- oder Rekonfigurationsbedarf nach jedem Kontextwechsel \cite{003:Infrastructure-from-Code}.

Neue Komplexität verlangt nach neuen Lösungen und Entwicklungsmethodiken haben sich im Laufe der Zeit vielfältig entwickelt und gewandelt. Eine Umfrage \textit{State of Developer Ecosystem} von \textit{JetBrains} richtet sich an Entwickler weltweit. 2023 gab über die Hälfte von ihnen an, an der Entwicklung von Infrastruktur beteiligt zu sein, ein Fünftel von ihnen habe eine Schlüsselrolle in diesem Bereich \cite{207:Developer-Ecosystem}. Während früher Wasserfallmodelle in der Softwareentwicklung üblich waren, werden Softwareteams zunehmend agiler. Vorgehensmodelle und Methodiken wie Scrum, \hyperref[sec:03-01_devops]{DevOps} oder \hyperref[sec:03-03_gitops]{GitOps} verändern Arbeitsweisen und Prozesse. \textit{Stack Overflow} ermittelte 2024, dass 69 \% der Entwickler \gls{ci} und \gls{cd} nutzen sowie 58 \% im Bereich \hyperref[sec:03-01_devops]{DevOps} unterwegs seien \cite{206:Developer-Survey-2024}. \hyperref[sec:03-03_gitops]{GitOps} ist eine relativ neue Methodik.

Der \textit{Hype Cycle of Emerging Technologies} des Marktforschungsunternehmens \textit{Gartner} liefert jährlich Daten zur Bewertung des Reifegrads und der Akzeptanz von Technologien. Unterstützen sollen diese Daten unter anderem Führungskräfte bei der Entwicklung nachhaltiger Unternehmensstrategien. Die Kurve des \textit{Gartner Hype Cycles} ist unterteilt in fünf verschiedene Phasen, die jede der auf der Kurve platzierten Technologien in eine Phase ihres Lebenszyklus einordnen. Gibt es einen Durchbruch in einem Forschungsgebiet oder löst ein Proof-of-Concept eine Öffentlichkeitswirkung aus, wird dies als \textbf{Technologischer Auslöser} (englisch \qit{Innovation Trigger}) bezeichnet. Die nächste Phase ist der \textbf{Gipfel der überzogenen Erwartungen} (englisch \qit{Peak of inflated Expactations}), während welchem öffentlich intensiv über Erfolge und Misserfolge berichtet wird. Einige Unternehmen beginnen in dieser Phase Investitionen in eine Technologie. Flachen die Erfolgs- und Misserfolgsmeldungen sowie in deren Folge auch das öffentliche Interesse ab, befindet sich die Technologie im \textbf{Tal der Enttäuschungen} (englisch \qit{Trough of Disillusionment}). Die vierte Phase, der \textbf{Pfad der Erleuchtungen} (englisch \qit{Slope of Enlightenment}) bringt ein besseres Technologieverständnis und mehr Unternehmensinvestitionen mit sich. Schlussendlich erreicht eine Technologie das \textbf{Plateau der Produktivität} (englisch \qit{Plateau of Productivity}). Die Technologie hat sich etabliert und ist gereift, erste Investitionen zahlen sich aus \cite{108:Gartner-Hype-Cycle}.

Im Jahr der Erstellung dieser Arbeit identifizierte der \textit{Gartner Hype Cycle 2024} (siehe \autoref{fig:g-02_gartner-hype-cycle-2024}) den großen Bereich Entwicklerproduktivität. Unter diesem Begriff befindet neben \gls{cloud-native}, Entwicklerportalen, \Gls{prompt-engineering} und Web Assembly auch \hyperref[sec:03-03_gitops]{GitOps} kurz vor dem Höhepunkt der Kurve, dem \textbf{Gipfel der überzohenen Erwartungen} \cite{106:Gartner-2024-Hype-Cycle-for-Emerging-Technologies}. Es ist also möglich, dass diese neue Technologie sich durchsetzt, die letzte Lebenszyklusphase erreicht und sich in Unternehmen etabliert.

\begin{figure}[h]
    \centering
    \includegraphics[width=0.75\textwidth]{g-02_gartner-hype-cycle-2024.png}
    \caption{Gartner Hype Cycle for Emerging Technologies 2024 \cite{106:Gartner-2024-Hype-Cycle-for-Emerging-Technologies}}
    \label{fig:g-02_gartner-hype-cycle-2024}
\end{figure}

Unter dem Überbegriff Entwicklerproduktivität benennt \textit{Gartner} unter anderem Produktivität und \Gls{developer-experience}. Diese Konzepte sollen die Zufriedenheit und Zusammenarbeit von Entwicklern erhöhen. Von möglichen neuen Technologien in diesem Feld wird erwartet, dass sie die Qualität von Sofwtareprodukten schnell und nachhaltig verbessern \cite{107:Spotlight-on-2024-Gartner-Hype-Cycle-for-Emerging-Technologies}.

Einen weiteren, ähnlich großen Einfluss auf das Software Engineering haben die Technologien der Cloud und Virtualisierung. Auch sie verändern Prozesse und Methoden, insbesondere beim Einsatz im Bereich der Entwicklung, also dem tatsächlichen Schreiben von Quellcode \cite{014:Managing-Container-based-Software-Development-Environments}.
