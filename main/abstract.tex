\clearpage

\begin{center}
    {\LARGE\textbf{Kurzfassung}}
\end{center}
\vspace{1em}

Modernes Software Engineering wird durch eine Vielzahl technischer Werkzeuge und organisatorischer Methoden unterstützt, welche Entwicklerinnen und Entwickler aktiv bei der Erstellung hochqualitativer Software begleiten. In der Praxis führt dies jedoch zu einer wachsenden Komplexität von Entwicklungs- und Bereitstellungsumgebungen. In beiden dieser Bereiche, Development und Deployment, bestehen unterschiedliche und teils sehr spezifische Anforderungen. Steigt die Menge verwendeter Technologien und Werkzeuge, die Toolchain eines Projekts, so vergrößert sich auch die Lücke zwischen den beiden Bereichen. Dieses im Rahmen dieser Arbeit als \q{Development-Deployment-Gap} bezeichnete Phänomen führt zu Herausforderungen, indem es die Entwicklung zeitaufwändig und fehleranfällig macht.

Motiviert durch ein Projekt im Application Management eines großen Automobilherstellers beschäftigt sich die vorliegende Arbeit mit der Frage, welche Ansätze bereits bestehen, um diesem Trend entgegenzuwirken und welche Anforderungen Entwicklerinnen und Entwickler an eine Lösung stellen. Auf Basis von Literaturrecherche und Experteninterviews wird eine \q{Toolchain-as-Code Strategie} entwickelt, deren Ziel eine Verlagerung und Zentralisierung von Konfigurationen jeglicher Art in Software Repositories ist, wodurch eine Durchgängigkeit von Toolchains erreicht werden soll. Die Untersuchung bezieht sich dabei auf Webentwicklungsprojekte mit Microservices Architekturen. Unter Berücksichtigung von Sicherheits-, Wartbarkeits- und Performanzaspekten sollen Best Practices für die Entwicklung mit Software Containern erarbeitet werden. Eine auf die Erkenntnisse der Arbeit aufbauende, prototypische Umsetzung der Strategie soll Realisierbarkeit und Mehrwerte aufzeigen sowie einen Implementierungsleitfaden bereitstellen.

\vfil

\begin{center}
    {\LARGE\textbf{Abstract}}
\end{center}
\vspace{1em}

Modern software engineering relies on a wide range of technical tools and organizational methods that support developers in building high-quality software, yet this variety also introduces challenges, particularly with growing complexity in development and deployment environments. Each area, development and deployment, comes with its own set of sometimes very specific requirements, and as projects adopt more technologies and tools, a \q{development-deployment-gap} can emerge, making processes more time-consuming and prone to errors.

Based on a project in the application management division of a major automotive manufacturer, this study examines current strategies to address this gap and identifies developer's key requirements for an effective solution. Using a combination of literature review and expert interviews, this work develops a \q{toolchain-as-code} strategy, that moves and centralizes configuration of tools in software repositories, leading to an improvement of overall toolchain consistency. This thesis aims to establish best practices for development with software containers in the context of web development and microservice architectures, while taking security, maintainability, and performance into account. A prototype based on the findings of this study will demonstrate the strategy’s feasibility and value, providing a practical guide for implementation.
