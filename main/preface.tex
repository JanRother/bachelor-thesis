\clearpage

{\LARGE\textbf{Vorwort}}
\vspace{1em}

\itshape

Die Idee zur vorliegenden Abschlussarbeit mit dem Ziel der Erreichung des akademischen Grades \degree\space im Rahmen eines Dualen Studiums im Studiengang \studies\space ist im Application Management eines großen Automobilherstellers entstanden. Sie soll einen Beitrag zur Verbesserung der Entwicklung und Bereitstellung von Software leisten und einen positiven Einfluss auf die tägliche Arbeit von Entwicklerinnen und Entwicklern haben.

Zum Zweck der besseren Lesbarkeit wird in dieser Abhandlung größtenteils auf die geschlechtsspezifische Schreibweise verzichtet. Das gewählte generische Maskulinum bezieht sich jedoch immer zugleich auf weibliche und männliche Personen, sofern dies für die Aussage erforderlich ist.

An dieser Stelle möchte ich allen an der Entstehung dieser Arbeit beteiligten Personen, die diese Arbeit in ihrer finalen Form erst möglich gemacht haben, meinen großen Dank aussprechen.

Besonders bedanken möchte ich mich bei meinem Professor und Erstbetreuer \textbf{Prof. Dr. Hans Grönniger} für die Begleitung bei der Entwicklung eines Themas und seine wertvollen Gedanken während der fachlichen Ausarbeitung. Ebenso danke ich meinem Zweitbetreuer \textbf{Dr. Daniel Fruhner} für die hilfreichen Impulse, die zahlreichen Stunden des Austauschs und die motivierenden Worte während der gesamten Zeit. Beiden danke ich, dass sie mir geduldig und ergebnisorientiert mit ihrer akademischen und technischen Expertise zur Seite standen.

Wichtige technische Impulse erhielt ich außerdem von den Kollegen, die sich kurzfristig Zeit nahmen, um als Experte die zentralen Fragen der Interviews zu beantworten. Dafür bedanke ich mich bei \textbf{Felix Affeldt}, \textbf{Barna Kocsis}, \textbf{Sven Frerichs} und \textbf{Jens Dornieden}.

Ich schätze sehr die Unterstützung meiner Freunde, die einige Stunden opferten, um in den letzten Tagen vor der Abgabe dieser Arbeit wichtige Anmerkungen und Verbesserungsvorschläge zu geben. Explizit danke ich \textbf{Jonas Beking} für sein Interesse und seine konstruktive Kritik. Ein ganz besonderer Dank gilt \textbf{Jan Loewe}, der nicht nur am Ende in Form seiner detaillierten Korrekturlesung unterstützte, sondern bereits während des Entstehungsprozesses seine fachliche Expertise einfließen ließ und oft die richtigen Ideen bei der Behebung hartnäckiger Fehler lieferte.

Ich bedanke mich außerdem bei meinen Kolleginnen und Kollegen, die den Entstehungsprozess dieser Arbeit interessiert begleitet haben und einige Ideen einfließen ließen. Ganz besonders danke ich hier \textbf{Lars Kröhn} für seinen unermüdlichen Einsatz und sein persönliches Engagement. Er ermöglichte es mir, mich voll und ganz auf die Erstellung dieser Arbeit zu konzentrieren und begleitete mich bereits während eines Großteils meines Studiums.

Meiner Familie danke ich für ihre stetige Ermutigung und ihr Vertrauen in mich. \textbf{Lars Rother}, \textbf{Mandy Rother} und \textbf{Finja Rother} haben jeden meiner wichtigen Schritte mit mir geteilt und mich liebevoll auch in schwierigen Zeiten auf meinem akademischen, beruflichen und privaten Weg begleitet. Von Herzen danke ich außerdem meiner Partnerin und Stütze \textbf{Ronja Rosenbach}. Ihr Verständnis und ihre Rücksicht, ihre Geduld und ihr Vertrauen sowie ihre aufbauenden Worte während der Entstehung dieser Arbeit haben einen wirklich unschätzbaren Wert für mich.

\normalfont

\vspace{1em}
\begin{flushright}
    \textbf{\studentAName,\xspace \documentMonthOfYear \xspace \documentYear}
\end{flushright}
