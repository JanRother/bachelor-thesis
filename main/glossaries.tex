% GLOBAL GLOSSARY



% ------------------------------------------------
% GLOSSARIES

\newglossaryentry{developer-experience}{
    name={Developer Experience},
    description={wird beeinflusst durch verschiedene Faktoren, die Personen, welche in Aktivitäten der Softwareentwicklung eingebunden sind, betreffen, wobei ein unmittelbarer Einfluss auf Ergebnisse von Softwareentwicklungsprojekten besteht \cite{017:Developer-Experience-Concept-and-Definition}}
}

\newglossaryentry{deployment}{
    name={Deployment},
    description={bezeichnet die Installation von gebauter Software in bestimmten Umgebungen, aber nicht notwendigerweise deren Bereitstellung für tatsächliche Nutzer \cite{005:Continous-Software-Engineering-and-Beyond}}
}

\newglossaryentry{cloud-native}{
    name={cloud-native},
    description={beschreibt einen Ansatz zum Bauen, Deployen und Verwalten von modernen Applikationen in Cloud Umgebungen \cite{011:Cloud-Native-CI-CD-Platform,110:Cloud-Native}, wobei Software entsteht, die skalierbar, flexibel und resilient ist \cite{110:Cloud-Native}}
}

\newglossaryentry{prompt-engineering}{
    name={Prompt Engineering},
    description={bezeichnet einen Prozess, bei welchem durch Optimierung und Veränderung der zur Generierung von Inhalten eingegebenen Daten die Qualität von Ergebnissen Generativer Künstlicher Intelligenzen verbessert werden kann \cite{900:Herausforderungen-Strategien-Kuenstliche-Intelligenz-Software-Engineering}}
}

\newglossaryentry{git}{
    name={Git},
    description={ist ein \Gls{vcs}, welches die Nachverfolgung von Änderungen in Dateien \cite{023:Setting-up-CI-CD-Pipeline-in-the-Cloud-for-Web-Application,301:About-GitHub-and-Git} und eine effizientere und gleichzeitigere Zusammenarbeit mehrerer Entwickler an einem Projekt ermöglicht \cite{023:Setting-up-CI-CD-Pipeline-in-the-Cloud-for-Web-Application}
    }
}



% ------------------------------------------------
% ACRONYMS

\newacronym[shortplural=IDEs, longplural=Integrated Development Environments]{ide}{IDE}{Integrated Development Environment}
\newacronym[shortplural=VCSs, longplural=Version Control Systems]{vcs}{VCS}{Version Control System}
\newacronym{ci}{CI}{Continous Integration}
\newacronym{cd}{CD}{Continous Deployment}
\newacronym{am}{AM}{Application Management}
\newacronym{it}{IT}{Informationstechnik}
\newacronym{te}{TE}{Technische Entwicklung}
