% GLOBAL GLOSSARY



% ------------------------------------------------
% GLOSSARIES

\newglossaryentry{developer-experience}{
    name={Developer Experience},
    description={wird beeinflusst durch verschiedene Faktoren, die Personen, welche in Aktivitäten der Softwareentwicklung eingebunden sind, betreffen, wobei ein unmittelbarer Einfluss auf Ergebnisse von Softwareentwicklungsprojekten besteht \cite{017:Developer-Experience-Concept-and-Definition}}
}

\newglossaryentry{deployment}{
    name={Deployment},
    description={bezeichnet die Installation von gebauter Software in bestimmten Umgebungen, aber nicht notwendigerweise deren Bereitstellung für tatsächliche Nutzer \cite{005:Continous-Software-Engineering-and-Beyond}}
}

\newglossaryentry{cloud-native}{
    name={cloud-native},
    description={beschreibt einen Ansatz zum Bauen, Deployen und Verwalten von modernen Applikationen in Cloud Umgebungen \cite{011:Cloud-Native-CI-CD-Platform,110:Cloud-Native}, wobei Software entsteht, die skalierbar, flexibel und resilient ist \cite{110:Cloud-Native}}
}

\newglossaryentry{prompt-engineering}{
    name={Prompt Engineering},
    description={bezeichnet einen Prozess, bei welchem durch Optimierung und Veränderung der zur Generierung von Inhalten eingegebenen Daten die Qualität von Ergebnissen Generativer Künstlicher Intelligenzen verbessert werden kann \cite{900:Herausforderungen-Strategien-Kuenstliche-Intelligenz-Software-Engineering}}
}

\newglossaryentry{git}{
    name={Git},
    description={ist ein \Gls{vcs}, welches die Nachverfolgung von Änderungen in Dateien \cite{023:Setting-up-CI-CD-Pipeline-in-the-Cloud-for-Web-Application,301:About-GitHub-and-Git} und eine effizientere und gleichzeitigere Zusammenarbeit mehrerer Entwickler an einem Projekt ermöglicht \cite{023:Setting-up-CI-CD-Pipeline-in-the-Cloud-for-Web-Application}
    }
}

\newglossaryentry{development}{
    name={Development},
    description={umfasst die hauptsächlichen Softwareentwicklungsaktivitäten wie Anforderungsanalyse, Design, Implementierung und Verifikation \cite{005:Continous-Software-Engineering-and-Beyond} beziehungsweise im Kontext dieser Arbeit \acrshort{iAa} \citeauthor{211:Coding-and-Development-Phase-in-Software-Engineering} den Teil von ihnen, in welchem die tatsächliche Schaffung der Software stattfindet, also die Entwicklung, Integration und Dokumentation von Quellcode}
}

\newglossaryentry{legacy-system}{
    name={Legacy System},
    plural={Legacy Systeme},
    description={ist ein historisch gewachsenes Altsystem, welches über lange Zeit genutzt, betrieben und gewartet wird und nach wie vor in Verwendung ist \cite{105:Legacy-Systeme-modernisieren}}
    descriptionplural={sind historisch gewachsene Altsysteme, welche über lange Zeit genutzt, betrieben und gewartet werden und nach wie vor in Verwendung sind \cite{105:Legacy-Systeme-modernisieren}}
}

\newglossaryentry{delivery}{
    name={delivery},
    description={baut auf \Gls{deployment} auf, stellt jedoch zusätzlich die Bereitstellung der Software für Kunden sicher \cite{005:Continous-Software-Engineering-and-Beyond}}
}

\newglossaryentry{codebase}{
    name={Codebase},
    description={kann eine oder mehrere Instanzen einer Revisionsverfolgungsdatenbank (in einem \Gls{vcs} wie beispielsweise \Gls{git}) sein \cite{101:The-Twelve-Factor-App}}
}



% ------------------------------------------------
% ACRONYMS

\newacronym{iAa}{i. A. a.}{in Anlehnung an}

\newacronym[shortplural=RQs, longplural=Forschungsfragen]{rq}{RQ}{Forschungsfrage}

\newacronym[shortplural=IDEs, longplural=Integrated Development Environments]{ide}{IDE}{Integrated Development Environment}
\newacronym[shortplural=VCSs, longplural=Version Control Systems]{vcs}{VCS}{Version Control System}
\newacronym{ci}{CI}{Continous Integration}
\newacronym{cd}{CD}{Continous Deployment}
\newacronym{cicd}{CI/CD}{Continous Integration / Continous Deployment}
\newacronym{am}{AM}{Application Management}
\newacronym{it}{IT}{Informationstechnik}
\newacronym{te}{TE}{Technische Entwicklung}
\newacronym[shortplural=VPNs, longplural=Virtual Private Networks]{vpn}{VPN}{Virtual Private Network}
\newacronym{html}{HTML}{Hyper Text Markup Language}
\newacronym{css}{CSS}{Cascading Style Sheets}
\newacronym{rest}{REST}{Representational State Transfer}
\newacronym{rpc}{RPC}{Remote Procedure Calls}
