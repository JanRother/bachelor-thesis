% GLOBAL GLOSSARY



% ------------------------------------------------
% GLOSSARIES

\newglossaryentry{developer-experience}{
    name={Developer Experience},
    description={wird beeinflusst durch verschiedene Faktoren, die Personen, welche in Aktivitäten der Softwareentwicklung eingebunden sind, betreffen, wobei ein unmittelbarer Einfluss auf Ergebnisse von Softwareentwicklungsprojekten besteht \cite{017:Developer-Experience-Concept-and-Definition}}
}

\newglossaryentry{deployment}{
    name={Deployment},
    description={bezeichnet die Installation von gebauter Software in bestimmten Umgebungen, aber nicht notwendigerweise deren Bereitstellung für tatsächliche Nutzer \cite{005:Continous-Software-Engineering-and-Beyond}}
}

\newglossaryentry{cloud-native}{
    name={cloud-native},
    description={beschreibt einen Ansatz zum Bauen, Deployen und Verwalten von modernen Applikationen in Cloud Umgebungen \cite{011:Cloud-Native-CI-CD-Platform,110:Cloud-Native}, wobei Software entsteht, die skalierbar, flexibel und resilient ist \cite{110:Cloud-Native}}
}

\newglossaryentry{prompt-engineering}{
    name={Prompt Engineering},
    description={bezeichnet einen Prozess, bei welchem durch Optimierung und Veränderung der zur Generierung von Inhalten eingegebenen Daten die Qualität von Ergebnissen Generativer Künstlicher Intelligenzen verbessert werden kann \cite{900:Herausforderungen-Strategien-Kuenstliche-Intelligenz-Software-Engineering}}
}

\newglossaryentry{git}{
    name={Git},
    description={ist ein \Gls{vcs}, welches die Nachverfolgung von Änderungen in Dateien \cite{023:Setting-up-CI-CD-Pipeline-in-the-Cloud-for-Web-Application,301:About-GitHub-and-Git} und eine effizientere und gleichzeitigere Zusammenarbeit mehrerer Entwickler an einem Projekt ermöglicht \cite{023:Setting-up-CI-CD-Pipeline-in-the-Cloud-for-Web-Application}
    }
}

\newglossaryentry{development}{
    name={Development},
    description={umfasst die hauptsächlichen Softwareentwicklungsaktivitäten wie Anforderungsanalyse, Design, Implementierung und Verifikation \cite{005:Continous-Software-Engineering-and-Beyond} beziehungsweise im Kontext dieser Arbeit \acrshort{iAa} \citeauthor{211:Coding-and-Development-Phase-in-Software-Engineering} den Teil dieser Aktivitäten, in welchem die tatsächliche Schaffung der Software stattfindet, also die Entwicklung, Integration und Dokumentation von Quellcode}
}

\newglossaryentry{legacy-system}{
    name={Legacy System},
    plural={Legacy Systeme},
    description={ist ein historisch gewachsenes Altsystem, welches über lange Zeit genutzt, betrieben und gewartet wird, nach wie vor in Verwendung ist \cite{105:Legacy-Systeme-modernisieren} und als Phänomen besonders häufig in größeren Organisationen auftritt},
    descriptionplural={sind historisch gewachsene Altsysteme, welche über lange Zeit genutzt, betrieben und gewartet werden, nach wie vor in Verwendung sind \cite{105:Legacy-Systeme-modernisieren} und als Phänomen besonders häufig in größeren Organisationen auftreten}
}

\newglossaryentry{delivery}{
    name={Delivery},
    description={baut auf \Gls{deployment} auf, stellt jedoch zusätzlich die Bereitstellung der Software für einen Kunden sicher \cite{005:Continous-Software-Engineering-and-Beyond}}
}

\newglossaryentry{codebase}{
    name={Codebase},
    description={kann eine oder mehrere Instanzen einer Revisionsverfolgungsdatenbank (in einem \Gls{vcs} wie beispielsweise \Gls{git}) sein \cite{101:The-Twelve-Factor-App}}
}

\newglossaryentry{container-registry}{
    name={Container Registry},
    plural={Container Registries},
    description={ist ein Repository oder eine Sammlung von Repositories, welches zur Ablage und zum Zugriff auf Container Images verwendet und direkt mit Containerization Tools wie \textit{Docker} verbunden wird, wobei es öffentlich oder privat sein kann \cite{111:Container-Registry}},
    descriptionplural={sind Repositories oder Sammlungen von Repositories, welche zur Ablage und zum Zugriff auf Container Images verwendet und direkt mit Containerization Tools wie \textit{Docker} verbunden werden, wobei sie öffentlich oder privat sein können \cite{111:Container-Registry}}
}

\newglossaryentry{github}{
    name={GitHub},
    description={ist eine cloud-basierte Plattform, auf der Quellcode gespeichert, geteilt und gemeinsam mit anderen bearbeitet werden kann \cite{301:About-GitHub-and-Git}}
}

\newglossaryentry{github-actions}{
    name={GitHub Actions},
    description={ist eine cloud-basierte Plattform für \acrfull{ci} und \acrfull{cd}, die das automatisieren von \Gls{build}, Test und \Gls{deployment} Pipelines ermöglicht, wobei individuelle Ereignisse als Auslöser für einen so genannten Workflow konfigurierbar sind \cite{302:Understanding-GitHub-Actions}}
}

\newglossaryentry{image}{
    name={Image},
    plural={Images},
    description={im Kontext dieser Arbeit bezeichnet eine als Container ausführbare Kompositionen von Softwarekomponenten, welche über eine Datei gebaut werden kann, die Anweisungen zum Aufbau und den installierten Paketen enthält \cite{016:Effectively-managing-all-of-those-Applications}},
    descriptionplural={im Kontext dieser Arbeit bezeichnen als Container ausführbare Kompositionen von Softwarekomponenten, welche über eine Datei gebaut werden können, die Anweisungen zum Aufbau und den installierten Paketen enthält \cite{016:Effectively-managing-all-of-those-Applications}}
}

\newglossaryentry{build}{
    name={Build},
    plural={Builds},
    description={beschreibt die Transformation eines Repositorys in eine oder mehrere ausführbare Dateien \cite{101:The-Twelve-Factor-App}},
    descriptionplural={beschreiben die Transformation eines Repositorys in eine oder mehrere ausführbare Dateien \cite{101:The-Twelve-Factor-App}}
}

\newglossaryentry{release}{
    name={Release},
    plural={Releases},
    description={kombiniert die Artefakte aus dem \Gls{build} mit der Konfiguration einer Applikation \cite{101:The-Twelve-Factor-App}},
    descriptionplural={kombinieren die Artefakte aus dem \Gls{build} mit der Konfiguration einer Applikation \cite{101:The-Twelve-Factor-App}}
}

\newglossaryentry{run}{
    name={Run},
    plural={Runs},
    description={bedeutet, dass die Prozesse einer Applikation und somit die Applikation selbst gestartet wird \cite{101:The-Twelve-Factor-App}},
    descriptionplural={bedeutet, dass die Prozesse einer Applikation und somit die Applikation selbst gestartet wird \cite{101:The-Twelve-Factor-App}}
}

\newglossaryentry{hostsystem}{
    name={Hostsystem},
    plural={Hostsysteme},
    description={bezeichnet hier einen Computer mit Betriebssystem, auf welchem beispielsweise Docker betrieben wird \cite{022:Automated-Cloud-Infrastructure-Continous-Integration-and-Continous-Delivery-using-Docker}}
    descriptionplural={bezeichnen hier Computer mit Betriebssystem, auf welchen beispielsweise Docker betrieben wird \cite{022:Automated-Cloud-Infrastructure-Continous-Integration-and-Continous-Delivery-using-Docker}}
}



% ------------------------------------------------
% ACRONYMS

\newacronym{etal}{et al.}{et alii / und andere}
\newacronym{iAa}{i. A. a.}{in Anlehnung an}
\newacronym{vgl}{vgl.}{vergleiche}
\newacronym{na}{N/A}{not available / nicht angegeben}

\newacronym[shortplural=RQs, longplural=Forschungsfragen]{rq}{RQ}{Forschungsfrage}
\newacronym[shortplural=IQs, longplural=Interviewfragen]{iq}{IQ}{Interviewfrage}
\newacronym[shortplural=IPs, longplural=Interviewpersonen]{ip}{IP}{Interviewpersonen}
\newacronym[shortplural=P., longplural=Punkte]{p}{P.}{Punkt}
\newacronym{dev}{DEV}{\Gls{development}}
\newacronym{dep}{DEP}{\Gls{deployment}}
\newacronym{cnt}{CNT}{Continuity, hier Durchgängigkeit}

\newacronym[shortplural=IDEs, longplural=Integrated Development Environments]{ide}{IDE}{Integrated Development Environment}
\newacronym[shortplural=VCSs, longplural=Version Control Systems]{vcs}{VCS}{Version Control System}
\newacronym{ci}{CI}{Continous Integration}
\newacronym{cd}{CD}{Continous Deployment}
\newacronym{cde}{CDE}{Continous Delivery}
\newacronym{cicd}{CI/CD}{Continous Integration / Continous Deployment}
\newacronym{am}{AM}{Application Management}
\newacronym{sdc}{SDC}{Software Development Center}
\newacronym{it}{IT}{Informationstechnik}
\newacronym{te}{TE}{Technische Entwicklung}
\newacronym[shortplural=VPNs, longplural=Virtual Private Networks]{vpn}{VPN}{Virtual Private Network}
\newacronym{html}{HTML}{Hyper Text Markup Language}
\newacronym{css}{CSS}{Cascading Style Sheets}
\newacronym{rest}{REST}{Representational State Transfer}
\newacronym{rpc}{RPC}{Remote Procedure Calls}
\newacronym{iac}{IaC}{Infrastructure-as-Code}
\newacronym{loc}{LoC}{Lines of Code}
\newacronym{mime}{MIME}{Multipurpose Internet Mail Extensions}
\newacronym{wsl}{WSL}{Windows Subsystem for Linux}
\newacronym{http}{HTTP}{Hyper Text Transfer Protocol}
\newacronym{https}{HTTPS}{Hyper Text Transfer Protocol Secure}
\newacronym[shortplural=SDKs, longplural=Software Development Kits]{sdk}{SDK}{Software Development Kit}
\newacronym{yaml}{YAML}{YAML ain't Markup Language}
\newacronym{gui}{GUI}{Graphical User Interface}
\newacronym{json}{JSON}{JavaScript Object Notation}
\newacronym{poc}{PoC}{Proof-of-Concept}
\newacronym{mlops}{MLOps}{Maschine Learning Operations}
\newacronym{ki}{KI}{Künstliche Intelligenz}
